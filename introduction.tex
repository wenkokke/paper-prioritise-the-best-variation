% !TeX root = priorities.tex
\documentclass[main.tex]{subfiles}

\begin{document}
\section{Introduction}

We introduce Priority GV (PGV), a~session-typed concurrent $\lambda$-calculus.

\begin{itemize}
\item
  Historically, calculi in the GV family~\cite{wadler15,lindleymorris15} have achieved deadlock freedom via a syntactic restriction, i.e., by combining channel creation and thread spawning into a single operation, called fork, corresponding to the logical cut in CP, itself the combination of a name restriction and a parallel composition.
  Unfortunately, this is overly restrictive, as it limits the possible communication structures to trees. The calculus GV is based on, on the other hand, did not satisfy deadlock freedom~\cite{gayvasconcelos12}.
\item
  Recent developments in CP have led to various approaches to decoupling these constructs, either while maintaining the strong correspondence to logic, as in Hypersequent CP~\cite[HCP]{kokkemontesi19popl,kokkemontesi19tlla}, or by weakening the correspondence to logic in exchange for a more expressive language, as in Priority CP~\cite[PCP]{dardhagay18}.
\item
  PCP decouples CP's cut into separate constructs for name restriction and parallel composition, and restores deadlock freedom by adding priorities \`{a} la~\cite{kobayashi06}.
\item
  Following the example set by PCP, we present Priority GV (PGV), a~variant of GV which decouples channel creation from thread creation, and restores deadlock freedom by adding priorities.
\item
  We demonstrate the correspondence between PGV and PCP by proving operational correspondence for a translation from PCP to PGV.
\item
  The benefits of using a functional language as opposed to a process calculus are that:
  \begin{itemize}
  \item 
    it supports higher-order functions (and therefore abstraction) not usually present in process calculi; and
  \item
    it allows us to derive extensions of the communication fragment of the language via well-understood extensions of the functional fragment, i.e., deriving internal/external choice from sum types.
  \end{itemize}
\item
  The benefits of using GV over other session-typed functional languages are that GV has strong ties to linear logic, via its relation to CP~\cite{wadler15}, and consequently it has strong formal properties, e.g., deadlock freedom.
\end{itemize}

\paragraph*{Contributions}
\begin{itemize}
\item PGV (typing rules, operational semantics, subject reduction, closed progress).
\item An updated version of PCP, in which we remove the commuting conversions, moving away from reduction as cut elimination, and towards reduction as one would expect for a process calculus.
\item A~translation from PCP into PGV (type preservation, operational correspondence).
\end{itemize}

\end{document}