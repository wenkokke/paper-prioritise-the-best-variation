\documentclass[main.tex]{subfiles}

\begin{document}
\section{Introduction}
Session types~\cite{honda93,takeuchihonda94,hondavasconcelos98} are types for protocols. Regular types ensure functions are used according to their specification. Session types ensure \emph{communication channels} are used according to their protocols. Session types have been studied in many settings. For instance, in the $\pi$-calculus~\cite{honda93,takeuchihonda94,hondavasconcelos98}, a foundational calculus for communication and concurrency, and in concurrent $\lambda$-calculi~\cite{gayvasconcelos12}, including the focus of our paper: Good Variation~\cite[GV]{wadler15,lindleymorris15}.

GV is a concurrent $\lambda$-calculus with \emph{binary} session types, where each channel is shared between exactly two processes. Binary session types guarantee two crucial properties \emph{communication safety}---\eg, if the protocol says to transmit an integer, you transmit an integer---and \emph{session fidelity}---\eg, if the protocol says send, you send. A third crucial property is \emph{deadlock freedom}, which ensures that processes do not have cyclic dependencies---\eg, when two processes wait for each other to send a value. Binary session types \emph{alone} are insufficient to rule out deadlocks arising from interleaved sessions, but several additional techniques have been developed to guarantee deadlock freedom in session-typed $\pi$-calculus and concurrent $\lambda$-calculus.

In the $\pi$-calculus literature, there is a growing line of work in developing Curry-Howard correspondences between session-typed $\pi$-calculus and linear logic~\cite{girard87}: Caires and Pfenning's $\pi$DILL~\cite{cairespfenning10} corresponds to dual intuitionistic linear logic~\cite{barber96}, and Wadler's Classical Processes~\cite[CP]{wadler14} corresponds to classical linear logic~\cite[CLL]{girard87}. Both calculi guarantee deadlock freedom, which they achieve by restricting structure of processes and shared channels to \emph{trees}, by combing name restriction and parallel composition into a single construct, corresponding to the logical cut. This ensures that two processes can only communicate via exactly one series of channels, which rules out interleavings of sessions, and guarantees deadlock freedom.
There are many downsides to combining name restriction and parallel composition---lack of modularity, difficulty typing structural congruence and formulating label-transition semantics, \etc---which have led to various approaches to decoupling these constructs. Hypersequent CP~\cite{MP18,kokkemontesi19popl,kokkemontesi19tlla} and Linear Compositional Choreographies~\cite{CarboneMS18} decouple them, but maintain the correspondence to CLL and allow only tree-structured processes. Priority CP~\cite[PCP]{dardhagay18extended} weakens the correspondence to CLL in exchange for a more expressive language which allows cyclic-structured processes. PCP decouples CP's cut rule into two separate constructs: one for parallel composition via a mix rule, and one for name restriction via a cycle rule. To restore deadlock freedom, PCP uses \emph{priorities}~\cite{kobayashi06,padovani14}. Priorities encode the \emph{order of actions} and rule out bad interleavings. Dardha and Gay~\cite{dardhagay18extended} prove cycle-elimination for PCP, adapting the cut-elimination proof for classical linear logic, and deadlock freedom follows as a corollary.

CP and GV are related via a pair of translations which satisfy simulation~\cite{lindleymorris16}, and which can be tweaked to satisfy reflection. The two calculi share the same strong guarantees. GV achieves deadlock freedom via a similar syntactic restriction: it combines channel creation and thread spawning into a single operation, called ``fork'', which is related to the cut construct in CP. Unfortunately, as with CP, this syntactic restriction has many downsides.

Our aims are to develop the expressiveness of GV but maintain deadlock freedom. We chose GV for several reasons, some of which apply more generally as benefits to working within a concurrent $\lambda$-calculus as opposed to a name-passing process calculus. Concurrent $\lambda$-calculi support higher-order functions, and have a capability for abstraction not usually present in process calculi. Within a concurrent $\lambda$-calculus, one can derive extensions of the communication capabilities of the language via well-understood extensions of the functional fragment, \eg, we can derive internal/external choice from sum types. Concurrent $\lambda$-calculi maintain a clear separation between the program which the user writes and the configurations which represent the state of the system as it evaluates the program. However, our main motivation is that results obtained for $\lambda$-calculi transfer more easily to real-world functional programming languages. \emph{Case in point}: we were able to embed the theory developed in this paper in Linear Haskell~\cite{bernardyboespflug18} without much effort~\cite{kokkedardha21hs}. For us, these advantages outweigh the benefits of process calculi, \eg, their succinctness.
The benefits of working with GV specifically, as opposed to other concurrent $\lambda$-calculi, is its relation to CP~\cite{wadler14}, and its strong formal properties. We thus pose our research question for GV:
\begin{quotation}
  \textbf{RQ:}
  Can we design a more expressive GV which guarantees deadlock freedom for cyclic-structured processes?
\end{quotation}

We follow the line of work from CP to Priority CP, and present Priority GV (PGV), a~variant of GV which decouples channel creation from thread spawning, thus allowing cyclic-structured processes, but which nonetheless guarantees deadlock freedom via priorities. This closes the circle of the connection between CP and GV~\cite{wadler14}, and their priority-based versions, PCP~\cite{dardhagay18extended} and PGV.
We make the following main contributions:
\begin{enumerate}[labelindent=0pt,labelwidth=1.333\parindent,align=left,labelsep*=0pt,leftmargin=!]
\item[(\secref{sec:pgv})] \textbf{Priority GV}. We present Priority GV (\secref{sec:pgv}, PGV), a session-typed functional language with priorities, and prove subject reduction (\cref{thm:pgv-subject-reduction-confs}) and progress (\cref{thm:pgv-closed-progress-confs}). We addresses several problems in the original GV language, most notably:
  \begin{enumerate*}
  \item PGV does not require the pseudo-type $S^\sharp$; and
  \item its structural congruence is type preserving.
  \end{enumerate*}
  PGV answers our research question positively as it allows cyclic-structured binary session-typed processes that are deadlock free.
\item[(\secref{sec:pcp})] \textbf{Translation from PCP to PGV}.
  We present a \emph{sound and complete encoding} of Priority CP~\cite{dardhagay18extended} in PGV (\secref{sec:pcp}). We prove the encoding preserves typing (\cref{thm:pcp-to-pgv-confs-preservation}) and satisfies operational correspondence (\cref{thm:pcp-to-pgv-operational-correspondence-soundness,thm:pcp-to-pgv-operational-correspondence-completeness}).
  To obtain a tight correspondence, we update PCP, moving away from commuting conversions and reduction as cut elimination towards reduction based on structural congruence, as it is standard in process calculi.
\end{enumerate}
\end{document}
