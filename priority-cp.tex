% !TeX root = priorities.tex
\documentclass[main.tex]{subfiles}

\begin{document}

\section{Relation to Priority CP}\label{sec:pcp}

We present a correspondence between Priority GV and an updated version of Priority CP~\cite[PCP]{dardhagay18}, which is Wadler's CP~\cite{wadler14} with priorities. This correspondence connects PGV to (a~relaxed variant of) classical linear logic.

\subsection{Revisiting Priority CP}
\begingroup
\usingnamespace{pcp}
\textbf{Types} ($\ty{A}$, $\ty{B}$) in PCP correspond to linear logic connectives annotated with priorities $\cs{o}\in\mathbb{N}$. Typing environments, duality, and the priority function $\pr(\cdot)$ are defined as expected (see~\cref{sec:pcp-environments,sec:pcp-duality,sec:pcp-priorities}).
\[
  \begin{array}{lcl}
    \ty{A}, \ty{B}
    & \Coloneqq & \ty{\tytens[\cs{o}]{A}{B}}
      \sep        \ty{\typarr[\cs{o}]{A}{B}}
      \sep        \ty{\tyone[\cs{o}]}
      \sep        \ty{\tybot[\cs{o}]}
      \sep        \ty{\typlus[\cs{o}]{A}{B}}
      \sep        \ty{\tywith[\cs{o}]{A}{B}}
      \sep        \ty{\tynil[\cs{o}]}
      \sep        \ty{\tytop[\cs{o}]}
  \end{array}
\]
\textbf{Processes} ($\tm{P}$, $\tm{Q}$) in PCP are defined by the following grammar.
\[
  \begin{array}[t]{lcl}
    \tm{P}, \tm{Q}
    & \Coloneqq & \tm{\link{x}{y}}
           \sep   \tm{\res{x}{y}{P}}
           \sep   \tm{(\ppar{P}{Q})}
           \sep   \tm{\halt}
    \\   & \sep & \tm{\send{x}{y}{P}}
           \sep   \tm{\close{x}{P}}
           \sep   \tm{\recv{x}{y}{P}}
           \sep   \tm{\wait{x}{P}}
    \\   & \sep & \tm{\inl{x}{P}}
           \sep   \tm{\inr{x}{P}}
           \sep   \tm{\offer{x}{P}{Q}}
           \sep   \tm{\absurd{x}}
  \end{array}
\]
Processes are typed by sequents $\seq{P}{\ty{\Gamma}}$, which correspond to the one-sided sequents in classical linear logic. Differently from PGV, in PCP we do not need to store the greatest priority on the sequent, as, due to the absence of higher-order functions, we cannot compose processes \emph{sequentially}.

PCP decomposes cut into \LabTirName{T-Res} and \LabTirName{T-Par} rules---logically corresponding to cycle and mix rules, respectively---and guarantees deadlock freedom by using priority constraints, \eg, as in \LabTirName{T-Send}. The full typing rules and operational semantics are available in \cref{fig:pcp-operational-semantics,fig:pcp-typing} in \cref{app:revisiting-PCP}:
\begin{mathpar}
  \small
  \inferrule*[lab=T-Res]{
    \seq{P}{\ty{\Gamma},\tmty{x}{A},\tmty{y}{\co{A}}}
  }{\seq{\res{x}{y}{P}}{\ty{\Gamma}}}
  \inferrule*[lab=T-Par]{
    \seq{P}{\ty{\Gamma}}
    \and
    \seq{Q}{\ty{\Delta}}
  }{\seq{\ppar{P}{Q}}{\ty{\Gamma},\ty{\Delta}}}
  \inferrule*[lab=T-Send]{
    \seq{P}{\ty{\Gamma},\tmty{y}{A},\tmty{x}{B}}
    \and
    \cs{o}<\minpr(\ty{\Gamma},\ty{A},\ty{B})
  }{\seq{\send{x}{y}{P}}{\ty{\Gamma},\tmty{x}{\tytens[\cs{o}]{A}{B}}}}
\end{mathpar}

The main change we make to PCP is \emph{removing commuting conversions} and defining an operational semantics based on structural congruence. Commuting conversions are necessary if we want our reduction strategy to correspond \emph{exactly} to cut (or cycle in \cite{dardhagay18}) elimination. 
However, from the perspective of process calculi, commuting conversions behave strangely: they allow an input/output action to be moved to the top of a process, thus potentially blocking actions which were previously possible (an example is given in \cref{app:commuting-conversions}). This makes CP, and PCP in \cite{dardhagay18}, non-confluent.
As Lindley and Morris~\cite{lindleymorris15} show, all communications that can be performed \emph{with} the use of commuting conversions, can also be performed \emph{without} them, if using structural congruence.
%Consider the commuting conversion $(\kappa_{\parr})$ for $\pcp{\tm{\recv{x}{y}{P}}}$:
%\begin{mathpar}
%  (\kappa_{\parr})
%  \quad
%  \tm{\res{z}{z'}{(\ppar{\recv{x}{y}{P}}{Q})}}
%  \red
%  \tm{\recv{x}{y}{\res{z}{z'}{(\ppar{P}{Q})}}}
%\end{mathpar}
%As a result of $(\kappa_{\parr})$, $\pcp{\tm{Q}}$ becomes blocked on $\pcp{\tm{\labrecv{x}{y}}}$, and any actions $\pcp{\tm{Q}}$ was able to perform become unavailable. Consequently, CP is non-confluent:
%\begin{mathpar}
%  \setlength{\arraycolsep}{2em}
%  \begin{array}{cc}
%    \multicolumn{2}{c}{%
%    \hspace*{10ex}
%    {\tm{\res{x}{x'}{(\ppar{\recv{a}{y}{P}}{\res{z}{z'}{(\ppar{\close{z}{\halt}}{\wait{z'}{Q}})}})}}}}
%    \\
%    \qquad\rotatebox[origin=c]{270}{$\red\hphantom{{}^+}$}
%    &
%    \rotatebox[origin=c]{270}{$\red^+$}\qquad
%    \\
%    {\tm{\recv{a}{y}{\res{x}{x'}{(\ppar{P}{\res{z}{z'}{(\ppar{\close{z}{\halt}}{\wait{z'}{Q}})}})}}}}
%    &
%    {\tm{\recv{a}{y}{\res{x}{x'}{(\ppar{P}{Q})}}}}
%  \end{array}
%\end{mathpar}
%

In  particular for PCP, commuting conversions break our intuition that an action with lower priority \emph{occurs before} an action with higher priority. To cite Dardha and Gay~\cite{dardhagay18} ``\emph{if a prefix on a channel endpoint $\pcp{\tm{x}}$ with priority $\cs{o}$ is pulled out at top level, then to preserve priority constraints in the typing rules [..], it is necessary to increase priorities of all actions after the prefix on $\pcp{\tm{x}}$}'' by $\cs{o+1}$.
One benefit of removing commuting conversions is that we no longer need to dynamically change the priorities during reduction, which means that the intuition for priorities holds true in our updated version of PCP. Furthermore, we can safely define reduction on untyped processes, which means that type and priority information is erasable!

We prove closed progress for our updated PCP. The proof is in \cref{prf:thm-pcp-closed-progress}.
\begin{compacttheorems}
  \begin{restatabletheorem}{thmpcpclosedprogress}[Progress, $\pcp{\red}$]
    \label{thm:pcp-closed-progress}
    \hfill\\%newline before theorem statement
    If $\pcp{\seq{P}{\emptyenv}}$, then either $\pcp{\tm{P}=\tm{\halt}}$ or there exists a $\pcp{\tm{Q}}$ such that $\pcp{\tm{P}\red\tm{Q}}$.
  \end{restatabletheorem}
\end{compacttheorems}
\endgroup

\subsection{Correspondence between PGV and PCP}
\begingroup
We illustrate the relation between PCP and PGV by giving a translation from PCP to PGV. The translation on types is defined as follows:
\[
  \begin{array}{lcl}
    \ty{\cpgvT{\pcp{\tytens[\cs{o}]{A}{B}}}}
    & = & \pgv{\ty{\tysend[\cs{o}]{\co{\cpgvT{A}}}{\cpgvT{B}}}}
    \\
    \ty{\cpgvT{\pcp{\typlus[\cs{o}]{A}{B}}}}
    & = & \pgv{\ty{\tyselect[\cs{o}]{\cpgvT{A}}{\cpgvT{B}}}}
  \end{array}
  \;
  \begin{array}{lcl}
    \ty{\cpgvT{\pcp{\tyone[\cs{o}]}}}
    & = & \ty{\pgv{\tyends[\cs{o}]}}
    \\
    \ty{\cpgvT{\pcp{\tynil[\cs{o}]}}}
    & = & \pgv{\ty{\tyselectemp[\cs{o}]}}
  \end{array}
  \;
  \begin{array}{lcl}
    \ty{\cpgvT{\pcp{\typarr[\cs{o}]{A}{B}}}}
    & = & \pgv{\ty{\tyrecv[\cs{o}]{\cpgvT{A}}{\cpgvT{B}}}}
    \\
    \ty{\cpgvT{\pcp{\tywith[\cs{o}]{A}{B}}}}
    & = & \pgv{\ty{\tyoffer[\cs{o}]{\cpgvT{A}}{\cpgvT{B}}}}
  \end{array}
  \;
  \begin{array}{lcl}
    \ty{\cpgvT{\pcp{\tybot[\cs{o}]}}}
    & = & \ty{\pgv{\tyendr[\cs{o}]}}
    \\
    \ty{\cpgvT{\pcp{\tytop[\cs{o}]}}}
    & = & \pgv{\ty{\tyofferemp[\cs{o}]}}
  \end{array}
\]

There are two separate translations on processes. The main translation, $\tm{\cpgvM{\cdot}}$, translates processes to \emph{terms}:
\begin{align*}
  &\pcp{\tm{\cpgvM{\link{x}{y}}}}
  &&= \pgv{\tm{\link\;{\pair{x}{y}}}} \\
  &\pcp{\tm{\cpgvM{\res{x}{y}{P}}}}
  &&= \pgv{\tm{\letpair{x}{y}{\new}{\cpgvM{P}}}} \\
  &\pcp{\tm{\cpgvM{\ppar{P}{Q}}}}
  &&= \pgv{\tm{\andthen{\spawn\;{(\lambda\unit.\cpgvM{P})}}{\cpgvM{Q}}}} \\
  &\pcp{\tm{\cpgvM{\halt}}}
  &&= \pgv{\tm{\unit}} \\
  &\pcp{\tm{\cpgvM{\close{x}{P}}}}
  &&= \pgv{\tm{\andthen{\close\;{x}}{\cpgvM{P}}}} \\
  &\pcp{\tm{\cpgvM{\wait{x}{P}}}}
  &&= \pgv{\tm{\andthen{\wait\;{x}}{\cpgvM{P}}}} \\
  &\pcp{\tm{\cpgvM{\send{x}{y}{P}}}}
  &&= \pgv{\tm{\letpair{y}{z}{\new}{\letbind{x}{\send\;{\pair{z}{x}}}{\cpgvM{P}}}}} \\
  &\pcp{\tm{\cpgvM{\recv{x}{y}{P}}}}
  &&= \pgv{\tm{\letpair{y}{x}{\recv\;{x}}{\cpgvM{P}}}} \\
  &\pcp{\tm{\cpgvM{\inl{x}{P}}}}
  &&= \pgv{\tm{\letbind{x}{\select{\labinl}\;{x}}{\cpgvM{P}}}} \\
  &\pcp{\tm{\cpgvM{\inr{x}{P}}}}
  &&= \pgv{\tm{\letbind{x}{\select{\labinr}\;{x}}{\cpgvM{P}}}} \\
  &\pcp{\tm{\cpgvM{\offer{x}{P}{Q}}}}
  &&= \pgv{\tm{\offer{x}{x}{\cpgvM{P}}{x}{\cpgvM{Q}}}} \\
  &\pcp{\tm{\cpgvM{\absurd{x}}}}
  &&= \pgv{\tm{\offeremp{x}}}
\end{align*}

Unfortunately, the operational correspondence along $\tm{\cpgvM{\cdot}}$ is unsound, as it translates $\nu$-binders and parallel compositions to $\pgv{\tm{\new}}$ and $\pgv{\tm{\spawn}}$, which can reduce to their equivalent configuration constructs using \LabTirName{E-New} and \LabTirName{E-Spawn}. The same goes for $\nu$-binders which are inserted when translating bound send to unbound send. For instance, the process $\pcp{\tm{\send{x}{y}{P}}}$ is blocked, but its translation uses $\pgv{\tm{\new}}$ and can reduce. To address this issue, we use a second translation, $\tm{\cpgvC{\cdot}}$, which is equivalent to $\tm{\cpgvM{\cdot}}$ followed by reductions using \LabTirName{E-New} and \LabTirName{E-Spawn}:
\begin{align*}
  &\pcp{\tm{\cpgvC{\res{x}{y}{P}}}}
  &&= \pgv{\tm{\res{x}{y}{\cpgvC{P}}}}
  \\
  &\pcp{\tm{\cpgvC{\ppar{P}{Q}}}}
  &&= \pgv{\tm{\ppar{\cpgvC{P}}{\cpgvC{Q}}}}
  \\
  &\pcp{\tm{\cpgvC{\send{x}{y}{P}}}}
  &&= \pgv{\tm{\res{y}{z}{(\child\;\letbind{x}{\send\;\pair{z}{x}}{\cpgvM{P}})}}}
  \\
  &\pcp{\tm{\cpgvC{\inl{x}{P}}}}
  &&= \pgv{\tm{\res{y}{z}{(\child\;\letbind{x}{\andthen{\close\;(\send\;\pair{\inl{y}}{x})}{z}}{\cpgvM{P}})}}}
  \\
  &\pcp{\tm{\cpgvC{\inr{x}{P}}}}
  &&= \pgv{\tm{\res{y}{z}{(\child\;\letbind{x}{\andthen{\close\;(\send\;\pair{\inr{y}}{x})}{z}}{\cpgvM{P}})}}}
  \\
  &\pcp{\tm{\cpgvC{P}}}
  &&= \pgv{\tm{\child{\cpgvM{P}}}},\quad\text{if none of the above apply}
\end{align*}

Typing environments are translated pointwise, and sequents $\pcp{\seq{P}{\ty{\Gamma}}}$ are translated as $\pgv{\cseq[\child]{\ty{\cpgvT{\ty{\Gamma}}}}{\cpgvC{P}}}$.
The translations $\tm{\cpgvM{\cdot}}$ and $\tm{\cpgvC{\cdot}}$ preserve typing, and the latter induces a sound and complete operational correspondence. The proofs are standard inductions and can be found in~\cref{prf:lem-pcp-to-pgv-terms-preservation,prf:lem-pcp-to-pgv-terms-preservation,prf:lem-pcp-to-pgv-cpgvM-to-cpgvC,prf:thm-pcp-to-pgv-operational-correspondence-soundness,prf:thm-pcp-to-pgv-operational-correspondence-completeness}.
\begin{compacttheorems}
  \begin{restatablelemma}{lempcptopgvtermspreservation}[Preservation, ${\tm{\cpgvM{\cdot}}}$]
    \label{lem:pcp-to-pgv-terms-preservation}
    If $\pcp{\seq{P}{\ty{\Gamma}}}$, then $\pgv{\tseq[\cs{p}]{\ty{\cpgvT{\Gamma}}}{\cpgvM{P}}{\tyunit}}$.
  \end{restatablelemma}
  \begin{restatabletheorem}{thmpcptopgvconfspreservation}[Preservation, ${\tm{\cpgvC{\cdot}}}$]
    \label{thm:pcp-to-pgv-confs-preservation}
    If $\pcp{\seq{P}{\ty{\Gamma}}}$, then $\pgv{\cseq[\child]{\ty{\cpgvT{\Gamma}}}{\cpgvC{P}}}$.
  \end{restatabletheorem}
  \begin{restatablelemma}{lempcptopgvcpgvMtocpgvC}
    \label{lem:pcp-to-pgv-cpgvM-to-cpgvC}
    For any $\pcp{\tm{P}}$, either:
    \begin{itemize}
    \item $\pgv{\tm{\child\;\cpgvM{P}}=\tm{\cpgvC{P}}}$; or
    \item   $\pgv{\tm{\child\;\cpgvM{P}}\cred^+\tm{\cpgvC{P}}}$, and for any $\pgv{\tm{\conf{C}}}$, if $\pgv{\tm{\child\;\cpgvM{P}}\cred\tm{\conf{C}}}$, then $\pgv{\tm{\conf{C}}\cred^\star\tm{\cpgvC{P}}}$.
    \end{itemize}
  \end{restatablelemma}
  \begin{restatabletheorem}{thmpcptopgvoperationalcorrespondencesoundness}%
    [Operational Correspondence, Soundness, ${\tm{\cpgvC{\cdot}}}$]
    \label{thm:pcp-to-pgv-operational-correspondence-soundness}
    \hfill\\%newline before theorem statement
    If $\pcp{\seq{P}{\ty{\Gamma}}}$ and $\pgv{\tm{\cpgvC{P}}\cred\tm{\conf{C}}}$, there exists a $\tm{Q}$ such that $\pcp{\tm{P}\red^+\tm{Q}}$ and $\pgv{\tm{\conf{C}}\cred^\star\tm{\cpgvC{Q}}}$
  \end{restatabletheorem}
  \begin{restatabletheorem}{thmpcptopgvoperationalcorrespondencecompleteness}%
    [Operational Correspondence, Completeness, ${\tm{\cpgvC{\cdot}}}$]
    \label{thm:pcp-to-pgv-operational-correspondence-completeness}
    \hfill\\%newline before theorem statement
    If $\pcp{\seq{P}{\ty{\Gamma}}}$ and $\pcp{\tm{P}\red\tm{Q}}$,
    then $\pgv{\tm{\cpgvC{P}}\cred^+\tm{\cpgvC{Q}}}$.
  \end{restatabletheorem}
\end{compacttheorems}
\endgroup
\end{document}

%%% Local Variables:
%%% TeX-master: "priorities"
%%% End:
