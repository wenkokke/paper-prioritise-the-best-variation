\documentclass[main.tex]{subfiles}

\begin{document}

\section{Priority CP}
\usingnamespace{pcp}

\begin{figure}[t]
  \centering
  \includegraphics[width=0.8\columnwidth]{scheduler}
  \vspace{-4mm}
  \caption{Cyclic Scheduler from \cite{DardhaG18}}
  \label{fig:scheduler}
  \vspace{-4mm}
  \end{figure}

In this section we present an updated version of Priority CP \cite{DardhaG18}, which is Wadler's Classical Processes \cite{W12} with \emph{priorities}.
Some details of the calculus are inspired by Carbone \emph{et al.} \cite{CLMSW16} and Caires and P\'{e}rez \cite{CP17}.

\subsection{Syntax}
\subsubsection{Priorities, types, and environments}
Types are based on CLL and range over $\ty{A}, \ty{B}, \ty{C}$.
Types are annotated with priorities which range over $\cs{o} \in \mathbb{N} \sqcup \{\omega\}$.
Let $\omega$ be a special element such that $\cs{o}<\omega$ for all $\cs{o}\in \mathbb{N}$.
Often, we will omit $\omega$ when not necessary.
\todo{check the omega thing is also part of our presentation}
Types $\ty{\tyone[\cs{o}]}$ and $\ty{\tybot[\cs{o}]}$
are used to type channel endpoints which are ready to be closed.
Types $\ty{\tynil[\cs{o}]}$ and $\ty{\tytop[\cs{o}]}$ are used to... 
Type $\ty{\tytens[\cs{o}]{A}{B}}$ (respectively, $\ty{\typarr[\cs{o}]{A}{B}}$) is used to type a channel endpoint that first {outputs} (respectively, {inputs}) a channel of type $\ty A$ and then proceeds as $\ty B$.
$\ty{\typlus[\cs{o}]{A}{B}}$ is used to type a channel endpoint over which we can {select} either the \emph{left} branch and proceed as $\ty A$ or the \emph{right} branch and proceed as $\ty B$.
$\ty{\tywith[\cs{o}]{A}{B}}$ is used to type a channel endpoint that can offer a both $\ty A$ and $\ty B$.
A typing context $\ty\Gamma$ is a partial function associating channel names to types.
\[
\begin{array}{lcl}
  \ty{A}, \ty{B}, \ty{C}
  & \coloneqq & \ty{\tyone[\cs{o}]}
    \sep        \ty{\tybot[\cs{o}]}
    \sep        \ty{\tynil[\cs{o}]}
    \sep        \ty{\tytop[\cs{o}]}
    \sep        \ty{\tytens[\cs{o}]{A}{B}}
    \sep        \ty{\typarr[\cs{o}]{A}{B}}
    \sep        \ty{\typlus[\cs{o}]{A}{B}}
    \sep        \ty{\tywith[\cs{o}]{A}{B}}
  \\
  \ty{\Gamma}, \ty{\Delta}
  & \coloneqq & \ty{\emptyenv}
    \sep        \ty{\Gamma}, \tmty{x}{A}
\end{array}
\]

Duality on types is a total function and is defined below. It preserves priorities of types.
\[
\begin{array}{lcl}
  \ty{\co{(\tyone[\cs{o}])}} & = & \ty{\tybot[\cs{o}]} \\
  \ty{\co{(\tybot[\cs{o}])}} & = & \ty{\tyone[\cs{o}]} \\
  \ty{\co{(\tynil[\cs{o}])}} & = & \ty{\tytop[\cs{o}]} \\
  \ty{\co{(\tytop[\cs{o}])}} & = & \ty{\tynil[\cs{o}]}
\end{array}
\qquad
\begin{array}{lcl}
  \ty{\co{(\tytens[\cs{o}]{A}{B})}} & = & \ty{\typarr[\cs{o}]{\co{A}}{\co{B}}} \\
  \ty{\co{(\typarr[\cs{o}]{A}{B})}} & = & \ty{\tytens[\cs{o}]{\co{A}}{\co{B}}} \\
  \ty{\co{(\typlus[\cs{o}]{A}{B})}} & = & \ty{\tywith[\cs{o}]{\co{A}}{\co{B}}} \\
  \ty{\co{(\tywith[\cs{o}]{A}{B})}} & = & \ty{\typlus[\cs{o}]{\co{A}}{\co{B}}}
\end{array}
\]

Finally, function $\pr(\cdot)$ defined over types returns the priority of that type; function $\minpr(\cdot)$ defined over typing contexts returns the \emph{minimum} priority of all types in that context.
Both functions are defined as expected below.
\[
\begin{array}{lclclcl}
  \pr(\ty{\tyone[\cs{o}]})        & = & \cs{o}  \\
  \pr(\ty{\tybot[\cs{o}]})        & = & \cs{o}  \\
  \pr(\ty{\tynil[\cs{o}]})        & = & \cs{o}  \\
  \pr(\ty{\tytop[\cs{o}]})        & = & \cs{o}  \\
  \\
  \minpr(\ty{\emptyenv})          & = & \varnothing
\end{array}
\qquad
\begin{array}{lclclcl}
  \pr(\ty{\tytens[\cs{o}]{A}{B}}) & = & \cs{o}  \\
  \pr(\ty{\typarr[\cs{o}]{A}{B}}) & = & \cs{o}  \\
  \pr(\ty{\typlus[\cs{o}]{A}{B}}) & = & \cs{o}  \\
  \pr(\ty{\tywith[\cs{o}]{A}{B}}) & = & \cs{o}  \\
  \\
  \minpr(\ty{\Gamma},\tmty{x}{A}) & = & \minpr(\ty{\Gamma})\sqcap\minpr(\ty{A})
\end{array}
\]

\subsubsection{Terms}
\[
\begin{array}[t]{lcl}
  \tm{P}, \tm{Q}
  & \coloneqq & \tm{\link{x}{y}}
         \sep   \tm{\res{x}{y}{P}}
         \sep   \tm{(\ppar{P}{Q})}
         \sep   \tm{\halt}
  \\   & \sep & \tm{\send{x}{y}{P}}
         \sep   \tm{\close{x}{P}}
         \sep   \tm{\recv{x}{y}{P}}
         \sep   \tm{\wait{x}{P}}
  \\   & \sep & \tm{\inl{x}{P}}
         \sep   \tm{\inr{x}{P}}
         \sep   \tm{\offer{x}{P}{Q}}
\end{array}
\]

\subsubsection{Syntactic Sugar}
We define \emph{unbound} send:
\[
  \tm{\usend{x}{y}{P}}\elabarrow\tm{\send{x}{z}{(\ppar{\link{y}{z}}{P})}}
\]

\subsection{Typing}

\subsubsection{Typing Rules}
\begin{mathpar}
  \inferrule*[lab=T-Link]{
  }{\seq{\link[\ty{A}]{x}{y}}{\tmty{x}{A},\tmty{y}{\co{A}}}}

  \inferrule*[lab=T-Res]{
    \seq{P}{\ty{\Gamma},\tmty{x}{A},\tmty{y}{\co{A}}}
  }{\seq{\res{x}{y}{P}}{\ty{\Gamma}}}
  \\
  \inferrule*[lab=T-Par]{
    \seq{P}{\ty{\Gamma}}
    \\
    \seq{Q}{\ty{\Delta}}
  }{\seq{\ppar{P}{Q}}{\ty{\Gamma},\ty{\Delta}}}

  \inferrule*[lab=T-Halt]{
  }{\seq{\halt}{\emptyenv}}
  \\
  \inferrule*[lab=T-Send]{
    \seq{P}{\ty{\Gamma},\tmty{y}{A},\tmty{x}{B}}
    \\
    \cs{o}<\minpr(\ty{\Gamma},\ty{A},\ty{B})
  }{\seq{\send{x}{y}{P}}{\ty{\Gamma},\tmty{x}{\tytens[\cs{o}]{A}{B}}}}

  \inferrule*[lab=T-Close]{
    \seq{P}{\ty{\Gamma}}
    \\
    \cs{o}<\minpr(\ty{\Gamma})
  }{\seq{\close{x}{P}}{\ty{\Gamma},\tmty{x}{\tyone[\cs{o}]}}}
  \\
  \inferrule*[lab=T-Recv]{
    \seq{P}{\ty{\Gamma},\tmty{y}{A},\tmty{x}{B}}
    \\
    \cs{o}<\minpr(\ty{\Gamma},\ty{A},\ty{B})
  }{\seq{\recv{x}{y}{P}}{\ty{\Gamma},\tmty{x}{\typarr[\cs{o}]{A}{B}}}}

  \inferrule*[lab=T-Wait]{
    \seq{P}{\ty{\Gamma}}
    \\
    \cs{o}<\minpr(\ty{\Gamma})
  }{\seq{\wait{x}{P}}{\ty{\Gamma},\tmty{x}{\tybot[\cs{o}]}}}
  \\
  \inferrule*[lab=T-Select-Inl]{
    \seq{P}{\ty{\Gamma},\tmty{x}{A}}
    \\
    \cs{o}<\minpr(\ty{\Gamma},\ty{A},\ty{B})
    \\
    \pr(\ty{A})=\pr(\ty{B})
  }{\seq{\inl{x}{P}}{\ty{\Gamma},\tmty{x}{\typlus[\cs{o}]{A}{B}}}}

  \inferrule*[lab=T-Select-Inr]{
    \seq{P}{\ty{\Gamma},\tmty{x}{B}}
    \\
    \cs{o}<\minpr(\ty{\Gamma},\ty{A},\ty{B})
    \\
    \pr(\ty{A})=\pr(\ty{B})
  }{\seq{\inr{x}{P}}{\ty{\Gamma},\tmty{x}{\typlus[\cs{o}]{A}{B}}}}

  \inferrule*[lab=T-Offer]{
    \seq{P}{\ty{\Gamma},\tmty{x}{A}}
    \\
    \seq{Q}{\ty{\Gamma},\tmty{x}{B}}
    \\
    \cs{o}<\minpr(\ty{\Gamma},\ty{A},\ty{B})
  }{\seq{\offer{x}{P}{Q}}{\ty{\Gamma},\tmty{x}{\tywith[\cs{o}]{A}{B}}}}
\end{mathpar}

\subsubsection{Typing Rules for Syntactic Sugar}
\begin{mathpar}
  \inferrule*[lab=T-Unbound-Send]{
    \seq{P}{\ty{\Gamma},\tmty{x}{B}}
  }{\seq{\usend{x}{y}{P}}{\ty{\Gamma},\tmty{x}{\tytens[\cs{o}]{A}{B}},\tmty{y}{\co{A}}}}
  \elabarrow
  \inferrule*{
    \inferrule*{
      \inferrule*{
      }{\seq{\link{y}{z}}{\tmty{y}{\co{A}},\tmty{z}{A}}}
      \\
      \seq{P}{\ty{\Gamma},\tmty{x}{B}}
    }{\seq{\ppar{\link{y}{z}}{P}}{\ty{\Gamma},\tmty{x}{B},\tmty{y}{\co{A}},\tmty{z}{A}}}
    \\
    \cs{o}<\minpr(\ty{\Gamma},\ty{A},\ty{B},\ty{\co{A}})
  }{\seq{\send{x}{z}{(\ppar{\link{y}{z}}{P})}}{\ty{\Gamma},\tmty{x}{\tytens[\cs{o}]{A}{B}},\tmty{y}{\co{A}}}}
\end{mathpar}

\subsection{Operational Semantics}

\subsubsection{Structural congruence}
\begin{mathpar}
  \begin{array}{llcl}
    \LabTirName{SC-LinkSwap}   & \tm{\link{x}{y}}
                               & \equiv & \tm{\link{y}{x}}
    \\
    \LabTirName{SC-ResLink}    & \tm{\res{x}{y}{\link{x}{y}}}
                               & \equiv & \tm{\halt}
    \\
    \LabTirName{SC-ResSwap}    & \tm{\res{x}{y}{P}}
                               & \equiv & \tm{\res{y}{x}{P}}
    \\
    \LabTirName{SC-ResComm}    & \tm{\res{x}{y}{\res{z}{w}{P}}}
                               & \equiv & \tm{\res{z}{w}{\res{x}{y}{P}}}
    \\
    \LabTirName{SC-ResExt}     & \tm{\res{x}{y}{(\ppar{P}{Q})}}
                               & \equiv & \tm{\ppar{\res{x}{y}{P}}{Q}},
                                          \text{ if }{\tm{x},\tm{y}\notin\fv(\tm{Q})}
    \\
    \LabTirName{SC-ParNil}     & \tm{\ppar{P}{\halt}}
                               & \equiv & \tm{P}
    \\
    \LabTirName{SC-ParComm}    & \tm{\ppar{P}{Q}}
                               & \equiv & \tm{\ppar{Q}{P}}
    \\
    \LabTirName{SC-ParAssoc}   & \tm{\ppar{P}{(\ppar{Q}{R})}}
                               & \equiv & \tm{\ppar{(\ppar{P}{Q})}{R}}
  \end{array}
\end{mathpar}

\subsubsection{Reduction}
\begin{mathpar}
  \begin{array}{llcl}
    \LabTirName{E-Link}    & \tm{\res{x}{y}{(\ppar{\link{w}{x}}{P})}}
                           & \red & \tm{\subst{P}{w}{x}}
    \\
    \LabTirName{E-Send}    & \tm{\res{x}{y}{(\ppar{\send{x}{z}{P}}{\recv{x}{w}{Q}})}}
                           & \red & \tm{\res{x}{y}{\res{z}{w}{(\ppar{P}{Q})}}}
    \\
    \LabTirName{E-Close}   & \tm{\res{x}{y}{(\ppar{\close{x}{P}}{\wait{y}{Q}})}}
                           & \red & \tm{\ppar{P}{Q}}
    \\
    \LabTirName{E-Select-Inl}
                           & \tm{\res{x}{y}{(\ppar{\inl{x}{P}}{\offer{x}{Q}{R}})}}
                           & \red & \tm{\res{x}{y}{(\ppar{P}{Q})}}
    \\
    \LabTirName{E-Select-Inr}
                           & \tm{\res{x}{y}{(\ppar{\inr{x}{P}}{\offer{x}{Q}{R}})}}
                           & \red & \tm{\res{x}{y}{(\ppar{P}{R})}}
  \end{array}
  \\
  \inferrule*[lab=E-Res]{
    \tm{P}\red\tm{P'}
  }{\tm{\res{x}{y}{P}}\red\tm{\res{x}{y}{P'}}}

  \inferrule*[lab=E-Par]{
    \tm{P}\red\tm{P'}
  }{\tm{\ppar{P}{Q}}\red\tm{\ppar{P'}{Q}}}

  \inferrule*[lab=E-LiftSC]{
    \tm{P}\equiv\tm{P'}
    \\
    \tm{P'}\red\tm{Q'}
    \\
    \tm{Q'}\equiv\tm{Q}
  }{\tm{P}\red\tm{Q}}
\end{mathpar}

\end{document}

%%% Local Variables:
%%% TeX-master: "priorities"
%%% End:
