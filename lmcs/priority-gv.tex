\section{Priority GV}\label{sec:pgv}
\usingnamespace{pgv}

We present Priority GV (PGV), a session-typed functional language based on GV~\cite{wadler15,lindleymorris15} which uses priorities \`{a} la Kobayashi and Padovani~\cite{kobayashi06,padovaninovara15} to enforce deadlock freedom.
Priority GV is more flexible than GV because it allows processes to share more than one communication channel.

We illustrate this with two programs in PGV, \cref{ex:cycle} and \cref{ex:deadlock}. Each program contains two processes---the main process, and the child process created by $\tm{\spawn}$---which communicate using \emph{two} channels. The child process receives a unit over the channel $\tm{x}/\tm{x'}$, and then sends a unit over the channel $\tm{y}/\tm{y'}$. The main process does one of two things:
\begin{enumerate}[label= (\alph*) ]
  \item in \cref{ex:cycle}, it sends a unit over the channel $\tm{x}/\tm{x'}$, and then waits to receive a unit over the channel $\tm{y}/\tm{y'}$;
  \item in \cref{ex:deadlock}, it does these in the opposite order, which results in a deadlock.
\end{enumerate}
PGV is more expressive than GV: \cref{ex:cycle} is typeable and guaranteed to be deadlock-free in PGV, but is not typeable in GV~\cite{wadler14} and not guaranteed deadlock-free in GV's predecessor~\cite{gayvasconcelos10}. We believe PGV is a non-conservative \emph{extension} of GV, as CP can be embedded in a Kobayashi-style system~\cite{dardhaperez15extended}.

\begin{figure}
  \begin{minipage}{0.5\linewidth}
    \begin{exa}\label{ex:cycle}
      % [Cyclic Structure]%
      \[\tm{%
          \begin{array}{l}
            \letpair{x}{x'}{\new\;\unit}{}
            \\
            \letpair{y}{y'}{\new\;\unit}{}
            \\
            \spawn\left(\lambda\unit.%
            \begin{array}{l}
                {\letpair{\unit}{x'}{\recv\;{x'}}{}}
                \\
                {\letbind{y}{\send\;{\pair{\unit}{y}}}{}}
                \\
                {\andthen{\wait\;{x'}}{\close\;{y}}}
              \end{array}%
            \right);
            \\
            \underline{\letbind{x}{\send\;{\pair{\unit}{x}}}{}}
            \\
            \underline{\letpair{\unit}{y'}{\recv\;{y'}}{}}
            \\
            {\andthen{\close\;{x}}{\wait\;{y'}}}
          \end{array}
        }\]
    \end{exa}
  \end{minipage}%
  \begin{minipage}{0.5\linewidth}
    \begin{exa}\label{ex:deadlock}
      % [Deadlock]%
      \[\tm{%
          \begin{array}{l}
            \letpair{x}{x'}{\new\;\unit}{}
            \\
            \letpair{y}{y'}{\new\;\unit}{}
            \\
            \spawn\left(\lambda\unit.%
            \begin{array}{l}
                {\letpair{\unit}{x'}{\recv\;{x'}}{}}
                \\
                {\letbind{y}{\send\;{\pair{\unit}{y}}}{}}
                \\
                {\andthen{\wait\;{x'}}{\close\;{y}}}
              \end{array}%
            \right);
            \\
            \underline{\letpair{\unit}{y'}{\recv\;{y'}}{}}
            \\
            \underline{\letbind{x}{\send\;{\pair{\unit}{x}}}{}}
            \\
            {\andthen{\close\;{x}}{\wait\;{y'}}}
          \end{array}
        }\]
    \end{exa}
  \end{minipage}
\end{figure}

\subsection{Syntax of Types and Terms}
\subsubsection*{Session types}
Session types ($\ty{S}$) are defined by the following grammar:
\[
  \begin{array}{lcl}
    \ty{S}
     & \Coloneqq & \ty{\tysend[\cs{o}]{T}{S}}
    \sep        \ty{\tyrecv[\cs{o}]{T}{S}}
    \sep        \ty{\tyends[\cs{o}]}
    \sep        \ty{\tyendr[\cs{o}]}
  \end{array}
\]

Session types $\ty{\tysend[\cs{o}]{T}{S}}$ and $\ty{\tyrecv[\cs{o}]{T}{S}}$ describe the endpoints of a channel over which we send or receive a value of type $\ty{T}$, and then proceed as $\ty{S}$. Types $\ty{\tyends[\cs{o}]}$ and $\ty{\tyendr[\cs{o}]}$ describe endpoints of a channel whose communication has finished, and over which we must synchronise before closing the channel. Each connective in a session type is annotated with a \emph{priority} $\cs{o}\in\mathbb{N}$.

\subsubsection*{Types}
Types ($\ty{T}$, $\ty{U}$) are defined by the following grammar:
\[
  \begin{array}{lcl}
    \ty{T}, \ty{U}
     & \Coloneqq & \ty{\typrod{T}{U}}
    \sep        \ty{\tyunit}
    \sep        \ty{\tysum{T}{U}}
    \sep        \ty{\tyvoid}
    \sep        \ty{\tylolli[\cs{p},\cs{q}]{T}{U}}
    \sep        \ty{S}
  \end{array}
\]

Types $\ty{\typrod{T}{U}}$, $\ty{\tyunit}$, $\ty{\tysum{T}{U}}$, and $\ty{\tyvoid}$ are the standard linear $\lambda$-calculus product type, unit type, sum type, and empty type.
Type $\ty{\tylolli[\cs{p},\cs{q}]{T}{U}}$ is the standard linear function type, annotated with \emph{priority bounds} $\cs{p},\cs{q}\in\mathbb{N}\cup\{\cs{\pbot},\cs{\ptop}\}$.
Every session type is also a type.
Given a function with type $\ty{\tylolli[\cs{p},\cs{q}]{T}{U}}$, $\cs{p}$ is a \emph{lower bound} on the priorities of the endpoints captured by the body of the function, and $\cs{q}$ is an \emph{upper bound} on the priority of the communications that take place as a result of applying the function. The type of {pure functions} $\ty{\tylolli{T}{U}}$, \ie those which perform no communications, is syntactic sugar for $\ty{\tylolli[\cs{\ptop},\cs{\pbot}]{T}{U}}$. The lower bound for a pure function is $\cs{\ptop}$ as pure functions never start communicating. For similar reasons, the upper bound for a pure function is $\cs{\pbot}$.

We postulate that the only function types---and, consequently, sequents---that are inhabited in PGV are pure functions and functions $\ty{\tylolli[\cs{p},\cs{q}]{T}{U}}$ for which $\cs{p}<\cs{q}$.

\subsubsection*{Typing Environments}
Typing environments $\ty{\Gamma}$, $\ty{\Delta}$, $\ty{\Theta}$ associate types to names. Environments are linear, so two environments can only be combined as $\ty{\Gamma}, \ty{\Delta}$ if their names are distinct, \ie $\fv(\ty{\Gamma})\cap\fv(\ty{\Delta})=\varnothing$.
\[
  \begin{array}{lcl}
    \ty{\Gamma}, \ty{\Delta}
     & \Coloneqq & \ty{\emptyenv}
    \sep        \ty{\Gamma}, \tmty{x}{T}
  \end{array}
\]

\subsubsection*{Type Duality}
Duality plays a crucial role in session types. The two endpoints of a channel are assigned dual types, ensuring that, for instance, whenever one program {sends} a value on a channel, the program on the other end is waiting to {receive}. Each session type $\ty{S}$ has a dual, written $\ty{\co{S}}$. Duality is an involutive function which {preserves priorities}:
\[
  \ty{\co{\tysend[\cs{o}]{T}{S}}} = \ty{\tyrecv[\cs{o}]{T}{\co{S}}}
  \qquad
  \ty{\co{\tyrecv[\cs{o}]{T}{S}}} = \ty{\tysend[\cs{o}]{T}{\co{S}}}
  \qquad
  \ty{\co{\tyends[\cs{o}]}} = \ty{\tyendr[\cs{o}]}
  \qquad
  \ty{\co{\tyendr[\cs{o}]}} = \ty{\tyends[\cs{o}]}
\]

\subsubsection*{Priorities}
Function $\pr(\cdot)$ returns the smallest priority of a session type. The type system guarantees that the top-most connective always holds the smallest priority, so we simply return the priority of the top-most connective:
\[
  \pr(\ty{\tysend[\cs{o}]{T}{S}}) = \cs{o}
  \qquad
  \pr(\ty{\tyrecv[\cs{o}]{T}{S}}) = \cs{o}
  \qquad
  \pr(\ty{\tyends[\cs{o}]})       = \cs{o}
  \qquad
  \pr(\ty{\tyendr[\cs{o}]})       = \cs{o}
\]

We extend the function $\pr(\cdot)$ to types and typing contexts by returning the smallest priority in the type or context, or $\cs{\top}$ if there is no priority. We use $\sqcap$ and $\sqcup$ to denote the minimum and maximum, respectively:
\[
  \begin{array}{lcl}
    \minpr(\ty{\typrod{T}{U}})                 & = & \minpr({\ty{T}})\sqcap\minpr({\ty{U}})  \\
    \minpr(\ty{\tysum{T}{U}})                  & = & \minpr({\ty{T}})\sqcap\minpr({\ty{U}})  \\
    \minpr(\ty{\tylolli[\cs{p},\cs{q}]{T}{U}}) & = & \cs{p}                                  \\
    \minpr(\ty{\Gamma}, \tmty{x}{A})           & = & \minpr(\ty{\Gamma})\sqcap\minpr(\ty{A})
  \end{array}
  \qquad
  \begin{array}{lcl}
    \minpr(\ty{\tyunit})   & = & \ptop       \\
    \minpr(\ty{\tyvoid})   & = & \ptop       \\
    \minpr(\ty{S})         & = & \pr(\ty{S}) \\
    \minpr(\ty{\emptyenv}) & = & \ptop
  \end{array}
\]

\subsubsection*{Terms}
Terms ($\tm{L}$, $\tm{M}$, $\tm{N}$) are defined by the following grammar:
\[
  \begin{array}{lcl}
    \tm{L}, \tm{M}, \tm{N}
     & \Coloneqq & \tm{x}
    \sep        \tm{K}
    \sep        \tm{\lambda x.M}
    \sep        \tm{M\;N}                 \\
     & \sep      & \tm{\unit}
    \sep        \tm{\andthen{M}{N}}
    \sep        \tm{\pair{M}{N}}
    \sep        \tm{\letpair{x}{y}{M}{N}} \\
     & \sep      & \tm{\inl{M}}
    \sep        \tm{\inr{M}}
    \sep        \tm{\casesum{L}{x}{M}{y}{N}}
    \sep        \tm{\absurd{M}}           \\
    \tm{K}
     & \Coloneqq & \tm{\link}
    \sep        \tm{\new}
    \sep        \tm{\spawn}
    \sep        \tm{\send}
    \sep        \tm{\recv}
    \sep        \tm{\close}
    \sep        \tm{\wait}
  \end{array}
\]

Let $\tm{x}$, $\tm{y}$, $\tm{z}$, and $\tm{w}$ range over variable names. Occasionally, we use $\tm{a}$, $\tm{b}$, $\tm{c}$, and $\tm{d}$. The term language is the standard linear $\lambda$-calculus with products, sums, and their units, extended with constants $\tm{K}$ for the communication primitives.

Constants are best understood in conjunction with their typing and reduction rules in~\cref{fig:pgv-typing,fig:pgv-operational-semantics}.

Briefly, $\tm{\link}$ links two endpoints together, forwarding messages from one to the other, $\tm{\new}$ creates a new channel and returns a pair of its endpoints, and $\tm{\spawn}$ spawns off its argument as a new thread.

The $\tm{\send}$ and $\tm{\recv}$ functions send and receive values on a channel.
However, since the typing rules for PGV ensure the linear usage of endpoints, they also return a new copy of the endpoint to continue the session.

The $\tm{\close}$ and $\tm{\wait}$ functions close a channel.

We use syntactic sugar to make terms more readable: we write $\tm{\letbind{x}{M}{N}}$ in place of $\tm{(\lambda x.N)\;M}$, $\tm{\lambda\unit.M}$ in place of $\tm{\lambda z.\andthen{z}{M}}$, and $\tm{\lambda\pair{x}{y}.M}$ in place of $\tm{\lambda z.\letpair{x}{y}{z}{M}}$.
We can recover GV's $\tm{\fork}$ as $\tm{\lambda x.\letpair{y}{z}{\new\;\unit}{\andthen{\spawn\;{(\lambda\unit.x\;y)}}{z}}}$.

\subsubsection*{Internal and External Choice}
Typically, session-typed languages feature constructs for internal and external choice. In GV, these can be defined in terms of the core language, by sending or receiving a value of a sum type~\cite{lindleymorris15}. We use the following syntactic sugar for internal ($\ty{\tyselect[\cs{o}]{S}{S'}}$) and external ($\ty{\tyoffer[\cs{o}]{S}{S'}}$) choice and their units:
\[
  \begin{array}{lcl}
    \ty{\tyselect[\cs{o}]{S}{S'}}
     & \elabarrow & \ty{\tysend[\cs{o}]{(\tysum{\co{S}}{\co{S'}})}{\tyends[\cs{o+1}]}} \\
    \ty{\tyoffer[\cs{o}]{S}{S'}}
     & \elabarrow & \ty{\tyrecv[\cs{o}]{(\tysum{S}{S'})}{\tyendr[\cs{o+1}]}}           \\
  \end{array}
  \qquad
  \begin{array}{lcl}
    \ty{\tyselectemp[\cs{o}]}
     & \elabarrow & \ty{\tysend[\cs{o}]{\tyvoid}{\tyends[\cs{o+1}]}} \\
    \ty{\tyofferemp[\cs{o}]}
     & \elabarrow & \ty{\tyrecv[\cs{o}]{\tyvoid}{\tyendr[\cs{o+1}]}} \\
  \end{array}
\]

As the syntax for units suggests, these are the binary and nullary forms of the more common n-ary choice constructs $\ty{{\oplus}^{\cs{o}}\{l_i:S_i\}_{i\in{I}}}$ and $\ty{{\with}^{\cs{o}}\{l_i:S_i\}_{i\in{I}}}$, which one may obtain generalising the sum types to variant types. For simplicity, we present only the binary and nullary forms.

Similarly, we use syntactic sugar for the term forms of choice, which combine sending and receiving with the introduction and elimination forms for the sum and empty types. There are two constructs for binary internal choice, expressed using the meta-variable $\tm{\ell}$ which ranges over $\{\tm{\labinl},\tm{\labinr}\}$. As there is no introduction for the empty type, there is no construct for nullary internal choice:
\[
  \begin{array}{lcl}
    \tm{\select{\ell}}
     & \elabarrow
     & \tm{\lambda x.\letpair{y}{z}{\new}{\andthen{\close\;(\send\;{\pair{\ell\;y}{x}})}{z}}}
    \\
    \multicolumn{3}{l}{%
      \tm{\offer{L}{x}{M}{y}{N}}\elabarrow}
    \\
    \multicolumn{3}{l}{%
    \qquad\tm{\letpair{z}{w}{\recv\;{L}}{\andthen{\wait\;{w}}{\casesum{z}{x}{M}{y}{N}}}}}
    \\
    \tm{\offeremp{L}}
     & \elabarrow
     & \tm{\letpair{z}{w}{\recv\;L}{\andthen{\wait\;{w}}{\absurd{z}}}}
  \end{array}
\]

\subsection{Operational Semantics}
\subsubsection*{Configurations}
Priority GV terms are evaluated as part of a configuration of processes. Configurations are defined by the following grammar:
\[
  \tm{\phi}
  \Coloneqq \tm{\main}
  \sep      \tm{\child}
  \hfill\qquad\hfill
  \tm{\conf{C}}, \tm{\conf{D}}, \tm{\conf{E}}
  \Coloneqq \tm{\phi\;M}
  \sep      \tm{\ppar{\conf{C}}{\conf{D}}}
  \sep      \tm{\res{x}{x'}{\conf{C}}}
\]

Configurations ($\tm{\conf{C}}$, $\tm{\conf{D}}$, $\tm{\conf{E}}$) consist of threads $\tm{\phi\;M}$, parallel compositions $\tm{\ppar{\conf{C}}{\conf{D}}}$, and name restrictions $\tm{\res{x}{x'}{\conf{C}}}$. To preserve the functional nature of PGV, where programs return a single value, we use flags ($\tm{\phi}$) to differentiate between the main thread, marked $\tm{\main}$, and child threads created by $\tm{\spawn}$, marked $\tm{\child}$. Only the main thread returns a value. We determine the flag of a configuration by combining the flags of all threads in that configuration:
\[
  \tm{\main}  + \tm{\child} = \tm{\main}
  \hfill\qquad\hfill
  \tm{\child} + \tm{\main}  = \tm{\main}
  \hfill\qquad\hfill
  \tm{\child} + \tm{\child} = \tm{\child}
  \hfill\qquad\hfill
  (\tm{\main}  + \tm{\main} \; \text{is undefined})
  \hfill\qquad\hfill
\]

To distinguish between the use of $\tm{\child}$ to mark child threads~\cite{lindleymorris15} and the use of the meta-variable $\cs{o}$ for priorities~\cite{dardhagay18extended}, they are typeset in a different font and colour.

\subsubsection*{Values}
Values ($\tm{V}$, $\tm{W}$), evaluation contexts ($\tm{E}$), thread evaluation contexts ($\tm{\conf{F}}$), and configuration contexts ($\tm{\conf{G}}$) are defined by the following grammars:
\[
  \begin{array}[t]{lcl}
    \tm{V}, \tm{W}
     & \Coloneqq & \tm{x}
    \sep        \tm{K}
    \sep        \tm{\lambda x.M}
    \sep        \tm{\unit}
    \sep        \tm{\pair{V}{W}}
    \sep        \tm{\inl{V}}
    \sep        \tm{\inr{V}}              \\
    \tm{E}
     & \Coloneqq & \tm{\hole}
    \sep        \tm{E\;M}
    \sep        \tm{V\;E}                 \\
     & \sep      & \tm{\andthen{E}{N}}
    \sep        \tm{\pair{E}{M}}
    \sep        \tm{\pair{V}{E}}
    \sep        \tm{\letpair{x}{y}{E}{M}} \\
     & \sep      & \tm{\inl{E}}
    \sep        \tm{\inr{E}}
    \sep        \tm{\casesum{E}{x}{M}{y}{N}}
    \sep        \tm{\absurd{E}}           \\
    \tm{\conf{F}}
     & \Coloneqq & \tm{\phi\;E}
    \\
    \tm{\conf{G}}
     & \Coloneqq & \tm{\hole}
    \sep        \tm{\ppar{\conf{G}}{\conf{C}}}
    \sep        \tm{\res{x}{y}{\conf{G}}}
  \end{array}
\]
Values are the subset of terms which cannot reduce further.
Evaluation contexts are one-hole term contexts, \ie, terms with exactly one hole, written $\tm{\hole}$. We write $\tm{\plug{E}{M}}$ for the evaluation context $\tm{E}$ with its hole replaced by the term $\tm{M}$. Evaluation contexts are specifically those one-hole term contexts under which term reduction can take place.
Thread contexts are a convenient way to lift the notion of evaluation contexts to threads. We write $\tm{\plug{\conf{F}}{M}}$ for the thread context $\tm{\conf{F}}$ with its hole replaced by the term $\tm{M}$.
Configuration contexts are one-hole configuration contexts, \ie, configurations with exactly one hole, written $\tm{\hole}$. Specifically, configuration contexts are those one-hole term contexts under which configuration reduction can take place. The definition for $\tm{\conf{G}}$ only gives the case in which the hole is in the left-most parallel process, \ie, it only defines $\tm{\ppar{\conf{G}}{\conf{C}}}$ and not $\tm{\ppar{\conf{C}}{\conf{G}}}$. The latter is not needed, as $\tm{\parallel}$ is symmetric under structural congruence, though it would be harmless to add. We write $\tm{\plug{\conf{G}}{\conf{C}}}$ for the evaluation context $\tm{\conf{G}}$ with its hole replaced by the term $\tm{\conf{C}}$.


\subsubsection*{Reduction Relation}
\begin{figure}[h]
  \paragraph*{Term reduction}
  \begin{mathpar}
    \begin{array}{llcl}
      \LabTirName{E-Lam}  & \tm{(\lambda x.M) \; V}
                          & \tred & \tm{\subst{M}{V}{x}}
      \\
      \LabTirName{E-Unit} & \tm{\letunit{\unit}{M}}
                          & \tred & \tm{M}
      \\
      \LabTirName{E-Pair} & \tm{\letpair{x}{y}{\pair{V}{W}}{M}}
                          & \tred & \tm{\subst{\subst{M}{V}{x}}{W}{y}}
      \\
      \LabTirName{E-Inl}  & \tm{\casesum{\inl{V}}{x}{M}{y}{N}}
                          & \tred & \tm{\subst{M}{V}{x}}
      \\
      \LabTirName{E-Inr}  & \tm{\casesum{\inr{V}}{x}{M}{y}{N}}
                          & \tred & \tm{\subst{N}{V}{y}}
    \end{array}
    \\
    \inferrule*[lab=E-Lift]{
      \tm{M}\tred\tm{M'}
    }{\tm{\plug{E}{M}}\tred\tm{\plug{E}{M'}}}
  \end{mathpar}
  \\
  \paragraph*{Structural congruence}
  \begin{mathpar}
    \begin{array}{llcl}
      \LabTirName{SC-LinkSwap}   & \tm{\plug{\conf{F}}{\link\;{\pair{x}{y}}}}
                                 & \equiv & \tm{\plug{\conf{F}}{\link\;{\pair{y}{x}}}}
      \\
      \LabTirName{SC-ResLink}    & \tm{\res{x}{y}{(\phi\;\link\;\pair{x}{y})}}
                                 & \equiv & \tm{\phi\;\unit}
      \\
      \LabTirName{SC-ResSwap}    & \tm{\res{x}{y}{\conf{C}}}
                                 & \equiv & \tm{\res{y}{x}{\conf{C}}}
      \\
      \LabTirName{SC-ResComm}    & \tm{\res{x}{y}{\res{z}{w}{\conf{C}}}}
                                 & \equiv & \tm{\res{z}{w}{\res{x}{y}{\conf{C}}}}
      \\
      \LabTirName{SC-ResExt}     & \tm{\res{x}{y}{(\ppar{\conf{C}}{\conf{D}}})}
                                 & \equiv & \tm{\ppar{\conf{C}}{\res{x}{y}{\conf{D}}}},
                                            \text{ if }{\tm{x},\tm{y}\notin\fv(\tm{\conf{C}})}
      \\
      \LabTirName{SC-ParNil}     & \tm{\ppar{\conf{C}}{\child{\unit}}}
                                 & \equiv & \tm{\conf{C}}
      \\
      \LabTirName{SC-ParComm}    & \tm{\ppar{\conf{C}}{\conf{D}}}
                                 & \equiv & \tm{\ppar{\conf{D}}{\conf{C}}}
      \\
      \LabTirName{SC-ParAssoc}   & \tm{\ppar{\conf{C}}{(\ppar{\conf{D}}{\conf{E}})}}
                                 & \equiv & \tm{\ppar{(\ppar{\conf{C}}{\conf{D}})}{\conf{E}}}
    \end{array}
  \end{mathpar}
  \\
  \paragraph*{Configuration reduction}
  \begin{mathpar}
    \begin{array}{llcl}
      \LabTirName{E-Link}  & \tm{\res{x}{y}{(\ppar{\plug{\conf{F}}{\link\;\pair{w}{x}}}{\conf{C}})}}
                           & \cred & \tm{\ppar{\plug{\conf{F}}{\unit}}{\subst{\conf{C}}{w}{y}}}
      \\
      \LabTirName{E-New}   & \tm{\plug{\conf{F}}{\new\;\unit}}
                           & \cred & \tm{\res{x}{y}{(\plug{\conf{F}}{\pair{x}{y}})}}
      \\
      \LabTirName{E-Spawn} & \tm{\plug{\conf{F}}{\spawn\;V}}
                           & \cred & \tm{\ppar{\plug{\conf{F}}{\unit}}{\child\;V\;\unit}}
      \\
      \LabTirName{E-Send}  & \tm{\res{x}{y}{(\ppar
                             {\plug{\conf{F}}{\send\;{\pair{V}{x}}}}
                             {\plug{\conf{F'}}{\recv\;{y}}})}}
                           & \cred & \tm{\res{x}{y}{(\ppar
                                     {\plug{\conf{F}}{x}}
                                     {\plug{\conf{F'}}{\pair{V}{y}}})}}
      \\
      \LabTirName{E-Close} & \tm{\res{x}{y}{(\ppar
                             {\plug{\conf{F}}{\wait\;{x}}}
                             {\plug{\conf{F'}}{\close\;{y}}})}}
                           & \cred & \tm{\ppar{\plug{\conf{F}}{\unit}}{\plug{\conf{F'}}{\unit}}}
    \end{array}
    \\
    \inferrule*[lab=E-LiftC]{
      \tm{\conf{C}}\cred\tm{\conf{C'}}
    }{\tm{\plug{\conf{G}}{\conf{C}}}\cred\tm{\plug{\conf{G}}{\conf{C'}}}}
  
    \inferrule*[lab=E-LiftM]{
      \tm{M}\tred\tm{M'}
    }{\tm{\plug{\conf{F}}{M}}\tred\tm{\plug{\conf{F}}{M'}}}
  
    \inferrule*[lab=E-LiftSC]{
      \tm{\conf{C}}\equiv\tm{\conf{C'}}
      \\
      \tm{\conf{C'}}\cred\tm{\conf{D'}}
      \\
      \tm{\conf{D'}}\equiv\tm{\conf{D}}
    }{\tm{\conf{C}}\cred\tm{\conf{D}}}
  \end{mathpar}
  \caption{Operational Semantics for PGV.}
  \label{fig:pgv-operational-semantics}
\end{figure}

%%% Local Variables:
%%% TeX-master: "../priorities"
%%% End:


We factor the reduction relation of PGV into a deterministic reduction on terms ($\tred$) and a non-deterministic reduction on configurations ($\cred$), see \cref{fig:pgv-operational-semantics}. We write $\tred^+$ and $\cred^+$ for the transitive closures, and $\tred^\star$ and $\cred^\star$ for the reflexive-transitive closures.

Term reduction is the standard call-by-value, left-to-right evaluation for GV, and only deviates from reduction for the linear $\lambda$-calculus in that it reduces terms to values {or} ready terms waiting to perform a communication action.

Configuration reduction resembles evaluation for a process calculus: \LabTirName{E-Link}, \LabTirName{E-Send}, and \LabTirName{E-Close} perform communications, \LabTirName{E-LiftC} allows reduction under configuration contexts, and \LabTirName{E-LiftSC} embeds a structural congruence $\equiv$. The remaining rules mediate between the process calculus and the functional language: \LabTirName{E-New} and \LabTirName{E-Spawn} evaluate the $\tm{\new}$ and $\tm{\spawn}$ constructs, creating the equivalent configuration constructs, and \LabTirName{E-LiftM} embeds term reduction.

Structural congruence satisfies the following axioms: $\LabTirName{SC-LinkSwap}$ allows swapping channels in the link process. $\LabTirName{SC-ResLink}$ allows restriction to be applied to link which is structurally equivalent to the terminated process, thus allowing elimination of unnecessary restrictions. $\LabTirName{SC-ResSwap}$ allows swapping channels and $\LabTirName{SC-ResComm}$ states that restriction is commutative. $\LabTirName{SC-ResExt}$ is the standard scope extrusion rule. Rules $\LabTirName{SC-ParNil}$, $\LabTirName{SC-ParComm}$ and $\LabTirName{SC-ParAssoc}$ state that parallel composition uses the terminated process as the neutral element; it is commutative and associative.

While our configuration reduction is based on the standard evaluation for GV, the increased expressiveness of PGV allows us to simplify the relation on two counts.
\begin{enumerate}[label=(\roman*)]
  \item
        \emph{We decompose the $\tm{\fork}$ construct}.
        In GV, $\tm{\fork}$ creates a new channel, spawns a child thread, and, when the child thread finishes, it closes the channel to its parent. In PGV, these are three separate operations: $\tm{\new}$, $\tm{\spawn}$, and $\tm{\close}$. We no longer require that every child thread finishes by returning a terminated channel. Consequently, we also simplify the evaluation of the $\tm{\link}$ construct.

        Intuitively, evaluating $\tm{\link}$ causes a substitution: if we have a channel bound as $\tm{\res{x}{y}{}}$, then $\tm{\link\;\pair{w}{x}}$ replaces all occurrences of $\tm{y}$ by $\tm{w}$. However, in GV, $\tm{\link}$ is required to return a terminated channel, which means that the semantics for $\tm{link}$ must {create} a fresh channel of type $\ty{\tyends}/\ty{\tyendr}$. The endpoint of type $\ty{\tyends}$ is returned by the $\tm{link}$ construct, and a $\tm{\wait}$ on the other endpoint guards the {actual} substitution. In PGV, evaluating $\tm{\link}$ simply causes a substitution.
  \item
        \emph{Our structural congruence is type preserving}. Consequently, we can embed it directly into the reduction relation. In GV, this is not the case, and subject reduction relies on proving that if the result of rewriting via $\equiv$ followed by reducing via $\cred$ is an ill-typed configuration, we can rewrite it to a well-typed configuration via $\equiv$.
\end{enumerate}


\subsection{Typing Rules}
\begin{figure}
  \begin{mdframed}
    \textbf{Static Typing Rules.}
    \begin{mathpar}
      \inferrule*[lab=T-Var]{
      }{\tseq[\cs{\pbot}]{\tmty{x}{T}}{x}{T}}
      
      \inferrule*[lab=T-Lam]{
        \tseq[\cs{q}]{\ty{\Gamma},\tmty{x}{T}}{M}{U}
      }{\tseq[\cs{\pbot}]{\ty{\Gamma}}{\lambda x.M}{\tylolli[\cs{\minpr(\ty{\Gamma})},\cs{q}]{T}{U}}}
      
      \inferrule*[lab=T-Const]{
      }{\tseq[\cs{\pbot}]{\emptyenv}{K}{T}}
      
      \inferrule*[lab=T-App]{
        \tseq[\cs{p}]{\ty{\Gamma}}{M}{\tylolli[\cs{p'},\cs{q'}]{T}{U}}
        \\
        \tseq[\cs{q}]{\ty{\Delta}}{N}{T}
        \\
        \cs{p}<\minpr(\ty{\Delta})
        \\
        \cs{q}<\cs{p'}
      }{\tseq[\cs{p}\sqcup\cs{q}\sqcup\cs{q'}]{\ty{\Gamma},\ty{\Delta}}{M\;N}{U}}
      \\
      \inferrule*[lab=T-Unit]{
      }{\tseq[\cs{\pbot}]{\emptyenv}{\unit}{\tyunit}}
      
      \inferrule*[lab=T-LetUnit]{
        \tseq[\cs{p}]{\ty{\Gamma}}{M}{\tyunit}
        \\
        \tseq[\cs{q}]{\ty{\Delta}}{N}{T}
        \\
        \cs{p}<\minpr(\ty{\Delta})
      }{\tseq[\cs{p}\sqcup\cs{q}]{\ty{\Gamma},\ty{\Delta}}{\andthen{M}{N}}{T}}
      
      \inferrule*[lab=T-Pair]{
        \tseq[\cs{p}]{\ty{\Gamma}}{M}{T}
        \\
        \tseq[\cs{q}]{\ty{\Delta}}{N}{U}
        \\
        \cs{p}<\minpr(\ty{\Delta})
      }{\tseq[\cs{p}\sqcup\cs{q}]{\ty{\Gamma},\ty{\Delta}}{\pair{M}{N}}{\typrod{T}{U}}}
      
      \inferrule*[lab=T-LetPair]{
        \tseq[\cs{p}]{\ty{\Gamma}}{M}{\typrod{T}{T'}}
        \\
        \tseq[\cs{q}]{\ty{\Delta},\tmty{x}{T},\tmty{y}{T'}}{N}{U}
        \\
        \cs{p}<\minpr(\ty{\Delta},\ty{T},\ty{T'})
      }{\tseq[\cs{p}\sqcup\cs{q}]{\ty{\Gamma},\ty{\Delta}}{\letpair{x}{y}{M}{N}}{U}}
      
      \inferrule*[lab=T-Inl]{
        \tseq[\cs{p}]{\ty{\Gamma}}{M}{T}
        \\
        \minpr(\ty{T})=\minpr(\ty{U})
      }{\tseq[\cs{p}]{\ty{\Gamma}}{\inl{M}}{\tysum{T}{U}}}
      
      \inferrule*[lab=T-Inr]{
        \tseq[\cs{p}]{\ty{\Gamma}}{M}{U}
        \\
        \minpr(\ty{T})=\minpr(\ty{U})
      }{\tseq[\cs{p}]{\ty{\Gamma}}{\inr{M}}{\tysum{T}{U}}}
      
      \inferrule*[lab=T-CaseSum]{
        \tseq[\cs{p}]{\ty{\Gamma}}{L}{\tysum{T}{T'}}
        \\
        \tseq[\cs{q}]{\ty{\Delta},\tmty{x}{T}}{M}{U}
        \\
        \tseq[\cs{q}]{\ty{\Delta},\tmty{y}{T'}}{N}{U}
        \\
        \cs{p}<\minpr(\ty{\Delta})
      }{\tseq[\cs{p}\sqcup\cs{q}]{\ty{\Gamma},\ty{\Delta}}{\casesum{L}{x}{M}{y}{N}}{U}}
      
      \inferrule*[lab=T-Absurd]{
        \tseq[\cs{p}]{\ty{\Gamma}}{M}{\tyvoid}
      }{\tseq[\cs{p}]{\ty{\Gamma},\ty{\Delta}}{\absurd{M}}{T}}
    \end{mathpar}
    \begin{center}
      We write $\tmty{K}{T}$ for $\tseq[\cs{\pbot}]{\emptyenv}{K}{T}$ in typing derivations, \\
      and omit $\ty{T}$ when it can be inferred from context.
    \end{center}    
    \textbf{Type Schemas for Constants.}
    \begin{mathpar}
      \tmty{\link}{\tylolli{\typrod{S}{\co{S}}}{\tyunit}}
      
      \tmty{\new}{\tylolli{\tyunit}{\typrod{S}{\co{S}}}}
      
      \tmty{\spawn}{\tylolli{(\tylolli[\cs{p},\cs{q}]{\tyunit}{\tyunit})}{\tyunit}}
      \\
      \tmty{\send}{\tylolli[\cs{\ptop},\cs{o}]{\typrod{T}{\tysend[\cs{o}]{T}{S}}}{S}}
      
      \tmty{\recv}{\tylolli[\cs{\ptop},\cs{o}]{\tyrecv[\cs{o}]{T}{S}}{\typrod{T}{S}}}
      \\
      \tmty{\close}{\tylolli[\cs{\ptop},\cs{o}]{\tyends[\cs{o}]}{\tyunit}}
      
      \tmty{\wait}{\tylolli[\cs{\ptop},\cs{o}]{\tyendr[\cs{o}]}{\tyunit}}
    \end{mathpar}
    \textbf{Runtime Typing Rules.}
    \begin{mathpar}
      \inferrule*[lab=T-Main]{
        \tseq[\cs{p}]{\ty{\Gamma}}{M}{T}
      }{\cseq[\main]{\ty{\Gamma}}{\main\;M}}
      
      \inferrule*[lab=T-Child]{
        \tseq[\cs{p}]{\ty{\Gamma}}{M}{\tyunit}
      }{\cseq[\child]{\ty{\Gamma}}{\child\;M}}
      
      \inferrule*[lab=T-Res]{
        \cseq[\phi]{\ty{\Gamma},\tmty{x}{S},\tmty{y}{\co{S}}}{\conf{C}}
      }{\cseq[\phi]{\ty{\Gamma}}{\res{x}{y}{\conf{C}}}}
      
      \inferrule*[lab=T-Par]{
        \cseq[\phi]{\ty{\Gamma}}{\conf{C}}
        \\
        \cseq[\phi']{\ty{\Delta}}{\conf{D}}
      }{\cseq[\phi+\phi']{\ty{\Gamma},\ty{\Delta}}{\ppar{\conf{C}}{\conf{D}}}}
    \end{mathpar}
    \caption{Typing Rules for PGV.}
    \label{fig:pgv-typing}
  \end{mdframed}
\end{figure}
%%% Local Variables:
%%% TeX-master: "../priorities"
%%% End:


\subsubsection*{Terms Typing}
%\Cref{fig:pgv-typing} gives the typing rules for PGV.
Typing rules for terms are at the top of \cref{fig:pgv-typing}.
Terms are typed by a judgement $\tseq[\cs{p}]{\ty{\Gamma}}{M}{T}$ stating that ``a term $\tm{M}$ has type $\ty{T}$ and an upper bound on its priority $\cs{p}$ under the typing environment $\ty{\Gamma}$''. Typing for the linear $\lambda$-calculus is standard. Linearity is ensured by splitting environments on branching rules, requiring that the environment in the variable rule consists of just the variable, and the environment in the constant and unit rules are empty. Constants $\tm{K}$ are typed using type schemas, which hold for any concrete assignment of types and priorities to their meta-variables. Instantiated type schemas are embedded into typing derivations using \LabTirName{T-Const} in~\cref{fig:pgv-typing}, \eg, the type schema for $\tm{\send}$ can be instantiated with $\cs{o}=\cs{2}$, $\ty{T}=\ty{\tyunit}$, and $\ty{S}=\ty{\tyvoid}$, and embedded using \LabTirName{T-Const} to give the following typing derivation:
\[
  \inferrule*{
  }{\tmty{\send}{\tylolli[\cs{\ptop},\cs{2}]{\typrod{\tyunit}{\tysend[\cs{2}]{\tyunit}{\tyvoid}}}{\tyvoid}}}
\]

The typing rules treat {all variables} as linear resources, even those of non-linear types such as $\ty{\tyunit}$, though they can easily be extended to allow values with unrestricted usage~\cite{wadler14}.

The only non-standard feature of the typing rules is the priority annotations. Priorities are based on {obligations/capabilities} used by Kobayashi~\cite{kobayashi06}, and simplified to single priorities following Padovani~\cite{padovani14}. The integration of priorities into GV is adapted from Padovani and Novara~\cite{padovaninovara15}. Paraphrasing Dardha and Gay~\cite{dardhagay18extended}, priorities obey the following two laws:
\begin{enumerate}[label= (\roman*) ]
  \item an action with lower priority happens before an action with higher priority; and
  \item communication requires \emph{equal} priorities for dual actions.
\end{enumerate}

In PGV, we keep track of a lower and upper bound on the priorities of a term, \ie, while evaluating the term, when it starts communicating, and when it finishes, respectively. The upper bound is written on the sequent and the lower bound is approximated from the typing environment, \eg, for $\tseq[\cs{p}]{\ty{\Gamma}}{M}{T}$ the upper bound is $\cs{p}$ and the lower bound is at least $\minpr(\ty{\Gamma})$. The latter is {correct} because a term cannot communicate at a priority earlier than the earliest priority amongst the channels it has access to. It is an \emph{approximation} on function terms, as these can ``skip'' communication by returning the corresponding channel unused. However, linearity prevents such functions from being used in well typed configurations: once the unused channel's priority has passed, it can no longer be used.

Typing rules for sequential constructs enforce sequentiality, \eg, the typing for $\tm{\andthen{M}{N}}$ has a side condition which requires that the upper bound of $\tm{M}$ is smaller than the lower bound of $\tm{N}$, \ie, $\tm{M}$ finishes before $\tm{N}$ starts. The typing rule for $\tm{\new}$ ensures that both endpoints of a channel share the same priorities. Together, these two constraints guarantee deadlock freedom.

To illustrate this, let's go back to the deadlocked program in \cref{ex:deadlock}. Crucially, it composes the terms below in parallel. While each of these terms itself is well typed, they impose opposite conditions on the priorities, so connecting $\ty{x}$ to $\ty{x'}$ and $\ty{y}$ to $\ty{y'}$ using \LabTirName{T-Res} is ill-typed, as there is no assignment to $\cs{o}$ and $\cs{o'}$ that can satsify both $\cs{o}<\cs{o'}$ and $\cs{o'}<\cs{o}$. (We omit the priorities on $\ty{\tyends}$ and $\ty{\tyendr}$.)
\begin{mathpar}
  \inferrule*{
  \tseq[\cs{o'}]
  {\tmty{y'}{\tyrecv[\cs{o'}]{\tyunit}{\tyendr}}}
  {\recv\;{y'}}
  {\typrod{\tyunit}{\tyendr}}
  \\
  \tseq[\cs{p}]
  {\tmty{x}{\tysend[\cs{o}]{\tyunit}{\tyends}},\tmty{y'}{\tyendr}}
  {\letbind{x}{\send\;{\pair{\unit}{x}}}{\dots}}
  {\tyunit}
  \and
  \cs{o'}<\cs{o}
  }{%
  \tseq[\cs{p}]
  {%
  \tmty{x}{\tysend[\cs{o}]{\tyunit}{\tyends}},
  \tmty{y'}{\tyrecv[\cs{o'}]{\tyunit}{\tyendr}}
  }
  {\letpair{\unit}{y'}{\recv\;{y'}}{\letbind{x}{\send\;{\pair{\unit}{x}}}{\dots}}}
  {\tyunit}
  }

  \inferrule*{
  \tseq[\cs{o}]
  {\tmty{x'}{\tyrecv[\cs{o}]{\tyunit}{\tyendr}}}
  {\recv\;{x'}}
  {\typrod{\tyunit}{\tyendr}}
  \\
  \tseq[\cs{q}]
  {\tmty{y}{\tysend[\cs{o'}]{\tyunit}{\tyends}},\tmty{x'}{\tyendr}}
  {\letbind{y}{\send\;{\pair{\unit}{y}}}{\dots}}
  {\tyunit}
  \and
  \cs{o}<\cs{o'}
  }{%
  \tseq[\cs{q}]
  {%
  \tmty{y}{\tysend[\cs{o'}]{\tyunit}{\tyends}},
  \tmty{x'}{\tyrecv[\cs{o}]{\tyunit}{\tyendr}}
  }
  {\letpair{\unit}{x'}{\recv\;{x'}}{\letbind{y}{\send\;{\pair{\unit}{y}}}{\dots}}}
  {\tyunit}
  }
\end{mathpar}

Closures suspend communication, so \LabTirName{T-Lam} stores the priority bounds of the function body on the function type, and \LabTirName{T-App} restores them. For instance, $\tm{\lambda{x}.\send\;\pair{x}{y}}$ is assigned the type $\ty{\tylolli[\cs{o},\cs{o}]{A}{S}}$, \ie, a~function which, when applied, starts and finishes communicating at priority $\cs{o}$.
\begin{mathpar}
  \inferrule*{
    \inferrule*{
      \inferrule*{
      }{
        \tmty{\send}{\tylolli[\ptop,\cs{o}]{\typrod{T}{\tysend[\cs{o}]{T}{S}}}{S}}
      }
      \and
      \inferrule*{%
        \inferrule*{
        }{
          \tseq[\pbot]{\tmty{x}{T}}{x}{T}
        }
        \and
        \inferrule*{
        }{
          \tseq[\pbot]
          {\tmty{x}{T},\tmty{y}{\tysend[\cs{o}]{T}{S}}}
          {y}{\tysend[\cs{o}]{T}{S}}
        }
      }{
        \tseq[\pbot]
        {\tmty{x}{T},\tmty{y}{\tysend[\cs{o}]{T}{S}}}
        {\pair{x}{y}}{\typrod{T}{\tysend[\cs{o}]{T}{S}}}
      }
    }{
      \tseq[\cs{o}]
      {\tmty{x}{T},\tmty{y}{\tysend[\cs{o}]{T}{S}}}
      {\send\;\pair{x}{y}}{S}
    }
  }{
    \tseq[\pbot]
    {\tmty{y}{\tysend[\cs{o}]{T}{S}}}
    {\lambda{x}.\send\;\pair{x}{y}}{\tylolli[\cs{o},\cs{o}]{T}{S}}
  }
\end{mathpar}

For simplicity, we assume priority annotations are not inferred, but provided as an input to type checking. However, for any term, priorities can be inferred, \eg, by using the topological ordering of the directed graph where the vertices are the priority meta-variables and the edges are the inequality constraints between the priority meta-variables in the typing derivation.

\subsubsection*{Configurations Typing}
Typing rules for configurations are at the bottom of \cref{fig:pgv-typing}. Configurations are typed by a judgement $\cseq[\phi]{\ty{\Gamma}}{\conf{C}}$ stating that ``a configuration $\tm{\conf{C}}$ with flag $\tm\phi$ is well typed under typing environment $\ty{\Gamma}$''. Configuration typing is based on the standard typing for GV. Terms are embedded either as main or as child threads. The priority bound from the term typing is discarded, as configurations contain no further blocking actions. Main threads are allowed to return a value, whereas child threads are required to return the unit value. Sequents are annotated with a flag $\tm\phi$, which ensures that there is at most one main thread.

While our configuration typing is based on the standard typing for GV, it differs on two counts:
\begin{enumerate}[label= (\roman*) ]
  \item \emph{we require that child threads return the unit value}, as opposed to a terminated channel; and
  \item \emph{we simplify typing for parallel composition}.
\end{enumerate}

In order to guarantee deadlock freedom, in GV each parallel composition must split \emph{exactly one} channel of the channel pseudo-type $\ty{S^\sharp}$ into two endpoints of type $\ty{S}$ and $\ty{\co{S}}$. Consequently, associativity of parallel composition does not preserve typing. In PGV, we guarantee deadlock freedom using priorities, which removes the need for the channel pseudo-type $\ty{S^\sharp}$, and simplifies typing for parallel composition, while restoring type preservation for the structural congruence.

\subsubsection*{Syntactic Sugar Typing}
\begin{figure}[t]
  \paragraph*{Typing Rules for Syntactic Sugar}
  \begin{mathpar}
    \inferrule*[lab=T-Seq]{
      \tseq[\cs{p}]{\ty{\Gamma}}{M}{\tyunit}
      \\
      \tseq[\cs{q}]{\ty{\Delta}}{N}{T}
      \\
      \cs{p}<\pr(\ty{\Delta})
    }{\tseq[\cs{p}\sqcup\cs{q}]{\ty{\Gamma},\ty{\Delta}}{\andthen{M}{N}}{T}}
    
    \inferrule*[lab=T-Let]{
      \tseq[\cs{p}]{\ty{\Gamma}}{M}{T}
      \\
      \tseq[\cs{q}]{\ty{\Delta},\tmty{x}{T}}{N}{U}
      \\
      \cs{p}<\pr(\ty{\Delta})
    }{\tseq[\cs{p}\sqcup\cs{q}]{\ty{\Gamma},\ty{\Delta}}{\letbind{x}{M}{N}}{U}}
    \\
    \inferrule*[lab=T-LamUnit]{
      {\tseq[\cs{o}]{\ty{\Gamma}}{M}{T}}
    }{\tseq[\cs{\pbot}]
      {\ty{\Gamma}}
      {\lambda\unit.M}
      {\tylolli[\cs{\pr(\ty{\Gamma})},\cs{o}]{\tyunit}{T}}}
    
    \inferrule*[lab=T-LamPair]
    {\tseq[\cs{o}]
      {\ty{\Gamma},\tmty{x}{T},\tmty{y}{T'}}
      {M}
      {U}}
    {\tseq[\cs{\pbot}]
      {\ty{\Gamma}}
      {\lambda\pair{x}{y}.M}
      {\tylolli[\cs{\pr(\ty{\Gamma})},\cs{o}]{\typrod{T}{T'}}{U}}}
    \\
    \inferrule*[lab=T-Fork]{
    }{\tseq[\cs{\pbot}]
      {\emptyenv}
      {\fork}
      {\tylolli[]{(\tylolli[\cs{p},\cs{q}]{S}{\tyunit})}{\co{S}}}}
    \\
    \inferrule*[lab=T-Select-Inl]{
      \pr(\ty{S})=\pr(\ty{S'})
    }{\tseq[\cs{\pbot}]{\emptyenv}{\select{\labinl}}{\tylolli[\cs{\ptop},\cs{o}]{\tyselect[\cs{o}]{S}{S'}}{S}}}
    
    \inferrule*[lab=T-Select-Inr]{
      \pr(\ty{S})=\pr(\ty{S'})
    }{\tseq[\cs{\pbot}]{\emptyenv}{\select{\labinr}}{\tylolli[\cs{\ptop},\cs{o}]{\tyselect[\cs{o}]{S}{S'}}{S'}}}
    
    \inferrule*[lab=T-Offer]{
      {\tseq[\cs{p}]
        {\ty{\Gamma}}
        {L}
        {\tyoffer[\cs{o}]{S}{S'}}}
      \\
      {\tseq[\cs{q}]
        {\ty{\Delta},\tmty{x}{S}}
        {M}
        {T}}
      \\
      {\tseq[\cs{q}]
        {\ty{\Delta},\tmty{y}{S'}}
        {N}
        {T}}
      \\
      \cs{o}\sqcup\cs{p}<\pr(\ty{\Delta},\ty{S},\ty{S'})
    }{\tseq[\cs{o}\sqcup\cs{p}\sqcup\cs{q}]
      {\ty{\Gamma},\ty{\Delta}}
      {\offer{L}{x}{M}{y}{N}}
      {T}}
    
    \inferrule*[lab=T-Offer-Absurd]{
      \tseq[\cs{p}]
      {\ty{\Gamma}}
      {L}
      {\tyofferemp[\cs{o}]}
      \\
      \cs{o}\sqcup\cs{p}<\pr(\ty{\Delta})
    }{\tseq[\cs{o}\sqcup\cs{p}]
      {\ty{\Gamma},\ty{\Delta}}
      {\offeremp{L}}
      {T}}
  \end{mathpar}
  \caption{Typing Rules for Syntactic Sugar for PGV.}
  \label{fig:pgv-typing-sugar}
\end{figure}
%%% Local Variables:
%%% TeX-master: "../priorities"
%%% End:

The following typing rules given in \cref{{fig:pgv-typing-sugar},fig:pgv-typing-sugar-select,fig:pgv-typing-sugar-offer}, cover syntactic sugar typing for PGV.

\section{Technical Developments}
\subsection{Subject Reduction}
Unlike with previous versions of GV, structural congruence, term reduction, and configuration reduction are all type preserving.

We must show that substitution preserves priority constraints. For this, we prove~\cref{lem:pgv-value-done}, which shows that values have finished all their communication, and that any priorities in the type of the value come from the typing environment.
\begin{lem}\label{lem:pgv-value-done}
  If $\tseq[\cs{p}]{\ty{\Gamma}}{V}{T}$, then $\cs{p}=\cs{\pbot}$, and $\minpr(\ty{\Gamma})=\minpr(\ty{T})$.
\end{lem}
\begin{proof}
  \label{prf:lem-pgv-value-done}
  By induction on the derivation of $\tseq[\cs{o}]{\ty{\Gamma}}{V}{T}$.

  \begin{case*}[\LabTirName{T-Lam}]
    Immediately.
    \begin{mathpar}
      \inferrule*{
        \tseq[\cs{q}]{\ty{\Gamma},\tmty{x}{T}}{M}{U}
      }{\tseq[\cs{\pbot}]{\ty{\Gamma}}{\lambda x.M}{\tylolli[\cs{\pr(\ty{\Gamma})},\cs{q}]{T}{U}}}
    \end{mathpar}
  \end{case*}
  \begin{case*}[\LabTirName{T-Unit}]
    Immediately.
    \begin{mathpar}
      \inferrule*{
      }{\tseq[\cs{\pbot}]{\emptyenv}{\unit}{\tyunit}}
    \end{mathpar}
  \end{case*}
  \begin{case*}[\LabTirName{T-Pair}]
    The induction hypotheses give us $\cs{p}=\cs{q}=\cs{\pbot}$, hence $\cs{p}\sqcup\cs{q}=\cs{\pbot}$, and $\pr(\ty{\Gamma})=\pr(\ty{T})$ and $\pr(\ty{\Delta})=\pr(\ty{U})$, hence $\pr(\ty{\Gamma},\ty{\Delta})=\pr(\ty{\Gamma})\sqcap\pr(\ty{\Delta})=\pr(\ty{T})\sqcap\pr(\ty{U})=\pr(\ty{\typrod{T}{U}})$.
    \begin{mathpar}
      \inferrule*{
        \tseq[\cs{p}]{\ty{\Gamma}}{V}{T}
        \\
        \tseq[\cs{q}]{\ty{\Delta}}{W}{U}
        \\
        \cs{p}<\pr(\ty{\Delta})
      }{\tseq[\cs{p}\sqcup\cs{q}]{\ty{\Gamma},\ty{\Delta}}{\pair{V}{W}}{\typrod{T}{U}}}
    \end{mathpar}
  \end{case*}
  \begin{case*}[\LabTirName{T-Inl}]
    The induction hypothesis gives us $\cs{p}=\cs{\pbot}$, and $\pr(\ty{\Gamma})=\pr{\ty{T}}$. We know $\pr(\ty{T})=\pr({\ty{U}})$, hence $\pr(\ty{\Gamma})=\pr(\ty{\tysum{T}{U}})$.
    \begin{mathpar}
      \inferrule*{
        \tseq[\cs{p}]{\ty{\Gamma}}{V}{T}
        \\
        \pr(\ty{T})=\pr(\ty{U})
      }{\tseq[\cs{p}]{\ty{\Gamma}}{\inl{V}}{\tysum{T}{U}}}
    \end{mathpar}
  \end{case*}
  \begin{case*}[\LabTirName{T-Inr}]
    The induction hypothesis gives us $\cs{p}=\cs{\pbot}$, and $\pr(\ty{\Gamma})=\pr{\ty{U}}$. We know $\pr(\ty{T})=\pr({\ty{U}})$, hence $\pr(\ty{\Gamma})=\pr(\ty{\tysum{T}{U}})$.
    \begin{mathpar}
      \inferrule*{
        \tseq[\cs{p}]{\ty{\Gamma}}{V}{U}
        \\
        \pr(\ty{T})=\pr(\ty{U})
      }{\tseq[\cs{p}]{\ty{\Gamma}}{\inr{V}}{\tysum{T}{U}}}
    \end{mathpar}
  \end{case*}
\end{proof}

%%% Local Variables:
%%% TeX-master: "../priorities"
%%% End:


\begin{lem}\label{lem:pgv-substitution}
  %[Substitution]
  If $\tseq[\cs{p}]{\ty{\Gamma},\tmty{x}{U'}}{M}{T}$ and $\tseq[\cs{q}]{\ty{\Theta}}{V}{U'}$, then $\tseq[\cs{p}]{\ty{\Gamma},\ty{\Theta}}{\subst{M}{V}{x}}{T}$.
\end{lem}
\begin{proof}
  By induction on the derivation of $\tseq[\cs{p}]{\ty{\Gamma},\tmty{x}{U'}}{M}{T}$.
  \begin{case*}[\LabTirName{T-Var}]
    By \cref{lem:value-done}, $\cs{q}=\cs{\pbot}$.
    \begin{mathpar}
      \inferrule*{
      }{\tseq[\cs{\pbot}]{\tmty{x}{U'}}{x}{U'}}
      \substarrow{V}{x}
      \tseq[\cs{\pbot}]{\ty{\Theta}}{V}{U'}
    \end{mathpar}
  \end{case*}
  \begin{case*}[\LabTirName{T-Lam}]
    By \cref{lem:value-done}, $\pr(\ty{\Theta})=\pr(\ty{U'})$, hence $\pr(\ty{\Gamma},\ty{\Theta})=\pr(\ty{\Gamma},\ty{U'})$.
    \begin{mathpar}
      \inferrule*{
        \tseq[\cs{o}]{\ty{\Gamma},\tmty{x}{U'},\tmty{y}{T}}{M}{U}
      }{\tseq[\cs{\pbot}]{\ty{\Gamma},\tmty{x}{U'}}
        {\lambda y.M}
        {\tylolli[\cs{\pr(\ty{\Gamma},\ty{U'})},\cs{o}]{T}{U}}}
      \substarrow{V}{x}
      \inferrule*{
        \tseq[\cs{o}]{\ty{\Gamma},\ty{\Theta},\tmty{y}{T}}{\subst{M}{V}{x}}{U}
      }{\tseq[\cs{\pbot}]{\ty{\Gamma},\ty{\Theta}}
        {\lambda y.\subst{M}{V}{x}}
        {\tylolli[\cs{\pr(\ty{\Gamma},\ty{\Theta})},\cs{o}]{T}{U}}}
    \end{mathpar}
  \end{case*}
  \begin{case*}[\LabTirName{T-App}]
    There are two subcases:
    \begin{subcase*}[$\tm{x}\in\tm{M}$]
      Immediately, from the induction hypothesis.
      \begin{mathpar}
        \inferrule*{
          \tseq[\cs{p}]{\ty{\Gamma},\tmty{x}{U'}}{M}{\tylolli[\cs{o},\cs{r}]{T}{U}}
          \\
          \tseq[\cs{q}]{\ty{\Delta}}{N}{T}
          \\
          \cs{p}<\pr(\ty{\Delta})
          \\
          \cs{q}<\cs{o}
        }{\tseq[\cs{p}\sqcup\cs{q}\sqcup\cs{r}]{\ty{\Gamma},\ty{\Delta},\tmty{x}{U'}}{M\;N}{U}}
        \substarrow{V}{x}
        \inferrule*{
          \tseq[\cs{p}]{\ty{\Gamma},\ty{\Theta}}{\subst{M}{V}{x}}{\tylolli[\cs{o},\cs{r}]{T}{U}}
          \\
          \tseq[\cs{q}]{\ty{\Delta}}{N}{T}
          \\
          \cs{p}<\pr(\ty{\Delta})
          \\
          \cs{q}<\cs{o}
        }{\tseq[\cs{p}\sqcup\cs{q}\sqcup\cs{r}]{\ty{\Gamma},\ty{\Delta},\ty{\Theta}}{(\subst{M}{V}{x})\;N}{U}}
      \end{mathpar}
    \end{subcase*}
    \begin{subcase*}[$\tm{x}\in\tm{N}$]
      By \cref{lem:value-done}, $\pr(\ty{\Theta})=\pr(\ty{U'})$, hence $\pr(\ty{\Delta},\ty{\Theta})=\pr(\ty{\Delta},\ty{U'})$.
      \begin{mathpar}
        \inferrule*{
          \tseq[\cs{p}]{\ty{\Gamma}}{M}{\tylolli[\cs{o},\cs{r}]{T}{U}}
          \\
          \tseq[\cs{q}]{\ty{\Delta},\tmty{x}{U'}}{N}{T}
          \\
          \cs{p}<\pr(\ty{\Delta},\ty{U'})
          \\
          \cs{q}<\cs{o}
        }{\tseq[\cs{p}\sqcup\cs{q}\sqcup\cs{r}]{\ty{\Gamma},\ty{\Delta},\tmty{x}{U'}}{M\;N}{U}}
        \substarrow{V}{x}
        \inferrule*{
          \tseq[\cs{p}]{\ty{\Gamma}}{M}{\tylolli[\cs{o},\cs{r}]{T}{U}}
          \\
          \tseq[\cs{q}]{\ty{\Delta},\ty{\Theta}}{\subst{N}{V}{x}}{T}
          \\
          \cs{p}<\pr(\ty{\Delta},\ty{\Theta})
          \\
          \cs{q}<\cs{o}
        }{\tseq[\cs{p}\sqcup\cs{q}\sqcup\cs{r}]{\ty{\Gamma},\ty{\Delta},\ty{\Theta}}{M\;(\subst{N}{V}{x})}{U}}
      \end{mathpar}
    \end{subcase*}
  \end{case*}
  \begin{case*}[\LabTirName{T-LetUnit}]
    There are two subcases:
    \begin{subcase*}[$\tm{x}\in\tm{M}$]
      Immediately, from the induction hypothesis.
      \begin{mathpar}
        \inferrule*{
          \tseq[\cs{p}]{\ty{\Gamma},\tmty{x}{U'}}{M}{\tyunit}
          \\
          \tseq[\cs{q}]{\ty{\Delta}}{N}{T}
          \\
          \cs{p}<\pr(\ty{\Delta})
        }{\tseq[\cs{p}\sqcup\cs{q}]{\ty{\Gamma},\ty{\Delta},\tmty{x}{U'}}{\letunit{M}{N}}{T}}
        \substarrow{V}{x}
        \inferrule*{
          \tseq[\cs{p}]{\ty{\Gamma},\ty{\Theta}}{\subst{M}{V}{x}}{\tyunit}
          \\
          \tseq[\cs{q}]{\ty{\Delta}}{N}{T}
          \\
          \cs{p}<\pr(\ty{\Delta})
        }{\tseq[\cs{p}\sqcup\cs{q}]{\ty{\Gamma},\ty{\Delta},\ty{\Theta}}{\letunit{\subst{M}{V}{x}}{N}}{T}}
      \end{mathpar}
    \end{subcase*}
    \begin{subcase*}[$\tm{x}\in\tm{N}$]
      By \cref{lem:value-done}, $\pr(\ty{\Theta})=\pr(\ty{U'})$, hence $\pr(\ty{\Delta},\ty{\Theta})=\pr(\ty{\Delta},\ty{U'})$.
      \begin{mathpar}
        \inferrule*{
          \tseq[\cs{p}]{\ty{\Gamma}}{M}{\tyunit}
          \\
          \tseq[\cs{q}]{\ty{\Delta},\tmty{x}{U'}}{N}{T}
          \\
          \cs{p}<\pr(\ty{\Delta},\ty{U'})
        }{\tseq[\cs{p}\sqcup\cs{q}]{\ty{\Gamma},\ty{\Delta},\tmty{x}{U'}}{\letunit{M}{N}}{T}}
        \substarrow{V}{x}
        \inferrule*{
          \tseq[\cs{p}]{\ty{\Gamma}}{M}{\tyunit}
          \\
          \tseq[\cs{q}]{\ty{\Delta},\ty{\Theta}}{\subst{N}{V}{x}}{T}
          \\
          \cs{p}<\pr(\ty{\Delta},\ty{\Theta})
        }{\tseq[\cs{p}\sqcup\cs{q}]{\ty{\Gamma},\ty{\Delta},\ty{\Theta}}{\letunit{M}{\subst{N}{V}{x}}}{T}}
      \end{mathpar}
    \end{subcase*}
  \end{case*}
  \begin{case*}[\LabTirName{T-Pair}]
    There are two subcases:
    \begin{subcase*}[$\tm{x}\in\tm{M}$]
      Immediately, from the induction hypothesis.
      \begin{mathpar}
        \inferrule*{
          \tseq[\cs{p}]{\ty{\Gamma},\tmty{x}{U'}}{M}{T}
          \\
          \tseq[\cs{q}]{\ty{\Delta}}{N}{U}
          \\
          \cs{p}<\pr(\ty{\Delta},\ty{U'})
        }{\tseq[\cs{p}\sqcup\cs{q}]{\ty{\Gamma},\ty{\Delta},\tmty{x}{U'}}{\pair{M}{N}}{\typrod{T}{U}}}
        \substarrow{V}{x}
        \inferrule*{
          \tseq[\cs{p}]{\ty{\Gamma},\ty{\Theta}}{\subst{M}{V}{x}}{T}
          \\
          \tseq[\cs{q}]{\ty{\Delta}}{N}{U}
          \\
          \cs{p}<\pr(\ty{\Delta},\ty{\Theta})
        }{\tseq[\cs{p}\sqcup\cs{q}]{\ty{\Gamma},\ty{\Delta},\ty{\Theta}}{\pair{\subst{M}{V}{x}}{N}}{\typrod{T}{U}}}
      \end{mathpar}
    \end{subcase*}
    \begin{subcase*}[$\tm{x}\in\tm{N}$]
      By \cref{lem:value-done}, $\pr(\ty{\Theta})=\pr(\ty{U'})$, hence $\pr(\ty{\Delta},\ty{\Theta})=\pr(\ty{\Delta},\ty{U'})$.
      \begin{mathpar}
        \inferrule*{
          \tseq[\cs{p}]{\ty{\Gamma}}{M}{T}
          \\
          \tseq[\cs{q}]{\ty{\Delta},\tmty{x}{U'}}{N}{U}
          \\
          \cs{p}<\pr(\ty{\Delta},\ty{U'})
        }{\tseq[\cs{p}\sqcup\cs{q}]{\ty{\Gamma},\ty{\Delta},\tmty{x}{U'}}{\pair{M}{N}}{\typrod{T}{U}}}
        \substarrow{V}{x}
        \inferrule*{
          \tseq[\cs{p}]{\ty{\Gamma}}{M}{T}
          \\
          \tseq[\cs{q}]{\ty{\Delta},\ty{\Theta}}{\subst{N}{V}{x}}{U}
          \\
          \cs{p}<\pr(\ty{\Delta},\ty{\Theta})
        }{\tseq[\cs{p}\sqcup\cs{q}]{\ty{\Gamma},\ty{\Delta},\ty{\Theta}}{\pair{M}{\subst{N}{V}{x}}}{\typrod{T}{U}}}
      \end{mathpar}
    \end{subcase*}
  \end{case*}
  \begin{case*}[\LabTirName{T-LetPair}]
    There are two subcases:
    \begin{subcase*}[$\tm{x}\in\tm{M}$]
      Immediately, from the induction hypothesis.
      \begin{mathpar}
        \inferrule*{
          \tseq[\cs{p}]{\ty{\Gamma},\tmty{x}{U'}}{M}{\typrod{T}{T'}}
          \\
          \tseq[\cs{q}]{\ty{\Delta},\tmty{y}{T},\tmty{z}{T'}}{N}{U}
          \\
          \cs{p}<\pr(\ty{\Delta},\ty{T},\ty{T'})
        }{\tseq[\cs{p}\sqcup\cs{q}]{\ty{\Gamma},\ty{\Delta},\tmty{x}{U'}}{\letpair{y}{z}{M}{N}}{U}}
        \substarrow{V}{x}
        \inferrule*{
          \tseq[\cs{p}]{\ty{\Gamma},\ty{\Theta}}{\subst{M}{V}{x}}{\typrod{T}{T'}}
          \\
          \tseq[\cs{q}]{\ty{\Delta},\tmty{y}{T},\tmty{z}{T'}}{N}{U}
          \\
          \cs{p}<\pr(\ty{\Delta},\ty{T},\ty{T'})
        }{\tseq[\cs{p}\sqcup\cs{q}]{\ty{\Gamma},\ty{\Delta},\ty{\Theta}}{\letpair{y}{z}{\subst{M}{V}{x}}{N}}{U}}
      \end{mathpar}
    \end{subcase*}
    \begin{subcase*}[$\tm{x}\in\tm{N}$]
      By \cref{lem:value-done}, $\pr(\ty{\Theta})=\pr(\ty{U'})$, hence $\pr(\ty{\Delta},\ty{\Theta},\ty{T},\ty{T'})=\pr(\ty{\Delta},\ty{U'},\ty{T},\ty{T'})$.
      \begin{mathpar}
        \inferrule*{
          \tseq[\cs{p}]{\ty{\Gamma}}{M}{\typrod{T}{T'}}
          \\
          \tseq[\cs{q}]{\ty{\Delta},\tmty{x}{U'},\tmty{y}{T},\tmty{z}{T'}}{N}{U}
          \\
          \cs{p}<\pr(\ty{\Delta},\ty{U'},\ty{T},\ty{T'})
        }{\tseq[\cs{p}\sqcup\cs{q}]{\ty{\Gamma},\ty{\Delta},\tmty{x}{U'}}{\letpair{y}{z}{M}{N}}{U}}
        \substarrow{V}{x}
        \inferrule*{
          \tseq[\cs{p}]{\ty{\Gamma}}{M}{\typrod{T}{T'}}
          \\
          \tseq[\cs{q}]{\ty{\Delta},\ty{\Theta},\tmty{y}{T},\tmty{z}{T'}}{\subst{N}{V}{x}}{U}
          \\
          \cs{p}<\pr(\ty{\Delta},\ty{\Theta},\ty{T},\ty{T'})
        }{\tseq[\cs{p}\sqcup\cs{q}]{\ty{\Gamma},\ty{\Delta},\ty{\Theta}}{\letpair{y}{z}{M}{\subst{N}{V}{x}}}{U}}
      \end{mathpar}
    \end{subcase*}
  \end{case*}
  \begin{case*}[\LabTirName{T-Absurd}]
    \begin{mathpar}
      \inferrule*{
        \tseq[o]{\ty{\Gamma},\tmty{x}{U'}}{M}{\tyvoid}
      }{\tseq[o]{\ty{\Gamma},\ty{\Delta},\tmty{x}{U'}}{\absurd{M}}{T}}
      \substarrow{V}{x}
      \inferrule*{
        \tseq[o]{\ty{\Gamma},\ty{\Theta}}{\subst{M}{V}{x}}{\tyvoid}
      }{\tseq[o]{\ty{\Gamma},\ty{\Delta},\ty{\Theta}}{\absurd{\subst{M}{V}{x}}}{T}}
    \end{mathpar}
  \end{case*}
  \begin{case*}[\LabTirName{T-Inl}]
    \begin{mathpar}
      \inferrule*{
        \tseq[o]{\ty{\Gamma},\tmty{x}{U'}}{M}{T}
        \\
        \pr(\ty{T})=\pr(\ty{U})
      }{\tseq[o]{\ty{\Gamma},\tmty{x}{U'}}{\inl{M}}{\tysum{T}{U}}}
      \substarrow{V}{x}
      \inferrule*{
        \tseq[o]{\ty{\Gamma},\ty{\Theta}}{\subst{M}{V}{x}}{T}
        \\
        \pr(\ty{T})=\pr(\ty{U})
      }{\tseq[o]{\ty{\Gamma},\ty{\Theta}}{\inl{\subst{M}{V}{x}}}{\tysum{T}{U}}}
    \end{mathpar}
  \end{case*}
  \begin{case*}[\LabTirName{T-Inr}]
    \begin{mathpar}
      \inferrule*{
        \tseq[o]{\ty{\Gamma},\tmty{x}{U'}}{M}{U}
        \\
        \pr(\ty{T})=\pr(\ty{U})
      }{\tseq[o]{\ty{\Gamma},\tmty{x}{U'}}{\inr{M}}{\tysum{T}{U}}}
      \substarrow{V}{x}
      \inferrule*{
        \tseq[o]{\ty{\Gamma},\ty{\Theta}}{\subst{M}{V}{x}}{U}
        \\
        \pr(\ty{T})=\pr(\ty{U})
      }{\tseq[o]{\ty{\Gamma},\ty{\Theta}}{\inr{\subst{M}{V}{x}}}{\tysum{T}{U}}}
    \end{mathpar}
  \end{case*}
  \begin{case*}[\LabTirName{T-CaseSum}]
    There are two subcases:
    \begin{subcase*}[$\tm{x}\in\tm{L}$]
      Immediately, from the induction hypothesis.
      \begin{mathpar}
        \inferrule*{
          \tseq[\cs{p}]{\ty{\Gamma},\tmty{x}{U'}}{L}{\tysum{T}{T'}}
          \\
          \tseq[\cs{q}]{\ty{\Delta},\tmty{y}{T}}{M}{U}
          \\
          \tseq[\cs{q}]{\ty{\Delta},\tmty{z}{T'}}{N}{U}
          \\
          \cs{p}<\pr(\ty{\Delta})
        }{\tseq[\cs{p}\sqcup\cs{q}]{\ty{\Gamma},\ty{\Delta},\tmty{x}{U'}}{\casesum{L}{y}{M}{z}{N}}{U}}
        \substarrow{V}{x}
        \inferrule*{
          \tseq[\cs{p}]{\ty{\Gamma},\ty{\Theta}}{\subst{L}{V}{x}}{\tysum{T}{T'}}
          \\
          \tseq[\cs{q}]{\ty{\Delta},\tmty{y}{T}}{M}{U}
          \\
          \tseq[\cs{q}]{\ty{\Delta},\tmty{z}{T'}}{N}{U}
          \\
          \cs{p}<\pr(\ty{\Delta})
        }{\tseq[\cs{p}\sqcup\cs{q}]{\ty{\Gamma},\ty{\Delta},\ty{\Theta}}{\casesum{\subst{L}{V}{x}}{y}{M}{z}{N}}{U}}
      \end{mathpar}
    \end{subcase*}
    \begin{subcase*}[$\tm{x}\in\tm{M}$ and $\tm{x}\in\tm{N}$]
      By \cref{lem:value-done}, $\pr(\ty{\Theta})=\pr(\ty{U'})$, hence $\pr(\ty{\Delta},\ty{\Theta},\ty{T})=\pr(\ty{\Delta},\ty{U'},\ty{T})$ and $\pr(\ty{\Delta},\ty{\Theta},\ty{T'})=\pr(\ty{\Delta},\ty{U'},\ty{T'})$.
      \begin{mathpar}
        \inferrule*{
          \tseq[\cs{p}]{\ty{\Gamma}}{L}{\tysum{T}{T'}}
          \\
          \tseq[\cs{q}]{\ty{\Delta},\tmty{x}{U'},\tmty{y}{T}}{M}{U}
          \\
          \tseq[\cs{q}]{\ty{\Delta},\tmty{x}{U'},\tmty{z}{T'}}{N}{U}
          \\
          \cs{p}<\pr(\ty{\Delta},\ty{U'})
        }{\tseq[\cs{p}\sqcup\cs{q}]{\ty{\Gamma},\ty{\Delta},\tmty{x}{U'}}{\casesum{L}{y}{M}{z}{N}}{U}}
        \substarrow{V}{x}
        \inferrule*{
          \tseq[\cs{p}]{\ty{\Gamma}}{L}{\tysum{T}{T'}}
          \\
          \tseq[\cs{q}]{\ty{\Delta},\ty{\Theta},\tmty{y}{T}}{\subst{M}{V}{x}}{U}
          \\
          \tseq[\cs{q}]{\ty{\Delta},\ty{\Theta},\tmty{z}{T'}}{\subst{N}{V}{x}}{U}
          \\
          \cs{p}<\pr(\ty{\Delta},\ty{\Theta})
        }{\tseq[\cs{p}\sqcup\cs{q}]{\ty{\Gamma},\ty{\Delta},\ty{\Theta}}{\casesum{L}{y}{\subst{M}{V}{x}}{z}{\subst{N}{V}{x}}}{U}}
      \end{mathpar}
    \end{subcase*}
  \end{case*}
  \noindent
  We omit the cases where $\tm{x}\not\in\tm{M}$.
\end{proof}

%%% Local Variables:
%%% TeX-master: "../priorities"
%%% End:


\begin{lem}\label{lem:pgv-subject-reduction-terms}
  %[Subject Reduction, $\tred$]%
  If $\tseq[\cs{p}]{\ty{\Gamma}}{M}{T}$ and $\tm{M}\tred\tm{M'}$,
  then $\tseq[\cs{p}]{\ty{\Gamma}}{M'}{T}$.
\end{lem}
\begin{proof}
  \label{prf:lem-pgv-subject-reduction-terms}
  By induction on the derivation of $\tm{M}\tred\tm{M'}$.

  \begin{case*}[\LabTirName{E-Lam}]
    By \cref{lem:pgv-substitution}.
    \begin{mathpar}
      \inferrule*{
        \inferrule*{
          \tseq[\cs{p}]{\ty{\Gamma},\tmty{x}{T}}{M}{U}
        }{\tseq[\cs{\pbot}]{\ty{\Gamma}}{\lambda x.M}{\tylolli[\cs{\pr(\ty{\Gamma})},\cs{p}]{T}{U}}}
        \\
        \tseq[\cs{\pbot}]{\ty{\Delta}}{V}{T}
      }{\tseq[\cs{p}]{\ty{\Gamma},\ty{\Delta}}{(\lambda x.M)\;V}{U}}
      \tred
      \tseq[\cs{p}]{\ty{\Gamma},\ty{\Delta}}{\subst{M}{V}{x}}{U}
    \end{mathpar}
  \end{case*}
  \begin{case*}[\LabTirName{E-Unit}]
    By \cref{lem:pgv-substitution}.
    \begin{mathpar}
      \inferrule*{
        \inferrule*{
        }{\tseq[\cs{\pbot}]{\emptyenv}{\unit}{\tyunit}}
        \\
        \tseq[\cs{p}]{\ty{\Gamma}}{M}{T}
      }{\tseq[\cs{p}]{\ty{\Gamma}}{\letunit{\unit}{M}}{T}}
      \tred
      \tseq[\cs{p}]{\ty{\Gamma}}{M}{T}
    \end{mathpar}
  \end{case*}
  \begin{case*}[\LabTirName{E-Pair}]
    By \cref{lem:pgv-substitution}.
    \begin{mathpar}
      \inferrule*{
        \inferrule*{
          \tseq[\cs{\pbot}]{\ty{\Gamma}}{V}{T}
          \\
          \tseq[\cs{\pbot}]{\ty{\Delta}}{W}{T'}
        }{\tseq[\cs{\pbot}]{\ty{\Gamma},\ty{\Delta}}{\pair{V}{W}}{\typrod{T}{T'}}}
        \\
        \tseq[\cs{p}]{\ty{\Theta},\tmty{x}{T},\tmty{y}{T'}}{M}{U}
      }{\tseq[]{\ty{\Gamma},\ty{\Delta},\ty{\Theta}}{\letpair{x}{y}{\pair{V}{W}}{M}}{U}}
      \\
      \begin{turn}{270}
        \tred
      \end{turn}
      \\
      \tseq[\cs{p}]{\ty{\Gamma},\ty{\Delta},\ty{\Theta}}{\subst{\subst{M}{V}{x}}{W}{y}}{U}
    \end{mathpar}
  \end{case*}
  \begin{case*}[\LabTirName{E-Inl}]
    By \cref{lem:pgv-substitution}.
    \begin{mathpar}
      \inferrule*{
        \inferrule*{
          \tseq[\cs{\pbot}]{\ty{\Gamma}}{V}{T}
        }{\tseq[\cs{\pbot}]{\ty{\Gamma}}{\inl{V}}{\tysum{T}{T'}}}
        \\
        \tseq[\cs{p}]{\ty{\Delta},\tmty{x}{T}}{M}{U}
        \\
        \tseq[\cs{p}]{\ty{\Delta},\tmty{y}{T'}}{N}{U}
      }{\tseq[\cs{p}]{\ty{\Gamma},\ty{\Delta}}{\casesum{\inl{V}}{x}{M}{y}{N}}{U}}
      \\
      \begin{turn}{270}
        \tred
      \end{turn}
      \\
      \tseq[\cs{p}]{\ty{\Gamma},\ty{\Delta}}{\subst{M}{V}{x}}{U}
    \end{mathpar}
  \end{case*}
  \begin{case*}[\LabTirName{E-Inr}]
    By \cref{lem:pgv-substitution}.
    \begin{mathpar}
      \inferrule*{
        \inferrule*{
          \tseq[\cs{\pbot}]{\ty{\Gamma}}{V}{T'}
        }{\tseq[\cs{\pbot}]{\ty{\Gamma}}{\inr{V}}{\tysum{T}{T'}}}
        \\
        \tseq[\cs{p}]{\ty{\Delta},\tmty{x}{T}}{M}{U}
        \\
        \tseq[\cs{p}]{\ty{\Delta},\tmty{y}{T'}}{N}{U}
      }{\tseq[\cs{p}]{\ty{\Gamma},\ty{\Delta}}{\casesum{\inr{V}}{x}{M}{y}{N}}{U}}
      \\
      \begin{turn}{270}
        \tred
      \end{turn}
      \\
      \tseq[\cs{p}]{\ty{\Gamma},\ty{\Delta}}{\subst{N}{V}{y}}{U}
    \end{mathpar}
  \end{case*}
  \begin{case*}[\LabTirName{E-Lift}]
    By induction on the evaluation context $\tm{E}$.
  \end{case*}
\end{proof}

%%% Local Variables:
%%% TeX-master: "../priorities"
%%% End:


\begin{lem}\label{lem:pgv-subject-congruence}
  %[Subject Congruence, $\equiv$]%
  If $\cseq[\phi]{\ty{\Gamma}}{\conf{C}}$ and $\tm{\conf{C}}\equiv\tm{\conf{C'}}$,
  then $\cseq[\phi]{\ty{\Gamma}}{\conf{C'}}$.
\end{lem}
\begin{proof}
  \label{prf:lem-pgv-subject-congruence}
  By induction on the derivation of $\tm{\conf{C}}\equiv\tm{\conf{C'}}$.

  \begin{case*}[\LabTirName{SC-LnkSwp}]
    \begin{mathpar}
      \inferrule*{
        \inferrule*[vdots=1.5em]{
          \inferrule*{
          }{\tmty{\link}{\tylolli{\typrod{S}{\co{S}}}{\tyunit}}}
          \\
          \inferrule*{
            \inferrule*{
            }{\tseq[\cs{\pbot}]{\tmty{x}{S}}{x}{S}}
            \\
            \inferrule*{
            }{\tseq[\cs{\pbot}]{\tmty{y}{\co{S}}}{y}{\co{S}}}
          }{\tseq[\cs{\pbot}]{\tmty{x}{S},\tmty{y}{\co{S}}}{\pair{x}{y}}{\typrod{S}{\co{S}}}}
        }{\tseq[\cs{\pbot}]{\tmty{x}{S},\tmty{y}{\co{S}}}{\link\;{\pair{x}{y}}}{\tyunit}}
      }{\cseq[\phi]{\ty{\Gamma},\tmty{x}{S},\tmty{y}{\co{S}}}{\plug{\conf{F}}{\link\;{\pair{x}{y}}}}}
      \\
      \begin{turn}{270}
        \ensuremath{\equiv}
      \end{turn}
      \\
      \inferrule*{
        \inferrule*[vdots=1.5em]{
          \inferrule*{
          }{\tmty{\link}{\tylolli{\typrod{\co{S}}{S}}{\tyunit}}}
          \\
          \inferrule*{
            \inferrule*{
            }{\tseq[\cs{\pbot}]{\tmty{y}{\co{S}}}{y}{\co{S}}}
            \\
            \inferrule*{
            }{\tseq[\cs{\pbot}]{\tmty{x}{S}}{x}{S}}
          }{\tseq[\cs{\pbot}]{\tmty{x}{S},\tmty{y}{\co{S}}}{\pair{y}{x}}{\typrod{S}{\co{S}}}}
        }{\tseq[\cs{\pbot}]{\tmty{x}{S},\tmty{y}{\co{S}}}{\link\;{\pair{y}{x}}}{\tyunit}}
      }{\cseq[\phi]{\ty{\Gamma},\tmty{x}{S},\tmty{y}{\co{S}}}{\plug{\conf{F}}{\link\;{\pair{y}{x}}}}}
    \end{mathpar}
  \end{case*}

  \begin{case*}[\LabTirName{SC-ResExt}]
    \begin{mathpar}
      \inferrule*{
        \inferrule*{
          \cseq{\ty{\Gamma}}{\conf{C}}
          \\
          \cseq{\ty{\Delta},\tmty{x}{S},\tmty{y}{\co{S}}}{\conf{D}}
        }{\cseq{\ty{\Gamma},\ty{\Delta},\tmty{x}{S},\tmty{y}{\co{S}}}{(\ppar{\conf{C}}{\conf{D}})}}
      }{\cseq{\ty{\Gamma},\ty{\Delta}}{\res{x}{y}{(\ppar{\conf{C}}{\conf{D}}})}}
      \equiv
      \inferrule*{
        \cseq{\ty{\Gamma}}{\conf{C}}
        \\
        \inferrule*{
          \cseq{\ty{\Delta},\tmty{x}{S},\tmty{y}{\co{S}}}{\conf{D}}
        }{\cseq{\ty{\Delta}}{\res{x}{y}{\conf{D}}}}
      }{\cseq{\ty{\Gamma},\ty{\Delta}}{\ppar{\conf{C}}{\res{x}{y}{\conf{D}}}}}
    \end{mathpar}
  \end{case*}

  \begin{case*}[\LabTirName{SC-ResSwp}]
    \todo{Write proof.}
  \end{case*}

  \begin{case*}[\LabTirName{SC-ResCom}]
    \begin{mathpar}
      \inferrule*{
        \inferrule*{
          \cseq[\phi]{\ty{\Gamma},\tmty{x}{S},\tmty{y}{\co{S}},\tmty{z}{S'},\tmty{w}{\co{S'}}}{\conf{C}}
        }{\cseq[\phi]{\ty{\Gamma},\tmty{x}{S},\tmty{y}{\co{S}}}{\res{z}{w}{\conf{C}}}}
      }{\cseq[\phi]{\ty{\Gamma}}{\res{x}{y}{\res{z}{w}{\conf{C}}}}}
      \equiv
      \inferrule*{
        \inferrule*{
          \cseq[\phi]{\ty{\Gamma},\tmty{x}{S},\tmty{y}{\co{S}},\tmty{z}{S'},\tmty{w}{\co{S'}}}{\conf{C}}
        }{\cseq[\phi]{\ty{\Gamma},\tmty{z}{S'},\tmty{w}{\co{S'}}}{\res{x}{y}{\conf{C}}}}
      }{\cseq[\phi]{\ty{\Gamma}}{\cseq{}{\res{z}{w}{\res{x}{y}{\conf{C}}}}}}
    \end{mathpar}
  \end{case*}

  \begin{case*}[\LabTirName{SC-ParNil}]
    \begin{mathpar}
      \inferrule*{
        \cseq[\phi]{\ty{\Gamma}}{\conf{C}}
        \\
        \inferrule*{
          \inferrule*{
          }{\tseq[\cs{\pbot}]{\emptyenv}{\unit}{\tyunit}}
        }{\cseq[\child]{\emptyenv}{\child{\unit}}}
      }{\cseq[\phi]{\ty{\Gamma}}{\ppar{\conf{C}}{\child{\unit}}}}
      \equiv
      \cseq[\phi]{\ty{\Gamma}}{\conf{C}}
    \end{mathpar}
  \end{case*}

  \begin{case*}[\LabTirName{SC-ParCom}]
    \begin{mathpar}
      \inferrule*{
        \cseq[\phi]{\ty{\Gamma}}{\conf{C}}
        \\
        \cseq[\phi']{\ty{\Delta}}{\conf{D}}
      }{\cseq[\phi+\phi']{\ty{\Gamma},\ty{\Delta}}{(\ppar{\conf{C}}{\conf{D}})}}
      \equiv
      \inferrule*{
        \cseq[\phi']{\ty{\Delta}}{\conf{D}}
        \\
        \cseq[\phi]{\ty{\Gamma}}{\conf{C}}
      }{\cseq[\phi'+\phi]{\ty{\Gamma},\ty{\Delta}}{(\ppar{\conf{D}}{\conf{C}})}}
    \end{mathpar}
  \end{case*}


  \begin{case*}[\LabTirName{SC-ParAsc}]
    \begin{mathpar}
      \inferrule*{
        \cseq[\phi]{\ty{\Gamma}}{\conf{C}}
        \\
        \inferrule*{
          \cseq[\phi']{\ty{\Delta}}{\conf{D}}
          \\
          \cseq[\phi'']{\ty{\Theta}}{\conf{E}}
        }{\cseq[\phi'+\phi'']{\ty{\Delta},\ty{\Theta}}{(\ppar{\conf{D}}{\conf{E}})}}
      }{\cseq[\phi+\phi'+\phi'']{\ty{\Gamma},\ty{\Delta},\ty{\Theta}}{\ppar{\conf{C}}{(\ppar{\conf{D}}{\conf{E}})}}}
      \equiv
      \inferrule*{
        \inferrule*{
          \cseq[\phi]{\ty{\Gamma}}{\conf{C}}
          \\
          \cseq[\phi']{\ty{\Delta}}{\conf{D}}
        }{\cseq[\phi+\phi']{\ty{\Gamma},\ty{\Delta}}{(\ppar{\conf{C}}{\conf{D}})}}
        \\
        \cseq[\phi'']{\ty{\Theta}}{\conf{E}}
      }{\cseq[\phi+\phi'+\phi'']{\ty{\Gamma},\ty{\Delta},\ty{\Theta}}{\ppar{(\ppar{\conf{C}}{\conf{D}})}{\conf{E}}}}
    \end{mathpar}
  \end{case*}
\end{proof}

%%% Local Variables:
%%% TeX-master: "../priorities"
%%% End:


\begin{thm}\label{thm:pgv-subject-reduction-confs}
  %[Subject Reduction, $\cred$]%
  If $\cseq[\phi]{\ty{\Gamma}}{\conf{C}}$ and $\tm{\conf{C}}\cred\tm{\conf{C'}}$,
  then $\cseq[\phi]{\ty{\Gamma}}{\conf{C'}}$.
\end{thm}
\begin{proof}
  \begin{case}[\LabTirName{E-New}]
    \small
    \begin{mathpar}
      \inferrule*{
        \inferrule*[vdots=1.5em]{
          \inferrule*{
          }{\tmty{\new}{\tylolli{\tyunit}{\typrod{S}{\co{S}}}}}
          \\
          \inferrule*{
          }{\tseq[\cs{\pbot}]{\emptyenv}{\unit}{\tyunit}}
        }{\tseq[\cs{\pbot}]{\emptyenv}{\new\;\unit}{\typrod{S}{\co{S}}}}
      }{\cseq[\phi]{\ty{\Gamma}}{\plug{\conf{F}}{\new\;\unit}}}
      \cred
      \inferrule*{
        \inferrule*{
          \inferrule*[vdots=1.5em]{
            \inferrule*{
            }{\tseq[\cs{\pbot}]{\tmty{x}{S}}{x}{S}}
            \\
            \inferrule*{
            }{\tseq[\cs{\pbot}]{\tmty{y}{\co{S}}}{y}{\co{S}}}
          }{\tseq[\cs{\pbot}]{\tmty{x}{S},\tmty{y}{\co{S}}}{\pair{x}{y}}{\typrod{S}{\co{S}}}}
        }{\cseq[\phi]{\ty{\Gamma},\tmty{x}{S},\tmty{y}{\co{S}}}{\plug{\conf{F}}{\pair{x}{y}}}}
      }{\cseq[\phi]{\ty{\Gamma}}{\res{x}{y}{\plug{\conf{F}}{\pair{x}{y}}}}}
  \end{mathpar}
  \end{case}
  \begin{case}[\LabTirName{E-Spawn}]
    \small
    \begin{mathpar}
      \inferrule*{
        \inferrule*[vdots=1.5em]{
          \inferrule*{
          }{\tmty{\spawn}{\tylolli{(\tylolli[\cs{p},\cs{q}]{\tyunit}{\tyunit})}{\tyunit}}}
          \tseq[\cs{\pbot}]{\ty{\Delta}}{V}{\tylolli[\cs{p},\cs{q}]{\tyunit}{\tyunit}}
        }{\tseq[\cs{\pbot}]{\ty{\Delta}}{\spawn\;V}{\tyunit}}
      }{\cseq[\phi]{\ty{\Gamma},\ty{\Delta}}{\plug{\conf{F}}{\spawn\;V}}}
      \\
      \begin{turn}{270}
        \cred
      \end{turn}
      \\
      \inferrule*{
        \inferrule*{
          \inferrule*[vdots=1.5em]{
          }{\tseq[\cs{\pbot}]{\emptyenv}{\unit}{\tyunit}}
        }{\cseq[\phi]{\ty{\Gamma}}{\plug{\conf{F}}{\unit}}}
        \\
        \inferrule*{
          \inferrule*{
            \tseq[\cs{\pbot}]{\ty{\Delta}}{V}{\tylolli[\cs{p},\cs{q}]{\tyunit}{\tyunit}}
            \\
            \inferrule*{
            }{\tseq[\cs{\pbot}]{\emptyenv}{\unit}{\tyunit}}
          }{\tseq[\cs{q}]{\ty{\Delta}}{V\;\unit}{\tyunit}}
        }{\cseq[\child]{\ty{\Delta}}{\child\;(V\;\unit)}}
      }{\cseq[\phi]{\ty{\Gamma},\ty{\Delta}}{\ppar{\plug{\conf{F}}{\unit}}{\child\;(V\;\unit)}}}
    \end{mathpar}
  \end{case}
  \begin{case}[\LabTirName{E-Send}]
    See \cref{fig:pgv-subject-reduction-cred-send}.
  \end{case}
  \begin{case}[\LabTirName{E-Close}]
    See \cref{fig:pgv-subject-reduction-cred-close}.
  \end{case}
  \begin{case}[\LabTirName{E-LiftC}]
    \todo{By induction on the evaluation context $\conf{G}$.}
  \end{case}
  \begin{case}[\LabTirName{E-LiftM}]
    By \cref{thm:pgv-subject-reduction-terms}.
  \end{case}
  \begin{case}[\LabTirName{E-LiftE}]
    By \cref{thm:pgv-subject-congruence}.
  \end{case}
\end{proof}
\begin{landscape}
\begin{figure}[ht!]
     \small
    \begin{mathpar}
      \inferrule*{
        \inferrule*{
          \inferrule*{
            \inferrule*[vdots=1.5em]{
              \inferrule*{
              }{\tmty{\send}{\tylolli[\cs{\ptop},\cs{o}]{\typrod{T}{\tysend[\cs{o}]{T}{S}}}{S}}}
              \inferrule*{
                \tseq[\cs{p}]{\ty{\Delta}}{V}{T}
                \\
                \inferrule*{
                }{\tseq[\cs{\pbot}]{\tmty{x}{\tysend[\cs{o}]{T}{S}}}{x}{\tysend[\cs{o}]{T}{S}}}
              }{\tseq[\cs{p}]{\ty{\Delta},\tmty{x}{\tysend[\cs{o}]{T}{S}}}
                {\pair{V}{x}}{\typrod{T}{\tysend[\cs{o}]{T}{S}}}}
            }{\tseq[\cs{p}\sqcup\cs{o}]{\ty{\Delta},\tmty{x}{\tysend[\cs{o}]{T}{S}}}{\send\;{\pair{V}{x}}}{S}}
          }{\cseq[\phi]
            {\ty{\Gamma},\ty{\Delta},\tmty{x}{\tysend[\cs{o}]{T}{S}}}
            {\plug{\conf{F}}{\send\;{\pair{V}{x}}}}}
          \\
          \inferrule*{
            \inferrule*[vdots=1.5em]{
              \inferrule*{
              }{\tmty{\recv}{\tylolli[\cs{\ptop},\cs{o}]{\tyrecv[\cs{o}]{T}{\co{S}}}{\typrod{T}{\co{S}}}}}
              \\
              \inferrule*{
              }{\tseq[\cs{\pbot}]{\tmty{y}{\tyrecv[\cs{o}]{T}{\co{S}}}}{y}{\tyrecv[\cs{o}]{T}{\co{S}}}}
            }{\tseq[\cs{o}]
              {\tmty{y}{\tyrecv[\cs{o}]{T}{\co{S}}}}
              {\recv\;y}
              {\typrod{T}{\co{S}}}}
          }{\cseq[\phi']
            {\ty{\Theta},\tmty{y}{\tyrecv[\cs{o}]{T}{\co{S}}}}
            {\plug{\conf{F'}}{\recv\;{y}}}}
        }{\cseq[\phi+\phi']
          {\ty{\Gamma},\ty{\Delta},\ty{\Theta},
            \tmty{x}{\tysend[\cs{o}]{T}{S}},\tmty{y}{\tyrecv[\cs{o}]{T}{\co{S}}}}
          {\ppar{\plug{\conf{F}}{\send\;{\pair{V}{x}}}}{\plug{\conf{F'}}{\recv\;{y}}}}}
      }{\cseq[\phi+\phi']
        {\ty{\Gamma},\ty{\Delta},\ty{\Theta}}
        {\res{x}{y}{(\ppar
            {\plug{\conf{F}}{\send\;{\pair{V}{x}}}}
            {\plug{\conf{F'}}{\recv\;{y}}})}}}
      \\
      \begin{turn}{270}
        \cred
      \end{turn}
      \\
      \inferrule*{
        \inferrule*{
          \inferrule*{
            \inferrule*[vdots=1.5em]{
            }{\tseq[\cs{\pbot}]{\tmty{x}{S}}{x}{S}}
          }{\cseq[\phi]{\ty{\Gamma},\tmty{x}{S}}{\plug{\conf{F}}{x}}}
          \\
          \inferrule*{
            \inferrule*[vdots=1.5em]{
              \tseq[\cs{p}]{\ty{\Delta}}{V}{T}
              \\
              \inferrule*{
              }{\tseq[\cs{\pbot}]{\ty{\Delta},\tmty{y}{\co{S}}}{y}{\co{S}}}
            }{\tseq[\cs{p}]{\ty{\Delta},\tmty{y}{\co{S}}}{\pair{V}{y}}{\typrod{T}{\co{S}}}}
          }{\cseq[\phi']
            {\ty{\Delta},\ty{\Theta},\tmty{y}{\co{S}}}
            {\plug{\conf{F'}}{\pair{V}{y}}}}
        }{\cseq[\phi+\phi']
          {\ty{\Gamma},\ty{\Delta},\ty{\Theta},
            \tmty{x}{S},\tmty{y}{\co{S}}}
          {\ppar
            {\plug{\conf{F}}{x}}
            {\plug{\conf{F'}}{\pair{V}{y}}}}}
      }{\cseq[\phi+\phi']
        {\ty{\Gamma},\ty{\Delta},\ty{\Theta}}
        {\res{x}{y}{(\ppar
            {\plug{\conf{F}}{x}}
            {\plug{\conf{F'}}{\pair{V}{y}}})}}}
    \end{mathpar}
  \caption{Subject Reduction (\LabTirName{E-Send})}
  \label{fig:pgv-subject-reduction-cred-send}
\end{figure}
\begin{figure}[ht!]
     \small
    \begin{mathpar}
      \inferrule*{
        \inferrule*{
          \inferrule*{
            \inferrule*[vdots=1.5em]{
              \inferrule*{
              }{\tmty{\close}{\tylolli[\cs{\ptop},\cs{o}]{\tyends[\cs{o}]}{\tyunit}}}
              \\
              \inferrule*{
              }{\tseq[\cs{\pbot}]{\tmty{x}{\tyends[\cs{o}]}}{x}{\tyends[\cs{o}]}}
            }{\tseq[\cs{o}]{\tmty{x}{\tyends[\cs{o}]}}{\close\;{x}}{\tyunit}}
          }{\cseq[\phi]
            {\ty{\Gamma},\tmty{x}{\tyends[\cs{o}]}}
            {\plug{\conf{F}}{\close\;{x}}}}
          \\
          \inferrule*{
            \inferrule*[vdots=1.5em]{
              \inferrule*{
              }{\tmty{\wait}{\tylolli[\cs{\ptop},\cs{o}]{\tyendr[\cs{o}]}{\tyunit}}}
              \\
              \inferrule*{
              }{\tseq[\cs{\pbot}]{\tmty{y}{\tyendr[\cs{o}]}}{y}{\tyendr[\cs{o}]}}
            }{\tseq[\cs{o}]{\tmty{y}{\tyendr[\cs{o}]}}{\wait\;{y}}{\tyunit}}
          }{\cseq[\phi']
            {\ty{\Delta},\tmty{y}{\tyendr[\cs{o}]}}
            {\plug{\conf{F'}}{\wait\;{y}}}}
        }{\cseq[\phi+\phi']
          {\ty{\Gamma},\ty{\Delta},
            \tmty{x}{\tyends[\cs{o}]},\tmty{y}{\tyendr[\cs{o}]}}
          {\ppar{\plug{\conf{F}}{\close\;{x}}}{\plug{\conf{F'}}{\wait\;{y}}}}}
      }{\cseq[\phi+\phi']
        {\ty{\Gamma},\ty{\Delta}}
        {\res{x}{y}{(\ppar
            {\plug{\conf{F}}{\close\;{x}}}
            {\plug{\conf{F'}}{\wait\;{y}}})}}}
      \\
      \begin{turn}{270}
        \cred
      \end{turn}
      \\
      \inferrule*{
        \inferrule*{
          \inferrule*[vdots=1.5em]{
          }{\tseq[\cs{\pbot}]{\emptyenv}{\unit}{\tyunit}}
        }{\cseq[\phi]{\ty{\Gamma}}{\plug{\conf{F}}{\unit}}}
        \\
        \inferrule*{
          \inferrule*[vdots=1.5em]{
          }{\tseq[\cs{\pbot}]{\emptyenv}{\unit}{\tyunit}}
        }{\cseq[\phi']
          {\ty{\Delta}}
          {\plug{\conf{F'}}{\unit}}}
      }{\cseq[\phi+\phi']
        {\ty{\Gamma},\ty{\Delta}}
        {\ppar{\plug{\conf{F}}{\unit}}{\plug{\conf{F'}}{\unit}}}}
    \end{mathpar}
  \caption{Subject Reduction (\LabTirName{E-Close})}
  \label{fig:pgv-subject-reduction-cred-close}
\end{figure}
\end{landscape}

%%% Local Variables:
%%% TeX-master: "../priorites"
%%% End:


\subsection{Progress and Deadlock Freedom}
PGV satisfies progress, as PGV configurations either reduce or are in normal form. However, the normal forms may seem surprising at first, as evaluating a well-typed PGV term does not necessarily produce {just} a value. If a term returns an endpoint, then its normal form contains a thread which is ready to communicate on the dual of that endpoint. This behaviour is not new to PGV.

Let us consider an example, adapted from Lindley and Morris~\cite{lindleymorris15}, in which a term returns an endpoint linked to an echo server. The echo server receives a value and sends it back unchanged. Consider the program which creates a new channel, with endpoints $\tm{x}$ and $\tm{x'}$, spawns off an echo server listening on $\tm{x}$, and then returns $\tm{x'}$:
\[
  \begin{array}{ll}
    \tm{\main}
     & \tm{\letpair{x}{x'}{\new\;\unit}{}}
    \\
     & \tm{\andthen{\spawn\;(\lambda\unit.\echo_{x})}{x'}}
  \end{array}
  \hfill\qquad\hfill
  \begin{array}{lcl}
    \tm{\echo_x}
     & \defeq
     & \tm{\letpair{y}{x}{\recv\;{x}}{}}
    \\
     &
     & \tm{\letbind{x}{\send\;\pair{y}{x}}{\close\;x}}
  \end{array}
\]

If we reduce the above program, we get $\tm{\res{x}{x'}{(\ppar{\child\;\echo_{x}}{\main\;x'})}}$. Clearly, no more evaluation is possible, even though the configuration contains the thread $\tm{\child\;\echo_{x}}$, which is blocked on $\tm{x}$. In~\cref{cor:pgv-closed-progress} we will show that if a term does not return an endpoint, it must produce {only} a value.

\emph{Actions} are terms which perform communication actions and which synchronise between two threads.
\begin{defi}\label{def:pgv-actions}
  %[Actions]%
  A~term \emph{acts on} an endpoint $\tm{x}$ if it is $\tm{\send\;\pair{V}{x}}$, $\tm{\recv\;{x}}$, $\tm{\close\;{x}}$, or $\tm{\wait\;{x}}$. A~term is an \emph{action} if it acts on some endpoint $\tm{x}$.
\end{defi}

\emph{Ready terms} are terms which perform communication actions, either by themselves, \eg creating a new channel or thread, or with another thread, \eg sending or receiving.
It is worth mentioning that the notion of readiness presented here is akin to {live} processes introduced by Caires and Pfenning \cite{cairespfenning10,DardhaP22}, and {poised} processes introduced by Pfenning and Griffith \cite{PfenningG15} and later used by Balzer \etal \cite{balzerpfenning17,balzertoninho19}. Ready processes like live/poised processes denote processes that are ready to communicate on their providing channel.
\begin{defi}\label{def:pgv-ready-actions}
  % [Ready Terms]%
  A~term $\tm{L}$ is \emph{ready} if it is of the form $\tm{\plug{E}{M}}$, where $\tm{M}$ is of the form $\tm{\new}$, $\tm{\spawn\;N}$, $\tm{\link\;\pair{x}{y}}$, or $\tm{M}$ acts on $\tm{x}$. In the latter case, we say that $\tm{L}$ is \emph{ready to act on} $\tm{x}$ or is \emph{blocked on}.
\end{defi}

\emph{Progress} for the term language is standard for GV, and deviates from progress for linear $\lambda$-calculus only in that terms may reduce to values or \emph{ready terms}, where the definition of ready terms encompasses all terms whose reduction is struck on some constant $\tm{K}$.

\begin{lem}\label{lem:pgv-open-progress-terms}
  %[Progress, $\tred$]%
  If $\tseq[\cs{p}]{\ty{\Gamma}}{M}{T}$ and $\ty{\Gamma}$ contains only session types, then:
  \begin{enumerate*}[label= (\roman*) ]
    \item $\tm{M}$ is a value;
    \item $\tm{M}\tred\tm{N}$ for some $\tm{N}$; or
    \item $\tm{M}$ is ready.
  \end{enumerate*}
\end{lem}

With ``$\ty{\Gamma}$ contains only session types'' we mean that for every $\tmty{x}{T}\in\ty{\Gamma}$, $\ty{T}$ is a session type, \ie, is of the form $\ty{S}$.

\emph{Canonical forms} deviate from those for GV, in that we opt to move all $\nu$-binders to the top. The standard GV canonical form, alternating $\nu$-binders and their corresponding parallel compositions, does not work for PGV, since multiple channels may be split across a single parallel composition.

A~configuration either reduces, or it is equivalent to configuration in normal form. Crucial to the normal form is that each term $\tm{M_i}$ is blocked on the corresponding channel $\tm{x_i}$, and hence no two terms act on dual endpoints. Furthermore, no term $\tm{M_i}$ can perform a communication action by itself, since those are excluded by the definition of actions.
Finally, as a corollary, we get that well-typed terms which do not return endpoints return {just} a value:

\begin{defi}\label{def:pgv-canonical-forms}
  %[Canonical Forms]%
  A~configuration $\tm{\conf{C}}$ is in canonical form if it is of the form
  $\tm{\res{x_1}{x'_1}{\dots\res{x_n}{x'_n}{(\child\;M_1\parallel\dots\parallel\child\;M_m\parallel\main\;N)}}}$
  where no term $\tm{M_i}$ is a value.
\end{defi}

\begin{lem}\label{lem:pgv-canonical-forms}
  % [Canonical Forms]%
  If $\cseq[\main]{\ty{\Gamma}}{\conf{C}}$, there exists some $\tm{\conf{D}}$ such that $\tm{\conf{C}}\equiv\tm{\conf{D}}$ and $\tm{\conf{D}}$ is in canonical form.
\end{lem}
\proof
We move any $\nu$-binders to the top using \LabTirName{SC-ResExt}, discard any superfluous occurrences of $\tm{\child\;\unit}$ using \LabTirName{SC-ParNil}, and move the main thread to the rightmost position using \LabTirName{SC-ParComm} and \LabTirName{SC-ParAssoc}.
\qed

\begin{defi}
  % [Normal Forms]%
  A~configuration $\tm{\conf{C}}$ is in normal form if it is of the form
  $\tm{\res{x_1}{x'_1}{\dots\res{x_n}{x'_n}{(\child\;M_1\parallel\dots\parallel\child\;M_m\parallel\main\;V)}}}$
  where each $\tm{M_i}$ is ready to act on $\tm{x_i}$.
\end{defi}

\begin{lem}\label{lem:pgv-ready-priority}
  If $\tseq[\cs{p}]{\ty{\Gamma}}{L}{T}$ is ready to act on $\tmty{x}{S}\in\ty{\Gamma}$, then the priority bound $\cs{p}$ is some priority $\cs{o}$, \ie not $\cs{\pbot}$ or $\cs{\ptop}$.
\end{lem}
\proof
Let $\tm{L}=\tm{\plug{E}{M}}$. By induction on the structure of $\tm{E}$. $\tm{M}$ has priority $\pr({\ty{S}})$, and each constructor of the evaluation context $\tm{E}$ passes on the \emph{maximum} of the priorities of its premises. No rule introduces the priority bound $\cs{\ptop}$ on the sequent.
\qed

\begin{thm}\label{thm:pgv-closed-progress-confs}
  % [Progress, $\cred$]%
  If $\cseq[\main]{\emptyenv}{\conf{C}}$ and $\tm{\conf{C}}$ is in canonical form, then either $\tm{\conf{C}}\cred\tm{\conf{D}}$ for some $\tm{\conf{D}}$; or $\tm{\conf{C}\equiv\conf{D}}$ for some $\tm{\conf{D}}$ in normal form.
\end{thm}
\begin{proof}
  \todo{Shouldn't the statement be such that the def of C comes right after stating C is in canonical form, rather than in the or branch?}

  \label{prf:thm-pgv-closed-progress-confs}
  Let $\tm{\conf{C}}=\tm{\res{x_1}{x'_1}{\dots\res{x_n}{x'_n}{(\child\;M_1\parallel\dots\parallel\child\;M_m\parallel\main\;N)}}}$.
  We apply \cref{lem:pgv-open-progress-terms} to each $\tm{M_i}$ and $\tm{N}$. If for any $\tm{M_i}$ or $\tm{N}$ we obtain a reduction $\tm{M_i}\tred\tm{M'_i}$ or $\tm{N}\tred\tm{N'}$, we apply \LabTirName{E-LiftM} and \LabTirName{E-LiftC} to obtain a reduction on $\tm{\conf{C}}$.
  Otherwise, each term $\tm{M_i}$ is ready, and $\tm{N}$ is either ready or a value.

  Pick the \emph{ready} term $\tm{L}\in\{\tm{M_1},\dots,\tm{M_m},\tm{N}\}$ with the smallest priority bound. There are four cases: \todo{the below is presented as if there are only 3 cases, not 4...}
  \begin{itemize}
  \item
    If $\tm{L}$ is a new $\tm{\plug{E}{\new}}$, we apply \LabTirName{E-New}.
  \item
    If $\tm{L}$ is a spawn $\tm{\plug{E}{\spawn\;M}}$, we apply \LabTirName{E-Spawn}.
  \item
    If $\tm{L}$ is a link $\tm{\plug{E}{\link\;\pair{y}{z}}}$ or $\tm{\plug{E}{\link\;\pair{z}{y}}}$, we apply \LabTirName{E-Link}.
  \end{itemize}
  Otherwise, $\tm{L}$ is ready to act on some endpoint $\tmty{y}{S}$. Let $\tmty{y'}{\co{S}}$ be the dual endpoint of $\tm{y}$. There must be a term $\tm{L'}\in\{\tm{M_1},\dots,\tm{M_m},\tm{N}\}$ which uses $\tm{y'}$.
  \todo{in the same lines as DF for PCP, justify here why it is true that "there must be a term..."}
  There are two cases:
  \begin{itemize}
  \item
    $\tm{L'}$ is ready. By~\cref{lem:pgv-ready-priority}, the priority of $\tm{L}$ is $\pr(\ty{S})$. By duality, $\pr(\ty{\co{S}})=\pr(\ty{S})$.

    We cannot have $\tm{L}=\tm{L'}$, otherwise the action on $\tm{y'}$ would be guarded by the action on $\tm{y}$, requiring $\pr(\ty{\co{S}})<\pr(\ty{S})$. The term $\tm{L'}$ must be ready to act on $\tm{y'}$, otherwise the action $\tm{y'}$ would be guarded by another action with priority smaller than $\pr{(\ty{S})}$, which contradicts our choice of $\tm{L}$ as having the smallest priority. Therefore, we have two terms ready to act on dual endpoints. We apply the appropriate reduction rule, \ie \LabTirName{E-Send} or \LabTirName{E-Close}.
  \item
    $\tm{L'}=\tm{N}$ and is a value. We rewrite $\tm{\conf{C}}$ to put $\tm{L}$ in the position corresponding to the endpoint it is blocked on, using \LabTirName{SC-ParComm}, \LabTirName{SC-ParAssoc}, and optionally \LabTirName{SC-ResSwap}. We then repeat the steps above with the term with the next smallest priority, until either we find a reduction, or the configuration has reached the desired normal form. (The argument based on the priority being the smallest continues to hold, since we know that neither $\tm{L}$ nor $\tm{L'}$ will be picked, and no other term uses $\tm{y}$ or $\tm{y'}$.)
  \end{itemize} 
\end{proof}

%%% Local Variables:
%%% TeX-master: "../priorities"
%%% End:


\begin{cor}\label{cor:pgv-closed-progress}
  If $\cseq[\phi]{\emptyenv}{\conf{C}}$, $\tm{\conf{C}}\centernot\cred$, and $\tm{\conf{C}}$ contains no endpoints, then $\tm{\conf{C}}\equiv\tm{\phi\;V}$ for some value $\tm{V}$.
\end{cor}


An immediate consequence of \cref{thm:pgv-closed-progress-confs} and \cref{cor:pgv-closed-progress} is that \emph{a term which does not return an endpoint will complete all its communication actions, thus satisfying deadlock freedom.}
