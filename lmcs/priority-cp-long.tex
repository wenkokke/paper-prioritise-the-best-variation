% !TeX root = priorities.tex
% \section{Priority GV}
% \begingroup
% \usingnamespace{pgv}
% \restatelemma{lempgvvaluedone}
% \begin{proof}
  \label{prf:lem-pgv-value-done}
  By induction on the derivation of $\tseq[\cs{o}]{\ty{\Gamma}}{V}{T}$.

  \begin{case*}[\LabTirName{T-Lam}]
    Immediately.
    \begin{mathpar}
      \inferrule*{
        \tseq[\cs{q}]{\ty{\Gamma},\tmty{x}{T}}{M}{U}
      }{\tseq[\cs{\pbot}]{\ty{\Gamma}}{\lambda x.M}{\tylolli[\cs{\pr(\ty{\Gamma})},\cs{q}]{T}{U}}}
    \end{mathpar}
  \end{case*}
  \begin{case*}[\LabTirName{T-Unit}]
    Immediately.
    \begin{mathpar}
      \inferrule*{
      }{\tseq[\cs{\pbot}]{\emptyenv}{\unit}{\tyunit}}
    \end{mathpar}
  \end{case*}
  \begin{case*}[\LabTirName{T-Pair}]
    The induction hypotheses give us $\cs{p}=\cs{q}=\cs{\pbot}$, hence $\cs{p}\sqcup\cs{q}=\cs{\pbot}$, and $\pr(\ty{\Gamma})=\pr(\ty{T})$ and $\pr(\ty{\Delta})=\pr(\ty{U})$, hence $\pr(\ty{\Gamma},\ty{\Delta})=\pr(\ty{\Gamma})\sqcap\pr(\ty{\Delta})=\pr(\ty{T})\sqcap\pr(\ty{U})=\pr(\ty{\typrod{T}{U}})$.
    \begin{mathpar}
      \inferrule*{
        \tseq[\cs{p}]{\ty{\Gamma}}{V}{T}
        \\
        \tseq[\cs{q}]{\ty{\Delta}}{W}{U}
        \\
        \cs{p}<\pr(\ty{\Delta})
      }{\tseq[\cs{p}\sqcup\cs{q}]{\ty{\Gamma},\ty{\Delta}}{\pair{V}{W}}{\typrod{T}{U}}}
    \end{mathpar}
  \end{case*}
  \begin{case*}[\LabTirName{T-Inl}]
    The induction hypothesis gives us $\cs{p}=\cs{\pbot}$, and $\pr(\ty{\Gamma})=\pr{\ty{T}}$. We know $\pr(\ty{T})=\pr({\ty{U}})$, hence $\pr(\ty{\Gamma})=\pr(\ty{\tysum{T}{U}})$.
    \begin{mathpar}
      \inferrule*{
        \tseq[\cs{p}]{\ty{\Gamma}}{V}{T}
        \\
        \pr(\ty{T})=\pr(\ty{U})
      }{\tseq[\cs{p}]{\ty{\Gamma}}{\inl{V}}{\tysum{T}{U}}}
    \end{mathpar}
  \end{case*}
  \begin{case*}[\LabTirName{T-Inr}]
    The induction hypothesis gives us $\cs{p}=\cs{\pbot}$, and $\pr(\ty{\Gamma})=\pr{\ty{U}}$. We know $\pr(\ty{T})=\pr({\ty{U}})$, hence $\pr(\ty{\Gamma})=\pr(\ty{\tysum{T}{U}})$.
    \begin{mathpar}
      \inferrule*{
        \tseq[\cs{p}]{\ty{\Gamma}}{V}{U}
        \\
        \pr(\ty{T})=\pr(\ty{U})
      }{\tseq[\cs{p}]{\ty{\Gamma}}{\inr{V}}{\tysum{T}{U}}}
    \end{mathpar}
  \end{case*}
\end{proof}

%%% Local Variables:
%%% TeX-master: "../priorities"
%%% End:


% \restatelemma{lempgvsubstitution}
% \begin{proof}
  By induction on the derivation of $\tseq[\cs{p}]{\ty{\Gamma},\tmty{x}{U'}}{M}{T}$.
  \begin{case*}[\LabTirName{T-Var}]
    By \cref{lem:value-done}, $\cs{q}=\cs{\pbot}$.
    \begin{mathpar}
      \inferrule*{
      }{\tseq[\cs{\pbot}]{\tmty{x}{U'}}{x}{U'}}
      \substarrow{V}{x}
      \tseq[\cs{\pbot}]{\ty{\Theta}}{V}{U'}
    \end{mathpar}
  \end{case*}
  \begin{case*}[\LabTirName{T-Lam}]
    By \cref{lem:value-done}, $\pr(\ty{\Theta})=\pr(\ty{U'})$, hence $\pr(\ty{\Gamma},\ty{\Theta})=\pr(\ty{\Gamma},\ty{U'})$.
    \begin{mathpar}
      \inferrule*{
        \tseq[\cs{o}]{\ty{\Gamma},\tmty{x}{U'},\tmty{y}{T}}{M}{U}
      }{\tseq[\cs{\pbot}]{\ty{\Gamma},\tmty{x}{U'}}
        {\lambda y.M}
        {\tylolli[\cs{\pr(\ty{\Gamma},\ty{U'})},\cs{o}]{T}{U}}}
      \substarrow{V}{x}
      \inferrule*{
        \tseq[\cs{o}]{\ty{\Gamma},\ty{\Theta},\tmty{y}{T}}{\subst{M}{V}{x}}{U}
      }{\tseq[\cs{\pbot}]{\ty{\Gamma},\ty{\Theta}}
        {\lambda y.\subst{M}{V}{x}}
        {\tylolli[\cs{\pr(\ty{\Gamma},\ty{\Theta})},\cs{o}]{T}{U}}}
    \end{mathpar}
  \end{case*}
  \begin{case*}[\LabTirName{T-App}]
    There are two subcases:
    \begin{subcase*}[$\tm{x}\in\tm{M}$]
      Immediately, from the induction hypothesis.
      \begin{mathpar}
        \inferrule*{
          \tseq[\cs{p}]{\ty{\Gamma},\tmty{x}{U'}}{M}{\tylolli[\cs{o},\cs{r}]{T}{U}}
          \\
          \tseq[\cs{q}]{\ty{\Delta}}{N}{T}
          \\
          \cs{p}<\pr(\ty{\Delta})
          \\
          \cs{q}<\cs{o}
        }{\tseq[\cs{p}\sqcup\cs{q}\sqcup\cs{r}]{\ty{\Gamma},\ty{\Delta},\tmty{x}{U'}}{M\;N}{U}}
        \substarrow{V}{x}
        \inferrule*{
          \tseq[\cs{p}]{\ty{\Gamma},\ty{\Theta}}{\subst{M}{V}{x}}{\tylolli[\cs{o},\cs{r}]{T}{U}}
          \\
          \tseq[\cs{q}]{\ty{\Delta}}{N}{T}
          \\
          \cs{p}<\pr(\ty{\Delta})
          \\
          \cs{q}<\cs{o}
        }{\tseq[\cs{p}\sqcup\cs{q}\sqcup\cs{r}]{\ty{\Gamma},\ty{\Delta},\ty{\Theta}}{(\subst{M}{V}{x})\;N}{U}}
      \end{mathpar}
    \end{subcase*}
    \begin{subcase*}[$\tm{x}\in\tm{N}$]
      By \cref{lem:value-done}, $\pr(\ty{\Theta})=\pr(\ty{U'})$, hence $\pr(\ty{\Delta},\ty{\Theta})=\pr(\ty{\Delta},\ty{U'})$.
      \begin{mathpar}
        \inferrule*{
          \tseq[\cs{p}]{\ty{\Gamma}}{M}{\tylolli[\cs{o},\cs{r}]{T}{U}}
          \\
          \tseq[\cs{q}]{\ty{\Delta},\tmty{x}{U'}}{N}{T}
          \\
          \cs{p}<\pr(\ty{\Delta},\ty{U'})
          \\
          \cs{q}<\cs{o}
        }{\tseq[\cs{p}\sqcup\cs{q}\sqcup\cs{r}]{\ty{\Gamma},\ty{\Delta},\tmty{x}{U'}}{M\;N}{U}}
        \substarrow{V}{x}
        \inferrule*{
          \tseq[\cs{p}]{\ty{\Gamma}}{M}{\tylolli[\cs{o},\cs{r}]{T}{U}}
          \\
          \tseq[\cs{q}]{\ty{\Delta},\ty{\Theta}}{\subst{N}{V}{x}}{T}
          \\
          \cs{p}<\pr(\ty{\Delta},\ty{\Theta})
          \\
          \cs{q}<\cs{o}
        }{\tseq[\cs{p}\sqcup\cs{q}\sqcup\cs{r}]{\ty{\Gamma},\ty{\Delta},\ty{\Theta}}{M\;(\subst{N}{V}{x})}{U}}
      \end{mathpar}
    \end{subcase*}
  \end{case*}
  \begin{case*}[\LabTirName{T-LetUnit}]
    There are two subcases:
    \begin{subcase*}[$\tm{x}\in\tm{M}$]
      Immediately, from the induction hypothesis.
      \begin{mathpar}
        \inferrule*{
          \tseq[\cs{p}]{\ty{\Gamma},\tmty{x}{U'}}{M}{\tyunit}
          \\
          \tseq[\cs{q}]{\ty{\Delta}}{N}{T}
          \\
          \cs{p}<\pr(\ty{\Delta})
        }{\tseq[\cs{p}\sqcup\cs{q}]{\ty{\Gamma},\ty{\Delta},\tmty{x}{U'}}{\letunit{M}{N}}{T}}
        \substarrow{V}{x}
        \inferrule*{
          \tseq[\cs{p}]{\ty{\Gamma},\ty{\Theta}}{\subst{M}{V}{x}}{\tyunit}
          \\
          \tseq[\cs{q}]{\ty{\Delta}}{N}{T}
          \\
          \cs{p}<\pr(\ty{\Delta})
        }{\tseq[\cs{p}\sqcup\cs{q}]{\ty{\Gamma},\ty{\Delta},\ty{\Theta}}{\letunit{\subst{M}{V}{x}}{N}}{T}}
      \end{mathpar}
    \end{subcase*}
    \begin{subcase*}[$\tm{x}\in\tm{N}$]
      By \cref{lem:value-done}, $\pr(\ty{\Theta})=\pr(\ty{U'})$, hence $\pr(\ty{\Delta},\ty{\Theta})=\pr(\ty{\Delta},\ty{U'})$.
      \begin{mathpar}
        \inferrule*{
          \tseq[\cs{p}]{\ty{\Gamma}}{M}{\tyunit}
          \\
          \tseq[\cs{q}]{\ty{\Delta},\tmty{x}{U'}}{N}{T}
          \\
          \cs{p}<\pr(\ty{\Delta},\ty{U'})
        }{\tseq[\cs{p}\sqcup\cs{q}]{\ty{\Gamma},\ty{\Delta},\tmty{x}{U'}}{\letunit{M}{N}}{T}}
        \substarrow{V}{x}
        \inferrule*{
          \tseq[\cs{p}]{\ty{\Gamma}}{M}{\tyunit}
          \\
          \tseq[\cs{q}]{\ty{\Delta},\ty{\Theta}}{\subst{N}{V}{x}}{T}
          \\
          \cs{p}<\pr(\ty{\Delta},\ty{\Theta})
        }{\tseq[\cs{p}\sqcup\cs{q}]{\ty{\Gamma},\ty{\Delta},\ty{\Theta}}{\letunit{M}{\subst{N}{V}{x}}}{T}}
      \end{mathpar}
    \end{subcase*}
  \end{case*}
  \begin{case*}[\LabTirName{T-Pair}]
    There are two subcases:
    \begin{subcase*}[$\tm{x}\in\tm{M}$]
      Immediately, from the induction hypothesis.
      \begin{mathpar}
        \inferrule*{
          \tseq[\cs{p}]{\ty{\Gamma},\tmty{x}{U'}}{M}{T}
          \\
          \tseq[\cs{q}]{\ty{\Delta}}{N}{U}
          \\
          \cs{p}<\pr(\ty{\Delta},\ty{U'})
        }{\tseq[\cs{p}\sqcup\cs{q}]{\ty{\Gamma},\ty{\Delta},\tmty{x}{U'}}{\pair{M}{N}}{\typrod{T}{U}}}
        \substarrow{V}{x}
        \inferrule*{
          \tseq[\cs{p}]{\ty{\Gamma},\ty{\Theta}}{\subst{M}{V}{x}}{T}
          \\
          \tseq[\cs{q}]{\ty{\Delta}}{N}{U}
          \\
          \cs{p}<\pr(\ty{\Delta},\ty{\Theta})
        }{\tseq[\cs{p}\sqcup\cs{q}]{\ty{\Gamma},\ty{\Delta},\ty{\Theta}}{\pair{\subst{M}{V}{x}}{N}}{\typrod{T}{U}}}
      \end{mathpar}
    \end{subcase*}
    \begin{subcase*}[$\tm{x}\in\tm{N}$]
      By \cref{lem:value-done}, $\pr(\ty{\Theta})=\pr(\ty{U'})$, hence $\pr(\ty{\Delta},\ty{\Theta})=\pr(\ty{\Delta},\ty{U'})$.
      \begin{mathpar}
        \inferrule*{
          \tseq[\cs{p}]{\ty{\Gamma}}{M}{T}
          \\
          \tseq[\cs{q}]{\ty{\Delta},\tmty{x}{U'}}{N}{U}
          \\
          \cs{p}<\pr(\ty{\Delta},\ty{U'})
        }{\tseq[\cs{p}\sqcup\cs{q}]{\ty{\Gamma},\ty{\Delta},\tmty{x}{U'}}{\pair{M}{N}}{\typrod{T}{U}}}
        \substarrow{V}{x}
        \inferrule*{
          \tseq[\cs{p}]{\ty{\Gamma}}{M}{T}
          \\
          \tseq[\cs{q}]{\ty{\Delta},\ty{\Theta}}{\subst{N}{V}{x}}{U}
          \\
          \cs{p}<\pr(\ty{\Delta},\ty{\Theta})
        }{\tseq[\cs{p}\sqcup\cs{q}]{\ty{\Gamma},\ty{\Delta},\ty{\Theta}}{\pair{M}{\subst{N}{V}{x}}}{\typrod{T}{U}}}
      \end{mathpar}
    \end{subcase*}
  \end{case*}
  \begin{case*}[\LabTirName{T-LetPair}]
    There are two subcases:
    \begin{subcase*}[$\tm{x}\in\tm{M}$]
      Immediately, from the induction hypothesis.
      \begin{mathpar}
        \inferrule*{
          \tseq[\cs{p}]{\ty{\Gamma},\tmty{x}{U'}}{M}{\typrod{T}{T'}}
          \\
          \tseq[\cs{q}]{\ty{\Delta},\tmty{y}{T},\tmty{z}{T'}}{N}{U}
          \\
          \cs{p}<\pr(\ty{\Delta},\ty{T},\ty{T'})
        }{\tseq[\cs{p}\sqcup\cs{q}]{\ty{\Gamma},\ty{\Delta},\tmty{x}{U'}}{\letpair{y}{z}{M}{N}}{U}}
        \substarrow{V}{x}
        \inferrule*{
          \tseq[\cs{p}]{\ty{\Gamma},\ty{\Theta}}{\subst{M}{V}{x}}{\typrod{T}{T'}}
          \\
          \tseq[\cs{q}]{\ty{\Delta},\tmty{y}{T},\tmty{z}{T'}}{N}{U}
          \\
          \cs{p}<\pr(\ty{\Delta},\ty{T},\ty{T'})
        }{\tseq[\cs{p}\sqcup\cs{q}]{\ty{\Gamma},\ty{\Delta},\ty{\Theta}}{\letpair{y}{z}{\subst{M}{V}{x}}{N}}{U}}
      \end{mathpar}
    \end{subcase*}
    \begin{subcase*}[$\tm{x}\in\tm{N}$]
      By \cref{lem:value-done}, $\pr(\ty{\Theta})=\pr(\ty{U'})$, hence $\pr(\ty{\Delta},\ty{\Theta},\ty{T},\ty{T'})=\pr(\ty{\Delta},\ty{U'},\ty{T},\ty{T'})$.
      \begin{mathpar}
        \inferrule*{
          \tseq[\cs{p}]{\ty{\Gamma}}{M}{\typrod{T}{T'}}
          \\
          \tseq[\cs{q}]{\ty{\Delta},\tmty{x}{U'},\tmty{y}{T},\tmty{z}{T'}}{N}{U}
          \\
          \cs{p}<\pr(\ty{\Delta},\ty{U'},\ty{T},\ty{T'})
        }{\tseq[\cs{p}\sqcup\cs{q}]{\ty{\Gamma},\ty{\Delta},\tmty{x}{U'}}{\letpair{y}{z}{M}{N}}{U}}
        \substarrow{V}{x}
        \inferrule*{
          \tseq[\cs{p}]{\ty{\Gamma}}{M}{\typrod{T}{T'}}
          \\
          \tseq[\cs{q}]{\ty{\Delta},\ty{\Theta},\tmty{y}{T},\tmty{z}{T'}}{\subst{N}{V}{x}}{U}
          \\
          \cs{p}<\pr(\ty{\Delta},\ty{\Theta},\ty{T},\ty{T'})
        }{\tseq[\cs{p}\sqcup\cs{q}]{\ty{\Gamma},\ty{\Delta},\ty{\Theta}}{\letpair{y}{z}{M}{\subst{N}{V}{x}}}{U}}
      \end{mathpar}
    \end{subcase*}
  \end{case*}
  \begin{case*}[\LabTirName{T-Absurd}]
    \begin{mathpar}
      \inferrule*{
        \tseq[o]{\ty{\Gamma},\tmty{x}{U'}}{M}{\tyvoid}
      }{\tseq[o]{\ty{\Gamma},\ty{\Delta},\tmty{x}{U'}}{\absurd{M}}{T}}
      \substarrow{V}{x}
      \inferrule*{
        \tseq[o]{\ty{\Gamma},\ty{\Theta}}{\subst{M}{V}{x}}{\tyvoid}
      }{\tseq[o]{\ty{\Gamma},\ty{\Delta},\ty{\Theta}}{\absurd{\subst{M}{V}{x}}}{T}}
    \end{mathpar}
  \end{case*}
  \begin{case*}[\LabTirName{T-Inl}]
    \begin{mathpar}
      \inferrule*{
        \tseq[o]{\ty{\Gamma},\tmty{x}{U'}}{M}{T}
        \\
        \pr(\ty{T})=\pr(\ty{U})
      }{\tseq[o]{\ty{\Gamma},\tmty{x}{U'}}{\inl{M}}{\tysum{T}{U}}}
      \substarrow{V}{x}
      \inferrule*{
        \tseq[o]{\ty{\Gamma},\ty{\Theta}}{\subst{M}{V}{x}}{T}
        \\
        \pr(\ty{T})=\pr(\ty{U})
      }{\tseq[o]{\ty{\Gamma},\ty{\Theta}}{\inl{\subst{M}{V}{x}}}{\tysum{T}{U}}}
    \end{mathpar}
  \end{case*}
  \begin{case*}[\LabTirName{T-Inr}]
    \begin{mathpar}
      \inferrule*{
        \tseq[o]{\ty{\Gamma},\tmty{x}{U'}}{M}{U}
        \\
        \pr(\ty{T})=\pr(\ty{U})
      }{\tseq[o]{\ty{\Gamma},\tmty{x}{U'}}{\inr{M}}{\tysum{T}{U}}}
      \substarrow{V}{x}
      \inferrule*{
        \tseq[o]{\ty{\Gamma},\ty{\Theta}}{\subst{M}{V}{x}}{U}
        \\
        \pr(\ty{T})=\pr(\ty{U})
      }{\tseq[o]{\ty{\Gamma},\ty{\Theta}}{\inr{\subst{M}{V}{x}}}{\tysum{T}{U}}}
    \end{mathpar}
  \end{case*}
  \begin{case*}[\LabTirName{T-CaseSum}]
    There are two subcases:
    \begin{subcase*}[$\tm{x}\in\tm{L}$]
      Immediately, from the induction hypothesis.
      \begin{mathpar}
        \inferrule*{
          \tseq[\cs{p}]{\ty{\Gamma},\tmty{x}{U'}}{L}{\tysum{T}{T'}}
          \\
          \tseq[\cs{q}]{\ty{\Delta},\tmty{y}{T}}{M}{U}
          \\
          \tseq[\cs{q}]{\ty{\Delta},\tmty{z}{T'}}{N}{U}
          \\
          \cs{p}<\pr(\ty{\Delta})
        }{\tseq[\cs{p}\sqcup\cs{q}]{\ty{\Gamma},\ty{\Delta},\tmty{x}{U'}}{\casesum{L}{y}{M}{z}{N}}{U}}
        \substarrow{V}{x}
        \inferrule*{
          \tseq[\cs{p}]{\ty{\Gamma},\ty{\Theta}}{\subst{L}{V}{x}}{\tysum{T}{T'}}
          \\
          \tseq[\cs{q}]{\ty{\Delta},\tmty{y}{T}}{M}{U}
          \\
          \tseq[\cs{q}]{\ty{\Delta},\tmty{z}{T'}}{N}{U}
          \\
          \cs{p}<\pr(\ty{\Delta})
        }{\tseq[\cs{p}\sqcup\cs{q}]{\ty{\Gamma},\ty{\Delta},\ty{\Theta}}{\casesum{\subst{L}{V}{x}}{y}{M}{z}{N}}{U}}
      \end{mathpar}
    \end{subcase*}
    \begin{subcase*}[$\tm{x}\in\tm{M}$ and $\tm{x}\in\tm{N}$]
      By \cref{lem:value-done}, $\pr(\ty{\Theta})=\pr(\ty{U'})$, hence $\pr(\ty{\Delta},\ty{\Theta},\ty{T})=\pr(\ty{\Delta},\ty{U'},\ty{T})$ and $\pr(\ty{\Delta},\ty{\Theta},\ty{T'})=\pr(\ty{\Delta},\ty{U'},\ty{T'})$.
      \begin{mathpar}
        \inferrule*{
          \tseq[\cs{p}]{\ty{\Gamma}}{L}{\tysum{T}{T'}}
          \\
          \tseq[\cs{q}]{\ty{\Delta},\tmty{x}{U'},\tmty{y}{T}}{M}{U}
          \\
          \tseq[\cs{q}]{\ty{\Delta},\tmty{x}{U'},\tmty{z}{T'}}{N}{U}
          \\
          \cs{p}<\pr(\ty{\Delta},\ty{U'})
        }{\tseq[\cs{p}\sqcup\cs{q}]{\ty{\Gamma},\ty{\Delta},\tmty{x}{U'}}{\casesum{L}{y}{M}{z}{N}}{U}}
        \substarrow{V}{x}
        \inferrule*{
          \tseq[\cs{p}]{\ty{\Gamma}}{L}{\tysum{T}{T'}}
          \\
          \tseq[\cs{q}]{\ty{\Delta},\ty{\Theta},\tmty{y}{T}}{\subst{M}{V}{x}}{U}
          \\
          \tseq[\cs{q}]{\ty{\Delta},\ty{\Theta},\tmty{z}{T'}}{\subst{N}{V}{x}}{U}
          \\
          \cs{p}<\pr(\ty{\Delta},\ty{\Theta})
        }{\tseq[\cs{p}\sqcup\cs{q}]{\ty{\Gamma},\ty{\Delta},\ty{\Theta}}{\casesum{L}{y}{\subst{M}{V}{x}}{z}{\subst{N}{V}{x}}}{U}}
      \end{mathpar}
    \end{subcase*}
  \end{case*}
  \noindent
  We omit the cases where $\tm{x}\not\in\tm{M}$.
\end{proof}

%%% Local Variables:
%%% TeX-master: "../priorities"
%%% End:


% \restatelemma{lempgvsubjectreductionterms}
% \begin{proof}
  \label{prf:lem-pgv-subject-reduction-terms}
  By induction on the derivation of $\tm{M}\tred\tm{M'}$.

  \begin{case*}[\LabTirName{E-Lam}]
    By \cref{lem:pgv-substitution}.
    \begin{mathpar}
      \inferrule*{
        \inferrule*{
          \tseq[\cs{p}]{\ty{\Gamma},\tmty{x}{T}}{M}{U}
        }{\tseq[\cs{\pbot}]{\ty{\Gamma}}{\lambda x.M}{\tylolli[\cs{\pr(\ty{\Gamma})},\cs{p}]{T}{U}}}
        \\
        \tseq[\cs{\pbot}]{\ty{\Delta}}{V}{T}
      }{\tseq[\cs{p}]{\ty{\Gamma},\ty{\Delta}}{(\lambda x.M)\;V}{U}}
      \tred
      \tseq[\cs{p}]{\ty{\Gamma},\ty{\Delta}}{\subst{M}{V}{x}}{U}
    \end{mathpar}
  \end{case*}
  \begin{case*}[\LabTirName{E-Unit}]
    By \cref{lem:pgv-substitution}.
    \begin{mathpar}
      \inferrule*{
        \inferrule*{
        }{\tseq[\cs{\pbot}]{\emptyenv}{\unit}{\tyunit}}
        \\
        \tseq[\cs{p}]{\ty{\Gamma}}{M}{T}
      }{\tseq[\cs{p}]{\ty{\Gamma}}{\letunit{\unit}{M}}{T}}
      \tred
      \tseq[\cs{p}]{\ty{\Gamma}}{M}{T}
    \end{mathpar}
  \end{case*}
  \begin{case*}[\LabTirName{E-Pair}]
    By \cref{lem:pgv-substitution}.
    \begin{mathpar}
      \inferrule*{
        \inferrule*{
          \tseq[\cs{\pbot}]{\ty{\Gamma}}{V}{T}
          \\
          \tseq[\cs{\pbot}]{\ty{\Delta}}{W}{T'}
        }{\tseq[\cs{\pbot}]{\ty{\Gamma},\ty{\Delta}}{\pair{V}{W}}{\typrod{T}{T'}}}
        \\
        \tseq[\cs{p}]{\ty{\Theta},\tmty{x}{T},\tmty{y}{T'}}{M}{U}
      }{\tseq[]{\ty{\Gamma},\ty{\Delta},\ty{\Theta}}{\letpair{x}{y}{\pair{V}{W}}{M}}{U}}
      \\
      \begin{turn}{270}
        \tred
      \end{turn}
      \\
      \tseq[\cs{p}]{\ty{\Gamma},\ty{\Delta},\ty{\Theta}}{\subst{\subst{M}{V}{x}}{W}{y}}{U}
    \end{mathpar}
  \end{case*}
  \begin{case*}[\LabTirName{E-Inl}]
    By \cref{lem:pgv-substitution}.
    \begin{mathpar}
      \inferrule*{
        \inferrule*{
          \tseq[\cs{\pbot}]{\ty{\Gamma}}{V}{T}
        }{\tseq[\cs{\pbot}]{\ty{\Gamma}}{\inl{V}}{\tysum{T}{T'}}}
        \\
        \tseq[\cs{p}]{\ty{\Delta},\tmty{x}{T}}{M}{U}
        \\
        \tseq[\cs{p}]{\ty{\Delta},\tmty{y}{T'}}{N}{U}
      }{\tseq[\cs{p}]{\ty{\Gamma},\ty{\Delta}}{\casesum{\inl{V}}{x}{M}{y}{N}}{U}}
      \\
      \begin{turn}{270}
        \tred
      \end{turn}
      \\
      \tseq[\cs{p}]{\ty{\Gamma},\ty{\Delta}}{\subst{M}{V}{x}}{U}
    \end{mathpar}
  \end{case*}
  \begin{case*}[\LabTirName{E-Inr}]
    By \cref{lem:pgv-substitution}.
    \begin{mathpar}
      \inferrule*{
        \inferrule*{
          \tseq[\cs{\pbot}]{\ty{\Gamma}}{V}{T'}
        }{\tseq[\cs{\pbot}]{\ty{\Gamma}}{\inr{V}}{\tysum{T}{T'}}}
        \\
        \tseq[\cs{p}]{\ty{\Delta},\tmty{x}{T}}{M}{U}
        \\
        \tseq[\cs{p}]{\ty{\Delta},\tmty{y}{T'}}{N}{U}
      }{\tseq[\cs{p}]{\ty{\Gamma},\ty{\Delta}}{\casesum{\inr{V}}{x}{M}{y}{N}}{U}}
      \\
      \begin{turn}{270}
        \tred
      \end{turn}
      \\
      \tseq[\cs{p}]{\ty{\Gamma},\ty{\Delta}}{\subst{N}{V}{y}}{U}
    \end{mathpar}
  \end{case*}
  \begin{case*}[\LabTirName{E-Lift}]
    By induction on the evaluation context $\tm{E}$.
  \end{case*}
\end{proof}

%%% Local Variables:
%%% TeX-master: "../priorities"
%%% End:


% \restatelemma{lempgvsubjectcongruence}
% \begin{proof}
  \label{prf:lem-pgv-subject-congruence}
  By induction on the derivation of $\tm{\conf{C}}\equiv\tm{\conf{C'}}$.

  \begin{case*}[\LabTirName{SC-LnkSwp}]
    \begin{mathpar}
      \inferrule*{
        \inferrule*[vdots=1.5em]{
          \inferrule*{
          }{\tmty{\link}{\tylolli{\typrod{S}{\co{S}}}{\tyunit}}}
          \\
          \inferrule*{
            \inferrule*{
            }{\tseq[\cs{\pbot}]{\tmty{x}{S}}{x}{S}}
            \\
            \inferrule*{
            }{\tseq[\cs{\pbot}]{\tmty{y}{\co{S}}}{y}{\co{S}}}
          }{\tseq[\cs{\pbot}]{\tmty{x}{S},\tmty{y}{\co{S}}}{\pair{x}{y}}{\typrod{S}{\co{S}}}}
        }{\tseq[\cs{\pbot}]{\tmty{x}{S},\tmty{y}{\co{S}}}{\link\;{\pair{x}{y}}}{\tyunit}}
      }{\cseq[\phi]{\ty{\Gamma},\tmty{x}{S},\tmty{y}{\co{S}}}{\plug{\conf{F}}{\link\;{\pair{x}{y}}}}}
      \\
      \begin{turn}{270}
        \ensuremath{\equiv}
      \end{turn}
      \\
      \inferrule*{
        \inferrule*[vdots=1.5em]{
          \inferrule*{
          }{\tmty{\link}{\tylolli{\typrod{\co{S}}{S}}{\tyunit}}}
          \\
          \inferrule*{
            \inferrule*{
            }{\tseq[\cs{\pbot}]{\tmty{y}{\co{S}}}{y}{\co{S}}}
            \\
            \inferrule*{
            }{\tseq[\cs{\pbot}]{\tmty{x}{S}}{x}{S}}
          }{\tseq[\cs{\pbot}]{\tmty{x}{S},\tmty{y}{\co{S}}}{\pair{y}{x}}{\typrod{S}{\co{S}}}}
        }{\tseq[\cs{\pbot}]{\tmty{x}{S},\tmty{y}{\co{S}}}{\link\;{\pair{y}{x}}}{\tyunit}}
      }{\cseq[\phi]{\ty{\Gamma},\tmty{x}{S},\tmty{y}{\co{S}}}{\plug{\conf{F}}{\link\;{\pair{y}{x}}}}}
    \end{mathpar}
  \end{case*}

  \begin{case*}[\LabTirName{SC-ResExt}]
    \begin{mathpar}
      \inferrule*{
        \inferrule*{
          \cseq{\ty{\Gamma}}{\conf{C}}
          \\
          \cseq{\ty{\Delta},\tmty{x}{S},\tmty{y}{\co{S}}}{\conf{D}}
        }{\cseq{\ty{\Gamma},\ty{\Delta},\tmty{x}{S},\tmty{y}{\co{S}}}{(\ppar{\conf{C}}{\conf{D}})}}
      }{\cseq{\ty{\Gamma},\ty{\Delta}}{\res{x}{y}{(\ppar{\conf{C}}{\conf{D}}})}}
      \equiv
      \inferrule*{
        \cseq{\ty{\Gamma}}{\conf{C}}
        \\
        \inferrule*{
          \cseq{\ty{\Delta},\tmty{x}{S},\tmty{y}{\co{S}}}{\conf{D}}
        }{\cseq{\ty{\Delta}}{\res{x}{y}{\conf{D}}}}
      }{\cseq{\ty{\Gamma},\ty{\Delta}}{\ppar{\conf{C}}{\res{x}{y}{\conf{D}}}}}
    \end{mathpar}
  \end{case*}

  \begin{case*}[\LabTirName{SC-ResSwp}]
    \todo{Write proof.}
  \end{case*}

  \begin{case*}[\LabTirName{SC-ResCom}]
    \begin{mathpar}
      \inferrule*{
        \inferrule*{
          \cseq[\phi]{\ty{\Gamma},\tmty{x}{S},\tmty{y}{\co{S}},\tmty{z}{S'},\tmty{w}{\co{S'}}}{\conf{C}}
        }{\cseq[\phi]{\ty{\Gamma},\tmty{x}{S},\tmty{y}{\co{S}}}{\res{z}{w}{\conf{C}}}}
      }{\cseq[\phi]{\ty{\Gamma}}{\res{x}{y}{\res{z}{w}{\conf{C}}}}}
      \equiv
      \inferrule*{
        \inferrule*{
          \cseq[\phi]{\ty{\Gamma},\tmty{x}{S},\tmty{y}{\co{S}},\tmty{z}{S'},\tmty{w}{\co{S'}}}{\conf{C}}
        }{\cseq[\phi]{\ty{\Gamma},\tmty{z}{S'},\tmty{w}{\co{S'}}}{\res{x}{y}{\conf{C}}}}
      }{\cseq[\phi]{\ty{\Gamma}}{\cseq{}{\res{z}{w}{\res{x}{y}{\conf{C}}}}}}
    \end{mathpar}
  \end{case*}

  \begin{case*}[\LabTirName{SC-ParNil}]
    \begin{mathpar}
      \inferrule*{
        \cseq[\phi]{\ty{\Gamma}}{\conf{C}}
        \\
        \inferrule*{
          \inferrule*{
          }{\tseq[\cs{\pbot}]{\emptyenv}{\unit}{\tyunit}}
        }{\cseq[\child]{\emptyenv}{\child{\unit}}}
      }{\cseq[\phi]{\ty{\Gamma}}{\ppar{\conf{C}}{\child{\unit}}}}
      \equiv
      \cseq[\phi]{\ty{\Gamma}}{\conf{C}}
    \end{mathpar}
  \end{case*}

  \begin{case*}[\LabTirName{SC-ParCom}]
    \begin{mathpar}
      \inferrule*{
        \cseq[\phi]{\ty{\Gamma}}{\conf{C}}
        \\
        \cseq[\phi']{\ty{\Delta}}{\conf{D}}
      }{\cseq[\phi+\phi']{\ty{\Gamma},\ty{\Delta}}{(\ppar{\conf{C}}{\conf{D}})}}
      \equiv
      \inferrule*{
        \cseq[\phi']{\ty{\Delta}}{\conf{D}}
        \\
        \cseq[\phi]{\ty{\Gamma}}{\conf{C}}
      }{\cseq[\phi'+\phi]{\ty{\Gamma},\ty{\Delta}}{(\ppar{\conf{D}}{\conf{C}})}}
    \end{mathpar}
  \end{case*}


  \begin{case*}[\LabTirName{SC-ParAsc}]
    \begin{mathpar}
      \inferrule*{
        \cseq[\phi]{\ty{\Gamma}}{\conf{C}}
        \\
        \inferrule*{
          \cseq[\phi']{\ty{\Delta}}{\conf{D}}
          \\
          \cseq[\phi'']{\ty{\Theta}}{\conf{E}}
        }{\cseq[\phi'+\phi'']{\ty{\Delta},\ty{\Theta}}{(\ppar{\conf{D}}{\conf{E}})}}
      }{\cseq[\phi+\phi'+\phi'']{\ty{\Gamma},\ty{\Delta},\ty{\Theta}}{\ppar{\conf{C}}{(\ppar{\conf{D}}{\conf{E}})}}}
      \equiv
      \inferrule*{
        \inferrule*{
          \cseq[\phi]{\ty{\Gamma}}{\conf{C}}
          \\
          \cseq[\phi']{\ty{\Delta}}{\conf{D}}
        }{\cseq[\phi+\phi']{\ty{\Gamma},\ty{\Delta}}{(\ppar{\conf{C}}{\conf{D}})}}
        \\
        \cseq[\phi'']{\ty{\Theta}}{\conf{E}}
      }{\cseq[\phi+\phi'+\phi'']{\ty{\Gamma},\ty{\Delta},\ty{\Theta}}{\ppar{(\ppar{\conf{C}}{\conf{D}})}{\conf{E}}}}
    \end{mathpar}
  \end{case*}
\end{proof}

%%% Local Variables:
%%% TeX-master: "../priorities"
%%% End:


% \restatetheorem{thmpgvsubjectreductionconfs}
% \begin{proof}
  \begin{case}[\LabTirName{E-New}]
    \small
    \begin{mathpar}
      \inferrule*{
        \inferrule*[vdots=1.5em]{
          \inferrule*{
          }{\tmty{\new}{\tylolli{\tyunit}{\typrod{S}{\co{S}}}}}
          \\
          \inferrule*{
          }{\tseq[\cs{\pbot}]{\emptyenv}{\unit}{\tyunit}}
        }{\tseq[\cs{\pbot}]{\emptyenv}{\new\;\unit}{\typrod{S}{\co{S}}}}
      }{\cseq[\phi]{\ty{\Gamma}}{\plug{\conf{F}}{\new\;\unit}}}
      \cred
      \inferrule*{
        \inferrule*{
          \inferrule*[vdots=1.5em]{
            \inferrule*{
            }{\tseq[\cs{\pbot}]{\tmty{x}{S}}{x}{S}}
            \\
            \inferrule*{
            }{\tseq[\cs{\pbot}]{\tmty{y}{\co{S}}}{y}{\co{S}}}
          }{\tseq[\cs{\pbot}]{\tmty{x}{S},\tmty{y}{\co{S}}}{\pair{x}{y}}{\typrod{S}{\co{S}}}}
        }{\cseq[\phi]{\ty{\Gamma},\tmty{x}{S},\tmty{y}{\co{S}}}{\plug{\conf{F}}{\pair{x}{y}}}}
      }{\cseq[\phi]{\ty{\Gamma}}{\res{x}{y}{\plug{\conf{F}}{\pair{x}{y}}}}}
  \end{mathpar}
  \end{case}
  \begin{case}[\LabTirName{E-Spawn}]
    \small
    \begin{mathpar}
      \inferrule*{
        \inferrule*[vdots=1.5em]{
          \inferrule*{
          }{\tmty{\spawn}{\tylolli{(\tylolli[\cs{p},\cs{q}]{\tyunit}{\tyunit})}{\tyunit}}}
          \tseq[\cs{\pbot}]{\ty{\Delta}}{V}{\tylolli[\cs{p},\cs{q}]{\tyunit}{\tyunit}}
        }{\tseq[\cs{\pbot}]{\ty{\Delta}}{\spawn\;V}{\tyunit}}
      }{\cseq[\phi]{\ty{\Gamma},\ty{\Delta}}{\plug{\conf{F}}{\spawn\;V}}}
      \\
      \begin{turn}{270}
        \cred
      \end{turn}
      \\
      \inferrule*{
        \inferrule*{
          \inferrule*[vdots=1.5em]{
          }{\tseq[\cs{\pbot}]{\emptyenv}{\unit}{\tyunit}}
        }{\cseq[\phi]{\ty{\Gamma}}{\plug{\conf{F}}{\unit}}}
        \\
        \inferrule*{
          \inferrule*{
            \tseq[\cs{\pbot}]{\ty{\Delta}}{V}{\tylolli[\cs{p},\cs{q}]{\tyunit}{\tyunit}}
            \\
            \inferrule*{
            }{\tseq[\cs{\pbot}]{\emptyenv}{\unit}{\tyunit}}
          }{\tseq[\cs{q}]{\ty{\Delta}}{V\;\unit}{\tyunit}}
        }{\cseq[\child]{\ty{\Delta}}{\child\;(V\;\unit)}}
      }{\cseq[\phi]{\ty{\Gamma},\ty{\Delta}}{\ppar{\plug{\conf{F}}{\unit}}{\child\;(V\;\unit)}}}
    \end{mathpar}
  \end{case}
  \begin{case}[\LabTirName{E-Send}]
    See \cref{fig:pgv-subject-reduction-cred-send}.
  \end{case}
  \begin{case}[\LabTirName{E-Close}]
    See \cref{fig:pgv-subject-reduction-cred-close}.
  \end{case}
  \begin{case}[\LabTirName{E-LiftC}]
    \todo{By induction on the evaluation context $\conf{G}$.}
  \end{case}
  \begin{case}[\LabTirName{E-LiftM}]
    By \cref{thm:pgv-subject-reduction-terms}.
  \end{case}
  \begin{case}[\LabTirName{E-LiftE}]
    By \cref{thm:pgv-subject-congruence}.
  \end{case}
\end{proof}
\begin{landscape}
\begin{figure}[ht!]
     \small
    \begin{mathpar}
      \inferrule*{
        \inferrule*{
          \inferrule*{
            \inferrule*[vdots=1.5em]{
              \inferrule*{
              }{\tmty{\send}{\tylolli[\cs{\ptop},\cs{o}]{\typrod{T}{\tysend[\cs{o}]{T}{S}}}{S}}}
              \inferrule*{
                \tseq[\cs{p}]{\ty{\Delta}}{V}{T}
                \\
                \inferrule*{
                }{\tseq[\cs{\pbot}]{\tmty{x}{\tysend[\cs{o}]{T}{S}}}{x}{\tysend[\cs{o}]{T}{S}}}
              }{\tseq[\cs{p}]{\ty{\Delta},\tmty{x}{\tysend[\cs{o}]{T}{S}}}
                {\pair{V}{x}}{\typrod{T}{\tysend[\cs{o}]{T}{S}}}}
            }{\tseq[\cs{p}\sqcup\cs{o}]{\ty{\Delta},\tmty{x}{\tysend[\cs{o}]{T}{S}}}{\send\;{\pair{V}{x}}}{S}}
          }{\cseq[\phi]
            {\ty{\Gamma},\ty{\Delta},\tmty{x}{\tysend[\cs{o}]{T}{S}}}
            {\plug{\conf{F}}{\send\;{\pair{V}{x}}}}}
          \\
          \inferrule*{
            \inferrule*[vdots=1.5em]{
              \inferrule*{
              }{\tmty{\recv}{\tylolli[\cs{\ptop},\cs{o}]{\tyrecv[\cs{o}]{T}{\co{S}}}{\typrod{T}{\co{S}}}}}
              \\
              \inferrule*{
              }{\tseq[\cs{\pbot}]{\tmty{y}{\tyrecv[\cs{o}]{T}{\co{S}}}}{y}{\tyrecv[\cs{o}]{T}{\co{S}}}}
            }{\tseq[\cs{o}]
              {\tmty{y}{\tyrecv[\cs{o}]{T}{\co{S}}}}
              {\recv\;y}
              {\typrod{T}{\co{S}}}}
          }{\cseq[\phi']
            {\ty{\Theta},\tmty{y}{\tyrecv[\cs{o}]{T}{\co{S}}}}
            {\plug{\conf{F'}}{\recv\;{y}}}}
        }{\cseq[\phi+\phi']
          {\ty{\Gamma},\ty{\Delta},\ty{\Theta},
            \tmty{x}{\tysend[\cs{o}]{T}{S}},\tmty{y}{\tyrecv[\cs{o}]{T}{\co{S}}}}
          {\ppar{\plug{\conf{F}}{\send\;{\pair{V}{x}}}}{\plug{\conf{F'}}{\recv\;{y}}}}}
      }{\cseq[\phi+\phi']
        {\ty{\Gamma},\ty{\Delta},\ty{\Theta}}
        {\res{x}{y}{(\ppar
            {\plug{\conf{F}}{\send\;{\pair{V}{x}}}}
            {\plug{\conf{F'}}{\recv\;{y}}})}}}
      \\
      \begin{turn}{270}
        \cred
      \end{turn}
      \\
      \inferrule*{
        \inferrule*{
          \inferrule*{
            \inferrule*[vdots=1.5em]{
            }{\tseq[\cs{\pbot}]{\tmty{x}{S}}{x}{S}}
          }{\cseq[\phi]{\ty{\Gamma},\tmty{x}{S}}{\plug{\conf{F}}{x}}}
          \\
          \inferrule*{
            \inferrule*[vdots=1.5em]{
              \tseq[\cs{p}]{\ty{\Delta}}{V}{T}
              \\
              \inferrule*{
              }{\tseq[\cs{\pbot}]{\ty{\Delta},\tmty{y}{\co{S}}}{y}{\co{S}}}
            }{\tseq[\cs{p}]{\ty{\Delta},\tmty{y}{\co{S}}}{\pair{V}{y}}{\typrod{T}{\co{S}}}}
          }{\cseq[\phi']
            {\ty{\Delta},\ty{\Theta},\tmty{y}{\co{S}}}
            {\plug{\conf{F'}}{\pair{V}{y}}}}
        }{\cseq[\phi+\phi']
          {\ty{\Gamma},\ty{\Delta},\ty{\Theta},
            \tmty{x}{S},\tmty{y}{\co{S}}}
          {\ppar
            {\plug{\conf{F}}{x}}
            {\plug{\conf{F'}}{\pair{V}{y}}}}}
      }{\cseq[\phi+\phi']
        {\ty{\Gamma},\ty{\Delta},\ty{\Theta}}
        {\res{x}{y}{(\ppar
            {\plug{\conf{F}}{x}}
            {\plug{\conf{F'}}{\pair{V}{y}}})}}}
    \end{mathpar}
  \caption{Subject Reduction (\LabTirName{E-Send})}
  \label{fig:pgv-subject-reduction-cred-send}
\end{figure}
\begin{figure}[ht!]
     \small
    \begin{mathpar}
      \inferrule*{
        \inferrule*{
          \inferrule*{
            \inferrule*[vdots=1.5em]{
              \inferrule*{
              }{\tmty{\close}{\tylolli[\cs{\ptop},\cs{o}]{\tyends[\cs{o}]}{\tyunit}}}
              \\
              \inferrule*{
              }{\tseq[\cs{\pbot}]{\tmty{x}{\tyends[\cs{o}]}}{x}{\tyends[\cs{o}]}}
            }{\tseq[\cs{o}]{\tmty{x}{\tyends[\cs{o}]}}{\close\;{x}}{\tyunit}}
          }{\cseq[\phi]
            {\ty{\Gamma},\tmty{x}{\tyends[\cs{o}]}}
            {\plug{\conf{F}}{\close\;{x}}}}
          \\
          \inferrule*{
            \inferrule*[vdots=1.5em]{
              \inferrule*{
              }{\tmty{\wait}{\tylolli[\cs{\ptop},\cs{o}]{\tyendr[\cs{o}]}{\tyunit}}}
              \\
              \inferrule*{
              }{\tseq[\cs{\pbot}]{\tmty{y}{\tyendr[\cs{o}]}}{y}{\tyendr[\cs{o}]}}
            }{\tseq[\cs{o}]{\tmty{y}{\tyendr[\cs{o}]}}{\wait\;{y}}{\tyunit}}
          }{\cseq[\phi']
            {\ty{\Delta},\tmty{y}{\tyendr[\cs{o}]}}
            {\plug{\conf{F'}}{\wait\;{y}}}}
        }{\cseq[\phi+\phi']
          {\ty{\Gamma},\ty{\Delta},
            \tmty{x}{\tyends[\cs{o}]},\tmty{y}{\tyendr[\cs{o}]}}
          {\ppar{\plug{\conf{F}}{\close\;{x}}}{\plug{\conf{F'}}{\wait\;{y}}}}}
      }{\cseq[\phi+\phi']
        {\ty{\Gamma},\ty{\Delta}}
        {\res{x}{y}{(\ppar
            {\plug{\conf{F}}{\close\;{x}}}
            {\plug{\conf{F'}}{\wait\;{y}}})}}}
      \\
      \begin{turn}{270}
        \cred
      \end{turn}
      \\
      \inferrule*{
        \inferrule*{
          \inferrule*[vdots=1.5em]{
          }{\tseq[\cs{\pbot}]{\emptyenv}{\unit}{\tyunit}}
        }{\cseq[\phi]{\ty{\Gamma}}{\plug{\conf{F}}{\unit}}}
        \\
        \inferrule*{
          \inferrule*[vdots=1.5em]{
          }{\tseq[\cs{\pbot}]{\emptyenv}{\unit}{\tyunit}}
        }{\cseq[\phi']
          {\ty{\Delta}}
          {\plug{\conf{F'}}{\unit}}}
      }{\cseq[\phi+\phi']
        {\ty{\Gamma},\ty{\Delta}}
        {\ppar{\plug{\conf{F}}{\unit}}{\plug{\conf{F'}}{\unit}}}}
    \end{mathpar}
  \caption{Subject Reduction (\LabTirName{E-Close})}
  \label{fig:pgv-subject-reduction-cred-close}
\end{figure}
\end{landscape}

%%% Local Variables:
%%% TeX-master: "../priorites"
%%% End:


% \begin{restatablelemma}{lempgvreadypriority}
%   \label{lem:pgv-ready-priority}
%   If $\tseq[\cs{p}]{\ty{\Gamma}}{L}{T}$ is ready to act on $\tmty{x}{S}\in\ty{\Gamma}$, then the priority bound $\cs{p}$ is some priority $\cs{o}$, \ie not $\cs{\pbot}$ or $\cs{\ptop}$.
% \end{restatablelemma}
% \begin{proof}
%   Let $\tm{L}=\tm{\plug{E}{M}}$. By induction on the structure of $\tm{E}$. $\tm{M}$ has priority $\pr({\ty{S}})$, and each constructor of the evaluation context $\tm{E}$ passes on the \emph{maximum} of the priorities of its premises. No rule introduces the priority bound $\cs{\ptop}$ on the sequent.
% \end{proof}

% \restatelemma{lempgvcanonicalforms}
% \begin{proof}
%   We move any $\nu$-binders to the top using \LabTirName{SC-ResExt}, discard any superfluous occurrences of $\tm{\child\;\unit}$ using \LabTirName{SC-ParNil}, and move the main thread to the rightmost position using \LabTirName{SC-ParComm} and \LabTirName{SC-ParAssoc}.
% \end{proof}

% \restatetheorem{thmpgvclosedprogressconfs}
% \begin{proof}
  \todo{Shouldn't the statement be such that the def of C comes right after stating C is in canonical form, rather than in the or branch?}

  \label{prf:thm-pgv-closed-progress-confs}
  Let $\tm{\conf{C}}=\tm{\res{x_1}{x'_1}{\dots\res{x_n}{x'_n}{(\child\;M_1\parallel\dots\parallel\child\;M_m\parallel\main\;N)}}}$.
  We apply \cref{lem:pgv-open-progress-terms} to each $\tm{M_i}$ and $\tm{N}$. If for any $\tm{M_i}$ or $\tm{N}$ we obtain a reduction $\tm{M_i}\tred\tm{M'_i}$ or $\tm{N}\tred\tm{N'}$, we apply \LabTirName{E-LiftM} and \LabTirName{E-LiftC} to obtain a reduction on $\tm{\conf{C}}$.
  Otherwise, each term $\tm{M_i}$ is ready, and $\tm{N}$ is either ready or a value.

  Pick the \emph{ready} term $\tm{L}\in\{\tm{M_1},\dots,\tm{M_m},\tm{N}\}$ with the smallest priority bound. There are four cases: \todo{the below is presented as if there are only 3 cases, not 4...}
  \begin{itemize}
  \item
    If $\tm{L}$ is a new $\tm{\plug{E}{\new}}$, we apply \LabTirName{E-New}.
  \item
    If $\tm{L}$ is a spawn $\tm{\plug{E}{\spawn\;M}}$, we apply \LabTirName{E-Spawn}.
  \item
    If $\tm{L}$ is a link $\tm{\plug{E}{\link\;\pair{y}{z}}}$ or $\tm{\plug{E}{\link\;\pair{z}{y}}}$, we apply \LabTirName{E-Link}.
  \end{itemize}
  Otherwise, $\tm{L}$ is ready to act on some endpoint $\tmty{y}{S}$. Let $\tmty{y'}{\co{S}}$ be the dual endpoint of $\tm{y}$. There must be a term $\tm{L'}\in\{\tm{M_1},\dots,\tm{M_m},\tm{N}\}$ which uses $\tm{y'}$.
  \todo{in the same lines as DF for PCP, justify here why it is true that "there must be a term..."}
  There are two cases:
  \begin{itemize}
  \item
    $\tm{L'}$ is ready. By~\cref{lem:pgv-ready-priority}, the priority of $\tm{L}$ is $\pr(\ty{S})$. By duality, $\pr(\ty{\co{S}})=\pr(\ty{S})$.

    We cannot have $\tm{L}=\tm{L'}$, otherwise the action on $\tm{y'}$ would be guarded by the action on $\tm{y}$, requiring $\pr(\ty{\co{S}})<\pr(\ty{S})$. The term $\tm{L'}$ must be ready to act on $\tm{y'}$, otherwise the action $\tm{y'}$ would be guarded by another action with priority smaller than $\pr{(\ty{S})}$, which contradicts our choice of $\tm{L}$ as having the smallest priority. Therefore, we have two terms ready to act on dual endpoints. We apply the appropriate reduction rule, \ie \LabTirName{E-Send} or \LabTirName{E-Close}.
  \item
    $\tm{L'}=\tm{N}$ and is a value. We rewrite $\tm{\conf{C}}$ to put $\tm{L}$ in the position corresponding to the endpoint it is blocked on, using \LabTirName{SC-ParComm}, \LabTirName{SC-ParAssoc}, and optionally \LabTirName{SC-ResSwap}. We then repeat the steps above with the term with the next smallest priority, until either we find a reduction, or the configuration has reached the desired normal form. (The argument based on the priority being the smallest continues to hold, since we know that neither $\tm{L}$ nor $\tm{L'}$ will be picked, and no other term uses $\tm{y}$ or $\tm{y'}$.)
  \end{itemize} 
\end{proof}

%%% Local Variables:
%%% TeX-master: "../priorities"
%%% End:


% \begin{figure}[t]
  \paragraph*{Typing Rules for Syntactic Sugar}
  \begin{mathpar}
    \inferrule*[lab=T-Seq]{
      \tseq[\cs{p}]{\ty{\Gamma}}{M}{\tyunit}
      \\
      \tseq[\cs{q}]{\ty{\Delta}}{N}{T}
      \\
      \cs{p}<\pr(\ty{\Delta})
    }{\tseq[\cs{p}\sqcup\cs{q}]{\ty{\Gamma},\ty{\Delta}}{\andthen{M}{N}}{T}}
    
    \inferrule*[lab=T-Let]{
      \tseq[\cs{p}]{\ty{\Gamma}}{M}{T}
      \\
      \tseq[\cs{q}]{\ty{\Delta},\tmty{x}{T}}{N}{U}
      \\
      \cs{p}<\pr(\ty{\Delta})
    }{\tseq[\cs{p}\sqcup\cs{q}]{\ty{\Gamma},\ty{\Delta}}{\letbind{x}{M}{N}}{U}}
    \\
    \inferrule*[lab=T-LamUnit]{
      {\tseq[\cs{o}]{\ty{\Gamma}}{M}{T}}
    }{\tseq[\cs{\pbot}]
      {\ty{\Gamma}}
      {\lambda\unit.M}
      {\tylolli[\cs{\pr(\ty{\Gamma})},\cs{o}]{\tyunit}{T}}}
    
    \inferrule*[lab=T-LamPair]
    {\tseq[\cs{o}]
      {\ty{\Gamma},\tmty{x}{T},\tmty{y}{T'}}
      {M}
      {U}}
    {\tseq[\cs{\pbot}]
      {\ty{\Gamma}}
      {\lambda\pair{x}{y}.M}
      {\tylolli[\cs{\pr(\ty{\Gamma})},\cs{o}]{\typrod{T}{T'}}{U}}}
    \\
    \inferrule*[lab=T-Fork]{
    }{\tseq[\cs{\pbot}]
      {\emptyenv}
      {\fork}
      {\tylolli[]{(\tylolli[\cs{p},\cs{q}]{S}{\tyunit})}{\co{S}}}}
    \\
    \inferrule*[lab=T-Select-Inl]{
      \pr(\ty{S})=\pr(\ty{S'})
    }{\tseq[\cs{\pbot}]{\emptyenv}{\select{\labinl}}{\tylolli[\cs{\ptop},\cs{o}]{\tyselect[\cs{o}]{S}{S'}}{S}}}
    
    \inferrule*[lab=T-Select-Inr]{
      \pr(\ty{S})=\pr(\ty{S'})
    }{\tseq[\cs{\pbot}]{\emptyenv}{\select{\labinr}}{\tylolli[\cs{\ptop},\cs{o}]{\tyselect[\cs{o}]{S}{S'}}{S'}}}
    
    \inferrule*[lab=T-Offer]{
      {\tseq[\cs{p}]
        {\ty{\Gamma}}
        {L}
        {\tyoffer[\cs{o}]{S}{S'}}}
      \\
      {\tseq[\cs{q}]
        {\ty{\Delta},\tmty{x}{S}}
        {M}
        {T}}
      \\
      {\tseq[\cs{q}]
        {\ty{\Delta},\tmty{y}{S'}}
        {N}
        {T}}
      \\
      \cs{o}\sqcup\cs{p}<\pr(\ty{\Delta},\ty{S},\ty{S'})
    }{\tseq[\cs{o}\sqcup\cs{p}\sqcup\cs{q}]
      {\ty{\Gamma},\ty{\Delta}}
      {\offer{L}{x}{M}{y}{N}}
      {T}}
    
    \inferrule*[lab=T-Offer-Absurd]{
      \tseq[\cs{p}]
      {\ty{\Gamma}}
      {L}
      {\tyofferemp[\cs{o}]}
      \\
      \cs{o}\sqcup\cs{p}<\pr(\ty{\Delta})
    }{\tseq[\cs{o}\sqcup\cs{p}]
      {\ty{\Gamma},\ty{\Delta}}
      {\offeremp{L}}
      {T}}
  \end{mathpar}
  \caption{Typing Rules for Syntactic Sugar for PGV.}
  \label{fig:pgv-typing-sugar}
\end{figure}
%%% Local Variables:
%%% TeX-master: "../priorities"
%%% End:

% \endgroup

\section{Relation to Priority CP}
\label{sec:pcp}

Thus far we have presented Priority GV (PGV) together with the relevant technical results. We remind the reader that this line of work of adding priorities, started with Priority CP (PCP) \cite{dardhagay18} where priorities are integrated in Wadler's Classical Processes (CP), which is a $\pi$-calculus leveraging the correspondence of session types as linear logic propositions \cite{wadler12}.
In his work, Wadler presents a connection (via encoding) of CP and GV. Following that work, we sat out to understand the connection between the priority versions of CP and GV, thus comparing PGV and PCP.
Before presenting our formal results, we will revisit PCP in the following section.

\begingroup
\usingnamespace{pcp}
\subsection{Revisiting Priority CP}
\label{app:revisiting-PCP}

\subsubsection*{Types}
Types ($\pcp{\ty{A}}, \pcp{\ty{B}}$) in PCP are based on classical linear logic propositions, and are defined by the following grammar:
\[
  \usingnamespace{pcp}
  \begin{array}{lcl}
    \ty{A}, \ty{B}
    & \Coloneqq & \ty{\tytens[\cs{o}]{A}{B}}
      \sep        \ty{\typarr[\cs{o}]{A}{B}}
      \sep        \ty{\tyone[\cs{o}]}
      \sep        \ty{\tybot[\cs{o}]}
      \sep        \ty{\typlus[\cs{o}]{A}{B}}
      \sep        \ty{\tywith[\cs{o}]{A}{B}}
      \sep        \ty{\tynil[\cs{o}]}
      \sep        \ty{\tytop[\cs{o}]}
  \end{array}
\]

Each connective is annotated with a priority $\cs{o}\in\mathbb{N}$.

Types $\pcp{\ty{\tytens[\cs{o}]{A}{B}}}$ and $\pcp{\ty{\typarr[\cs{o}]{A}{B}}}$ type the endpoints of a channel over which we send or receive a channel of type $\pcp{\ty{A}}$, and then proceed as type $\pcp{\ty{B}}$. Types $\pcp{\ty{\tyone[\cs{o}]}}$ and $\pcp{\ty{{\tybot}[\cs{o}]}}$ type the endpoints of a channel whose session has terminated, and over which we send or receive a \emph{ping} before closing the channel. These two types act as units for $\pcp{\ty{\tytens[\cs{o}]{A}{B}}}$ and $\pcp{\ty{\typarr[\cs{o}]{A}{B}}}$, respectively.

Types $\pcp{\ty{\typlus[\cs{o}]{A}{B}}}$ and $\pcp{\ty{\tywith[\cs{o}]{A}{B}}}$ type the endpoints of a channel over which we can receive or send a choice between two branches $\pcp{\ty{A}}$ or $\pcp{\ty{B}}$. We have opted for a simplified version of choice and followed the original Wadler's CP \cite{wadler14}, however types $\ty{\oplus}$ and $\ty{\with}$ can be trivially generalised to $\pcp{\ty{\oplus^{\cs{o}}\{l_i:A_i\}_{i\in I}}}$ and $\pcp{\ty{\with^{\cs{o}}\{l_i:A_i\}_{i\in I}}}$, respectively, as in the original PCP \cite{dardhagay18extended}.

Types $\pcp{\ty{\tynil[\cs{o}]}}$ and $\pcp{\ty{\tytop[\cs{o}]}}$ type the endpoints of a channel over which we can send or receive a choice between \emph{no options}. These two types act as units for $\pcp{\ty{\typlus[\cs{o}]{A}{B}}}$ and $\pcp{\ty{\tywith[\cs{o}]{A}{B}}}$, respectively.

\subsubsection*{Typing Environments}
\label{sec:pcp-environments}
Typing environments $\pcp{\ty{\Gamma}}$, $\pcp{\ty{\Delta}}$ associate names to types. Environments are linear, so two environments can only be combined as $\pcp{\ty{\Gamma}}, \pcp{\ty{\Delta}}$ if their names are distinct, \ie $\pcp{\fv(\ty{\Gamma})\cap\fv(\ty{\Delta})=\varnothing}$.
\[
  \usingnamespace{pcp}
  \begin{array}{lcl}
    \ty{\Gamma}, \ty{\Delta}
    & \Coloneqq & \ty{\emptyenv}
      \sep        \ty{\Gamma}, \tmty{x}{A}
  \end{array}
\]

\subsubsection*{Type Duality}
\label{sec:pcp-duality}
Duality is an involutive function on types which preserves priorities:
\[
  \usingnamespace{pcp}
  \setlength{\arraycolsep}{1pt}
  \begin{array}{lcl}
    \ty{\co{(\tyone[\cs{o}])}} & = & \ty{\tybot[\cs{o}]} \\
    \ty{\co{(\tybot[\cs{o}])}} & = & \ty{\tyone[\cs{o}]}
  \end{array}
  \quad
  \begin{array}{lcl}
    \ty{\co{(\tytens[\cs{o}]{A}{B})}} & = & \ty{\typarr[\cs{o}]{\co{A}}{\co{B}}} \\
    \ty{\co{(\typarr[\cs{o}]{A}{B})}} & = & \ty{\tytens[\cs{o}]{\co{A}}{\co{B}}}
  \end{array}
  \quad
  \begin{array}{lcl}
    \ty{\co{(\tynil[\cs{o}])}} & = & \ty{\tytop[\cs{o}]} \\
    \ty{\co{(\tytop[\cs{o}])}} & = & \ty{\tynil[\cs{o}]}
  \end{array}
  \quad
  \begin{array}{lcl}
    \ty{\co{(\typlus[\cs{o}]{A}{B})}} & = & \ty{\tywith[\cs{o}]{\co{A}}{\co{B}}} \\
    \ty{\co{(\tywith[\cs{o}]{A}{B})}} & = & \ty{\typlus[\cs{o}]{\co{A}}{\co{B}}}
  \end{array}
\]

\subsubsection*{Priorities}
\label{sec:pcp-priorities}
The function $\pr(\cdot)$ returns smallest priority of a type. As with PGV, the type system guarantees that the top-most connective always holds the smallest priority.  The function $\minpr(\cdot)$ returns the \emph{minimum} priority of all types a typing context, or $\cs{\ptop}$ if the context is empty:
\[
  \usingnamespace{pcp}
  \setlength{\arraycolsep}{1pt}
  \begin{array}{lclclcl}
    \pr(\ty{\tyone[\cs{o}]})        & = & \cs{o} \\
    \pr(\ty{\tybot[\cs{o}]})        & = & \cs{o}
  \end{array}
  \qquad
  \begin{array}{lclclcl}
    \pr(\ty{\tytens[\cs{o}]{A}{B}}) & = & \cs{o} \\
    \pr(\ty{\typarr[\cs{o}]{A}{B}}) & = & \cs{o}
  \end{array}
  \qquad
  \begin{array}{lclclcl}
    \pr(\ty{\tynil[\cs{o}]})        & = & \cs{o} \\
    \pr(\ty{\tytop[\cs{o}]})        & = & \cs{o}
  \end{array}
  \qquad
  \begin{array}{lclclcl}
    \pr(\ty{\typlus[\cs{o}]{A}{B}}) & = & \cs{o} \\
    \pr(\ty{\tywith[\cs{o}]{A}{B}}) & = & \cs{o}
  \end{array}
\]
\[
  \minpr(\ty{\emptyenv})          = \cs{\ptop}
  \quad
  \minpr(\ty{\Gamma},\tmty{x}{T}) = \minpr(\ty{\Gamma})\sqcap\minpr(\ty{T})
\]

\subsubsection*{Terms}
Processes ($\pcp{\tm{P}}$, $\pcp{\tm{Q}}$) in PCP are defined by the following grammar.
\[
  \usingnamespace{pcp}
  \begin{array}[t]{lcl}
    \tm{P}, \tm{Q}
    & \Coloneqq & \tm{\link{x}{y}}
           \sep   \tm{\res{x}{y}{P}}
           \sep   \tm{(\ppar{P}{Q})}
           \sep   \tm{\halt}
    \\   & \sep & \tm{\send{x}{y}{P}}
           \sep   \tm{\close{x}{P}}
           \sep   \tm{\recv{x}{y}{P}}
           \sep   \tm{\wait{x}{P}}
    \\   & \sep & \tm{\inl{x}{P}}
           \sep   \tm{\inr{x}{P}}
           \sep   \tm{\offer{x}{P}{Q}}
           \sep   \tm{\absurd{x}}
  \end{array}
\]

Process $\pcp{\tm{\link{x}{y}}}$ links endpoints $\pcp{\tm{x}}$ and $\pcp{\tm{y}}$ and forwards communication from one to the other. $\pcp{\tm{\res{x}{y}{P}}}$, $\pcp{\tm{(\ppar{P}{Q})}}$ and $\pcp{\tm{\halt}}$ denote respectively the restriction processes where channel endpoints $\pcp{\tm{x}}$ and $\pcp{\tm{y}}$ are bound together and with scope $\pcp{\tm{P}}$, the parallel composition of processes $\pcp{\tm{P}}$ and $\pcp{\tm{Q}}$ and the terminated process.

Processes $\pcp{\tm{\send{x}{y}{P}}}$ and $\pcp{\tm{\recv{x}{y}{P}}}$ send or receive over channel $\pcp{\tm{x}}$ a value $\pcp{\tm{y}}$ and proceed as process $\pcp{\tm{P}}$. Processes $\pcp{\tm{\close{x}{P}}}$ and $\pcp{\tm{\wait{x}{P}}}$ send and receive an empty value---denoting the closure of channel $\pcp{\tm{x}}$, and continue as $\pcp{\tm{P}}$.

Processes $\pcp{\tm{\inl{x}{P}}}$ and $\pcp{\tm{\inr{x}{P}}}$ make a left and right choice, respectively and proceed as process $\pcp{\tm{P}}$. Dually, $\pcp{\tm{\offer{x}{P}{Q}}}$ offers both left and right branches, with continuations $\pcp{\tm{P}}$ and $\pcp{\tm{Q}}$, and $\pcp{\tm{\absurd{x}}}$ is the empty offer.

We write \emph{unbound} send as $\pcp{\tm{\usend{x}{y}{P}}}$, which is syntactic sugar for $\pcp{\tm{\send{x}{z}{(\ppar{\link{y}{z}}{P})}}}$. Alternatively, we could take $\pcp{\tm{\usend{x}{y}{P}}}$ as primitive, and let $\pcp{\tm{\send{x}{y}{P}}}$ be syntactic sugar for $\pcp{\tm{\res{y}{z}{(\usend{x}{z}{P})}}}$. CP takes \emph{bound} sending as primitive, as it is impossible to eliminate the top-level cut in terms such as $\pcp{\tm{\res{y}{z}{(\usend{x}{z}{P})}}}$, even with commuting conversions. In our setting without commuting conversions and with more permissive normal forms, this is no longer an issue, but, for simplicity, we keep bound sending as primitive.

\subsubsection*{On Commuting Conversions}
\label{app:commuting-conversions}

The main change we make to PCP is \emph{removing commuting conversions}. Commuting conversions are necessary if we want our reduction strategy to correspond \emph{exactly} to cut (or cycle in \cite{dardhagay18extended}) elimination. However, as Lindley and Morris~\cite{lindleymorris15} show, all communications that can be performed \emph{with} the use of commuting conversions, can also be performed \emph{without} them, but using structural congruence.

From the perspective of process calculi, commuting conversions behave strangely.
Consider the commuting conversion $(\kappa_{\parr})$ for $\pcp{\tm{\recv{x}{y}{P}}}$:
\begin{mathpar}
  (\kappa_{\parr})
  \quad
  \tm{\res{z}{z'}{(\ppar{\recv{x}{y}{P}}{Q})}}
  \red
  \tm{\recv{x}{y}{\res{z}{z'}{(\ppar{P}{Q})}}}
\end{mathpar}
As a result of $(\kappa_{\parr})$, $\pcp{\tm{Q}}$ becomes blocked on $\pcp{\tm{\labrecv{x}{y}}}$, and any actions $\pcp{\tm{Q}}$ was able to perform become unavailable. Consequently, CP is non-confluent:
\begin{mathpar}
  \setlength{\arraycolsep}{2em}
  \begin{array}{cc}
    \multicolumn{2}{c}{%
    \hspace*{10ex}
    {\tm{\res{x}{x'}{(\ppar{\recv{a}{y}{P}}{\res{z}{z'}{(\ppar{\close{z}{\halt}}{\wait{z'}{Q}})}})}}}}
    \\
    \qquad\rotatebox[origin=c]{270}{$\red\hphantom{{}^+}$}
    &
    \rotatebox[origin=c]{270}{$\red^+$}\qquad
    \\
    {\tm{\recv{a}{y}{\res{x}{x'}{(\ppar{P}{\res{z}{z'}{(\ppar{\close{z}{\halt}}{\wait{z'}{Q}})}})}}}}
    &
    {\tm{\recv{a}{y}{\res{x}{x'}{(\ppar{P}{Q})}}}}
  \end{array}
\end{mathpar}

In PCP, commuting conversions break our intuition that an action with lower priority occurs before an action with higher priority. To cite Dardha and Gay~\cite{dardhagay18extended} ``\emph{if a prefix on a channel endpoint $\pcp{\tm{x}}$ with priority $\cs{o}$ is pulled out at top level, then to preserve priority constraints in the typing rules [..], it is necessary to increase priorities of all actions after the prefix on $\pcp{\tm{x}}$}'' by $\cs{o+1}$.

\subsection{Operational Semantics}
The operational semantics for PCP, given in \cref{fig:pcp-operational-semantics}, is defined as a reduction relation $\pcp{\red}$ on processes (bottom) and uses structural congruence (top). Each of the axioms of structural congruence corresponds to the axiom of the same name for PGV. We write $\pcp{\red^+}$ for the transitive closures, and $\pcp{\red^\star}$ for the reflexive-transitive closures.

The reduction relation is given by a set of axioms and inference rules for context closure. Reduction occurs under restriction. $\LabTirName{E-Link}$ reduces a parallel composition with a link into a substitution. $\LabTirName{E-Send}$ is the main communication rule, where send and receive processes sychronise and reduce to the corresponding continuations. $\LabTirName{E-Close}$ follows the previous rule and it closes the channel identified by endpoints $\pcp{\tm{x}}$ and $\pcp{\tm{y}}$. $\LabTirName{E-Select-Inl}$ and $\LabTirName{E-Select-Inr}$ are generalised versions of $\LabTirName{E-Send}$. They state respectively that a left and right selection synchronises with a choice offering and reduces to the corresponding continuations. The last three rules state that reduction is closed under restriction, parallel composition and structural congruence, respectively.

\begin{figure}
  \begin{mdframed}
    \small
    \paragraph*{Structural congruence}
    \begin{mathpar}
      \begin{array}{llcl}
        \LabTirName{SC-LinkSwap}   & \tm{\link{x}{y}}
                                   & \equiv & \tm{\link{y}{x}}
        \\
        \LabTirName{SC-ResLink}    & \tm{\res{x}{y}{\link{x}{y}}}
                                   & \equiv & \tm{\halt}
        \\
        \LabTirName{SC-ResSwap}    & \tm{\res{x}{y}{P}}
                                   & \equiv & \tm{\res{y}{x}{P}}
        \\
        \LabTirName{SC-ResComm}    & \tm{\res{x}{y}{\res{z}{w}{P}}}
                                   & \equiv & \tm{\res{z}{w}{\res{x}{y}{P}}}
        \\
        \LabTirName{SC-ResExt}     & \tm{\res{x}{y}{(\ppar{P}{Q})}}
                                   & \equiv & \tm{\ppar{P}{\res{x}{y}{Q}}},
                                              \text{ if }{\tm{x},\tm{y}\notin\fv(\tm{P})}
        \\
        \LabTirName{SC-ParNil}     & \tm{\ppar{P}{\halt}}
                                   & \equiv & \tm{P}
        \\
        \LabTirName{SC-ParComm}    & \tm{\ppar{P}{Q}}
                                   & \equiv & \tm{\ppar{Q}{P}}
        \\
        \LabTirName{SC-ParAssoc}   & \tm{\ppar{P}{(\ppar{Q}{R})}}
                                   & \equiv & \tm{\ppar{(\ppar{P}{Q})}{R}}
      \end{array}
    \end{mathpar}
    
    \paragraph*{Reduction}
    \begin{mathpar}
      \begin{array}{llcl}
        \LabTirName{E-Link}    & \tm{\res{x}{y}{(\ppar{\link{w}{x}}{P})}}
                               & \red & \tm{\subst{P}{w}{x}}
        \\
        \LabTirName{E-Send}    & \tm{\res{x}{y}{(\ppar{\send{x}{z}{P}}{\recv{x}{w}{Q}})}}
                               & \red & \tm{\res{x}{y}{\res{z}{w}{(\ppar{P}{Q})}}}
        \\
        \LabTirName{E-Close}   & \tm{\res{x}{y}{(\ppar{\close{x}{P}}{\wait{y}{Q}})}}
                               & \red & \tm{\ppar{P}{Q}}
        \\
        \LabTirName{E-Select-Inl}
                               & \tm{\res{x}{y}{(\ppar{\inl{x}{P}}{\offer{x}{Q}{R}})}}
                               & \red & \tm{\res{x}{y}{(\ppar{P}{Q})}}
        \\
        \LabTirName{E-Select-Inr}
                               & \tm{\res{x}{y}{(\ppar{\inr{x}{P}}{\offer{x}{Q}{R}})}}
                               & \red & \tm{\res{x}{y}{(\ppar{P}{R})}}
      \end{array}
      \\
      \inferrule*[lab=E-LiftRes]{
        \tm{P}\red\tm{P'}
      }{\tm{\res{x}{y}{P}}\red\tm{\res{x}{y}{P'}}}
    
      \inferrule*[lab=E-LiftPar]{
        \tm{P}\red\tm{P'}
      }{\tm{\ppar{P}{Q}}\red\tm{\ppar{P'}{Q}}}
    
      \inferrule*[lab=E-LiftSC]{
        \tm{P}\equiv\tm{P'}
        \\
        \tm{P'}\red\tm{Q'}
        \\
        \tm{Q'}\equiv\tm{Q}
      }{\tm{P}\red\tm{Q}}
    \end{mathpar}
    \caption{Operational Semantics for PCP.}
    \label{fig:pcp-operational-semantics}
  \end{mdframed}
\end{figure}
%%% Local Variables:
%%% TeX-master: "../priorities"
%%% End:


\subsection{Typing Rules}
\Cref{fig:pcp-typing} gives the typing rules for our version of PCP. A typing judgement $\pcp{\seq{P}{\Gamma}}$ states that ``process $\pcp{\tm{P}}$ is well typed under the typing context $\pcp{\ty{\Gamma}}$''.

\begin{figure}[H]
  \begin{mdframed}
    \begin{mathpar}
      \inferrule*[lab=T-Link]{
      }{\seq{\link[\ty{A}]{x}{y}}{\tmty{x}{A},\tmty{y}{\co{A}}}}
      
      \inferrule*[lab=T-Res]{
        \seq{P}{\ty{\Gamma},\tmty{x}{A},\tmty{y}{\co{A}}}
      }{\seq{\res{x}{y}{P}}{\ty{\Gamma}}}
      \\
      \inferrule*[lab=T-Par]{
        \seq{P}{\ty{\Gamma}}
        \\
        \seq{Q}{\ty{\Delta}}
      }{\seq{\ppar{P}{Q}}{\ty{\Gamma},\ty{\Delta}}}
      
      \inferrule*[lab=T-Halt]{
      }{\seq{\halt}{\emptyenv}}
      \\
      \inferrule*[lab=T-Send]{
        \seq{P}{\ty{\Gamma},\tmty{y}{A},\tmty{x}{B}}
        \\
        \cs{o}<\minpr(\ty{\Gamma},\ty{A},\ty{B})
      }{\seq{\send{x}{y}{P}}{\ty{\Gamma},\tmty{x}{\tytens[\cs{o}]{A}{B}}}}
      
      \inferrule*[lab=T-Close]{
        \seq{P}{\ty{\Gamma}}
        \\
        \cs{o}<\minpr(\ty{\Gamma})
      }{\seq{\close{x}{P}}{\ty{\Gamma},\tmty{x}{\tyone[\cs{o}]}}}
      \\
      \inferrule*[lab=T-Recv]{
        \seq{P}{\ty{\Gamma},\tmty{y}{A},\tmty{x}{B}}
        \\
        \cs{o}<\minpr(\ty{\Gamma},\ty{A},\ty{B})
      }{\seq{\recv{x}{y}{P}}{\ty{\Gamma},\tmty{x}{\typarr[\cs{o}]{A}{B}}}}
      
      \inferrule*[lab=T-Wait]{
        \seq{P}{\ty{\Gamma}}
        \\
        \cs{o}<\minpr(\ty{\Gamma})
      }{\seq{\wait{x}{P}}{\ty{\Gamma},\tmty{x}{\tybot[\cs{o}]}}}
      \\
      \inferrule*[lab=T-Select-Inl]{
        \seq{P}{\ty{\Gamma},\tmty{x}{A}}
        \\
        \cs{o}<\minpr(\ty{\Gamma},\ty{A},\ty{B})
        \\
        \pr(\ty{A})=\pr(\ty{B})
      }{\seq{\inl{x}{P}}{\ty{\Gamma},\tmty{x}{\typlus[\cs{o}]{A}{B}}}}
      
      \inferrule*[lab=T-Select-Inr]{
        \seq{P}{\ty{\Gamma},\tmty{x}{B}}
        \\
        \cs{o}<\minpr(\ty{\Gamma},\ty{A},\ty{B})
        \\
        \pr(\ty{A})=\pr(\ty{B})
      }{\seq{\inr{x}{P}}{\ty{\Gamma},\tmty{x}{\typlus[\cs{o}]{A}{B}}}}
      
      \inferrule*[lab=T-Offer]{
        \seq{P}{\ty{\Gamma},\tmty{x}{A}}
        \\
        \seq{Q}{\ty{\Gamma},\tmty{x}{B}}
        \\
        \cs{o}<\minpr(\ty{\Gamma},\ty{A},\ty{B})
      }{\seq{\offer{x}{P}{Q}}{\ty{\Gamma},\tmty{x}{\tywith[\cs{o}]{A}{B}}}}
      
      \inferrule*[lab=T-Offer-Absurd]{
        \cs{o}<\pr(\ty{\Gamma})
      }{\seq{\absurd{x}}{\ty{\Gamma},\tmty{x}{\tytop[\cs{o}]}}}
    \end{mathpar}
    \caption{Typing Rules for PCP.}
    \label{fig:pcp-typing}
  \end{mdframed}
\end{figure}
%%% Local Variables:
%%% TeX-master: "../priorities"
%%% End:


\textsc{T-Link} states that the link process $\pcp{\tm{\link{x}{y}}}$ is well typed under channels $\pcp{\tm{x}}$ and $\pcp{\tm{y}}$ having dual types, respectively $\pcp{\ty{A}}$ and $\pcp{\ty{\co{A}}}$. \textsc{T-Res} states that the restriction process $\pcp{\tm{\res{x}{y}{P}}}$ is well typed under typing context $\pcp{\ty{\Gamma}}$ if process $\pcp{\tm{P}}$ is well typed in $\pcp{\ty{\Gamma}}$ augmented with channel endpoints $\pcp{\tm{x}}$ and $\pcp{\tm{y}}$ having dual types, respectively $\pcp{\ty{A}}$ and $\pcp{\ty{\co{A}}}$. \textsc{T-Par} states that the parallel composition of processes $\pcp{\tm{P}}$ and $\pcp{\tm{Q}}$ is well typed under the disjoint union of their respective typing contexts. \textsc{T-Halt} states that the terminated process $\pcp{\tm{\halt}}$ is well typed in the empty context.

\textsc{T-Send} and \textsc{T-Recv} state that the sending and receiving of a bound name $\pcp{\tm{y}}$ over a channel $\pcp{\tm{x}}$ is well typed under $\pcp{\ty{\Gamma}}$ and $\pcp{\tm{x}}$ of type $\pcp{\ty{\tytens[\cs{o}]{A}{B}}}$, respectively $\pcp{\typarr[\cs{o}]{A}{B}}$. Priority $\cs{o}$ is the smallest among all priorities of the types used by the output or input process, captured by the side condition $\pcp{\cs{o}<\minpr(\ty{\Gamma},\ty{A},\ty{B})}$.

Rules \textsc{T-Close} and \textsc{T-Wait} type the closure of channel $\pcp{\tm{x}}$ and are in the same lines as the previous two rules, requiring that the priority of channel $\pcp{\tm{x}}$ is the smallest among all priorities in $\pcp{\ty{\Gamma}}$.

\textsc{T-Select-Inl} and \textsc{T-Select-Inr} type respectively the left $\pcp{\tm{\inl{x}{P}}}$ and right $\pcp{\tm{\inr{x}{P}}}$ choice performed on channel $\pcp{\tm{x}}$. \textsc{T-Offer} and \textsc{T-Offer-Absurd} type the offering of a choice, or empty choice, on channel $\pcp{\tm{x}}$. In all the above rules the priority $\cs{o}$ of channel $\pcp{\tm{x}}$ is the smallest with respect to the typing context $\pcp{\cs{o}<\minpr(\ty{\Gamma})}$ and types involved in the choice $\pcp{\cs{o}<\minpr(\ty{\Gamma},\ty{A},\ty{B})}$.

\Cref{fig:pcp-typing-sugar} shows how syntactic sugar in PCP is well typed.
\begin{figure}[t]
  \begin{mathpar}
    \inferrule*[lab=T-Unbound-Send]{
      \seq{P}{\ty{\Gamma},\tmty{x}{B}}
    }{\seq{\usend{x}{y}{P}}{\ty{\Gamma},\tmty{x}{\tytens[\cs{o}]{A}{B}},\tmty{y}{\co{A}}}}
    \elabarrow
    \inferrule*{
      \inferrule*{
        \inferrule*{
        }{\seq{\link{y}{z}}{\tmty{y}{\co{A}},\tmty{z}{A}}}
        \\
        \seq{P}{\ty{\Gamma},\tmty{x}{B}}
      }{\seq{\ppar{\link{y}{z}}{P}}{\ty{\Gamma},\tmty{x}{B},\tmty{y}{\co{A}},\tmty{z}{A}}}
      \\
      \cs{o}<\minpr(\ty{\Gamma},\ty{A},\ty{B},\ty{\co{A}})
    }{\seq{\send{x}{z}{(\ppar{\link{y}{z}}{P})}}{\ty{\Gamma},\tmty{x}{\tytens[\cs{o}]{A}{B}},\tmty{y}{\co{A}}}}
  \end{mathpar}
  \caption{Typing Rules for Syntactic Sugar for PCP.}
  \label{fig:pcp-typing-sugar}
\end{figure}
%%% Local Variables:
%%% TeX-master: "../priorities"
%%% End:


Finally, since our reduction relation is a strict subset of the reduction relation in the original~\cite{dardhagay18extended}, we defer to their proofs. We prove progress for our version of PCP, see~\cref{prf:thm-pcp-closed-progress}.


\subsection{PCP and PLL}
In this subsection, we highlight the connection between PCP and linear logic.
Dardha and Gay \cite{dardhagay18} present PCP--consequently also PCP given in this paper--in a way which can be viewed both as Classical Processes with restriction (T-Res) and parallel composition (T-Par) typing rules, and as a new version of linear logic, which they call Priority Linear Logic (PLL).
PLL builds on CLL by replacing the cut rule with two logical rules: a mix and a cycle rule---here corresponding to T-Par and T-Res, respectively. Dardha and Gay \cite[\S 4]{dardhagay18} prove cycle-elimination, in the same lines as cut-elimination for CLL. As a corollary of cycle-elimination for PLL, we obtain deadlock freedom for PCP (Theorem 3 in \cite[\S 4]{dardhagay18}). In summary, PLL is an extension of CLL and the authors show the correspondence of PCP and PLL. Notice however, that PCP is not in correspondence with CLL itself, since processes in PCP are graphs, whether CLL induces trees.

\subsection{Technical Developments}
\begin{definition}[Actions]
A~process acts on an endpoint $\tm{x}$ if it is $\tm{\link{x}{y}}$, $\tm{\link{y}{x}}$, $\tm{\send{x}{y}{P}}$, $\tm{\recv{x}{y}{P}}$, $\tm{\close{x}{P}}$, $\tm{\wait{x}{P}}$, $\tm{\inl{x}{P}}$, $\tm{\inr{x}{P}}$, $\tm{\offer{x}{P}{Q}}$, or $\tm{\absurd{x}}$. A~process is an action if it acts on some endpoint $\tm{x}$.
\end{definition}
\begin{definition}[Canonical Forms]
\label{def:pcp-canonical-forms}
A~process $\tm{P}$ is in canonical form if it is either $\tm{\halt}$ or of the form $\tm{\res{x_1}{x'_1}{\dots\res{x_n}{x'_n}{(P_1 \parallel\dots\parallel P_m)}}}$ where $m>0$ and each $\tm{P_j}$ is an action.
\end{definition}
\begin{lemma}[Canonical Forms]
\label{lem:pcp-canonical-forms}
If $\seq{P}{\ty{\Gamma}}$, there exists some $\tm{Q}$ such that $\tm{P}\equiv\tm{Q}$ and $\tm{Q}$ is in canonical form.
\end{lemma}
\begin{proof}
If $\tm{P}=\tm{\halt}$, we are done. Otherwise, we move any $\nu$-binders to the top using \LabTirName{SC-ResExt}, and discard any superfluous occurrences of $\tm{\halt}$ using \LabTirName{SC-ParNil}.
\end{proof}

%\restatetheorem{thmpcpclosedprogress}
\begin{restatabletheorem}{thmpcpclosedprogress}[Progress, $\pcp{\red}$]%
  \label{thm:pcp-closed-progress}
  \hfill\\%newline before theorem statement
  If $\pcp{\seq{P}{\emptyenv}}$, then either $\pcp{\tm{P}=\tm{\halt}}$ or there exists a $\pcp{\tm{Q}}$ such that $\pcp{\tm{P}\red\tm{Q}}$.
\end{restatabletheorem}
\begin{proof}
  \label{prf:thm-pcp-closed-progress}
  By~\cref{lem:pcp-canonical-forms}, we rewrite $\tm{P}$ to canonical form. If the resulting process is $\tm{\halt}$, we are done. Otherwise, it is of the form
  \[
    \seq{\res{x_1}{x'_1}{\dots\res{x_n}{x'_n}{(P_1 \parallel\dots\parallel P_m)}}}{\emptyenv}
  \]
  where $m>0$ and each $\seq{P_i}{\ty{\Gamma_i}}$ is an action.

  Our proof follows the same reasoning by Kobayashi~\cite{kobayashi06} used in the proof of deadlock freedom for closed processes (Theorem 2). 

  Consider processes $\tm {P_1 \parallel\dots\parallel P_m}$. Among them, we pick the process with the smallest priority $\minpr{(\ty{\Gamma_i})}$ for all $i$. Let this process be let this be $\tm{P_i}$ and the priority of the top prefix be $\cs o$. $\tm{P_i}$ acts on some endpoint $\tmty{y}{A}\in\ty{\Gamma_i}$. We must have $\minpr{(\ty{\Gamma_i})}=\pr{(\ty{A})} = \cs o$, since the other actions in $\tm{P_i}$ are guarded by the action on $\tmty{y}{A}$, thus satisfying law (i) of priorities.

  If $\tm{P_i}$ is a link $\tm{\link{y}{z}}$ or $\tm{\link{z}{y}}$, we apply \LabTirName{E-Link}.

  Otherwise, $\tm{P_i}$ is an input/branching or output/selection action on endpoint $\tm y$ of type $\ty A$ with priority $\cs o$. Since process $\tm P$ is closed and consequently it respects law (ii) of priorities, there must be a co-action $y'$ of type $\ty{\co{A}}$  where $\tm{y}$ and $\tm{y'}$ are dual endpoints of the same channel (by application of rule \textsc{T-Res}). By duality, $\pr{(\ty{A})}=\pr{(\ty{\co{A}})}= \cs o$. In the following we show that: $y'$ is the subject of a top level action of a process $P_j$ with $i\neq j$. This allows for the communication among $\tm{P_i}$ and $\tm{P_j}$ to happen immediately over channel endpoints $\tm{y}$ and $\tm{y'}$.
  
  Suppose that $\tm{y'}$ is an action not in a different parallel process $P_j$ but rather of $P_i$ itself. That means that the action on $\tm{y'}$ must be prefixed by the action on $\tm{y}$, which is top level in $\tm{P_i}$. To respect law (i) of priorities we must have $\cs o < \cs o$, which is absurd. This means that $\tm{y'}$ is in another parallel process $\tm{P_j}$ for $i\neq j$.

  Suppose that $\tm{y'}$ in $\tm{P_j}$ is not at top level. In order to respect law (i) of priorities, it means that $\tm{y'}$ is prefixed by actions that are smaller than its priority $\cs o$. This leads to a contradiction because stated that $\cs o$ is the smallest priority. Hence, $y'$ must be the subject of a top level action.
  
  We have two processes, acting on dual endpoints. We apply the appropriate reduction rule, \ie \LabTirName{E-Send}, \LabTirName{E-Close}, \LabTirName{E-Select-Inl}, or \LabTirName{E-Select-Inr}.
\end{proof}

%%% Local Variables:
%%% TeX-master: "../priorities"
%%% End:

\endgroup

\subsection{Correspondence between PGV and PCP}
\begingroup
We illustrate the relation between PCP and PGV by defining a translation from PCP to PGV. The translation on types is defined as follows:
\[
  \begin{array}{lcl}
    \ty{\cpgvT{\pcp{\tytens[\cs{o}]{A}{B}}}}
    & = & \pgv{\ty{\tysend[\cs{o}]{\co{\cpgvT{A}}}{\cpgvT{B}}}}
    \\
    \ty{\cpgvT{\pcp{\typlus[\cs{o}]{A}{B}}}}
    & = & \pgv{\ty{\tyselect[\cs{o}]{\cpgvT{A}}{\cpgvT{B}}}}
  \end{array}
  \hfill\quad\hfill%
  \begin{array}{lcl}
    \ty{\cpgvT{\pcp{\tyone[\cs{o}]}}}
    & = & \ty{\pgv{\tyends[\cs{o}]}}
    \\
    \ty{\cpgvT{\pcp{\tynil[\cs{o}]}}}
    & = & \pgv{\ty{\tyselectemp[\cs{o}]}}
  \end{array}
\]%
\[%
  \begin{array}{lcl}
    \ty{\cpgvT{\pcp{\typarr[\cs{o}]{A}{B}}}}
    & = & \pgv{\ty{\tyrecv[\cs{o}]{\cpgvT{A}}{\cpgvT{B}}}}
    \\
    \ty{\cpgvT{\pcp{\tywith[\cs{o}]{A}{B}}}}
    & = & \pgv{\ty{\tyoffer[\cs{o}]{\cpgvT{A}}{\cpgvT{B}}}}
  \end{array}
  \hfill\quad\hfill%
  \begin{array}{lcl}
    \ty{\cpgvT{\pcp{\tybot[\cs{o}]}}}
    & = & \ty{\pgv{\tyendr[\cs{o}]}}
    \\
    \ty{\cpgvT{\pcp{\tytop[\cs{o}]}}}
    & = & \pgv{\ty{\tyofferemp[\cs{o}]}}
  \end{array}
\]

There are two separate translations on processes. The main translation, $\tm{\cpgvM{\cdot}}$, translates processes to \emph{terms}:
\begin{align*}
  &\pcp{\tm{\cpgvM{\link{x}{y}}}}
  &&= \pgv{\tm{\link\;{\pair{x}{y}}}} \\
  &\pcp{\tm{\cpgvM{\res{x}{y}{P}}}}
  &&= \pgv{\tm{\letpair{x}{y}{\new\;\unit}{\cpgvM{P}}}} \\
  &\pcp{\tm{\cpgvM{\ppar{P}{Q}}}}
  &&= \pgv{\tm{\andthen{\spawn\;{(\lambda\unit.\cpgvM{P})}}{\cpgvM{Q}}}} \\
  &\pcp{\tm{\cpgvM{\halt}}}
  &&= \pgv{\tm{\unit}} \\
  &\pcp{\tm{\cpgvM{\close{x}{P}}}}
  &&= \pgv{\tm{\andthen{\close\;{x}}{\cpgvM{P}}}} \\
  &\pcp{\tm{\cpgvM{\wait{x}{P}}}}
  &&= \pgv{\tm{\andthen{\wait\;{x}}{\cpgvM{P}}}} \\
  &\pcp{\tm{\cpgvM{\send{x}{y}{P}}}}
  &&= \pgv{\tm{\letpair{y}{z}{\new\;\unit}{\letbind{x}{\send\;{\pair{z}{x}}}{\cpgvM{P}}}}} \\
  &\pcp{\tm{\cpgvM{\recv{x}{y}{P}}}}
  &&= \pgv{\tm{\letpair{y}{x}{\recv\;{x}}{\cpgvM{P}}}} \\
  &\pcp{\tm{\cpgvM{\inl{x}{P}}}}
  &&= \pgv{\tm{\letbind{x}{\select{\labinl}\;{x}}{\cpgvM{P}}}} \\
  &\pcp{\tm{\cpgvM{\inr{x}{P}}}}
  &&= \pgv{\tm{\letbind{x}{\select{\labinr}\;{x}}{\cpgvM{P}}}} \\
  &\pcp{\tm{\cpgvM{\offer{x}{P}{Q}}}}
  &&= \pgv{\tm{\offer{x}{x}{\cpgvM{P}}{x}{\cpgvM{Q}}}} \\
  &\pcp{\tm{\cpgvM{\absurd{x}}}}
  &&= \pgv{\tm{\offeremp{x}}}
\end{align*}

Unfortunately, the operational correspondence along $\tm{\cpgvM{\cdot}}$ is unsound, as it translates $\nu$-binders and parallel compositions to $\pgv{\tm{\new}}$ and $\pgv{\tm{\spawn}}$, which can reduce to their equivalent configuration constructs using \LabTirName{E-New} and \LabTirName{E-Spawn}. The same goes for $\nu$-binders which are inserted when translating bound send to unbound send. For instance, the process $\pcp{\tm{\send{x}{y}{P}}}$ is blocked, but its translation uses $\pgv{\tm{\new}}$ and can reduce. To address this issue, we use a second translation, $\tm{\cpgvC{\cdot}}$, which is equivalent to translating with $\tm{\cpgvM{\cdot}}$ then reducing with \LabTirName{E-New} and \LabTirName{E-Spawn}:
\begin{align*}
  &\pcp{\tm{\cpgvC{\res{x}{y}{P}}}}
  &&= \pgv{\tm{\res{x}{y}{\cpgvC{P}}}}
  \\
  &\pcp{\tm{\cpgvC{\ppar{P}{Q}}}}
  &&= \pgv{\tm{\ppar{\cpgvC{P}}{\cpgvC{Q}}}}
  \\
  &\pcp{\tm{\cpgvC{\send{x}{y}{P}}}}
  &&= \pgv{\tm{\res{y}{z}{(\child\;\letbind{x}{\send\;\pair{z}{x}}{\cpgvM{P}})}}}
  \\
  &\pcp{\tm{\cpgvC{\inl{x}{P}}}}
  &&= \pgv{\tm{\res{y}{z}{(\child\;\letbind{x}{\andthen{\close\;(\send\;\pair{\inl{y}}{x})}{z}}{\cpgvM{P}})}}}
  \\
  &\pcp{\tm{\cpgvC{\inr{x}{P}}}}
  &&= \pgv{\tm{\res{y}{z}{(\child\;\letbind{x}{\andthen{\close\;(\send\;\pair{\inr{y}}{x})}{z}}{\cpgvM{P}})}}}
  \\
  &\pcp{\tm{\cpgvC{P}}}
  &&= \pgv{\tm{\child{\cpgvM{P}}}},\quad\text{if none of the above apply}
\end{align*}
Typing environments are translated pointwise, and sequents $\pcp{\seq{P}{\ty{\Gamma}}}$ are translated as $\pgv{\cseq[\child]{\ty{\cpgvT{\ty{\Gamma}}}}{\cpgvC{P}}}$, where $\tm{\pgv{\child}}$ indicates a child thread, since translated processes do not have a main thread.
The translations $\tm{\cpgvM{\cdot}}$ and $\tm{\cpgvC{\cdot}}$ preserve typing, and the latter induces a sound and complete operational correspondence.

%\restatelemma{lempcptopgvtermspreservation}
\begin{lemma}[Preservation, ${\tm{\cpgvM{\cdot}}}$]%
  \label{lem:pcp-to-pgv-terms-preservation}
  If $\pcp{\seq{P}{\ty{\Gamma}}}$, then $\pgv{\tseq[\cs{p}]{\ty{\cpgvT{\Gamma}}}{\cpgvM{P}}{\tyunit}}$.
\end{lemma}
\begin{proof}
  \label{prf:lem-pcp-to-pgv-terms-preservation}
  By induction on the derivation of $\pcp{\seq{P}{\ty{\Gamma}}}$.
  \begin{case*}[\LabTirName{T-Link}, \LabTirName{T-Res}, \LabTirName{T-Par}, and \LabTirName{T-Halt}]
    See \cref{fig:pcp-to-pgv-preservation}.
  \end{case*}
  \begin{case*}[\LabTirName{T-Close}, and \LabTirName{T-Wait}]
    See \cref{fig:pcp-to-pgv-preservation-close-and-wait}.
  \end{case*}
  \begin{case*}[\LabTirName{T-Send}]
    See \cref{fig:pcp-to-pgv-preservation-send}.
  \end{case*}
  \begin{case*}[\LabTirName{T-Recv}]
    See \cref{fig:pcp-to-pgv-preservation-recv}.
  \end{case*}
  \begin{case*}[\LabTirName{T-Select-Inl}, \LabTirName{T-Select-Inr}, and \LabTirName{T-Offer}]
    See \cref{fig:pcp-to-pgv-preservation-select-and-offer}.
  \end{case*}
\end{proof}
\begin{landscape}
\begin{figure}
\small
\begin{mathpar}
  % Translation for Link
  \pcp{\inferrule*[lab=T-Link]{
    }{\seq{\link[\ty{A}]{x}{y}}{\tmty{x}{A}, \tmty{y}{\co{A}}}}}
  \cpgvMarrow
  \pgv{\inferrule*{
      \inferrule*{
      }{\tmty{\link}{\tylolli[]{\typrod{\cpgvT{A}}{\co{\cpgvT{A}}}}{\tyunit}}}
      \\
      \inferrule*{
        \inferrule*{
        }{\tseq[\cs{\pbot}]
          {\tmty{x}{\cpgvT{A}}}
          {x}
          {\cpgvT{A}}}
        \\
        \inferrule*{
        }{\tseq[\cs{\pbot}]
          {\tmty{y}{\co{\cpgvT{A}}}}
          {y}
          {\co{\cpgvT{A}}}}
      }{\tseq[\cs{\pbot}]
        {\tmty{x}{\cpgvT{A}},\tmty{y}{\co{\cpgvT{A}}}}
        {\pair{x}{y}}
        {\typrod{\cpgvT{A}}{\co{\cpgvT{A}}}}}
    }{\tseq[\cs{\pbot}]
      {\tmty{x}{\cpgvT{A}},\tmty{y}{\co{\cpgvT{A}}}}
      {\link\;{\pair{x}{y}}}
      {\tyunit}}}

  % Translation for Res
  \pcp{\inferrule*[lab=T-Res]{
      \seq{P}{\ty{\Gamma},\tmty{x}{A},\tmty{y}{\co{A}}}
    }{\seq{\res{x}{y}{P}}{\ty{\Gamma}}}}
  \cpgvMarrow
  \pgv{\inferrule*{
      \inferrule*{
        \inferrule*{
        }{\tmty
          {\new}
          {\tylolli{\tyunit}{{\typrod{\cpgvT{A}}{\co{\cpgvT{A}}}}}}}
        \\
        \inferrule*{
        }{\tseq[\cs{\pbot}]
          {\emptyenv}
          {\unit}
          {\tyunit}}
      }{\tseq[\cs{\pbot}]
        {\emptyenv}
        {\new\;\unit}
        {\typrod{\cpgvT{A}}{\co{\cpgvT{A}}}}}
      \\
      {\tseq[\cs{p}]
        {\ty{\cpgvT{\Gamma}},\tmty{x}{\cpgvT{A}},\tmty{y}{\co{\cpgvT{A}}}}
        {\cpgvM{P}}
        {\tyunit}}
    }{\tseq[\cs{p}]
      {\ty{\cpgvT{\Gamma}}}
      {\letpair{x}{y}{\new\;\unit}{\cpgvM{P}}}
      {\tyunit}}}

  % Translation for Mix
  \pcp{\inferrule*[lab=T-Par]{
      \seq{\tm{P}}{\ty{\Gamma}}
      \\
      \seq{\tm{Q}}{\ty{\Delta}}
    }{\seq{\tm{\ppar{P}{Q}}}{\ty{\Gamma},\ty{\Delta}}}}
  \cpgvMarrow
  \pgv{\inferrule*{
      \inferrule*{
        \inferrule*{
        }{\tmty{\spawn}{\tylolli{(\tylolli[\cs{\pr{(\ty{\Gamma})}},\cs{p}]{\tyunit}{\tyunit})}{\tyunit}}}
        \\
        \inferrule*{
          \tseq[\cs{p}]
          {\ty{\cpgvT{\Gamma}}}
          {\tm{\cpgvM{P}}}
          {\tyunit}
        }{\tseq[\cs{\pbot}]
          {\ty{\cpgvT{\Gamma}}}
          {\lambda\unit.\cpgvM{P}}
          {\tylolli[\cs{\pr{(\ty{\Gamma})}},\cs{p}]{\tyunit}{\tyunit}}}
      }{\tseq[\cs{\pbot}]
        {\ty{\cpgvT{\Gamma}}}
        {\tm{\spawn\;{(\lambda\unit.\cpgvM{P})}}}
        {\tyunit}}
      \\
      \tseq[\cs{q}]
      {\ty{\cpgvT{\Delta}}}
      {\tm{\cpgvM{Q}}}
      {\tyunit}
    }{\tseq[\cs{q}]
      {\ty{\cpgvT{\Gamma}},\ty{\cpgvT{\Delta}}}
      {\andthen{\spawn\;{(\lambda\unit.\cpgvM{P})}}{\cpgvM{Q}}}
      {\tyunit}}}
  \\
  % Translation for Halt
  \pcp{\inferrule*[lab=T-Halt]{
    }{\seq{\tm{\halt}}{\emptyenv}}}
  \cpgvMarrow
  \pgv{\inferrule*{
    }{\tseq[\cs{\pbot}]{\emptyenv}{\unit}{\tyunit}}}
\end{mathpar}
\caption{Translation $\cpgvM{\cdot}$ preserves typing (\LabTirName{T-Link}, \LabTirName{T-Res}, \LabTirName{T-Par}, and \LabTirName{T-Halt}).}
\label{fig:pcp-to-pgv-preservation}
\end{figure}
\begin{figure}
\begin{mathpar}
  % Translation for Close
  \pcp{\inferrule*[lab=T-Close]{
      \seq{\tm{P}}{\ty{\Gamma}}
      \\
      \cs{o}<\pr(\ty{\Gamma})
    }{\seq{\tm{\close{x}{P}}}{\ty{\Gamma},\tmty{x}{\tyone[\cs{o}]}}}}
  \cpgvMarrow
  \pgv{\inferrule*{
      \inferrule*{
        \inferrule*{
        }{\tmty{\close}{\tylolli[\cs{\ptop},\cs{o}]{\tyends[\cs{o}]}{\tyunit}}}
        \\
        \inferrule*{
        }{\tseq[\cs{\pbot}]
          {\tmty{x}{\tyends[\cs{o}]}}
          {x}
          {\tyends[\cs{o}]}}
      }{\tseq[\cs{o}]
        {\tmty{x}{\tyends[\cs{o}]}}
        {\close\;{x}}
        {\tyunit}}
      \\
      \tseq[\cs{p}]{\ty{\cpgvT{\Gamma}}}{\cpgvM{P}}{\tyunit}
      \\
      \cs{o}<\pr(\ty{\cpgvT{\Gamma}})
    }{\tseq[\cs{o}\sqcup\cs{p}]
      {\ty{\cpgvT{\Gamma}},\tmty{x}{\tyends[\cs{o}]}}
      {\andthen{\close\;{x}}{\cpgvM{P}}}
      {\tyunit}}}
    \\
  % Translation for Wait
  \pcp{\inferrule*[lab=T-Wait]{
      \seq{\tm{P}}{\ty{\Gamma}}
      \\
      \cs{o}<\pr(\ty{\Gamma})
    }{\seq{\tm{\wait{x}{P}}}{\ty{\Gamma},\tmty{x}{\tyone[\cs{o}]}}}}
  \cpgvMarrow
  \pgv{\inferrule*{
      \inferrule*{
        \inferrule*{
        }{\tmty{\wait}{\tylolli[\cs{\ptop},\cs{o}]{\tyendr[\cs{o}]}{\tyunit}}}
        \\
        \inferrule*{
        }{\tseq[\cs{\pbot}]
          {\tmty{x}{\tyendr[\cs{o}]}}
          {x}
          {\tyendr[\cs{o}]}}
      }{\tseq[\cs{o}]
        {\tmty{x}{\tyendr[\cs{o}]}}
        {\wait\;{x}}
        {\tyunit}}
      \\
      \tseq[\cs{p}]{\ty{\cpgvT{\Gamma}}}{\cpgvM{P}}{\tyunit}
      \\
      \cs{o}<\pr(\ty{\cpgvT{\Gamma}})
    }{\tseq[\cs{o}\sqcup\cs{p}]
      {\ty{\cpgvT{\Gamma}},\tmty{x}{\tyendr[\cs{o}]}}
      {\andthen{\wait\;{x}}{\cpgvM{P}}}
      {\tyunit}}}
\end{mathpar}
\caption{Translation $\cpgvM{\cdot}$ preserves typing (\LabTirName{T-Close} and \LabTirName{T-Wait}).}
\label{fig:pcp-to-pgv-preservation-close-and-wait}
\end{figure}
\begin{figure}
\begin{mathpar}
  % Translation for Send
  \pcp{\inferrule*[lab=T-Send]{
      \seq{\tm{P}}{\ty{\Gamma},\tmty{y}{A},\tmty{x}{B}}
      \\
      \cs{o}<\pr(\ty{\Gamma},\ty{A},\ty{B})
    }{\seq{\tm{\send{x}{y}{P}}}{\ty{\Gamma},\tmty{x}{\tytens[\cs{o}]{A}{B}}}}}
  \cpgvMarrow
  \\
  \pgv{\inferrule*[lab=(a)]{
      \inferrule*{
      }{\tmty
        {\new}
        {\tylolli{\tyunit}{{\typrod{\cpgvT{A}}{\co{\cpgvT{A}}}}}}}
      \\
      \inferrule*{
      }{\tseq[\cs{\pbot}]
        {\emptyenv}
        {\unit}
        {\tyunit}}
    }{\tseq[\cs{\pbot}]
      {\emptyenv}
      {\new\;\unit}
      {\typrod{\cpgvT{A}}{\co{\cpgvT{A}}}}}}
  \\
  \pgv{\inferrule*[lab=(b)]{
      \inferrule*{
      }{\tmty{\send}{\tylolli[\cs{\ptop},\cs{o}]
          {\typrod{\co{\cpgvT{A}}}{\tysend[\cs{o}]{\co{\cpgvT{A}}}{\cpgvT{B}}}}
          {\cpgvT{B}}}}
      \\
      \inferrule*{
        \inferrule*{
        }{\tseq[\cs{\pbot}]
          {\tmty{z}{\co{\cpgvT{A}}}}
          {x}
          {\co{\cpgvT{A}}}}
        \\
        \inferrule*{
        }{\tseq[\cs{\pbot}]
          {\tmty{x}{\tysend[\cs{o}]{\co{\cpgvT{A}}}{\cpgvT{B}}}}
          {x}
          {\tysend[\cs{o}]{\co{\cpgvT{A}}}{\cpgvT{B}}}}
      }{\tseq[\cs{\pbot}]
        {\tmty{x}{\tysend[\cs{o}]{\co{\cpgvT{A}}}{\cpgvT{B}}},\tmty{z}{\co{\cpgvT{A}}}}
        {\pair{z}{x}}
        {\typrod{\co{\cpgvT{A}}}{\tysend[\cs{o}]{\co{\cpgvT{A}}}{\cpgvT{B}}}}}
    }{\tseq[\cs{o}]
      {\tmty{x}{\tysend[\cs{o}]{\co{\cpgvT{A}}}{\cpgvT{B}}},\tmty{z}{\co{\cpgvT{A}}}}
      {\send\;{\pair{z}{x}}}
      {\cpgvT{B}}}}
  \\
  \pgv{\inferrule*{
      \LabTirName{(a)}
      \\
      \inferrule*{
        \LabTirName{(b)}
        \\
        {\tseq[\cs{p}]
          {\ty{\cpgvT{\Gamma}},\tmty{y}{\cpgvT{A}},\tmty{x}{\cpgvT{B}}}
          {\cpgvM{P}}
          {\tyunit}}
        \\
        \cs{o}<\pr(\ty{\cpgvT{\Gamma}},\ty{\cpgvT{A}},\ty{\cpgvT{B}})
      }{\tseq[\cs{o}\sqcup\cs{p}]
        {\ty{\cpgvT{\Gamma}},%
          \tmty{x}{\tysend[\cs{o}]{\co{\cpgvT{A}}}{\cpgvT{B}}},%
          \tmty{y}{\cpgvT{A}},\tmty{z}{\co{\cpgvT{A}}}}
        {\letbind{x}{\send\;{\pair{z}{x}}}{\cpgvM{P}}}
        {\tyunit}}
    }{\tseq[\cs{o}\sqcup\cs{p}]
      {\ty{\cpgvT{\Gamma}},\tmty{x}{\tysend[\cs{o}]{\co{\cpgvT{A}}}{\cpgvT{B}}}}
      {\letpair{y}{z}{\new\;\unit}{\letbind{x}{\send\;{\pair{z}{x}}}{\cpgvM{P}}}}
      {\tyunit}}}
\end{mathpar}
\caption{Translation $\cpgvM{\cdot}$ preserves typing (\LabTirName{T-Send}).}
\label{fig:pcp-to-pgv-preservation-send}
\end{figure}
\begin{figure}
\begin{mathpar}
  % Translation for Recv
  \pcp{\inferrule*[lab=T-Recv]{
      \seq{P}{\ty{\Gamma},\tmty{y}{A},\tmty{x}{B}}
      \\
      \cs{o}<\pr(\ty{\Gamma},\ty{A},\ty{B})
    }{\seq{\recv{x}{y}{P}}{\ty{\Gamma},\tmty{x}{\typarr[\cs{o}]{A}{B}}}}}
  \cpgvMarrow
  \\
  \pgv{\inferrule*[lab=(a)]{
      \inferrule*{
      }{\tmty{\recv}{\tylolli[\cs{\ptop},\cs{o}]
          {\tyrecv[\cs{o}]{{\cpgvT{A}}}{\cpgvT{B}}}
          {\typrod{{\cpgvT{A}}}{\cpgvT{B}}}}}
      \\
      \inferrule*{
      }{\tseq[\cs{\pbot}]
        {\tmty{x}{\tyrecv[\cs{o}]{{\cpgvT{A}}}{\cpgvT{B}}}}
        {x}
        {\tyrecv[\cs{o}]{{\cpgvT{A}}}{\cpgvT{B}}}}
    }{\tseq[\cs{o}]
      {\tmty{x}{\tyrecv[\cs{o}]{{\cpgvT{A}}}{\cpgvT{B}}}}
      {\recv\;{x}}{\typrod{\cpgvT{A}}{\cpgvT{B}}}}}
  \\
  \pgv{\inferrule*{
      \LabTirName{(a)}
      \\
      {\tseq[\cs{p}]
        {\ty{\cpgvT{\Gamma}},\tmty{y}{\cpgvT{A}},\tmty{x}{\cpgvT{B}}}
        {\cpgvM{P}}
        {\tyunit}}
      \\
      \cs{o}<\pr(\ty{\cpgvT{\Gamma}},\ty{\cpgvT{A}},\ty{\cpgvT{B}})
    }{\tseq[\cs{o}\sqcup\cs{p}]
      {\ty{\cpgvT{\Gamma}},%
        \tmty{x}{\tyrecv[\cs{o}]{{\cpgvT{A}}}{\cpgvT{B}}},\tmty{y}{\cpgvT{A}},\tmty{z}{{\cpgvT{A}}}}
      {\letbind{x}{\recv{x}}{\cpgvM{P}}}{\tyunit}}}
\end{mathpar}
\caption{Translation $\cpgvM{\cdot}$ preserves typing (\LabTirName{T-Recv}).}
\label{fig:pcp-to-pgv-preservation-recv}
\end{figure}
\begin{figure}
\begin{mathpar}
  % Translation for Select-Inl
  \pcp{\inferrule*[lab=T-Select-Inl]{
      \seq{\tm{P}}{\ty{\Gamma},\tmty{x}{A}}
      \\
      \cs{o}<\pr(\ty{\Gamma})
    }{\seq{\inl{x}{P}}{\ty{\Gamma},\tmty{x}{\typlus[\cs{o}]{A}{B}}}}}
  \cpgvMarrow
  \pgv{\inferrule*{
      \inferrule*{
        \inferrule*{
        }{\tmty
          {\select{\labinl}}
          {\tylolli[\cs{\ptop},\cs{o}]{\tyselect[\cs{o}]{\cpgvT{A}}{\cpgvT{B}}}{\cpgvT{A}}}}
        \\
        \inferrule*{
        }{\tseq[\cs{\pbot}]
          {\tmty{x}{\tyselect[\cs{o}]{\cpgvT{A}}{\cpgvT{B}}}}
          {x}
          {\tyselect[\cs{o}]{\cpgvT{A}}{\cpgvT{B}}}}
      }{\tseq[\cs{o}]
        {\tmty{x}{\tyselect[\cs{o}]{\cpgvT{A}}{\cpgvT{B}}}}
        {\select{\labinl}\;{x}}
        {\cpgvT{A}}}
      \\
      {\tseq[\cs{p}]
        {\ty{\Gamma},\tmty{x}{\cpgvT{A}}}
        {\cpgvM{P}}
        {\tyunit}}
      \\
      \cs{o}<\pr(\ty{\Gamma})
    }{\tseq[\cs{o}\sqcup\cs{p}]
      {\ty{\Gamma},\tmty{x}{\tyselect[\cs{o}]{\cpgvT{A}}{\cpgvT{B}}}}
      {\letbind{x}{\select{\labinl}\;{x}}{\cpgvM{P}}}
      {\tyunit}}}
  \\
  % Translation for Select-Inl
  \pcp{\inferrule*[lab=T-Select-Inr]{
      \seq{\tm{P}}{\ty{\Gamma},\tmty{x}{A}}
      \\
      \cs{o}<\pr(\ty{\Gamma})
    }{\seq{\inr{x}{P}}{\ty{\Gamma},\tmty{x}{\typlus[\cs{o}]{A}{B}}}}}
  \cpgvMarrow
  \pgv{\inferrule*{
      \inferrule*{
        \inferrule*{
        }{\tmty
          {\select{\labinr}}
          {\tylolli[\cs{\ptop},\cs{o}]{\tyselect[\cs{o}]{\cpgvT{A}}{\cpgvT{B}}}{\cpgvT{B}}}}
        \\
        \inferrule*{
        }{\tseq[\cs{\pbot}]
          {\tmty{x}{\tyselect[\cs{o}]{\cpgvT{A}}{\cpgvT{B}}}}
          {x}
          {\tyselect[\cs{o}]{\cpgvT{A}}{\cpgvT{B}}}}
      }{\tseq[\cs{o}]
        {\tmty{x}{\tyselect[\cs{o}]{\cpgvT{A}}{\cpgvT{B}}}}
        {\select{\labinr}\;{x}}
        {\cpgvT{B}}}
      \\
      {\tseq[\cs{p}]
        {\ty{\Gamma},\tmty{x}{\cpgvT{B}}}
        {\cpgvM{P}}
        {\tyunit}}
      \\
      \cs{o}<\pr(\ty{\Gamma})
    }{\tseq[\cs{o}\sqcup\cs{p}]
      {\ty{\Gamma},\tmty{x}{\tyselect[\cs{o}]{\cpgvT{A}}{\cpgvT{B}}}}
      {\letbind{x}{\select{\labinr}\;{x}}{\cpgvM{P}}}
      {\tyunit}}}
  \\
  % Translation for Offer
  \pcp{\inferrule*[lab=T-Offer]{
      \seq{P}{\ty{\Gamma},\tmty{x}{A}}
      \\
      \seq{Q}{\ty{\Gamma},\tmty{x}{B}}
      \\
      \cs{o}<\pr(\ty{\Gamma},\ty{A},\ty{B})
    }{\seq{\offer{x}{P}{Q}}{\ty{\Gamma},\tmty{x}{\tywith[\cs{o}]{A}{B}}}}}
  \cpgvMarrow
  \pgv{\inferrule*{
      \inferrule*{
      }{\tseq[\cs{\pbot}]
        {\tmty{x}{\tyoffer[\cs{o}]{\cpgvT{A}}{\cpgvT{B}}}}
        {x}
        {\tyoffer[\cs{o}]{\cpgvT{A}}{\cpgvT{B}}}}
      \\
      \tseq[\cs{p}]{\ty{\cpgvT{\Gamma}},\tmty{x}{\cpgvT{A}}}{\cpgvM{P}}{\tyunit}
      \\
      \tseq[\cs{p}]{\ty{\cpgvT{\Gamma}},\tmty{x}{\cpgvT{B}}}{\cpgvM{Q}}{\tyunit}
      \\
      \cs{o}<\pr(\ty{\cpgvT{\Gamma}},\ty{\cpgvT{A}},\ty{\cpgvT{B}})
    }{\tseq[\cs{o}\sqcup\cs{p}]
      {\ty{\cpgvT{\Gamma}},\tmty{x}{\tyoffer[\cs{o}]{\cpgvT{A}}{\cpgvT{B}}}}
      {\offer{x}{x}{\cpgvM{P}}{x}{\cpgvM{Q}}}
      {\tyunit}}}
\end{mathpar}
\caption{Translation $\cpgvM{\cdot}$ preserves typing (\LabTirName{T-Select-Inl}, \LabTirName{T-Select-Inr}, and \LabTirName{T-Offer}).}
\label{fig:pcp-to-pgv-preservation-select-and-offer}
\end{figure}
\end{landscape}

%%% Local Variables:
%%% TeX-master: "../priorities"
%%% End:

%\restatetheorem{thmpcptopgvconfspreservation}
\begin{theorem}[Preservation, ${\tm{\cpgvC{\cdot}}}$]%
  \label{thm:pcp-to-pgv-confs-preservation}
  If $\pcp{\seq{P}{\ty{\Gamma}}}$, then $\pgv{\cseq[\child]{\ty{\cpgvT{\Gamma}}}{\cpgvC{P}}}$.
\end{theorem}
\begin{proof}
  \begin{case*}[\LabTirName{T-Cut}]
    Immediately, from the induction hypothesis.
    \small
    \begin{mathpar}
      \pcp{\inferrule*[lab=T-Cut]{
          \seq{\ty{\Gamma},\tmty{x}{A},\tmty{y}{\co{A}}}{\tm{P}}
        }{\seq{\ty{\Gamma}}{\tm{\res{x}{y}{P}}}}}
      \cpgvcarrow
      \pgv{\inferrule*{
          \cseq[\child]{\ty{\cpgv{\Gamma}},\tmty{x}{\cpgv{A}},\tmty{y}{\cpgv{B}}}{\tm{\cpgvc{P}}}
        }{\cseq[\child]{\ty{\cpgv{\Gamma}}}{\tm{\res{x}{y}{\cpgvc{P}}}}}}
    \end{mathpar}
  \end{case*}
  \begin{case*}[\LabTirName{T-Mix}]
    Immediately, from the induction hypotheses.
    \begin{mathpar}
      \pcp{\inferrule*[lab=T-Mix]{
          \seq{\ty{\Gamma}}{\tm{P}}
          \\
          \seq{\ty{\Delta}}{\tm{Q}}
        }{\seq{\ty{\Gamma},\ty{\Delta}}{\ppar{P}{Q}}}}
      \cpgvcarrow
      \pgv{\inferrule*{
          \cseq[\child]{\ty{\cpgv{\Gamma}}}{\tm{\cpgvc{P}}}
          \\
          \cseq[\child]{\ty{\cpgv{\Delta}}}{\tm{\cpgvc{Q}}}
        }{\cseq[\child]{\ty{\cpgv{\Gamma}},\ty{\cpgv{\Delta}}}{\tm{\ppar{\cpgvc{P}}{\cpgvc{Q}}}}}}
    \end{mathpar}
  \end{case*}
  \begin{case*}[\LabTirName{*}]
    By \cref{lem:pcp-to-pgv-terms-preservation}
    \begin{mathpar}
      \pcp{\seq{\ty{\Gamma}}{\tm{P}}}
      \cpgvcarrow
      \pgv{\inferrule*{
          \tseq[\cs{p}]{\ty{\cpgv{\Gamma}}}{\tm{\cpgv{P}}}{\tyunit}
        }{\cseq[\child]{\ty{\cpgv{\Gamma}}}{\tm{\child\;\cpgv{P}}}}}
    \end{mathpar}
  \end{case*}
\end{proof}

%%% Local Variables:
%%% TeX-master: "../priorities"
%%% End:


%\restatetheorem{thmpcptopgvoperationalcorrespondencesoundness}
\begin{theorem}%
  [Operational Correspondence, Soundness, ${\tm{\cpgvC{\cdot}}}$]%
  \label{thm:pcp-to-pgv-operational-correspondence-soundness}
  \hfill\\%newline before theorem statement
  If $\pcp{\seq{P}{\ty{\Gamma}}}$ and $\pgv{\tm{\cpgvC{P}}\cred\tm{\conf{C}}}$, there exists a $\tm{Q}$ such that $\pcp{\tm{P}\red^+\tm{Q}}$ and $\pgv{\tm{\conf{C}}\cred^\star\tm{\cpgvC{Q}}}$
\end{theorem}
\begin{proof}
  \label{prf:thm-pcp-to-pgv-operational-correspondence-soundness}
  By induction on the derivation of $\pgv{\tm{\cpgvC{P}}\cred\tm{\conf{C}}}$.
  We omit the cases which cannot occur as their left-hand side term forms are not in the image of the translation function, \ie \LabTirName{E-New}, \LabTirName{E-Spawn}, and \LabTirName{E-LiftM}.

  \begin{case*}[\LabTirName{E-Link}]
    \[\pgv{%
        \tm{\res{x}{x'}{(\ppar{\plug{\conf{F}}{\link\;\pair{w}{x}}}{\conf{C}})}}
        \cred
        \tm{\ppar{\plug{\conf{F}}{\unit}}{\subst{\conf{C}}{w}{x'}}}
      }\]
    The source for $\pgv{\tm{\link\;\pair{w}{x}}}$ \emph{must} be $\pcp{\tm{\link{w}{x}}}$. None of the translation rules introduce an evaluation context around the recursive call, hence $\pgv{\tm{\conf{F}}}$ must be the empty context. Let $\pcp{\tm{P}}$ be the source term for $\pgv{\tm{\conf{C}}}$, \ie $\pgv{\tm{\cpgvC{P}}=\tm{\conf{C}}}$. Hence, we have:
    \begin{mathpar}
      \begin{tikzcd}[cramped, column sep=tiny]
        \pcp{\tm{\res{x}{x'}{(\ppar{\link{w}{x}}{P})}}}
        \arrow[r, "\pcp{\red}"]
        \arrow[d, "\cpgvC{\cdot}"]
        &
        \pcp{\tm{\subst{P}{w}{x'}}}
        \arrow[dd, "\cpgvC{\cdot}"]
        \\
        \pgv{\tm{\res{x}{x'}{(\ppar{\child\;\link\;\pair{w}{x}}{\cpgvC{P}})}}}
        \arrow[d, "\pgv{\cred^+}"]
        \\
        \pgv{\tm{\subst{\cpgvC{P}}{w}{x'}}}
        \arrow[r, "\pcp{=}"]
        &
        \pgv{\tm{\cpgvC{\subst{P}{w}{x'}}}}
      \end{tikzcd}
    \end{mathpar}
  \end{case*}
  \begin{case*}[\LabTirName{E-Send}]
    \[\pgv{%
        \tm{\res{x}{x'}{(\ppar{\plug{\conf{F}}{\send\;{\pair{V}{x}}}}{\plug{\conf{F'}}{\recv\;{x'}}})}}
        \cred
        \tm{\res{x}{x'}{(\ppar{\plug{\conf{F}}{x}}{\plug{\conf{F'}}{\pair{V}{x'}}})}}
      }\]
    There are three possible sources for $\pgv{\tm{\send}}$ and $\pgv{\tm{\recv}}$: $\pcp{\tm{\send{x}{y}{P}}}$ and $\pcp{\tm{\recv{x'}{y'}{Q}}}$; $\pcp{\tm{\inl{x}{P}}}$ and $\pcp{\tm{\offer{x'}{Q}{R}}}$; or $\pcp{\tm{\inr{x}{P}}}$ and $\pcp{\tm{\offer{x'}{Q}{R}}}$.
    \begin{subcase*}[$\pcp{\tm{\send{x}{y}{P}}}$ and $\pcp{\tm{\recv{x'}{y'}{Q}}}$]
      None of the translation rules introduce an evaluation context around the recursive call, hence $\pgv{\tm{\conf{F}}}$ must be $\pgv{\tm{\child\;\letbind{x}{\hole}{\cpgvM{P}}}}$. Similarly, $\pgv{\tm{\conf{F'}}}$ must be $\pgv{\tm{\child\;\letpair{y'}{x'}{\hole}{\cpgvM{Q}}}}$. The value $\pgv{\tm{V}}$ must be an endpoint $\tm{y}$, bound by the name restriction $\pgv{\tm{\res{y}{y'}{}}}$ introduced by the translation. Hence, we have:
      \begin{mathpar}
          \begin{tikzcd}[cramped, column sep=tiny]
            \pcp{\tm{\res{x}{x'}{(\ppar{\send{x}{y}{P}}{\recv{x'}{y'}{Q}})}}}
            \arrow[d, "\cpgvC{\cdot}"]
            \arrow[r, "\pcp{\red}"]
            &
            \pcp{\tm{\res{x}{x'}{\res{y}{y'}{(\ppar{P}{Q})}}}}
            \arrow[dd, "\cpgvC{\cdot}"]
            \\
            \pgv{\tm{\res{x}{x'}{\res{y}{y'}{}}\left(
                  \begin{array}{l}
                    \child\;\letbind{x}{\send\;{\pair{y}{x}}}{\cpgvM{P}}
                    \parallel
                    \\
                    \child\;\letpair{y'}{x'}{\recv\;{x'}}{\cpgvM{Q}}
                  \end{array}
                \right)}}
            \arrow[d, "\pgv{\equiv\cred^+}"]
            \\
            \pgv{\tm{\res{x}{x'}{\res{y}{y'}{(\ppar{\child\;\cpgvM{P}}{\child\;\cpgvM{Q}})}}}}
            \arrow[r, "\pgv{\cred^\star}", "\text{(by \cref{lem:pcp-to-pgv-cpgvM-to-cpgvC})}"']
            &
            \pgv{\tm{\res{x}{x'}{\res{y}{y'}{(\ppar{\cpgvC{P}}{\cpgvC{Q}})}}}}
          \end{tikzcd}
      \end{mathpar}
    \end{subcase*}
    \begin{subcase*}[$\pcp{\tm{\inl{x}{P}}}$ and $\pcp{\tm{\offer{x'}{Q}{R}}}$]
      None of the translation rules introduce an evaluation context around the recursive call, hence $\pgv{\tm{\conf{F}}}$ must be $$\pgv{\tm{\child\;\letbind{x}{\andthen{\close\;\hole}{y}}{\cpgvM{P}}}}.$$ Similarly, $\pgv{\tm{\conf{F'}}}$ must be $$\pgv{\tm{\child\;\letpair{y'}{x'}{\hole}{\andthen{\wait\;x'}{\casesum{y'}{y'}{\cpgvM{Q}}{y'}{\cpgvM{R}}}}}}.$$ Hence, we have:
      \begin{mathpar}
        \begin{tikzcd}[cramped, column sep=tiny]
          \pcp{\tm{\res{x}{x'}{(\ppar{\inl{x}{P}}{\offer{x}{Q}{R}})}}}
          \arrow[d, "\cpgvM{\cdot}"]
          \arrow[r, "\pcp{\red}"]
          &
          \pcp{\tm{\res{x}{x'}{(\ppar{P}{Q})}}}
          \arrow[dd, "\cpgvC{\cdot}"]
          \\
          \pgv{\tm{\res{x}{x'}{}\left(
                \begin{array}{l}
                  \child\;\letbind{x}{\select{\labinl}\;{x}}{\cpgvM{P}}
                  \parallel
                  \\
                  \child\;\offer{x'}{x'}{\cpgvM{Q}}{x'}{\cpgvM{R}}
                \end{array}
              \right)}}
          \arrow[d, "\pgv{\cred^+}"]
          \\ 
          \pgv{\tm{\res{x}{x'}{(\ppar{\child\;\cpgvM{P}}{\child\;\cpgvM{Q}})}}}
          \arrow[r, "\pgv{\cred^\star}", "\text{(by \cref{lem:pcp-to-pgv-cpgvM-to-cpgvC})}"']
          & 
          \pgv{\tm{\res{x}{x'}{(\ppar{\cpgvC{P}}{\cpgvC{Q}})}}}
        \end{tikzcd}
      \end{mathpar}
    \end{subcase*}
    \begin{subcase*}[$\pcp{\tm{\inr{x}{P}}}$ and $\pcp{\tm{\offer{x'}{Q}{R}}}$]
      None of the translation rules introduce an evaluation context around the recursive call, hence $\pgv{\tm{\conf{F}}}$ must be $$\pgv{\tm{\child\;\letbind{x}{\andthen{\close\;\hole}{y}}{\cpgvM{P}}}}.$$ Similarly, $\pgv{\tm{\conf{F'}}}$ must be $$\pgv{\tm{\child\;\letpair{y'}{x'}{\hole}{\andthen{\wait\;x'}{\casesum{y'}{y'}{\cpgvM{Q}}{y'}{\cpgvM{R}}}}}}.$$ Hence, we have:
      \begin{mathpar}
        \begin{tikzcd}[cramped, column sep=tiny]
          \pcp{\tm{\res{x}{x'}{(\ppar{\inr{x}{P}}{\offer{x}{Q}{R}})}}}
          \arrow[d, "\cpgvM{\cdot}"]
          \arrow[r, "\pcp{\red}"]
          &
          \pcp{\tm{\res{x}{x'}{(\ppar{P}{Q})}}}
          \arrow[dd, "\cpgvC{\cdot}"]
          \\
          \pgv{\tm{\res{x}{x'}{}\left(
                \begin{array}{l}
                  \child\;\letbind{x}{\select{\labinr}\;{x}}{\cpgvM{P}}
                  \parallel
                  \\
                  \child\;\offer{x'}{x'}{\cpgvM{Q}}{x'}{\cpgvM{R}}
                \end{array}
              \right)}}
          \arrow[d, "\pgv{\cred^+}"]
          \\ 
          \pgv{\tm{\res{x}{x'}{(\ppar{\child\;\cpgvM{P}}{\child\;\cpgvM{R}})}}}
          \arrow[r, "\pgv{\cred^\star}", "\text{(by \cref{lem:pcp-to-pgv-cpgvM-to-cpgvC})}"']
          & 
          \pgv{\tm{\res{x}{x'}{(\ppar{\cpgvC{P}}{\cpgvC{R}})}}}
        \end{tikzcd}
      \end{mathpar}
    \end{subcase*}
  \end{case*}
  \begin{case*}[\LabTirName{E-Close}]
    \[\pgv{%
        \tm{\res{x}{x'}{(\ppar{\plug{\conf{F}}{\wait\;{x}}}{\plug{\conf{F'}}{\close\;{x'}}})}}
        \cred
        \tm{\ppar{\plug{\conf{F}}{\unit}}{\plug{\conf{F'}}{\unit}}}
      }\]
    The source for $\pgv{\tm{\wait}}$ and $\pgv{\tm{\close}}$ \emph{must} be $\pcp{\tm{\wait{x}{P}}}$ and $\pcp{\tm{\close{x'}{Q}}}$.

    (The translation for $\pcp{\tm{\offer{x}{P}{Q}}}$ also introduces a $\pgv{\tm{\wait}}$, but it is blocked on another communication, and hence cannot be the first communication on a translated term. The translations for $\pcp{\tm{\inl{x}{P}}}$ and $\pcp{\tm{\inr{x}{P}}}$ also introduce a $\pgv{\tm{\close}}$, but these are similarly blocked.)

    None of the translation rules introduce an evaluation context around the recursive call, hence $\pgv{\tm{\conf{F}}}$ must be $\pgv{\tm{\andthen{\hole}{\cpgvM{P}}}}$. Similarly, $\pgv{\tm{\conf{F'}}}$ must be $\pgv{\tm{\andthen{\hole}{\cpgvM{Q}}}}$. Hence, we have:
    \begin{mathpar}
      \begin{tikzcd}
        \pcp{\tm{\res{x}{x'}{(\ppar{\close{x}{P}}{\wait{x'}{Q}})}}} 
        \arrow[d, "\cpgvM{\cdot}"]
        \arrow[r, "\pcp{\red}"]
        &
        \pcp{\tm{\ppar{P}{Q}}}
        \arrow[dd, "\cpgvC{\cdot}"]
        \\
        \pgv{\tm{\res{x}{x'}{(\ppar
              {\child\;\andthen{\close\;{x}}{\cpgvM{P}}} 
              {\child\;\andthen{\wait\;{x'}}{\cpgvM{Q}}}
              )}}}
        \arrow[d, "\pgv{\cred^+}"]
        \\ 
        \pgv{\tm{\ppar{\child\;\cpgvM{P}}{\child\;\cpgvM{Q}}}}
        \arrow[r, "\pgv{\cred^\star}", "\text{(by \cref{lem:pcp-to-pgv-cpgvM-to-cpgvC})}"']
        & 
        \pgv{\tm{\ppar{\cpgvC{P}}{\cpgvC{Q}}}}
      \end{tikzcd}
    \end{mathpar}
  \end{case*}
  \begin{case*}[\LabTirName{E-LiftC}]
    By the induction hypothesis and \LabTirName{E-LiftC}.
  \end{case*}
  \begin{case*}[\LabTirName{E-LiftSC}]
    By the induction hypothesis, \LabTirName{E-LiftSC},
    and \cref{lem:pcp-to-pgv-confs-operational-correspondence-equiv}.
  \end{case*}
\end{proof}

%%% Local Variables:
%%% TeX-master: "../priorities"
%%% End:



%\restatelemma{lempcptopgvcpgvMtocpgvC}
\begin{lemma}%
  \label{lem:pcp-to-pgv-cpgvM-to-cpgvC}
  For any $\pcp{\tm{P}}$, either:
  \begin{itemize}
  \item $\pgv{\tm{\child\;\cpgvM{P}}=\tm{\cpgvC{P}}}$; or
  \item $\pgv{\tm{\child\;\cpgvM{P}}\cred^+\tm{\cpgvC{P}}}$, and for any $\pgv{\tm{\conf{C}}}$, if $\pgv{\tm{\child\;\cpgvM{P}}\cred\tm{\conf{C}}}$, then $\pgv{\tm{\conf{C}}\cred^\star\tm{\cpgvC{P}}}$.
  \end{itemize}
\end{lemma}
\begin{proof}
  \label{prf:lem-pcp-to-pgv-cpgvM-to-cpgvC}
  By induction on the structure of $\pcp{\tm{P}}$.

  \begin{case*}[$\pcp{\tm{\res{x}{y}{P}}}$]
    We have:
    \begin{mathpar}
      \begin{array}{lrll}
        \pcp{\tm{\cpgvM{\res{x}{y}{P}}}}
        & =
        & \pgv{\tm{\child\;\letpair{x}{y}{\new}{\cpgvM{P}}}}
        \\
        & \pgv{\cred^+}
        & \pgv{\tm{\res{x}{y}{(\child\;\cpgvM{P})}}}
        \\
        & \pgv{\cred^\star}
        & \pgv{\tm{\res{x}{y}{\cpgvC{P}}}}
        \\
        & =
        & \pcp{\tm{\cpgvC{\res{x}{y}{P}}}}
      \end{array}
    \end{mathpar}
  \end{case*}
  \begin{case*}[$\pcp{\tm{\ppar{P}{Q}}}$]
    \begin{mathpar}
      \begin{array}{lrll}
        \pcp{\tm{\cpgvM{\ppar{P}{Q}}}}
        & =
        & \pgv{\tm{\child\;\andthen{\spawn\;(\lambda\unit.\cpgvM{P})}{\cpgvM{Q}}}}
        \\
        & \pgv{\cred^+}
        & \pgv{\tm{\ppar{\child\;\cpgvM{P}}{\child\;\cpgvM{Q}}}}
        \\
        & \pgv{\cred^\star}
        & \pgv{\tm{\ppar{\cpgvC{P}}{\cpgvC{Q}}}}
        \\
        & =
        & \pcp{\tm{\cpgvC{\ppar{P}{Q}}}}
      \end{array}
    \end{mathpar}
  \end{case*}
  \begin{case*}[$\pcp{\tm{\send{x}{y}{P}}}$]
    \begin{mathpar}
      \begin{array}{lrll}
        \pcp{\tm{\cpgvM{\send{x}{y}{P}}}}
        & =
        & \pgv{\tm{\letpair{y}{z}{\new}{\letbind{x}{\send\;{\pair{z}{x}}}{\cpgvM{P}}}}}
        \\
        & \pgv{\cred^+}
        & \pgv{\tm{\res{y}{z}{(\child\;\letbind{x}{\send\;\pair{z}{x}}{\cpgvM{P}})}}}
        \\
        & =
        & \pcp{\tm{\cpgvC{\send{x}{y}{P}}}}
      \end{array}
    \end{mathpar}
  \end{case*}
  \begin{case*}[$\pcp{\tm{\inl{x}{P}}}$]
    \begin{mathpar}
      \begin{array}{lrll}
        \pcp{\tm{\cpgvM{\send{x}{y}{P}}}}
        & =
        & \pgv{\tm{\letbind{x}{\select{\labinl}\;{x}}{\cpgvM{P}}}}
        \\
        & \elabarrow
        & \pgv{\tm{\letbind{x}{\letpair{y}{z}{\new}{\andthen{\close\;(\send\;{\pair{\inl{y}}{x}})}{z}}}{\cpgvM{P}}}}
        \\
        & \pgv{\cred^+}
        & \pgv{\tm{\res{y}{z}{(\child\;\letbind{x}{\andthen{\close\;(\send\;\pair{\inl{y}}{x})}{z}}{\cpgvM{P}})}}}
        \\
        & =
        & \pcp{\tm{\cpgvC{\send{x}{y}{P}}}}
      \end{array}
    \end{mathpar}
  \end{case*}
  \begin{case*}[$\pcp{\tm{\inr{x}{P}}}$]
    \begin{mathpar}
      \begin{array}{lrll}
        \pcp{\tm{\cpgvM{\send{x}{y}{P}}}}
        & =
        & \pgv{\tm{\letbind{x}{\select{\labinr}\;{x}}{\cpgvM{P}}}}
        \\
        & \elabarrow
        & \pgv{\tm{\letbind{x}{\letpair{y}{z}{\new}{\andthen{\close\;(\send\;{\pair{\inr{y}}{x}})}{z}}}{\cpgvM{P}}}}
        \\
        & \pgv{\cred^+}
        & \pgv{\tm{\res{y}{z}{(\child\;\letbind{x}{\andthen{\close\;(\send\;\pair{\inr{y}}{x})}{z}}{\cpgvM{P}})}}}
        \\
        & =
        & \pcp{\tm{\cpgvC{\send{x}{y}{P}}}}
      \end{array}
    \end{mathpar}
  \end{case*}
  \begin{case*}[$*$]
    In all other cases, $\pgv{\tm{\child\;\cpgvM{P}}=\tm{\cpgvC{P}}}$.
  \end{case*}
\end{proof}

%%% Local Variables:
%%% TeX-master: "../priorities"
%%% End:


\begin{lemma}%
  \label{lem:pcp-to-pgv-confs-operational-correspondence-equiv}
  If $\pcp{\seq{P}{\ty{\Gamma}}}$ and $\pcp{\tm{P}\equiv\tm{Q}}$,
  then $\pgv{\tm{\cpgvC{P}}\equiv\tm{\cpgvC{Q}}}$.
\end{lemma}
\begin{proof}
  Every axiom of the structural congruence in PCP maps directly to the axiom of the same name in PGV.
\end{proof}

%\restatetheorem{thmpcptopgvoperationalcorrespondencecompleteness}
\begin{theorem}%
  [Operational Correspondence, Completeness, ${\tm{\cpgvC{\cdot}}}$]%
  \label{thm:pcp-to-pgv-operational-correspondence-completeness}
  \hfill\\%newline before theorem statement
  If $\pcp{\seq{P}{\ty{\Gamma}}}$ and $\pcp{\tm{P}\red\tm{Q}}$,
  then $\pgv{\tm{\cpgvC{P}}\cred^+\tm{\cpgvC{Q}}}$.
\end{theorem}
\begin{proof}
  \label{prf:thm-pcp-to-pgv-operational-correspondence-completeness}
  By induction on the derivation of $\pcp{\tm{P}\red\tm{Q}}$.

  \begin{case*}[\LabTirName{E-Link}]
    \begin{mathpar}
      \begin{tikzcd}[cramped]
        \pcp{\tm{\res{x}{x'}{(\ppar{\link{w}{x}}{P})}}}
        \arrow[r, "\pcp{\red}"]
        \arrow[d, "\cpgvC{\cdot}"]
        &
        \pcp{\tm{\subst{P}{w}{x'}}}
        \arrow[dd, "\cpgvC{\cdot}"]
        \\
        \pgv{\tm{\res{x}{x'}{(\ppar{\child\;\link\;\pair{w}{x}}{\cpgvC{P}})}}}
        \arrow[d, "\pgv{\cred^+}"]
        \\
        \pgv{\tm{\subst{\cpgvC{P}}{w}{x'}}}
        \arrow[r, "\pcp{=}"]
        &
        \pgv{\tm{\cpgvC{\subst{P}{w}{x'}}}}
      \end{tikzcd}
    \end{mathpar}
  \end{case*}
  \begin{case*}[\LabTirName{E-Send}]
    \begin{mathpar}
      \begin{tikzcd}
        \pcp{\tm{\res{x}{x'}{(\ppar{\send{x}{y}{P}}{\recv{x'}{y'}{Q}})}}}
        \arrow[d, "\cpgvM{\cdot}"]
        \arrow[r, "\pcp{\red}"]
        &
        \pcp{\tm{\res{x}{x'}{\res{y}{y'}{(\ppar{P}{Q})}}}}
        \arrow[dd, "\cpgvC{\cdot}"]
        \\
        \pgv{\tm{
            \setlength{\arraycolsep}{0pt}
            \res{x}{x'}{}\left(
              \begin{array}{l}
                \child\bigg(
                  \begin{array}{l}
                    \letpair{y}{y'}{\new}{}
                    \\
                    \letbind{x}{\send\;{\pair{y}{x}}}{\cpgvM{P}}
                  \end{array}
                \bigg)
                \parallel
                \\
                \child\hphantom{\bigg(}
                \letpair{y'}{x'}{\recv\;{x'}}{\cpgvM{Q}}
              \end{array}
            \right)}}
        \arrow[d, "\pgv{\cred^+}"]
        \\
        \pgv{\tm{\res{x}{x'}{\res{y}{y'}{(\ppar{\child\;\cpgvM{P}}{\child\;\cpgvM{Q}})}}}}
        \arrow[r, "\pgv{\cred^\star}", "\text{(by \cref{lem:pcp-to-pgv-cpgvM-cpgvC-completeness})}"']
        &
        \pgv{\tm{\res{x}{x'}{\res{y}{y'}{(\ppar{\cpgvC{P}}{\cpgvC{Q}})}}}}
      \end{tikzcd}
    \end{mathpar}
  \end{case*}
  \begin{case*}[\LabTirName{E-Close}]
    \begin{mathpar}
      \begin{tikzcd}
        \pcp{\tm{\res{x}{x'}{(\ppar{\close{x}{P}}{\wait{x'}{Q}})}}} 
        \arrow[d, "\cpgvM{\cdot}"]
        \arrow[r, "\pcp{\red}"]
        &
        \pcp{\tm{\ppar{P}{Q}}}
        \arrow[dd, "\cpgvC{\cdot}"]
        \\
        \pgv{\tm{\res{x}{x'}{(\ppar
              {\child\;\andthen{\close\;{x}}{\cpgvM{P}}} 
              {\child\;\andthen{\wait\;{x'}}{\cpgvM{Q}}}
              )}}}
        \arrow[d, "\pgv{\cred^+}"]
        \\ 
        \pgv{\tm{\ppar{\child\;\cpgvM{P}}{\child\;\cpgvM{Q}}}}
        \arrow[r, "\pgv{\cred^\star}", "\text{(by \cref{lem:pcp-to-pgv-cpgvM-cpgvC-completeness})}"']
        & 
        \pgv{\tm{\ppar{\cpgvC{P}}{\cpgvC{Q}}}}
      \end{tikzcd}
    \end{mathpar}
  \end{case*}
  \begin{case*}[\LabTirName{E-Select-Inl}]
    \begin{mathpar}
      \begin{tikzcd}
        \pcp{\tm{\res{x}{x'}{(\ppar{\inl{x}{P}}{\offer{x}{Q}{R}})}}}
        \arrow[d, "\cpgvM{\cdot}"]
        \arrow[r, "\pcp{\red}"]
        &
        \pcp{\tm{\res{x}{x'}{(\ppar{P}{Q})}}}
        \arrow[dd, "\cpgvC{\cdot}"]
        \\
        \pgv{\tm{\res{x}{x'}{}\left(
              \begin{array}{l}
                \child\;\letbind{x}{\select{\labinl}\;{x}}{\cpgvM{P}}
                \parallel
                \\
                \child\;\offer{x'}{x'}{\cpgvM{Q}}{x'}{\cpgvM{R}}
              \end{array}
            \right)}}
        \arrow[d, "\pgv{\cred^+}"]
        \\ 
        \pgv{\tm{\res{x}{x'}{(\ppar{\child\;\cpgvM{P}}{\child\;\cpgvM{Q}})}}}
        \arrow[r, "\pgv{\cred^\star}", "\text{(by \cref{lem:pcp-to-pgv-cpgvM-cpgvC-completeness})}"']
        & 
        \pgv{\tm{\res{x}{x'}{(\ppar{\cpgvC{P}}{\cpgvC{Q}})}}}
      \end{tikzcd}
    \end{mathpar}
  \end{case*}
  \begin{case*}[\LabTirName{E-Select-Inr}]
    \begin{mathpar}
      \begin{tikzcd}
        \pcp{\tm{\res{x}{x'}{(\ppar{\inr{x}{P}}{\offer{x}{Q}{R}})}}}
        \arrow[d, "\cpgvM{\cdot}"]
        \arrow[r, "\pcp{\red}"]
        &
        \pcp{\tm{\res{x}{x'}{(\ppar{P}{R})}}}
        \arrow[dd, "\cpgvC{\cdot}"]
        \\
        \pgv{\tm{\res{x}{x'}{}\left(
              \begin{array}{l}
                \child\;\letbind{x}{\select{\labinr}\;{x}}{\cpgvM{P}}
                \parallel
                \\
                \child\;\offer{x'}{x'}{\cpgvM{Q}}{x'}{\cpgvM{R}}
              \end{array}
            \right)}}
        \arrow[d, "\pgv{\cred^+}"]
        \\ 
        \pgv{\tm{\res{x}{x'}{(\ppar{\child\;\cpgvM{P}}{\child\;\cpgvM{R}})}}}
        \arrow[r, "\pgv{\cred^\star}", "\text{(by \cref{lem:pcp-to-pgv-cpgvM-cpgvC-completeness})}"']
        & 
        \pgv{\tm{\res{x}{x'}{(\ppar{\cpgvC{P}}{\cpgvC{R}})}}}
      \end{tikzcd}
    \end{mathpar}
  \end{case*}
  \begin{case*}[\LabTirName{E-LiftRes}]
    By the induction hypothesis and \LabTirName{E-LiftC}.
  \end{case*}
  \begin{case*}[\LabTirName{E-LiftPar}]
    By the induction hypotheses and \LabTirName{E-LiftC}.
  \end{case*}
  \begin{case*}[\LabTirName{E-LiftSC}]
    By the induction hypothesis, \LabTirName{E-LiftSC}, and \cref{lem:pcp-to-pgv-confs-operational-correspondence-equiv}.
  \end{case*}
\end{proof}

%%% Local Variables:
%%% TeX-master: "../priorities"
%%% End:

\endgroup

\section{Milner's Cyclic Scheduler}
\label{sec:milner}
As an example of a deadlock-free cyclic process, Dardha and Gay~\cite{dardhagay18extended} introduce an implementation of Milner's cyclic scheduler~\cite{milner89} in Priority CP. We reproduce that scheduler here, and show its translation to Priority GV.

\begingroup
\usingnamespace{pcp}
\begin{example}[Milner's Cyclic Scheduler, PCP]
  \label{ex:pcp-scheduler}
  A set of processes $\tm{\proc{i}}$, $1\leq{i}\leq{n}$, is scheduled to perform some tasks in cyclic order, starting with $\tm{\proc1}$, ending with $\tm{\proc{n}}$, and notifying $\tm{\proc1}$ when all processes have finished.

  Our scheduler $\tm{\sched}$ consists of set of agents $\tm{\agent{i}}$, $1\leq{i}\leq{n}$, each representing their respective process. Each process $\tm{\proc{i}}$ waits for the signal to start their task on $\tm{a'_i}$, and signals completion on $\tm{b'_i}$. Each agent signals their process to start on $\tm{a_i}$, waits for their process to finish on $\tm{b_i}$, and then signals for the next agent to continue on $\tm{c_i}$. The agent $\tm{\agent1}$ initiates, then waits for every other process to finish, and signals $\tm{\proc1}$ on $\tm{d}$. Every other agent $\tm{\agent{i}}$, $2\leq{i}\leq{n}$ waits on $\tm{c'_{i-1}}$ for the signal to start. Each of the channels in the scheduler is of a terminated type, and is merely used to synchronise.

  Below is a diagram of our scheduler instantiated with three processes:
  \begin{center}
    \begin{tikzpicture}
      \node[shape=circle,draw=black]                       (agent1) {\tm{\agent1}};
      \node[shape=circle,draw=black,below right=of agent1] (agent2) {\tm{\agent2}};
      \node[shape=circle,draw=black,below left =of agent1] (agent3) {\tm{\agent3}};

      \node[shape=circle,draw=black,above      =of agent1] (proc1) {\tm{\proc1}};
      \node[shape=circle,draw=black,below right=of agent2] (proc2) {\tm{\proc2}};
      \node[shape=circle,draw=black,below left =of agent3] (proc3) {\tm{\proc3}};

      \draw[->]
      (agent1) -- node[pos=0.35,above] {$c_1$}
      node[pos=1.0,above] {$c'_1$} ++ (agent2);
      \draw[->]
      (agent2) -- node[pos=0.1,above] {$c_2$}
      node[pos=0.9,above] {$c'_2$} ++ (agent3);
      \draw[->]
      (agent3) -- node[pos=0.0,above] {$c_3$}
      node[pos=0.65,above] {$c'_3$} ++ (agent1);

      \draw[implies-implies,double]
      (agent1) -- node[pos=0.15,left] {$a_1$} node[pos=0.85,left] {$a'_1$}
      node[pos=0.15,right] {$b_1$} node[pos=0.85,right] {$b'_1$} ++ (proc1);
      \draw[implies-implies,double]
      (agent2) -- node[pos=0.15,above right] {$a_2$} node[pos=0.85,above right] {$a'_2$}
      node[pos=0.15,below left] {$b_2$} node[pos=0.85,below left] {$b'_2$} ++ (proc2);
      \draw[implies-implies,double]
      (agent3) -- node[pos=0.15,above left] {$a_3$} node[pos=0.85,above left] {$a'_3$}
      node[pos=0.15,below right] {$b_3$} node[pos=0.85,below right] {$b'_3$} ++ (proc3);

      \draw[->]
      (agent1) to[bend left=45] node[pos=0.15,left] {$d$} node[pos=0.85,left] {$d'$} (proc1);

      \draw[->,dashed]
      (proc1.east) to[bend left] node[pos=0.15,align=center,above right] {optional\\data transfer} (proc2.north);
      \draw[->,dashed]
      (proc2.west) to[bend left] (proc3.east);
      \draw[->,dashed]
      (proc3.north) to[bend left] (proc1.west);
    \end{tikzpicture}
  \end{center}

    We implement the scheduler as follows, using $\tm{\prod_{I}P_i}$ to denote the parallel composition of the processes $\tm{P_i}$, $\tm{i}\in\tm{I}$, and $\tm{\plug{P}{Q}}$ to denote the plugging of $\tm{Q}$ in the one-hole process-context $\tm{P}$. The process-contexts $\tm{P_i}$ represent the computations performed by each process $\tm{\proc{i}}$. The process-contexts $\tm{Q_i}$ represent any post-processing, and any possible data transfer from $\tm{\proc{i}}$ to $\tm{\proc{i+1}}$. Finally, $\tm{Q_1}$ should contain $\tm{\labwait{d'}}$.
\[
  \begin{array}{lrlr}
    \tm{\sched}
    & \defeq & \tm{\res{a_1}{a'_1}{\dots{\res{a_n}{a'_n}{}}}}
               \tm{\res{b_1}{b'_1}{\dots{\res{b_n}{b'_n}{}}}}
               \tm{\res{c_1}{c'_1}{\dots{\res{c_n}{c'_n}{}}}}
               \tm{\res{d}{d'}{}}
    \\ &     & \tm{(
               \ppar{\proc1}{\agent1}
               \parallel
               \prod_{2\leq{i}\leq{n}}(\ppar{\proc{i}}{\wait{c'_{i-1}}{\agent{i}}})
               )}
    \\
    \tm{\agent1}
    & \defeq & \tm{\close{a_i}{\wait{b_i}{\close{c_i}{\wait{c'_n}{\close{d}{\halt}}}}}}
    \\
    \tm{\agent{i}}
    & \defeq & \tm{\close{a_i}{\wait{b_i}{\close{c_i}{\halt}}}}
    \\
    \tm{\proc{i}}
    & \defeq & \tm{\wait{a'_i}{\plug{P_i}{\close{b'_i}{Q_i}}}}
  \end{array}
\]
\end{example}
\endgroup

\begingroup
\usingnamespace{pgv}
\begin{example}[Milner's Cyclic Scheduler, PGV]
  \label{ex:pgv-scheduler}
  The PGV scheduler has exactly the same behaviour as the PCP version in~\cref{ex:pcp-scheduler}. It is implemented as follows, using $\tm{\prod_{I}\conf{C}_i}$ to denote the parallel composition of the processes $\tm{\conf{C}_i}$, $\tm{i}\in\tm{I}$, and $\tm{\plug{M}{N}}$ to denote the plugging of $\tm{N}$ in the one-hole term-context $\tm{M}$. For simplicity, we let $\tm{\sched}$ be a configuration. The terms $\tm{M_i}$ represent the computations performed by each process $\tm{\proc{i}}$. The terms $\tm{N_i}$ represent any post-processing, and any possible data transfer from $\tm{\proc{i}}$ to $\tm{\proc{i+1}}$. Finally, $\tm{N_1}$ should contain $\tm{\wait\;{d'}}$.
  \[
    \begin{array}{lrlr}
      \tm{\sched}
      & \defeq & \tm{\res{a_1}{a'_1}{\dots{\res{a_n}{a'_n}{}}}}
                 \tm{\res{b_1}{b'_1}{\dots{\res{b_n}{b'_n}{}}}}
                 \tm{\res{c_1}{c'_1}{\dots{\res{c_n}{c'_n}{}}}}
                 \tm{\res{d}{d'}{}}
      \\ &     & \tm{\begin{array}{lll}
                       (
                       & \phi\;\proc1
                         \parallel
                         \child\;\andthen{\agent1}{\andthen{\wait\;{c'_n}}{\close\;{d}}}
                       \\
                       \parallel
                       &
                         \prod_{2\leq{i}\leq{n}}
                         (\child\;\proc{i} \parallel \child\;\andthen{\wait\;{c'_{i-1}}}{\agent{i}})
                       & )
                     \end{array}}
      \\
      \tm{\agent{i}}
      & \defeq & \tm{\andthen{\close\;{a_i}}{\andthen{\wait\;{b_i}}{\close\;{c_i}}}}
      \\
      \tm{\proc{i}}
      & \defeq & \tm{\andthen{\wait\;{a'_i}}{\plug{M_i}{\andthen{\close\;{b'_i}}{N_i}}}}
    \end{array}
  \]
\end{example}
\endgroup

If $\pcp{\tm{\cpgvM{P_i}}}=\pgv{\tm{M_i}}$ and $\pcp{\tm{\cpgvM{Q_i}}}=\pgv{\tm{N_i}}$, then the translation of $\pcp{\tm{\sched}}$ (\cref{ex:pcp-scheduler}), $\pcp{\tm{\cpgvC{\sched\,}}}$, is exactly $\pgv{\tm{\sched}}$ (\cref{ex:pgv-scheduler}).


%%% Local Variables:
%%% TeX-master: "priorities"
%%% End:
