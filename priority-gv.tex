\documentclass[main.tex]{subfiles}

\begin{document}

\section{Priority GV}\label{sec:pgv}
\usingnamespace{pgv}

We present Priority GV (PGV), a session-typed functional language based on GV~\cite{wadler15,lindleymorris15} which uses priorities \`{a} la Kobayashi and Padovani~\cite{kobayashi06,padovaninovara15} to enforce deadlock freedom.
Priority GV offers a more fine-grained analysis of communication structures, and by separating channel creation form thread spawning it allows cyclic structures.
We illustrate this with two programs in PGV, \cref{ex:cycle,ex:deadlock}. Each program contains two processes---the main process, and the child process created by $\tm{\spawn}$---which communicate using \emph{two} channels. The child process receives a unit over the channel $\tm{x}/\tm{x'}$, and then sends a unit over the channel $\tm{y}/\tm{y'}$. The main process does one of two things:
\begin{enumerate*}[label=(\alph*)]
\item in \cref{ex:cycle}, it sends a unit over the channel $\tm{x}/\tm{x'}$, and then waits to receive a unit over the channel $\tm{y}/\tm{y'}$;
\item in \cref{ex:deadlock}, it does these in the opposite order, which results in a deadlock.
\end{enumerate*}
We will show that only \cref{ex:cycle} is typeable in PGV, while it is not typable in GV~\cite{wadler14} nor guaranteed to be deadlock-free by its predecessor~\cite{gayvasconcelos12}.

\begin{center}
\begin{minipage}{0.5\linewidth}
\begin{example}[Cyclic Structure]
  \label{ex:cycle}
  \[\tm{%
    \begin{array}{l}
      \letpair{x}{x'}{\new}{}
      \\
      \letpair{y}{y'}{\new}{}
      \\
      \spawn\left(%
      \begin{array}{l}
        {\letpair{\unit}{x'}{\recv\;{x'}}{}}
        \\
        {\letbind{y}{\send\;{\pair{\unit}{y}}}{}}
        \\
        {\andthen{\wait\;{x'}}{\close\;{y}}}
      \end{array}%
      \right);
      \\
      \underline{\letbind{x}{\send\;{\pair{\unit}{x}}}{}}
      \\
      \underline{\letpair{\unit}{y'}{\recv\;{y'}}{}}
      \\
      {\andthen{\close\;{x}}{\wait\;{y'}}}
    \end{array}
  }\]
\end{example}
\end{minipage}%
\begin{minipage}{0.5\linewidth}
\begin{example}[Deadlock]
  \label{ex:deadlock}
  \[\tm{%
    \begin{array}{l}
      \letpair{x}{x'}{\new}{}
      \\
      \letpair{y}{y'}{\new}{}
      \\
      \spawn\left(%
      \begin{array}{l}
        {\letpair{\unit}{x'}{\recv\;{x'}}{}}
        \\
        {\letbind{y}{\send\;{\pair{\unit}{y}}}{}}
        \\
        {\andthen{\wait\;{x'}}{\close\;{y}}}
      \end{array}%
      \right);
      \\
      \underline{\letpair{\unit}{y'}{\recv\;{y'}}{}}
      \\
      \underline{\letbind{x}{\send\;{\pair{\unit}{x}}}{}}
      \\
      {\andthen{\close\;{x}}{\wait\;{y'}}}
    \end{array}
   }\]
\end{example}
\end{minipage}
\end{center}

\subsubsection*{Session types}
Session types ($\ty{S}$) are defined by the following grammar:
\[
\begin{array}{lcl}
  \ty{S}
  & \Coloneqq & \ty{\tysend[\cs{o}]{T}{S}}
    \sep        \ty{\tyrecv[\cs{o}]{T}{S}}
    \sep        \ty{\tyends[\cs{o}]}
    \sep        \ty{\tyendr[\cs{o}]}
\end{array}
\]
Session types $\ty{\tysend[\cs{o}]{T}{S}}$ and $\ty{\tyrecv[\cs{o}]{T}{S}}$ describe the endpoints of a channel over which we send or receive a value of type $\ty{T}$, and then proceed as $\ty{S}$. Types $\ty{\tyends[\cs{o}]}$ and $\ty{\tyendr[\cs{o}]}$ describe endpoints of a channel whose communication has finished, and over which we must synchronise before closing the channel. Each connective in a session type is annotated with a \emph{priority} $\cs{o}\in\mathbb{N}$.

\subsubsection*{Types}
Types ($\ty{T}$, $\ty{U}$) are defined by the following grammar:
\[
\begin{array}{lcl}
  \ty{T}, \ty{U}
  & \Coloneqq & \ty{\typrod{T}{U}}
    \sep        \ty{\tyunit}
    \sep        \ty{\tysum{T}{U}}
    \sep        \ty{\tyvoid}
    \sep        \ty{\tylolli[\cs{p},\cs{q}]{T}{U}}
    \sep        \ty{S}
\end{array}
\]
Types $\ty{\typrod{T}{U}}$, $\ty{\tyunit}$, $\ty{\tysum{T}{U}}$, and $\ty{\tyvoid}$ are the standard linear $\lambda$-calculus product type, unit type, sum type, and empty type.
Type $\ty{\tylolli[\cs{p},\cs{q}]{T}{U}}$ is the standard linear function type, annotated with \emph{priority bounds} $\cs{p},\cs{q}\in\mathbb{N}\cup\{\cs{\pbot},\cs{\ptop}\}$.
Every session type is also a type.
Given a function with type $\ty{\tylolli[\cs{p},\cs{q}]{T}{U}}$, $\cs{p}$ is a \emph{lower bound} on the priorities of the endpoints captured by the body of the function, and $\cs{q}$ is an \emph{upper bound} on the priority of the communications that take place as a result of applying the function. The type of \emph{pure functions} $\ty{\tylolli{T}{U}}$, \ie those which perform no communications, is syntactic sugar for $\ty{\tylolli[\cs{\ptop},\cs{\pbot}]{T}{U}}$.

\subsubsection*{Environments}
Typing environments $\ty{\Gamma}$, $\ty{\Delta}$ associate types to names. Environments are linear, so two environments can only be combined as $\ty{\Gamma}, \ty{\Delta}$ if their names are distinct, \ie $\fv(\ty{\Gamma})\cap\fv(\ty{\Delta})=\varnothing$.
\[
\begin{array}{lcl}
  \ty{\Gamma}, \ty{\Delta}
  & \Coloneqq & \ty{\emptyenv}
    \sep        \ty{\Gamma}, \tmty{x}{T}
\end{array}
\]

\subsubsection*{Duality}
Duality plays a crucial role in session types. The two endpoints of a channel are assigned dual types, ensuring that, for instance, whenever one program \emph{sends} a value on a channel, the program on the other end is waiting to \emph{receive}. Each session type $\ty{S}$ has a dual, written $\ty{\co{S}}$. Duality is an involutive function which \emph{preserves priorities}:
\[
  \ty{\co{\tysend[\cs{o}]{T}{S}}} = \ty{\tyrecv[\cs{o}]{T}{\co{S}}}
  \qquad
  \ty{\co{\tyrecv[\cs{o}]{T}{S}}} = \ty{\tysend[\cs{o}]{T}{\co{S}}}
  \qquad
  \ty{\co{\tyends[\cs{o}]}} = \ty{\tyendr[\cs{o}]}
  \qquad
  \ty{\co{\tyendr[\cs{o}]}} = \ty{\tyends[\cs{o}]}
\]

\subsubsection*{Priorities}
Function $\pr(\cdot)$ returns the smallest priority of a session type. The type system guarantees that the top-most connective always holds the smallest priority, so we simply return the priority of the top-most connective:
\[
  \pr(\ty{\tysend[\cs{o}]{T}{S}}) = \cs{o}
  \qquad
  \pr(\ty{\tyrecv[\cs{o}]{T}{S}}) = \cs{o}
  \qquad
  \pr(\ty{\tyends[\cs{o}]})       = \cs{o}
  \qquad
  \pr(\ty{\tyendr[\cs{o}]})       = \cs{o}
\]
We extend the function $\pr(\cdot)$ to types and typing contexts by returning the smallest priority in the type or context, or $\cs{\top}$ if there is no priority. We use $\sqcap$ and $\sqcup$ to denote the minimum and maximum:
\[
\begin{array}{lcl}
  \minpr(\ty{\typrod{T}{U}})                 & = & \minpr({\ty{T}})\sqcap\minpr({\ty{U}}) \\
  \minpr(\ty{\tysum{T}{U}})                  & = & \minpr({\ty{T}})\sqcap\minpr({\ty{U}}) \\
  \minpr(\ty{\tylolli[\cs{p},\cs{q}]{T}{U}}) & = & \cs{p} \\
  \minpr(\ty{\Gamma}, \tmty{x}{A})           & = & \minpr(\ty{\Gamma})\sqcap\minpr(\ty{A})
\end{array}
\qquad
\begin{array}{lcl}
  \minpr(\ty{\tyunit})                       & = & \ptop \\
  \minpr(\ty{\tyvoid})                       & = & \ptop \\
  \minpr(\ty{S})                             & = & \pr(\ty{S}) \\
  \minpr(\ty{\emptyenv})                     & = & \ptop
\end{array}
\]

\subsubsection*{Terms}
Terms ($\tm{L}$, $\tm{M}$, $\tm{N}$) are defined by the following grammar:
\[
\begin{array}{lcl}
  \tm{L}, \tm{M}, \tm{N}
  & \Coloneqq & \tm{x}
    \sep        \tm{K}
    \sep        \tm{\lambda x.M}
    \sep        \tm{M\;N} \\
  & \sep      & \tm{\unit}
    \sep        \tm{\andthen{M}{N}}
    \sep        \tm{\pair{M}{N}}
    \sep        \tm{\letpair{x}{y}{M}{N}} \\
  & \sep      & \tm{\inl{M}}
    \sep        \tm{\inr{M}}
    \sep        \tm{\casesum{L}{x}{M}{y}{N}}
    \sep        \tm{\absurd{M}} \\
  \tm{K}
  & \Coloneqq & \tm{\link}
    \sep        \tm{\new}
    \sep        \tm{\spawn}
    \sep        \tm{\send}
    \sep        \tm{\recv}
    \sep        \tm{\close}
    \sep        \tm{\wait}
\end{array}
\]
Let $\tm{x}$, $\tm{y}$, $\tm{z}$, and $\tm{w}$ range over variable names. Occasionally, we use $\tm{a}$, $\tm{b}$, $\tm{c}$, and $\tm{d}$. The term language is the standard linear $\lambda$-calculus with products, sums, and their units, extended with constants $\tm{K}$ for the communication primitives.

Constants are best understood in conjunction with their typing and reduction rules in~\cref{fig:pgv-typing,fig:pgv-operational-semantics}.
Briefly, $\tm{\link}$ links two endpoints together, forwarding messages from one to the other, $\tm{\new}$ creates a new channel and returns a pair of its endpoints, and $\tm{\spawn}$ spawns off its argument as a new thread.
The $\tm{\send}$ and $\tm{\recv}$ functions send and receive values on a channel.
However, since the typing rules for PGV ensure the linear usage of endpoints, they also return a new copy of the endpoint to continue the session.
The $\tm{\close}$ and $\tm{\wait}$ functions close a channel.
We use syntactic sugar to make terms more readable: we write $\tm{\letbind{x}{M}{N}}$ in place of $\tm{(\lambda x.N)\;M}$, $\tm{\lambda\unit.M}$ in place of $\tm{\lambda z.\andthen{z}{M}}$, and $\tm{\lambda\pair{x}{y}.M}$ in place of $\tm{\lambda z.\letpair{x}{y}{z}{M}}$.
We recover $\tm{\fork}$ as $\tm{\lambda x.\letpair{y}{z}{\new\;\unit}{\andthen{\spawn\;{(\lambda\unit.x\;y)}}{z}}}$.

\subsubsection*{Internal and External Choice}
Typically, session-typed languages feature constructs for internal and external choice. In GV, these can be defined in terms of the core language, by sending or receiving a value of a sum type~\cite{lindleymorris15}. We use the following syntactic sugar for internal ($\ty{\tyselect[\cs{o}]{S}{S'}}$) and external ($\ty{\tyoffer[\cs{o}]{S}{S'}}$) choice and their units:
\[
\begin{array}{lcl}
  \ty{\tyselect[\cs{o}]{S}{S'}}
  & \elabarrow & \ty{\tysend[\cs{o}]{(\tysum{\co{S}}{\co{S'}})}{\tyends[\cs{o+1}]}} \\
  \ty{\tyoffer[\cs{o}]{S}{S'}}
  & \elabarrow & \ty{\tyrecv[\cs{o}]{(\tysum{S}{S'})}{\tyendr[\cs{o+1}]}} \\
\end{array}
\qquad
\begin{array}{lcl}
  \ty{\tyselectemp[\cs{o}]}
  & \elabarrow & \ty{\tysend[\cs{o}]{\tyvoid}{\tyends[\cs{o+1}]}} \\
  \ty{\tyofferemp[\cs{o}]}
  & \elabarrow & \ty{\tyrecv[\cs{o}]{\tyvoid}{\tyendr[\cs{o+1}]}} \\
\end{array}
\]
As the syntax for units suggests, these are the binary and nullary forms of the more common n-ary choice constructs $\ty{{\oplus}^{\cs{o}}\{l_i:S_i\}_{i\in{I}}}$ and $\ty{{\with}^{\cs{o}}\{l_i:S_i\}_{i\in{I}}}$, which one may obtain generalising the sum types to variant types. For simplicity, we present only the binary and nullary forms.

Similarly, we use syntactic sugar for the term forms of choice, which combine sending and receiving with the introduction and elimination forms for the sum and empty types. There are two constructs for binary internal choice, expressed using the meta-variable $\tm{\ell}$ which ranges over $\{\tm{\labinl},\tm{\labinr}\}$. As there is no introduction for the empty type, there is no construct for nullary internal choice:
\[
\begin{array}{l}
  \tm{\select{\ell}}\elabarrow \\
  \qquad\tm{\lambda x.\letpair{y}{z}{\new}{\andthen{\close\;(\send\;{\pair{\ell\;y}{x}})}{z}}} \\
  \tm{\offer{L}{x}{M}{y}{N}}\elabarrow \\
  \qquad\tm{\letpair{z}{w}{\recv\;{L}}{\andthen{\wait\;{w}}{\casesum{z}{x}{M}{y}{N}}}} \\
  \tm{\offeremp{L}}\elabarrow \\
  \qquad\tm{\letpair{z}{w}{\recv\;L}{\andthen{\wait\;{w}}{\absurd{z}}}}
\end{array}
\]

\subsubsection*{Operational Semantics}
Priority GV terms are evaluated as part of a configuration of processes. Configurations are defined by the following grammar:
\[
\begin{array}[t]{lcl}
  \tm{\phi}
  & \Coloneqq & \tm{\main}
    \sep        \tm{\child}
  \\
  \tm{\conf{C}}, \tm{\conf{D}}, \tm{\conf{E}}
  & \Coloneqq & \tm{\phi\;M}
    \sep        \tm{\ppar{\conf{C}}{\conf{D}}}
    \sep        \tm{\res{x}{x'}{\conf{C}}}
\end{array}
\]
Configurations ($\tm{\conf{C}}$, $\tm{\conf{D}}$, $\tm{\conf{E}}$) consist of threads $\tm{\phi\;M}$, parallel compositions $\tm{\ppar{\conf{C}}{\conf{D}}}$, and name restrictions $\tm{\res{x}{x'}{\conf{C}}}$. To preserve the functional nature of PGV, where programs return a single value, we use flags ($\tm{\phi}$) to differentiate between the main thread, marked $\tm{\main}$, and child threads created by $\tm{\spawn}$, marked $\tm{\child}$. Only the main thread returns a value. We determine the flag of a configuration by combining the flags of all threads in that configuration:
\[
  \tm{\main}  + \tm{\child} = \tm{\main}
  \quad
  \tm{\child} + \tm{\main}  = \tm{\main}
  \quad
  \tm{\child} + \tm{\child} = \tm{\child}
  \quad
  (\tm{\main}  + \tm{\main} \; \text{is undefined})
\]
The use of $\tm{\child}$ for child threads~\cite{lindleymorris15} overlaps with the use of the meta-variable $\cs{o}$ for priorities~\cite{dardhagay18}. Both are used to annotate sequents: flags appear on the sequent in configuration typing, and priorities in term typing. To distinguish the two symbols, they are typeset in a different font and a different colour.

Values ($\tm{V}$, $\tm{W}$), evaluation contexts ($\tm{E}$), thread evaluation contexts ($\tm{\conf{F}}$), and configuration contexts ($\tm{\conf{G}}$) are defined by the following grammars:
\[
\begin{array}[t]{lcl}
  \tm{V}, \tm{W}
  & \Coloneqq & \tm{x}
    \sep        \tm{K}
    \sep        \tm{\lambda x.M}
    \sep        \tm{\unit}
    \sep        \tm{\pair{V}{W}}
    \sep        \tm{\inl{V}}
    \sep        \tm{\inr{V}} \\
  \tm{E}
  & \Coloneqq & \tm{\hole}
    \sep        \tm{E\;M}
    \sep        \tm{V\;E} \\
  & \sep      & \tm{\andthen{E}{N}}
    \sep        \tm{\pair{E}{M}}
    \sep        \tm{\pair{V}{E}}
    \sep        \tm{\letpair{x}{y}{E}{M}} \\
  & \sep      & \tm{\inl{E}}
    \sep        \tm{\inr{E}}
    \sep        \tm{\casesum{E}{x}{M}{y}{N}}
    \sep        \tm{\absurd{E}} \\
  \tm{\conf{F}}
  & \Coloneqq & \tm{\phi\;E}
  \\
  \tm{\conf{G}}
  & \Coloneqq & \tm{\hole}
    \sep        \tm{\ppar{\conf{G}}{\conf{C}}}
    \sep        \tm{\res{x}{y}{\conf{G}}}
\end{array}
\]

\begin{figure}[h]
  \paragraph*{Term reduction}
  \begin{mathpar}
    \begin{array}{llcl}
      \LabTirName{E-Lam}  & \tm{(\lambda x.M) \; V}
                          & \tred & \tm{\subst{M}{V}{x}}
      \\
      \LabTirName{E-Unit} & \tm{\letunit{\unit}{M}}
                          & \tred & \tm{M}
      \\
      \LabTirName{E-Pair} & \tm{\letpair{x}{y}{\pair{V}{W}}{M}}
                          & \tred & \tm{\subst{\subst{M}{V}{x}}{W}{y}}
      \\
      \LabTirName{E-Inl}  & \tm{\casesum{\inl{V}}{x}{M}{y}{N}}
                          & \tred & \tm{\subst{M}{V}{x}}
      \\
      \LabTirName{E-Inr}  & \tm{\casesum{\inr{V}}{x}{M}{y}{N}}
                          & \tred & \tm{\subst{N}{V}{y}}
    \end{array}
    \\
    \inferrule*[lab=E-Lift]{
      \tm{M}\tred\tm{M'}
    }{\tm{\plug{E}{M}}\tred\tm{\plug{E}{M'}}}
  \end{mathpar}
  \\
  \paragraph*{Structural congruence}
  \begin{mathpar}
    \begin{array}{llcl}
      \LabTirName{SC-LinkSwap}   & \tm{\plug{\conf{F}}{\link\;{\pair{x}{y}}}}
                                 & \equiv & \tm{\plug{\conf{F}}{\link\;{\pair{y}{x}}}}
      \\
      \LabTirName{SC-ResLink}    & \tm{\res{x}{y}{(\phi\;\link\;\pair{x}{y})}}
                                 & \equiv & \tm{\phi\;\unit}
      \\
      \LabTirName{SC-ResSwap}    & \tm{\res{x}{y}{\conf{C}}}
                                 & \equiv & \tm{\res{y}{x}{\conf{C}}}
      \\
      \LabTirName{SC-ResComm}    & \tm{\res{x}{y}{\res{z}{w}{\conf{C}}}}
                                 & \equiv & \tm{\res{z}{w}{\res{x}{y}{\conf{C}}}}
      \\
      \LabTirName{SC-ResExt}     & \tm{\res{x}{y}{(\ppar{\conf{C}}{\conf{D}}})}
                                 & \equiv & \tm{\ppar{\conf{C}}{\res{x}{y}{\conf{D}}}},
                                            \text{ if }{\tm{x},\tm{y}\notin\fv(\tm{\conf{C}})}
      \\
      \LabTirName{SC-ParNil}     & \tm{\ppar{\conf{C}}{\child{\unit}}}
                                 & \equiv & \tm{\conf{C}}
      \\
      \LabTirName{SC-ParComm}    & \tm{\ppar{\conf{C}}{\conf{D}}}
                                 & \equiv & \tm{\ppar{\conf{D}}{\conf{C}}}
      \\
      \LabTirName{SC-ParAssoc}   & \tm{\ppar{\conf{C}}{(\ppar{\conf{D}}{\conf{E}})}}
                                 & \equiv & \tm{\ppar{(\ppar{\conf{C}}{\conf{D}})}{\conf{E}}}
    \end{array}
  \end{mathpar}
  \\
  \paragraph*{Configuration reduction}
  \begin{mathpar}
    \begin{array}{llcl}
      \LabTirName{E-Link}  & \tm{\res{x}{y}{(\ppar{\plug{\conf{F}}{\link\;\pair{w}{x}}}{\conf{C}})}}
                           & \cred & \tm{\ppar{\plug{\conf{F}}{\unit}}{\subst{\conf{C}}{w}{y}}}
      \\
      \LabTirName{E-New}   & \tm{\plug{\conf{F}}{\new\;\unit}}
                           & \cred & \tm{\res{x}{y}{(\plug{\conf{F}}{\pair{x}{y}})}}
      \\
      \LabTirName{E-Spawn} & \tm{\plug{\conf{F}}{\spawn\;V}}
                           & \cred & \tm{\ppar{\plug{\conf{F}}{\unit}}{\child\;V\;\unit}}
      \\
      \LabTirName{E-Send}  & \tm{\res{x}{y}{(\ppar
                             {\plug{\conf{F}}{\send\;{\pair{V}{x}}}}
                             {\plug{\conf{F'}}{\recv\;{y}}})}}
                           & \cred & \tm{\res{x}{y}{(\ppar
                                     {\plug{\conf{F}}{x}}
                                     {\plug{\conf{F'}}{\pair{V}{y}}})}}
      \\
      \LabTirName{E-Close} & \tm{\res{x}{y}{(\ppar
                             {\plug{\conf{F}}{\wait\;{x}}}
                             {\plug{\conf{F'}}{\close\;{y}}})}}
                           & \cred & \tm{\ppar{\plug{\conf{F}}{\unit}}{\plug{\conf{F'}}{\unit}}}
    \end{array}
    \\
    \inferrule*[lab=E-LiftC]{
      \tm{\conf{C}}\cred\tm{\conf{C'}}
    }{\tm{\plug{\conf{G}}{\conf{C}}}\cred\tm{\plug{\conf{G}}{\conf{C'}}}}
  
    \inferrule*[lab=E-LiftM]{
      \tm{M}\tred\tm{M'}
    }{\tm{\plug{\conf{F}}{M}}\tred\tm{\plug{\conf{F}}{M'}}}
  
    \inferrule*[lab=E-LiftSC]{
      \tm{\conf{C}}\equiv\tm{\conf{C'}}
      \\
      \tm{\conf{C'}}\cred\tm{\conf{D'}}
      \\
      \tm{\conf{D'}}\equiv\tm{\conf{D}}
    }{\tm{\conf{C}}\cred\tm{\conf{D}}}
  \end{mathpar}
  \caption{Operational Semantics for PGV.}
  \label{fig:pgv-operational-semantics}
\end{figure}

%%% Local Variables:
%%% TeX-master: "../priorities"
%%% End:


We factor the reduction relation of PGV into a deterministic reduction on terms ($\tred$) and a non-deterministic reduction on configurations ($\cred$), see \cref{fig:pgv-operational-semantics}. We write $\tred^+$ and $\cred^+$ for the transitive closures, and $\tred^\star$ and $\cred^\star$ for the reflexive-transitive closures.

Term reduction is the standard call-by-value, left-to-right evaluation for GV, and only deviates from reduction for the linear $\lambda$-calculus in that it reduces terms to values \emph{or} ready terms waiting to perform a communication action.

Configuration reduction resembles evaluation for a process calculus: \LabTirName{E-Link}, \LabTirName{E-Send}, and \LabTirName{E-Close} perform communications, \LabTirName{E-LiftC} allows reduction under configuration contexts, and \LabTirName{E-LiftSC} embeds a structural congruence $\equiv$. The remaining rules mediate between the process calculus and the functional language: \LabTirName{E-New} and \LabTirName{E-Spawn} evaluate the $\tm{\new}$ and $\tm{\spawn}$ constructs, creating the equivalent configuration constructs, and \LabTirName{E-LiftM} embeds term reduction.

Structural congruence satisfies the following axioms: $\LabTirName{SC-LinkSwap}$ allows swapping channels in the link process. $\LabTirName{SC-ResLink}$ allows restriction to applied to link which is structurally equivalent to the terminated process, thus allowing elimination of unnecessary restrictions. $\LabTirName{SC-ResSwap}$ allows swapping channels and $\LabTirName{SC-ResComm}$ states that restriction is commutative. $\LabTirName{SC-ResExt}$ is the standard scope extrusion rule. Rules $\LabTirName{SC-ParNil}$, $\LabTirName{SC-ParComm}$ and $\LabTirName{SC-ParAssoc}$ state that parallel composition uses the terminated process as the neutral element; it is commutative and associative.

While our configuration reduction is based on the standard evaluation for GV, the increased expressiveness of PGV allows us to simplify the relation on two counts.
\begin{enumerate*}[label=(\alph*)]
\item
\emph{We decompose the $\tm{\fork}$ construct}.
In GV, $\tm{\fork}$ creates a new channel, spawns a child thread, and, when the child thread finishes, it closes the channel to its parent. In PGV, these are three separate operations: $\tm{\new}$, $\tm{\spawn}$, and $\tm{\close}$. We no longer require that every child thread finishes by returning a terminated channel. Consequently, we also simplify the evaluation of the $\tm{\link}$ construct.
Intuitively, evaluating $\tm{\link}$ causes a substitution: if we have a channel bound as $\tm{\res{x}{y}{}}$, then $\tm{\link\;\pair{w}{x}}$ replaces all occurrences of $\tm{y}$ by $\tm{w}$. However, in GV, $\tm{\link}$ is required to return a terminated channel, which means that the semantics for $\tm{link}$ must \emph{create} a fresh channel of type $\ty{\tyends}/\ty{\tyendr}$. The endpoint of type $\ty{\tyends}$ is returned by the $\tm{link}$ construct, and a $\tm{\wait}$ on the other endpoint guards the \emph{actual} substitution. In PGV, evaluating $\tm{\link}$ simply causes a substitution.
\item
\emph{Our structural congruence is type preserving}. Consequently, we can embed it directly into the reduction relation. In GV, this is not the case, and subject reduction relies on proving that if $\equiv\cred$ ends up in an ill-typed configuration, we can rewrite it to a well-typed configuration using $\equiv$.
\end{enumerate*}


\subsubsection*{Typing}
\Cref{fig:pgv-typing} gives the typing rules for PGV.
Typing rules for terms are at the top of \cref{fig:pgv-typing}. Terms are typed by a judgement $\tseq[\cs{p}]{\ty{\Gamma}}{M}{T}$ stating that ``a term $\tm{M}$ has type $\ty{T}$ and an upper bound on its priority $\cs{p}$ under the typing environment $\ty{\Gamma}$''. Typing for the linear $\lambda$-calculus is standard. Linearity is ensured by splitting environments on branching rules, requiring that the environment in the variable rule consists of just the variable, and the environment in the constant and unit rules are empty. Constants $\tm{K}$ are typed using type schemas, and embedded using \LabTirName{T-Const} (mid of \cref{fig:pgv-typing}). The typing rules treat \emph{all variables} as linear resources, even those of non-linear types such as $\ty{\tyunit}$. However, the rules can easily be extended to allow values with unrestricted usage~\cite{wadler14}.

\begin{figure}
  \begin{mdframed}
    \textbf{Static Typing Rules.}
    \begin{mathpar}
      \inferrule*[lab=T-Var]{
      }{\tseq[\cs{\pbot}]{\tmty{x}{T}}{x}{T}}
      
      \inferrule*[lab=T-Lam]{
        \tseq[\cs{q}]{\ty{\Gamma},\tmty{x}{T}}{M}{U}
      }{\tseq[\cs{\pbot}]{\ty{\Gamma}}{\lambda x.M}{\tylolli[\cs{\minpr(\ty{\Gamma})},\cs{q}]{T}{U}}}
      
      \inferrule*[lab=T-Const]{
      }{\tseq[\cs{\pbot}]{\emptyenv}{K}{T}}
      
      \inferrule*[lab=T-App]{
        \tseq[\cs{p}]{\ty{\Gamma}}{M}{\tylolli[\cs{p'},\cs{q'}]{T}{U}}
        \\
        \tseq[\cs{q}]{\ty{\Delta}}{N}{T}
        \\
        \cs{p}<\minpr(\ty{\Delta})
        \\
        \cs{q}<\cs{p'}
      }{\tseq[\cs{p}\sqcup\cs{q}\sqcup\cs{q'}]{\ty{\Gamma},\ty{\Delta}}{M\;N}{U}}
      \\
      \inferrule*[lab=T-Unit]{
      }{\tseq[\cs{\pbot}]{\emptyenv}{\unit}{\tyunit}}
      
      \inferrule*[lab=T-LetUnit]{
        \tseq[\cs{p}]{\ty{\Gamma}}{M}{\tyunit}
        \\
        \tseq[\cs{q}]{\ty{\Delta}}{N}{T}
        \\
        \cs{p}<\minpr(\ty{\Delta})
      }{\tseq[\cs{p}\sqcup\cs{q}]{\ty{\Gamma},\ty{\Delta}}{\andthen{M}{N}}{T}}
      
      \inferrule*[lab=T-Pair]{
        \tseq[\cs{p}]{\ty{\Gamma}}{M}{T}
        \\
        \tseq[\cs{q}]{\ty{\Delta}}{N}{U}
        \\
        \cs{p}<\minpr(\ty{\Delta})
      }{\tseq[\cs{p}\sqcup\cs{q}]{\ty{\Gamma},\ty{\Delta}}{\pair{M}{N}}{\typrod{T}{U}}}
      
      \inferrule*[lab=T-LetPair]{
        \tseq[\cs{p}]{\ty{\Gamma}}{M}{\typrod{T}{T'}}
        \\
        \tseq[\cs{q}]{\ty{\Delta},\tmty{x}{T},\tmty{y}{T'}}{N}{U}
        \\
        \cs{p}<\minpr(\ty{\Delta},\ty{T},\ty{T'})
      }{\tseq[\cs{p}\sqcup\cs{q}]{\ty{\Gamma},\ty{\Delta}}{\letpair{x}{y}{M}{N}}{U}}
      
      \inferrule*[lab=T-Inl]{
        \tseq[\cs{p}]{\ty{\Gamma}}{M}{T}
        \\
        \minpr(\ty{T})=\minpr(\ty{U})
      }{\tseq[\cs{p}]{\ty{\Gamma}}{\inl{M}}{\tysum{T}{U}}}
      
      \inferrule*[lab=T-Inr]{
        \tseq[\cs{p}]{\ty{\Gamma}}{M}{U}
        \\
        \minpr(\ty{T})=\minpr(\ty{U})
      }{\tseq[\cs{p}]{\ty{\Gamma}}{\inr{M}}{\tysum{T}{U}}}
      
      \inferrule*[lab=T-CaseSum]{
        \tseq[\cs{p}]{\ty{\Gamma}}{L}{\tysum{T}{T'}}
        \\
        \tseq[\cs{q}]{\ty{\Delta},\tmty{x}{T}}{M}{U}
        \\
        \tseq[\cs{q}]{\ty{\Delta},\tmty{y}{T'}}{N}{U}
        \\
        \cs{p}<\minpr(\ty{\Delta})
      }{\tseq[\cs{p}\sqcup\cs{q}]{\ty{\Gamma},\ty{\Delta}}{\casesum{L}{x}{M}{y}{N}}{U}}
      
      \inferrule*[lab=T-Absurd]{
        \tseq[\cs{p}]{\ty{\Gamma}}{M}{\tyvoid}
      }{\tseq[\cs{p}]{\ty{\Gamma},\ty{\Delta}}{\absurd{M}}{T}}
    \end{mathpar}
    \begin{center}
      We write $\tmty{K}{T}$ for $\tseq[\cs{\pbot}]{\emptyenv}{K}{T}$ in typing derivations, \\
      and omit $\ty{T}$ when it can be inferred from context.
    \end{center}    
    \textbf{Type Schemas for Constants.}
    \begin{mathpar}
      \tmty{\link}{\tylolli{\typrod{S}{\co{S}}}{\tyunit}}
      
      \tmty{\new}{\tylolli{\tyunit}{\typrod{S}{\co{S}}}}
      
      \tmty{\spawn}{\tylolli{(\tylolli[\cs{p},\cs{q}]{\tyunit}{\tyunit})}{\tyunit}}
      \\
      \tmty{\send}{\tylolli[\cs{\ptop},\cs{o}]{\typrod{T}{\tysend[\cs{o}]{T}{S}}}{S}}
      
      \tmty{\recv}{\tylolli[\cs{\ptop},\cs{o}]{\tyrecv[\cs{o}]{T}{S}}{\typrod{T}{S}}}
      \\
      \tmty{\close}{\tylolli[\cs{\ptop},\cs{o}]{\tyends[\cs{o}]}{\tyunit}}
      
      \tmty{\wait}{\tylolli[\cs{\ptop},\cs{o}]{\tyendr[\cs{o}]}{\tyunit}}
    \end{mathpar}
    \textbf{Runtime Typing Rules.}
    \begin{mathpar}
      \inferrule*[lab=T-Main]{
        \tseq[\cs{p}]{\ty{\Gamma}}{M}{T}
      }{\cseq[\main]{\ty{\Gamma}}{\main\;M}}
      
      \inferrule*[lab=T-Child]{
        \tseq[\cs{p}]{\ty{\Gamma}}{M}{\tyunit}
      }{\cseq[\child]{\ty{\Gamma}}{\child\;M}}
      
      \inferrule*[lab=T-Res]{
        \cseq[\phi]{\ty{\Gamma},\tmty{x}{S},\tmty{y}{\co{S}}}{\conf{C}}
      }{\cseq[\phi]{\ty{\Gamma}}{\res{x}{y}{\conf{C}}}}
      
      \inferrule*[lab=T-Par]{
        \cseq[\phi]{\ty{\Gamma}}{\conf{C}}
        \\
        \cseq[\phi']{\ty{\Delta}}{\conf{D}}
      }{\cseq[\phi+\phi']{\ty{\Gamma},\ty{\Delta}}{\ppar{\conf{C}}{\conf{D}}}}
    \end{mathpar}
    \caption{Typing Rules for PGV.}
    \label{fig:pgv-typing}
  \end{mdframed}
\end{figure}
%%% Local Variables:
%%% TeX-master: "../priorities"
%%% End:


The only non-standard feature of the typing rules is the priority annotations. Priorities are based on \emph{obligations/capabilities} used by Kobayashi~\cite{kobayashi06}, and simplified to single priorities following Padovani~\cite{padovani14}. The integration of priorities into GV is adapted from Padovani and Novara~\cite{padovaninovara15}. Paraphrasing Dardha and Gay~\cite{dardhagay18}, priorities obey the following two laws:
\begin{enumerate*}[label= (\roman*) ]
\item an action with lower priority happens before an action with higher priority; and
\item communication requires \emph{equal} priorities for dual actions.
\end{enumerate*}

In PGV, we keep track of a lower and upper bound on the priorities of a term, \ie, while evaluating the term, when does it start communicating, and when does it finish. The upper bound is written on the sequent, whereas the lower bound is approximated from the typing environment. Typing rules for sequential constructs enforce sequentially, \eg, the typing for $\tm{\andthen{M}{N}}$ has a side condition which requires that the upper bound of $\tm{M}$ is smaller than the lower bound of $\tm{N}$, \ie, $\tm{M}$ finishes before $\tm{N}$ starts. The typing rule for $\tm{\new}$ ensures that both endpoints of a channel share the same priorities. Together, these two constraints guarantee deadlock freedom.

To illustrate this, let's go back to the deadlocked program in \cref{ex:deadlock}. Crucially, it composes the terms below in parallel. While each of these terms itself is well-typed, they impose opposite conditions on the priorities, so their composition is ill-typed. (We omit the priorities on $\ty{\tyends}$ and $\ty{\tyendr}$.)
\begin{mathpar}
  \small
  \inferrule*{
    \tseq[\cs{o'}]
    {\tmty{y'}{\tyrecv[\cs{o'}]{\tyunit}{\tyendr}}}
    {\recv\;{y'}}
    {\typrod{\tyunit}{\tyendr}}
    \\
    \tseq[\cs{p}]
    {\tmty{x}{\tysend[\cs{o}]{\tyunit}{\tyends}},\tmty{y'}{\tyendr}}
    {\letbind{x}{\send\;{\pair{\unit}{x}}}{\dots}}
    {\tyunit}
    \and
    \cs{o'}<\cs{o}
  }{%
    \tseq[\cs{p}]
    {%
      \tmty{x}{\tysend[\cs{o}]{\tyunit}{\tyends}},
      \tmty{y'}{\tyrecv[\cs{o'}]{\tyunit}{\tyendr}}
    }
    {\letpair{\unit}{y'}{\recv\;{y'}}{\letbind{x}{\send\;{\pair{\unit}{x}}}{\dots}}}
    {\tyunit}
  }

  \inferrule*{
    \tseq[\cs{o}]
    {\tmty{x'}{\tyrecv[\cs{o}]{\tyunit}{\tyendr}}}
    {\recv\;{x'}}
    {\typrod{\tyunit}{\tyendr}}
    \\
    \tseq[\cs{q}]
    {\tmty{y}{\tysend[\cs{o'}]{\tyunit}{\tyends}},\tmty{x'}{\tyendr}}
    {\letbind{y}{\send\;{\pair{\unit}{y}}}{\dots}}
    {\tyunit}
    \and
    \cs{o}<\cs{o'}
  }{%
    \tseq[\cs{q}]
    {%
      \tmty{y}{\tysend[\cs{o'}]{\tyunit}{\tyends}},
      \tmty{x'}{\tyrecv[\cs{o}]{\tyunit}{\tyendr}}
    }
    {\letpair{\unit}{x'}{\recv\;{x'}}{\letbind{y}{\send\;{\pair{\unit}{y}}}{\dots}}}
    {\tyunit}
  }
\end{mathpar}
Closures suspend communication, so \LabTirName{T-Lam} stores the priority bounds of the function body on the function type, and \LabTirName{T-App} restores them. For instance, $\tm{\lambda{x}.\send\;\pair{x}{y}}$ is assigned the type $\ty{\tylolli[\cs{o},\cs{o}]{A}{S}}$, \ie, a~function which, when applied, starts and finishes communicating at priority $\cs{o}$.
\begin{mathpar}
  \small
  \inferrule*{
    \inferrule*{
      \inferrule*{
      }{
        \tmty{\send}{\tylolli[\ptop,\cs{o}]{\typrod{A}{\tysend[\cs{o}]{A}{S}}}{S}}
      }
      \and
      \inferrule*{%
        \inferrule*{
        }{
          \tseq[\pbot]{\tmty{x}{A}}{x}{A}
        }
        \and
        \inferrule*{
        }{
          \tseq[\pbot]
          {\tmty{x}{A},\tmty{y}{\tysend[\cs{o}]{A}{S}}}
          {y}{\tysend[\cs{o}]{A}{S}}
        }
      }{
        \tseq[\pbot]
        {\tmty{x}{A},\tmty{y}{\tysend[\cs{o}]{A}{S}}}
        {\pair{x}{y}}{\typrod{A}{\tysend[\cs{o}]{A}{S}}}
      }
    }{
      \tseq[\cs{o}]
      {\tmty{x}{A},\tmty{y}{\tysend[\cs{o}]{A}{S}}}
      {\send\;\pair{x}{y}}{S}
    }
  }{
    \tseq[\pbot]
    {\tmty{y}{\tysend[\cs{o}]{A}{S}}}
    {\lambda{x}.\send\;\pair{x}{y}}{\tylolli[\cs{o},\cs{o}]{A}{S}}
  }
\end{mathpar}

Typing rules for configurations are at the bottom of \cref{fig:pgv-typing}. Configurations are typed by a judgement $\cseq[\phi]{\ty{\Gamma}}{\conf{C}}$ stating that ``a configuration $\tm{\conf{C}}$ with flag $\phi$ is well typed under typing environment $\ty{\Gamma}$''. Configuration typing is based on the standard typing for GV. Terms are embedded either as main or as child threads. The priority bound from the term typing is discarded, as configurations contain no further blocking actions. Main threads are allowed to return a value, whereas child threads are required to return the unit value. Sequents are annotated with a flag $\phi$, which ensures that there is at most one main thread.

While our configuration typing is based on the standard typing for GV, it differs on two counts:
\begin{enumerate*}[label= (\roman*) ]
\item \emph{we require that child threads return the unit value}, as opposed to a terminated channel; and
\item \emph{we simplify typing for parallel composition}.
\end{enumerate*}
In order to guarantee deadlock freedom, in GV each parallel composition must split \emph{exactly one} channel of the channel pseudo-type $\ty{S^\sharp}$ into two endpoints of type $\ty{S}$ and $\ty{\co{S}}$. Consequently, associativity of parallel composition does not preserve typing. In PGV, we guarantee deadlock freedom using priorities, which removes the need for the channel pseudo-type $\ty{S^\sharp}$, and simplifies typing for parallel composition, while restoring type preservation for the structural congruence.


\subsubsection*{Subject reduction}
Unlike with previous versions of GV, structural congruence, term reduction, and configuration reduction are all type preserving.

We must show that substitution preserves priority constraints. For this, we prove~\cref{lem:pgv-value-done}, which shows that values have finished all their communication, and that any priorities in the type of the value come from the typing environment. The proofs are by structural induction and can be found in \cref{prf:lem-pgv-value-done,prf:lem-pgv-substitution,prf:lem-pgv-subject-reduction-terms,prf:lem-pgv-subject-congruence,prf:thm-pgv-subject-reduction-confs}.
\begin{compacttheorems}
  \begin{restatablelemma}{lempgvvaluedone}%
    \label{lem:pgv-value-done}
    If $\tseq[\cs{p}]{\ty{\Gamma}}{V}{T}$, then $\cs{p}=\cs{\pbot}$, and $\minpr(\ty{\Gamma})=\minpr(\ty{T})$.
  \end{restatablelemma}
  \begin{restatablelemma}{lempgvsubstitution}[Substitution]%
    \label{lem:pgv-substitution}
    \hfill\\%newline before theorem statement
    If $\tseq[\cs{p}]{\ty{\Gamma},\tmty{x}{U'}}{M}{T}$ and $\tseq[\cs{q}]{\ty{\Theta}}{V}{U'}$, then $\tseq[\cs{p}]{\ty{\Gamma},\ty{\Theta}}{\subst{M}{V}{x}}{T}$.
  \end{restatablelemma}
  \begin{restatablelemma}{lempgvsubjectreductionterms}[Subject Reduction, $\tred$]%
    \label{lem:pgv-subject-reduction-terms}
    \hfill\\%newline before theorem statement
    If $\tseq[\cs{p}]{\ty{\Gamma}}{M}{T}$ and $\tm{M}\tred\tm{M'}$,
    then $\tseq[\cs{p}]{\ty{\Gamma}}{M'}{T}$.
  \end{restatablelemma}
  \begin{restatablelemma}{lempgvsubjectcongruence}[Subject Congruence, $\equiv$]%
    \label{lem:pgv-subject-congruence}
    \hfill\\%newline before theorem statement
    If $\cseq[\phi]{\ty{\Gamma}}{\conf{C}}$ and $\tm{\conf{C}}\equiv\tm{\conf{C'}}$,
    then $\cseq[\phi]{\ty{\Gamma}}{\conf{C'}}$.
  \end{restatablelemma}
  \begin{restatabletheorem}{thmpgvsubjectreductionconfs}[Subject Reduction, $\cred$]%
    \label{thm:pgv-subject-reduction-confs}
    \hfill\\%newline before theorem statement
    If $\cseq[\phi]{\ty{\Gamma}}{\conf{C}}$ and $\tm{\conf{C}}\cred\tm{\conf{C'}}$,
    then $\cseq[\phi]{\ty{\Gamma}}{\conf{C'}}$.
  \end{restatabletheorem}
\end{compacttheorems}

\subsubsection*{Progress and deadlock freedom}
PGV satisfies progress, as PGV configurations either reduce or are in normal form. However, the normal forms may seem surprising at first, as evaluating a well-typed PGV term does not necessarily produce \emph{just} a value. If a term returns an endpoint, then its normal form contains a thread which is ready to communicate on the dual of that endpoint. This behaviour is not new to PGV. Let us consider an example, adapted from Lindley and Morris~\cite{lindleymorris15}, in which a term returns an endpoint linked to an echo server. The echo server receives a value and sends it back unchanged:
\[
  \tm{\echo_x} \defeq \tm{\letpair{y}{x}{\recv\;{x}}{\letbind{x}{\send\;\pair{y}{x}}{\close\;x}}}
\]
Consider the program which creates a new channel, with endpoints $\tm{x}$ and $\tm{x'}$, spawns off an echo server listening on $\tm{x'}$, and then returns $\tm{x}$:
\[
  \tm{\main\;\letpair{x}{x'}{\new}{\andthen{\spawn\;(\lambda\unit.\echo_{x'})}}{x}}
\]
If we reduce the above program, we get $\tm{\res{x}{x'}{(\ppar{\main\;x}{\child\;\echo_{x'}})}}$. Clearly, no more evaluation is possible, even though the configuration contains the thread $\tm{\child\;\echo_{x'}}$, which is blocked on $\tm{x'}$. In~\cref{cor:pgv-closed-progress} we will show that if a term does not return an endpoint, it must produce \emph{only} a value.

\emph{Actions} are terms which perform communication actions and which synchronise between two threads.
\begin{definition}[Actions]%
  \label{def:pgv-actions}
  A~term \emph{acts on} an endpoint $\tm{x}$ if it is $\tm{\send\;\pair{V}{x}}$, $\tm{\recv\;{x}}$, $\tm{\close\;{x}}$, or $\tm{\wait\;{x}}$. A~term is an \emph{action} if it acts on some endpoint $\tm{x}$.
\end{definition}

\emph{Ready terms} are terms which are ready to perform a communication action, either by themselves, \eg creating a new channel or thread, or with another thread, \eg sending or receiving.
\begin{definition}[Ready Terms]%
  \label{def:pgv-ready-actions}
  A~term $\tm{L}$ is \emph{ready} if it is of the form $\tm{\plug{E}{M}}$, where $\tm{M}$ is of the form $\tm{\new}$, $\tm{\spawn\;N}$, $\tm{\link\;\pair{x}{y}}$, or $\tm{M}$ acts on $\tm{x}$. In the latter case, we say that $\tm{L}$ is \emph{ready to act on} $\tm{x}$.
\end{definition}

Progress for the term language is standard for GV, and deviates from progress for linear $\lambda$-calculus only in that terms may reduce to values or \emph{ready terms}:
\begin{restatablelemma}{lempgvopenprogressterms}[Progress, $\tred$]%
  \label{lem:pgv-open-progress-terms}
  If $\tseq[\cs{p}]{\ty{\Gamma}}{M}{T}$ and $\ty{\Gamma}$ contains only session types, then:
  \begin{enumerate*}[label=\emph{(\alph*)}]
  \item $\tm{M}$ is a value;
  \item $\tm{M}\tred\tm{N}$ for some $\tm{N}$; or
  \item $\tm{M}$ is ready.
  \end{enumerate*}
\end{restatablelemma}

Canonical forms deviate from those for GV, in that we opt to move all $\nu$-binders to the top. The standard GV canonical form, alternating $\nu$-binders and their corresponding parallel compositions, does not work for PGV, since multiple channels may be split across a single parallel composition.

A~configuration either reduces, or it is equivalent to configuration in normal form. Crucial to the normal form is that each term $\tm{M_i}$ is blocked on the corresponding channel $\tm{x_i}$, and hence no two terms act on dual endpoints. Furthermore, no term $\tm{M_i}$ can perform a communication action by itself, since those are excluded by the definition of actions.
Finally, as a corollary, we get that well-typed terms which do not return endpoints return \emph{just} a value:

\begin{compacttheorems}
  \begin{definition}[Canonical Forms]%
    \label{def:pgv-canonical-forms}
    A~configuration $\tm{\conf{C}}$ is in canonical form if it is of the following form, where no term $\tm{M_i}$ is a value:
    \[
      \tm{\res{x_1}{x'_1}{\dots\res{x_n}{x'_n}{(\child\;M_1\parallel\dots\parallel\child\;M_m\parallel\main\;N)}}}.
    \]
  \end{definition}
  \begin{definition}[Normal Forms]%
    A~configuration $\tm{\conf{C}}$ is in normal form if it is of the following form, where each $\tm{M_i}$ is ready to act on $\tm{x_i}$:
    \[
      \tm{\res{x_1}{x'_1}{\dots\res{x_n}{x'_n}{(\child\;M_1\parallel\dots\parallel\child\;M_m\parallel\main\;V)}}}.
    \]
  \end{definition}
  \begin{restatablelemma}{lempgvcanonicalforms}[Canonical Forms]%
    \label{lem:pgv-canonical-forms}
    If $\cseq[\main]{\ty{\Gamma}}{\conf{C}}$, there exists some $\tm{\conf{D}}$ such that $\tm{\conf{C}}\equiv\tm{\conf{D}}$ and $\tm{\conf{D}}$ is in canonical form.
  \end{restatablelemma}
  \begin{restatabletheorem}{thmpgvclosedprogressconfs}[Progress, $\cred$]%
    \label{thm:pgv-closed-progress-confs}
    If $\cseq[\main]{\emptyenv}{\conf{C}}$ and $\tm{\conf{C}}$ is in canonical form, then either $\tm{\conf{C}}\cred\tm{\conf{D}}$ for some $\tm{\conf{D}}$; or $\tm{\conf{C}\equiv\conf{D}}$ for some $\tm{\conf{D}}$ in normal form.
  \end{restatabletheorem}
  \begin{proof}
    Our proof follows the reasoning by Kobayashi~\cite{kobayashi06} used in the proof of deadlock freedom for closed processes (Theorem 2). Details are in~\Cref{prf:thm-pgv-closed-progress-confs}.
  \end{proof}
  \begin{corollary}%
    \label{cor:pgv-closed-progress}
    If $\cseq[\main]{\emptyenv}{\conf{C}}$, $\tm{\conf{C}}\centernot\cred$, and $\tm{\conf{C}}$ contains no endpoints, then $\tm{\conf{C}}\equiv\tm{\phi\;V}$ for some value $\tm{V}$.
  \end{corollary}
\end{compacttheorems}
It follows immediately from \cref{thm:pgv-closed-progress-confs} and \cref{cor:pgv-closed-progress} that a term which does not return an endpoint will complete all its communication actions, thus satisfying deadlock freedom.
\end{document}

%%% Local Variables:
%%% TeX-master: "priorities"
%%% End:
