\documentclass[main.tex]{subfiles}

\begin{document}

\section{Priority GV}
\usingnamespace{pgv}

\subsection{Syntax}

\paragraph*{Priorities, types, and environments}
Types are divided into \emph{session types} $\ty{S}$ and \emph{value types} $\ty{T}$, $\ty{U}$.

Session types are annotated with \emph{priorities} $\cs{o}\in\mathbb{N}$.
The types $\ty{\tysend[\cs{o}]{T}{S}}$ and $\ty{\tyrecv[\cs{o}]{T}{S}}$ type the endpoints of a channel over which we send or receive a value of type $\ty{T}$, and then proceed as $\ty{S}$.
The types $\ty{\tyends[\cs{o}]}$ and $\ty{\tyendr[\cs{o}]}$ type endpoints of a channel whose session has finished, and over which we send or receive a ping before closing the channel.
\[
\begin{array}{lcl}
  \ty{S}
  & \coloneqq & \ty{\tysend[\cs{o}]{T}{S}}
    \sep        \ty{\tyrecv[\cs{o}]{T}{S}}
    \sep        \ty{\tyends[\cs{o}]}
    \sep        \ty{\tyendr[\cs{o}]}
\end{array}
\]
Value types are based on the types of the linear $\lambda$-calculus.
The types $\ty{\typrod{T}{U}}$, $\ty{\tyunit}$, $\ty{\tysum{T}{U}}$, and $\ty{\tyvoid}$ are the standard linear $\lambda$-calculus product type, unit type, sum type, and empty type.
The type $\ty{\tylolli[\cs{p},\cs{q}]{T}{U}}$ is the standard linear function type, annotated with \emph{priority bounds} $\cs{p},\cs{q}\in\mathbb{N}\cup\{\cs{\pbot},\cs{\ptop}\}$.
Session types are value types, as endpoints are values.
\[
\begin{array}{lcl}
  \ty{T}, \ty{U}
  & \coloneqq & \ty{\typrod{T}{U}}
    \sep        \ty{\tyunit}
    \sep        \ty{\tysum{T}{U}}
    \sep        \ty{\tyvoid}
    \sep        \ty{\tylolli[\cs{p},\cs{q}]{T}{U}}
    \sep        \ty{S}
\end{array}
\]
Given a function with type $\ty{\tylolli[\cs{p},\cs{q}]{T}{U}}$, the priority bound $\cs{p}$ is a \emph{lower bound} on the priorities of the channel names captured by the body of the function, and the priority bound $\cs{q}$ is an \emph{upper bound} on the priority of the communications that take place as a result of applying the function.
We discuss this in more detail when we give the typing rules.
We define syntactic sugar for the type of \emph{pure functions}.
\[
  \ty{\tylolli{T}{U}} \elabarrow \ty{\tylolli[\cs{\ptop},\cs{\pbot}]{T}{U}}
\]
A typing context $\ty\Gamma$ is a partial function associating names to types.
\[
\begin{array}{lcl}
  \ty{\Gamma}, \ty{\Delta}
  & \coloneqq & \ty{\emptyenv}
    \sep        \ty{\Gamma}, \tmty{x}{T}
\end{array}
\]
\[
\begin{array}{lcl}
  \ty{\co{\tysend[\cs{o}]{T}{S}}} & = & \ty{\tyrecv[\cs{o}]{T}{\co{S}}} \\
  \ty{\co{\tyrecv[\cs{o}]{T}{S}}} & = & \ty{\tysend[\cs{o}]{T}{\co{S}}}
\end{array}
\qquad
\begin{array}{lcl}
  \ty{\co{\tyends[\cs{o}]}} & = & \ty{\co{\tyendr[\cs{o}]}} \\
  \ty{\co{\tyendr[\cs{o}]}} & = & \ty{\co{\tyends[\cs{o}]}}
\end{array}
\]
\[
\begin{array}{lcl}
  \pr(\ty{\tysend[\cs{o}]{T}{S}})            & = & \cs{o} \\
  \pr(\ty{\tyrecv[\cs{o}]{T}{S}})            & = & \cs{o} \\
\end{array}
\qquad
\begin{array}{lcl}
  \pr(\ty{\tyends[\cs{o}]})                  & = & \cs{o} \\
  \pr(\ty{\tyendr[\cs{o}]})                  & = & \cs{o} \\
\end{array}
\]
\[
\begin{array}{lcl}
  \minpr(\ty{\typrod{T}{U}})                 & = & \minpr({\ty{T}})\sqcap\minpr({\ty{U}}) \\
  \minpr(\ty{\tysum{T}{U}})                  & = & \minpr({\ty{T}})\sqcap\minpr({\ty{U}}) \\
  \minpr(\ty{\tylolli[\cs{p},\cs{q}]{T}{U}}) & = & \cs{p} \\
  \\
  \minpr(\ty{\Gamma}, \tmty{x}{A})           & = & \minpr(\ty{\Gamma})\sqcap\pr(\ty{A})
\end{array}
\qquad
\begin{array}{lcl}
  \minpr(\ty{\tyunit})                       & = & \ptop \\
  \minpr(\ty{\tyvoid})                       & = & \ptop \\
  \minpr(\ty{S})                             & = & \pr(\ty{S}) \\
  \\
  \minpr(\ty{\emptyenv})                     & = & \ptop
\end{array}
\]

\paragraph*{Static Term Syntax}
We use $\tm{x}$, $\tm{y}$, $\tm{z}$, and $\tm{w}$ to range over variable names. Occasionally, we use $\tm{a}$, $\tm{b}$, and $\tm{c}$ to range over \emph{free} channel endpoints.
\[
\begin{array}{lcl}
  \tm{L}, \tm{M}, \tm{N}
  & \coloneqq & \tm{x}
    \sep        \tm{K}
    \sep        \tm{\lambda x.M}
    \sep        \tm{M\;N} \\
  & \sep      & \tm{\unit}
    \sep        \tm{\letunit{M}{N}} \\
  & \sep      & \tm{\pair{M}{N}}
    \sep        \tm{\letpair{x}{y}{M}{N}} \\
  & \sep      & \tm{\inl{M}}
    \sep        \tm{\inr{M}}
    \sep        \tm{\casesum{L}{x}{M}{y}{N}} \\
  & \sep      & \tm{\absurd{M}} \\
  \tm{K}
  & \coloneqq & \tm{\link}
    \sep        \tm{\new}
    \sep        \tm{\spawn}
    \sep        \tm{\send}
    \sep        \tm{\recv}
    \sep        \tm{\wait}
    \sep        \tm{\close}
\end{array}
\]

\paragraph*{Syntactic Sugar for Terms}
\[
\begin{array}{lcl}
  \tm{\andthen{M}{N}}
  & \elabarrow & \tm{\letunit{M}{N}} \\
  \tm{\letbind{x}{M}{N}}
  & \elabarrow & \tm{(\lambda x.N)\;M} \\
  \tm{\lambda\unit.M}
  & \elabarrow & \tm{\lambda z.\letunit{z}{M}} \\
  \tm{\lambda\pair{x}{y}.M}
  & \elabarrow & \tm{\lambda z.\letpair{x}{y}{z}{M}} \\
  \tm{\fork}
  & \elabarrow & \tm{\lambda x.\letpair{y}{z}{\new}{\andthen{\spawn\;{(\lambda\unit.x\;y)}}{z}}}
\end{array}
\]

\paragraph*{Syntactic Sugar for Internal and External Choice}
\[
\begin{array}{lcl}
  \ty{\tyselect[\cs{o}]{S}{S'}}
  & \elabarrow & \ty{\tysend[\cs{o}]{(\tysum{\co{S}}{\co{S'}})}{\tyends[\cs{o+1}]}} \\
  \ty{\tyoffer[\cs{o}]{S}{S'}}
  & \elabarrow & \ty{\tyrecv[\cs{o}]{(\tysum{S}{S'})}{\tyendr[\cs{o+1}]}} \\
\end{array}
\qquad
\begin{array}{lcl}
  \ty{\tyselectemp[\cs{o}]}
  & \elabarrow & \ty{\tysend[\cs{o}]{\tyvoid}{\tyends[\cs{o+1}]}} \\
  \ty{\tyofferemp[\cs{o}]}
  & \elabarrow & \ty{\tyrecv[\cs{o}]{\tyvoid}{\tyendr[\cs{o+1}]}} \\
\end{array}
\]
\[
\begin{array}{lcl}
  \tm{\select{\ell}}
  & \elabarrow & \tm{\lambda x.\letpair{y}{z}{\new}{\andthen{\close\;(\send\;{\pair{\ell\;y}{x}})}{z}}} \\
  \multicolumn{3}{l}{\tm{\offer{L}{x}{M}{y}{N}}\elabarrow} \\
  \multicolumn{3}{l}{
  \qquad\tm{\letpair{z}{w}{\recv\;{L}}{\andthen{\wait\;{w}}{\casesum{z}{x}{M}{y}{N}}}}} \\
  \tm{\offeremp{L}}
  & \elabarrow & \tm{\letpair{z}{w}{\recv\;L}{\andthen{\wait\;{w}}{\absurd{z}}}}
\end{array}
\]

\paragraph*{Runtime Term Syntax}
\[
\begin{array}[t]{lcl}
  \tm{\conf{C}}, \tm{\conf{D}}, \tm{\conf{E}}
  & \coloneqq & \tm{\phi\;M}
    \sep        \tm{\ppar{\conf{C}}{\conf{D}}}
    \sep        \tm{\res{x}{y}{\conf{C}}}
  \\
  \tm{\phi}
  & \coloneqq & \tm{\main}
    \sep        \tm{\child}
  \\
  \tm{V}, \tm{W}
  & \coloneqq & \tm{x}
    \sep        \tm{K}
    \sep        \tm{\lambda x.M}
    \sep        \tm{\unit}
    \sep        \tm{\pair{M}{N}}
    \sep        \tm{\inl{M}}
    \sep        \tm{\inr{M}}
  \\
  \tm{E}
  & \coloneqq & \tm{\hole} \\
  & \sep      & \tm{E\;M}
    \sep        \tm{V\;E} \\
  & \sep      & \tm{\letunit{E}{N}} \\
  & \sep      & \tm{\pair{E}{M}}
    \sep        \tm{\pair{V}{E}}
    \sep        \tm{\letpair{x}{y}{E}{M}} \\
  & \sep      & \tm{\inl{E}}
    \sep        \tm{\inr{E}}
    \sep        \tm{\casesum{E}{x}{M}{y}{N}} \\
  & \sep      & \tm{\absurd{E}} \\
  \\
  \tm{\conf{F}}
  & \coloneqq & \tm{\phi\;E}
  \\
  \tm{\conf{G}}
  & \coloneqq & \tm{\hole} \\
  & \sep      & \tm{\ppar{\conf{G}}{\conf{C}}}
    \sep        \tm{\res{x}{y}{\conf{G}}}
\end{array}
\]
\[
\begin{array}{ll}
  \tm{\main}  + \tm{\child} = \tm{\main}
  \\
  \tm{\child} + \tm{\main}  = \tm{\main}
\end{array}
\quad
\begin{array}{ll}
  \tm{\child} + \tm{\child} = \tm{\child}
  \\
  \tm{\main}  + \tm{\main} \; \text{undefined}
\end{array}
\]

\subsection{Typing}
See \cref{fig:pgv-typing}.
\begin{figure}[h]
  \paragraph*{Static Typing Rules}
  \begin{mathpar}
    \inferrule*[lab=T-Var]{
    }{\tseq[\cs{\pbot}]{\tmty{x}{T}}{x}{T}}
    
    \inferrule*[lab=T-Lam]{
      \tseq[\cs{o}]{\ty{\Gamma},\tmty{x}{T}}{M}{U}
    }{\tseq[\cs{\pbot}]{\ty{\Gamma}}{\lambda x.M}{\tylolli[\cs{\pr(\ty{\Gamma})},\cs{o}]{T}{U}}}
    
    \inferrule*[lab=T-Const]{
    }{\tseq[\cs{\pbot}]{\emptyenv}{K}{T}}
    
    \inferrule*[lab=T-App]{
      \tseq[\cs{p}]{\ty{\Gamma}}{M}{\tylolli[\cs{o},\cs{r}]{T}{U}}
      \\
      \tseq[\cs{q}]{\ty{\Delta}}{N}{T}
      \\
      \cs{p}<\pr(\ty{\Delta})
      \\
      \cs{q}<\cs{o}
    }{\tseq[\cs{p}\sqcup\cs{q}\sqcup\cs{o}]{\ty{\Gamma},\ty{\Delta}}{M\;N}{U}}
    \\
    \inferrule*[lab=T-Unit]{
    }{\tseq[\cs{\pbot}]{\emptyenv}{\unit}{\tyunit}}
    
    \inferrule*[lab=T-LetUnit]{
      \tseq[\cs{p}]{\ty{\Gamma}}{M}{\tyunit}
      \\
      \tseq[\cs{q}]{\ty{\Delta}}{N}{T}
      \\
      \cs{p}<\pr(\ty{\Delta})
    }{\tseq[\cs{p}\sqcup\cs{q}]{\ty{\Gamma},\ty{\Delta}}{\letunit{M}{N}}{T}}
    
    \inferrule*[lab=T-Pair]{
      \tseq[\cs{p}]{\ty{\Gamma}}{M}{T}
      \\
      \tseq[\cs{q}]{\ty{\Delta}}{N}{U}
      \\
      \cs{p}<\pr(\ty{\Delta})
    }{\tseq[\cs{p}\sqcup\cs{q}]{\ty{\Gamma},\ty{\Delta}}{\pair{M}{N}}{\typrod{T}{U}}}
    
    \inferrule*[lab=T-LetPair]{
      \tseq[\cs{p}]{\ty{\Gamma}}{M}{\typrod{T}{T'}}
      \\
      \tseq[\cs{q}]{\ty{\Delta},\tmty{x}{T},\tmty{y}{T'}}{N}{U}
      \\
      \cs{p}<\pr(\ty{\Delta},\ty{T},\ty{T'})
    }{\tseq[\cs{p}\sqcup\cs{q}]{\ty{\Gamma},\ty{\Delta}}{\letpair{x}{y}{M}{N}}{U}}
    
    \inferrule*[lab=T-Inl]{
      \tseq[o]{\ty{\Gamma}}{M}{T}
      \\
      \pr(\ty{T})=\pr(\ty{U})
    }{\tseq[o]{\ty{\Gamma}}{\inl{M}}{\tysum{T}{U}}}
    
    \inferrule*[lab=T-Inr]{
      \tseq[o]{\ty{\Gamma}}{M}{U}
      \\
      \pr(\ty{T})=\pr(\ty{U})
    }{\tseq[o]{\ty{\Gamma}}{\inr{M}}{\tysum{T}{U}}}
    
    \inferrule*[lab=T-CaseSum]{
      \tseq[\cs{p}]{\ty{\Gamma}}{L}{\tysum{T}{T'}}
      \\
      \tseq[\cs{q}]{\ty{\Delta},\tmty{x}{T}}{M}{U}
      \\
      \tseq[\cs{q}]{\ty{\Delta},\tmty{y}{T'}}{N}{U}
      \\
      \cs{p}<\pr(\ty{\Delta})
    }{\tseq[\cs{p}\sqcup\cs{q}]{\ty{\Gamma},\ty{\Delta}}{\casesum{L}{x}{M}{y}{N}}{U}}
    
    \inferrule*[lab=T-Absurd]{
      \tseq[o]{\ty{\Gamma}}{M}{\tyvoid}
    }{\tseq[o]{\ty{\Gamma},\ty{\Delta}}{\absurd{M}}{T}}
  \end{mathpar}
  We write $\tmty{K}{T}$ for $\tseq[\cs{\pbot}]{\emptyenv}{K}{T}$ in typing derivations.
  \\
  \paragraph*{Type Schemas for Constants}
  \begin{mathpar}
    \tmty{\link}{\tylolli{\typrod{S}{\co{S}}}{\tyunit}}
    
    \tmty{\new}{\tylolli{\tyunit}{\typrod{S}{\co{S}}}}
    
    \tmty{\spawn}{\tylolli{(\tylolli[\cs{p},\cs{q}]{\tyunit}{\tyunit})}{\tyunit}}
    \\
    \tmty{\send}{\tylolli[\cs{\ptop},\cs{o}]{\typrod{T}{\tysend[\cs{o}]{T}{S}}}{S}}
    
    \tmty{\recv}{\tylolli[\cs{\ptop},\cs{o}]{\tyrecv[\cs{o}]{T}{S}}{\typrod{T}{S}}}
    \\
    \tmty{\close}{\tylolli[\cs{\ptop},\cs{o}]{\tyends[\cs{o}]}{\tyunit}}
    
    \tmty{\wait}{\tylolli[\cs{\ptop},\cs{o}]{\tyendr[\cs{o}]}{\tyunit}}
  \end{mathpar}

  \paragraph*{Runtime Typing Rules}
  \begin{mathpar}
    \inferrule*[lab=T-Main]{
      \tseq[\cs{p}]{\ty{\Gamma}}{M}{T}
    }{\cseq[\main]{\ty{\Gamma}}{\main\;M}}
    
    \inferrule*[lab=T-Child]{
      \tseq[\cs{p}]{\ty{\Gamma}}{M}{\tyunit}
    }{\cseq[\child]{\ty{\Gamma}}{\child\;M}}
    
    \inferrule*[lab=T-Res]{
      \cseq[\phi]{\ty{\Gamma},\tmty{x}{S},\tmty{y}{\co{S}}}{\conf{C}}
    }{\cseq[\phi]{\ty{\Gamma}}{\res{x}{y}{\conf{C}}}}
    
    \inferrule*[lab=T-Par]{
      \cseq[\phi]{\ty{\Gamma}}{\conf{C}}
      \\
      \cseq[\phi']{\ty{\Delta}}{\conf{D}}
    }{\cseq[\phi+\phi']{\ty{\Gamma},\ty{\Delta}}{\ppar{\conf{C}}{\conf{D}}}}
  \end{mathpar}
  \caption{Typing Rules for PGV.}
  \label{fig:pgv-typing}
\end{figure}

%%% Local Variables:
%%% TeX-master: "../priorities"
%%% End:


\subsection{Operational Semantics}
See \cref{fig:pgv-operational-semantics}.
\begin{figure}[t]
  \paragraph*{Term reduction}
  \begin{mathpar}
    \begin{array}{llcl}
      \LabTirName{E-Lam}  & \tm{(\lambda x.M) \; V}
                          & \tred & \tm{\subst{M}{V}{x}}
      \\
      \LabTirName{E-Unit} & \tm{\letunit{\unit}{M}}
                          & \tred & \tm{M}
      \\
      \LabTirName{E-Pair} & \tm{\letpair{x}{y}{\pair{V}{W}}{M}}
                          & \tred & \tm{\subst{\subst{M}{V}{x}}{W}{y}}
      \\
      \LabTirName{E-Inl}  & \tm{\casesum{\inl{V}}{x}{M}{y}{N}}
                          & \tred & \tm{\subst{M}{V}{x}}
      \\
      \LabTirName{E-Inr}  & \tm{\casesum{\inr{V}}{x}{M}{y}{N}}
                          & \tred & \tm{\subst{N}{V}{y}}
    \end{array}
    \\
    \inferrule*[lab=E-Lift]{
      \tm{M}\tred\tm{M'}
    }{\tm{\plug{E}{M}}\tred\tm{\plug{E}{M'}}}
  \end{mathpar}
  \\
  \paragraph*{Structural congruence}
  \begin{mathpar}
    \begin{array}{llcl}
      \LabTirName{SC-LinkSwap}   & \tm{\plug{\conf{F}}{\link\;{\pair{x}{y}}}}
                                 & \equiv & \tm{\plug{\conf{F}}{\link\;{\pair{y}{x}}}}
      \\
      \LabTirName{SC-ResLink}    & \tm{\res{x}{y}{(\phi\;\link\;\pair{x}{y})}}
                                 & \equiv & \tm{\phi\;\unit}
      \\
      \LabTirName{SC-ResSwap}    & \tm{\res{x}{y}{\conf{C}}}
                                 & \equiv & \tm{\res{y}{x}{\conf{C}}}
      \\
      \LabTirName{SC-ResComm}    & \tm{\res{x}{y}{\res{z}{w}{\conf{C}}}}
                                 & \equiv & \tm{\res{z}{w}{\res{x}{y}{\conf{C}}}}
      \\
      \LabTirName{SC-ResExt}     & \tm{\res{x}{y}{(\ppar{\conf{C}}{\conf{D}}})}
                                 & \equiv & \tm{\ppar{\conf{C}}{\res{x}{y}{\conf{D}}}},
                                            \text{ if }{\tm{x},\tm{y}\notin\fv(\tm{\conf{C}})}
      \\
      \LabTirName{SC-ParNil}     & \tm{\ppar{\conf{C}}{\child{\unit}}}
                                 & \equiv & \tm{\conf{C}}
      \\
      \LabTirName{SC-ParComm}    & \tm{\ppar{\conf{C}}{\conf{D}}}
                                 & \equiv & \tm{\ppar{\conf{D}}{\conf{C}}}
      \\
      \LabTirName{SC-ParAssoc}   & \tm{\ppar{\conf{C}}{(\ppar{\conf{D}}{\conf{E}})}}
                                 & \equiv & \tm{\ppar{(\ppar{\conf{C}}{\conf{D}})}{\conf{E}}}
    \end{array}
  \end{mathpar}
  \\
  \paragraph*{Configuration reduction}
  \begin{mathpar}
    \begin{array}{llcl}
      \LabTirName{E-Link}  & \tm{\res{x}{y}{(\ppar{\plug{\conf{F}}{\link\;\pair{w}{x}}}{\conf{C}})}}
                           & \cred & \tm{\ppar{\plug{\conf{F}}{\unit}}{\subst{\conf{C}}{w}{y}}}
      \\
      \LabTirName{E-New}   & \tm{\plug{\conf{F}}{\new\;\unit}}
                           & \cred & \tm{\res{x}{y}{(\plug{\conf{F}}{\pair{x}{y}})}}
      \\
      \LabTirName{E-Spawn} & \tm{\plug{\conf{F}}{\spawn\;V}}
                           & \cred & \tm{\ppar{\plug{\conf{F}}{\unit}}{\child\;V\;\unit}}
      \\
      \LabTirName{E-Send}  & \tm{\res{x}{y}{(\ppar
                             {\plug{\conf{F}}{\send\;{\pair{V}{x}}}}
                             {\plug{\conf{F'}}{\recv\;{y}}})}}
                           & \cred & \tm{\res{x}{y}{(\ppar
                                     {\plug{\conf{F}}{x}}
                                     {\plug{\conf{F'}}{\pair{V}{y}}})}}
      \\
      \LabTirName{E-Close} & \tm{\res{x}{y}{(\ppar
                             {\plug{\conf{F}}{\wait\;{x}}}
                             {\plug{\conf{F'}}{\close\;{y}}})}}
                           & \cred & \tm{\ppar{\plug{\conf{F}}{\unit}}{\plug{\conf{F'}}{\unit}}}
    \end{array}
    \\
    \inferrule*[lab=E-LiftC]{
      \tm{\conf{C}}\cred\tm{\conf{C'}}
    }{\tm{\plug{\conf{G}}{\conf{C}}}\cred\tm{\plug{\conf{G}}{\conf{C'}}}}
  
    \inferrule*[lab=E-LiftM]{
      \tm{M}\tred\tm{M'}
    }{\tm{\plug{\conf{F}}{M}}\tred\tm{\plug{\conf{F}}{M'}}}
  
    \inferrule*[lab=E-LiftSC]{
      \tm{\conf{C}}\equiv\tm{\conf{C'}}
      \\
      \tm{\conf{C'}}\cred\tm{\conf{D'}}
      \\
      \tm{\conf{D'}}\equiv\tm{\conf{D}}
    }{\tm{\conf{C}}\cred\tm{\conf{D}}}
  \end{mathpar}
  \caption{Operational Semantics for PGV.}
  \label{fig:pgv-operational-semantics}
\end{figure}
%%% Local Variables:
%%% TeX-master: "../priorities"
%%% End:


\subsection{Metatheory}

\begin{restatablelemma}{lempgvvaluedone}
  \label{lem:pgv-value-done}
  If $\tseq[\cs{p}]{\ty{\Gamma}}{V}{T}$, then $\cs{p}=\cs{\pbot}$, and $\minpr(\ty{\Gamma})=\minpr(\ty{T})$.
\end{restatablelemma}
\begin{proof}
  By induction on the derivation of $\tseq[\cs{p}]{\ty{\Gamma}}{V}{T}$.
  (Details in \cref{prf:lem-pgv-value-done}.)
\end{proof}

\begin{restatablelemma}{lempgvsubstitution}[Substitution]
  \label{lem:pgv-substitution}
  \hfill\\%newline before theorem statement
  If $\tseq[\cs{p}]{\ty{\Gamma},\tmty{x}{U'}}{M}{T}$ and $\tseq[\cs{q}]{\ty{\Theta}}{V}{U'}$, then $\tseq[\cs{p}]{\ty{\Gamma},\ty{\Theta}}{\subst{M}{V}{x}}{T}$.
\end{restatablelemma}
\begin{proof}
  By induction on the derivation of $\tseq[\cs{p}]{\ty{\Gamma},\tmty{x}{U'}}{M}{T}$. Whenever the contexts in the premises and conclusion differ, we use \cref{lem:pgv-value-done} to prove that the priorities are preserved.
  (Details in \cref{prf:lem-pgv-substitution}.)
\end{proof}

\begin{restatablelemma}{lempgvsubjectreductionterms}[Subject Reduction, $\tred$]
  \label{lem:pgv-subject-reduction-terms}
  \hfill\\%newline before theorem statement
  If $\tseq[\cs{p}]{\ty{\Gamma}}{M}{T}$ and $\tm{M}\tred\tm{M'}$,
  then $\tseq[\cs{p}]{\ty{\Gamma}}{M'}{T}$.
\end{restatablelemma}
\begin{proof}
  By induction on the derivation of $\tm{M}\tred\tm{M'}$.
  (Details in \cref{prf:lem-pgv-subject-reduction-terms}.)
\end{proof}

\begin{restatablelemma}{lempgvsubjectcongruence}[Subject Congruence, $\equiv$]
  \label{lem:pgv-subject-congruence}
  \hfill\\%newline before theorem statement
  If $\cseq[\phi]{\ty{\Gamma}}{\conf{C}}$ and $\tm{\conf{C}}\equiv\tm{\conf{C'}}$,
  then $\cseq[\phi]{\ty{\Gamma}}{\conf{C'}}$.
\end{restatablelemma}
\begin{proof}
  By induction on the derivation of $\tm{\conf{C}}\equiv\tm{\conf{C'}}$.
  (Details in \cref{prf:lem-pgv-subject-congruence}.)
\end{proof}

\begin{restatabletheorem}{thmpgvsubjectreductionconfs}[Subject Reduction, $\cred$]
  \label{thm:pgv-subject-reduction-confs}
  \hfill\\%newline before theorem statement
  If $\cseq[\phi]{\ty{\Gamma}}{\conf{C}}$ and $\tm{\conf{C}}\cred\tm{\conf{C'}}$,
  then $\cseq[\phi]{\ty{\Gamma}}{\conf{C'}}$.
\end{restatabletheorem}
\begin{proof}
  By induction on the derivation of $\tm{\conf{C}}\cred\tm{\conf{C'}}$.
  (Details in \cref{prf:thm-pgv-subject-reduction-confs}.)
\end{proof}

\begin{definition}[Actions]
  \label{def:pgv-actions}
  A~term acts on an endpoint $\tm{x}$ if it is in one of the following forms:
  \begin{multicols}{3}
    \begin{itemize}[noitemsep,topsep=0pt,parsep=0pt,partopsep=0pt]
    \item $\tm{\link\;\pair{x}{y}}$ 
    \item $\tm{\link\;\pair{y}{x}}$
    \item $\tm{\send\;\pair{x}{y}}$
    \item $\tm{\recv\;{x}}$
    \item $\tm{\close\;{x}}$
    \item $\tm{\wait\;{x}}$
    \end{itemize}
  \end{multicols}
  \noindent
  A~term is an action if it acts on some endpoint $\tm{x}$.
\end{definition}

\begin{definition}[Ready Actions]
  \label{def:pgv-ready-actions}
  A~term is ready to act on an endpoint $\tm{x}$ if it is of the form $\tm{\plug{E}{M}}$, where $\tm{M}$ acts on $\tm{x}$. A~term is ready if it is ready to act on some endpoint $\tm{x}$.
\end{definition}

\begin{restatablelemma}{lempgvprogressterms}[Open Progress, $\tred$]

\end{restatablelemma}

\begin{definition}[Canonical Forms]
  \label{def:pgv-canonical-forms}
  A~configuration $\tm{\conf{C}}$ is in canonical form if it is of the following form:
  \[
    \tm{\res{x_1}{x'_1}{\dots\res{x_n}{x'_n}{(\child\;M_1\parallel\dots\parallel\child\;M_m\parallel\main\;N)}}}
  \]
\end{definition}

\begin{restatablelemma}{lempgvcanonicalforms}[Canonical Forms]
  \label{lem:pgv-canonical-forms}
  If $\cseq[\main]{\ty{\Gamma}}{\conf{C}}$, there exists some $\tm{\conf{D}}$ such that $\tm{\conf{C}}\equiv\tm{\conf{D}}$ and $\tm{\conf{D}}$ is in canonical form.
\end{restatablelemma}
\begin{proof}
  We move any $\nu$-binders to the top using \LabTirName{SC-ResExt}, discard any superfluous occurrences of $\tm{\child\;\unit}$ using \LabTirName{SC-ParNil}, and move the main thread to the rightmost position using \LabTirName{SC-ParComm} and \LabTirName{SC-ParAssoc}.
\end{proof}

\begin{restatabletheorem}{thmpgvclosedprogressconfs}[Closed Progress, $\cred$]
  \label{lem:pgv-closed-progress-confs}
  If $\cseq[\main]{\emptyenv}{\conf{C}}$ and $\tm{\conf{C}}$ is in canonical form, then either $\tm{\conf{C}}=\tm{\main\;V}$ or there exists a $\tm{\conf{D}}$ such that $\tm{\conf{C}}\cred\tm{\conf{D}}$.
\end{restatabletheorem}
\begin{proof}
  \label{prf:thm-pgv-closed-progress}
  Let $\tm{\conf{C}}=\tm{\res{x_1}{x'_1}{\dots\res{x_n}{x'_n}{(\child\;M_1\parallel\dots\parallel\child\;M_m\parallel\main\;N)}}}$. If any term $\tm{M_i}$ or $\tm{N}$ is \emph{not} ready, 
\end{proof}

  % Pick the process $\tm{P_i}$ with the smallest priority $\minpr{(\ty{\Gamma_i})}$. $\tm{P_i}$ acts on some endpoint $\tmty{y}{A}\in\ty{\Gamma_i}$. We must have $\minpr{(\ty{\Gamma_i})}=\pr{(\ty{A})}$, since the other actions in $\tm{P_i}$ are guarded by the action on $\tmty{y}{A}$.

  % If $\tm{P_j}$ is a link $\tm{\link{y}{z}}$ or $\tm{\link{z}{y}}$, we apply \LabTirName{E-Link}.

  % Otherwise, there must be a process $\seq{P_j}{\ty{\Gamma_j}}$ where $\tmty{y'}{\co{A}}\in\ty{\Gamma_j}$ and $\tm{y}$ and $\tm{y'}$ are dual endpoints of the same channel, since $\tm{y}$ is bound. By duality, $\pr{(\ty{A})}=\pr{(\ty{\co{A}})}$. We must have $\pr{(\ty{\co{A}})}=\minpr{(\ty{\Gamma_j})}$, since we chose $\tm{P_i}$ to have the smallest priority $\minpr{(\Gamma_i)}$. We cannot have $\tm{P_i}=\tm{P_j}$, otherwise the action on $\tm{y'}$ would be guarded by the action on $\tm{y}$, requiring $\pr{(\ty{A})}<\pr{(\ty{\co{A}})}$. The process $\tm{P_j}$ must act on $\tm{y'}$, otherwise the action on $\tm{y'}$ would be guarded by another action with priority smaller than $\pr{(\ty{A})}$. We have two processes, acting on dual endpoints. We apply the appropriate reduction rule, \ie \LabTirName{E-Send}, \LabTirName{E-Close}, \LabTirName{E-Select-Inl}, or \LabTirName{E-Select-Inr}.


\end{document}

%%% Local Variables:
%%% TeX-master: "priorities"
%%% End:
