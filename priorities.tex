\documentclass[citecolor=red,linkcolor=blue,runningheads]{llncs}
\usepackage[english]{babel}

% Document preparation
\newcommand{\todo}[1]{\noindent\textcolor{red}{\textbf{TODO:}~#1}}
\newcommand{\simon}[1]{\footnote{Simon:~#1}}

% Typesetting
\usepackage[all]{foreign}
\usepackage{etoolbox}
\usepackage{rotating}
% \usepackage{pdflscape} % activate for automatically rotated pages in PDF
\usepackage{lscape}
\usepackage{enumitem}
\usepackage{multicol}
\setlist[description]{style=unboxed,font=\normalfont\itshape}
\usepackage[strict]{changepage}
\usepackage{tikz}
\usepackage{mdframed}
\usepackage{float}
\floatplacement{figure}{tp}

% Math symbols
\usepackage[fleqn]{amsmath}
\let\:\undefined% conflicts with namespc
\usepackage{amssymb}
\usepackage{centernot}
\usepackage{stmaryrd}
\usepackage{cmll}
\usepackage{mathtools}
\usepackage{environ}
\usepackage{tikz-cd}
\usetikzlibrary{arrows,automata,shapes,shapes.geometric}

% Subfiles and namespaces
\usepackage{subfiles}
\newcommand{\onlyinsubfile}[1]{#1}
\newcommand{\notinsubfile}[1]{}
\usepackage{namespc}

% Theorem environments
\spnewtheorem*{case*}{Case}{\itshape}{\rmfamily}
\spnewtheorem*{subcase*}{Subcase}{\itshape}{\rmfamily}

% Restatable theorems and lemmas (LLNCS-compatible)
\newcommand{\restatetheorem}[1]{%
  \begingroup
  \renewcommand{\thetheorem}{\ref{#1}}%
  \expandafter\expandafter\expandafter\theorem
  \csname restatabletheorem@#1\endcsname
  \endtheorem
  \endgroup
}
\NewEnviron{restatabletheorem}[1]{%
  \global\expandafter\xdef\csname restatabletheorem@#1\endcsname{%
    \unexpanded\expandafter{\BODY}%
  }%
  \expandafter\theorem\BODY\unskip\label{#1}\endtheorem
}
\newcommand{\restatelemma}[1]{%
  \begingroup
  \renewcommand{\thelemma}{\ref{#1}}%
  \expandafter\expandafter\expandafter\lemma
  \csname restatablelemma@#1\endcsname
  \endlemma
  \endgroup
}
\NewEnviron{restatablelemma}[1]{%
  \global\expandafter\xdef\csname restatablelemma@#1\endcsname{%
    \unexpanded\expandafter{\BODY}%
  }%
  \expandafter\lemma\BODY\unskip\label{#1}\endlemma
}

% Inference rules
\usepackage{mathpartir}
\mprset{sep=1em}

% Sort macros
\usepackage{xcolor}
\newcommand{\tm}[1]{\ensuremath{{\color[HTML]{a40038}#1}}}
\newcommand{\ty}[1]{\ensuremath{{\color[HTML]{00007a}#1}}}
\newcommand{\cs}[1]{\ensuremath{{\color[HTML]{009180}#1}}}
\newcommand{\tmty}[2]{\ensuremath{\tm{#1}:\ty{#2}}}

% Common definitions
\newcommand{\sep}{\;\mid\;}
\newcommand{\emptyenv}{\varnothing}
\newcommand{\subst}[4][]{\ifstrempty{#1}{\ensuremath{#2\{#3/#4\}}}{\ensuremath{#2(\{#3/#4\}\cup#1)}}}
\newcommand{\plug}[2]{\ensuremath{#1[#2]}}
\newcommand{\minpr}[0]{\ensuremath{\text{min}_{\pr}}}
\newcommand{\maxpr}[0]{\ensuremath{\text{max}_{\pr}}}
\providecommand{\pbot}[0]{\ensuremath{{\bot\!}}}
\providecommand{\ptop}[0]{\ensuremath{{\top\!}}}
\newcommand{\aleq}[0]{\mathbin{{=_{\alpha}}}}
\newcommand{\defeq}[0]{\triangleq}
\newcommand{\elabarrow}[0]{\triangleq}% deprecated
\newcommand{\hole}[0]{\ensuremath{\square}}
\DeclareMathOperator{\pr}{pr}
\DeclareMathOperator{\fv}{fv}
\DeclareMathOperator{\fn}{fn}
\DeclareMathOperator{\bn}{bn}
\DeclareMathOperator{\cn}{cn}
\DeclareMathOperator{\ready}{ready}
\DeclareMathOperator{\blocked}{blocked}
\DeclareMathOperator{\dom}{dom}
\DeclareMathOperator{\cod}{cod}
\DeclareMathAlphabet{\rel}{OMS}{cmsy}{m}{n}
\DeclareMathAlphabet{\conf}{OMS}{cmsy}{m}{n}

% Translations
\newcommand{\substarrow}[2]{\xRightarrow{\scriptscriptstyle\{\tm{#1}/\tm{#2}\}}}
\DeclarePairedDelimiter{\cpgvT}{\llparenthesis}{\rrparenthesis}
\DeclarePairedDelimiter{\cpgvM}{\llparenthesis}{\rrparenthesis_{\!\scriptscriptstyle{M}}}
\DeclarePairedDelimiter{\cpgvC}{\llparenthesis}{\rrparenthesis_{\!\scriptscriptstyle{\conf{C}}}}
\newcommand{\cpgvMarrow}[0]{\xRightarrow{\;\raisebox{2pt}{\ensuremath{{\cpgvM{\cdot}}}}\;}}
\newcommand{\cpgvCarrow}[0]{\xRightarrow{\;\raisebox{2pt}{\ensuremath{{\cpgvC{\cdot}}}}\;}}

% Priority CP
\namespace*{pcp}{
  % Types
  \providecommand{\tyone}[1][]{\ensuremath{\mathbf{1}^{#1}}}
  \providecommand{\tynil}[1][]{\ensuremath{\mathbf{0}^{#1}}}
  \providecommand{\tytop}[1][]{\ensuremath{\top^{#1}}}
  \providecommand{\tybot}[1][]{\ensuremath{\bot^{#1}}}
  \providecommand{\typlus}[3][]{\ensuremath{{#2}\mathbin{\oplus^{#1}}{#3}}}
  \providecommand{\tywith}[3][]{\ensuremath{{#2}\mathbin{\with^{#1}}{#3}}}
  \providecommand{\tytens}[3][]{\ensuremath{{#2}\mathbin{\otimes^{#1}}{#3}}}
  \providecommand{\typarr}[3][]{\ensuremath{{#2}\mathbin{\parr^{#1}}{#3}}}
  \providecommand{\co}[1]{\ensuremath{#1^\bot}}
  % Terms
  \providecommand{\res}[4][]{\ensuremath{(\nu{#2}^{#1}{#3}){#4}}}
  \providecommand{\ppar}[2]{\ensuremath{#1\parallel#2}}
  \providecommand{\halt}[0]{\ensuremath{\mathbf{0}}}
  \providecommand{\link}[3][]{\lablink[#1]{#2}{#3}}
  \providecommand{\send}[3]{\ensuremath{\labsend{#1}{#2}.#3}}
  \providecommand{\usend}[3]{\ensuremath{#1\langle{#2}\rangle.#3}}
  \providecommand{\recv}[3]{\ensuremath{\labrecv{#1}{#2}.#3}}
  \providecommand{\close}[2]{\ensuremath{\labclose{#1}.#2}}
  \providecommand{\wait}[2]{\ensuremath{\labwait{#1}.#2}}
  \providecommand{\inl}[2]{\ensuremath{\labselinl{#1}.{#2}}}
  \providecommand{\inr}[2]{\ensuremath{\labselinr{#1}.{#2}}}
  \providecommand{\offer}[3]{\ensuremath{{#1}\triangleright\{\text{inl}:#2;\text{inr}:#3\}}}
  \providecommand{\absurd}[1]{\ensuremath{{#1}\triangleright\{\}}}
  % Typing judgements
  \providecommand{\seq}[2]{\ensuremath{\tm{#1}\vdash{#2}}}
  % Equivalences
  \providecommand{\bbq}[0]{\approxeq}
  \providecommand{\sbis}[0]{\sim}
  \providecommand{\bis}[0]{\approx}
  % Reduction relations
  \providecommand{\red}[0]{\ensuremath{\Longrightarrow}}
  % Labelled transition
  \providecommand{\obs}[2]{\ensuremath{\tm{#1}{\Downarrow_{\tm{#2}}}}}
  \providecommand{\slto}[1]{\ensuremath{\xRightarrow{\tm{#1}}}}
  \providecommand{\stto}[0]{\ensuremath{\xRightarrow{\tm{\;\tau\;}}}}
  \providecommand{\lto}[1]{\ensuremath{\xrightarrow{\tm{#1}}}}
  \providecommand{\tto}[0]{\ensuremath{\xrightarrow{\tm{\;\tau\;}}}}
  \providecommand{\lablink}[3][]{\ensuremath{{#2}{\leftrightarrow}^{#1}{#3}}}
  \providecommand{\labsend}[2]{\ensuremath{#1[#2]}}
  \providecommand{\labrecv}[2]{\ensuremath{#1(#2)}}
  \providecommand{\labclose}[1]{\labsend{#1}{}}
  \providecommand{\labwait}[1]{\labrecv{#1}{}}
  \providecommand{\labselinl}[1]{\ensuremath{{#1}\triangleleft\text{inl}}}
  \providecommand{\labselinr}[1]{\ensuremath{{#1}\triangleleft\text{inr}}}
  \providecommand{\laboffinl}[1]{\ensuremath{{#1}\triangleright\text{inl}}}
  \providecommand{\laboffinr}[1]{\ensuremath{{#1}\triangleright\text{inr}}}
  % Cyclic Scheduler
  \providecommand{\sched}{\ensuremath{Sched}}
  \providecommand{\agent}[1]{\ensuremath{Agent_{#1}}}
  \providecommand{\proc}[1]{\ensuremath{Proc_{#1}}}
}{}
\newcommand{\pcp}[1]{\namespace*{pcp}{}{#1}}

% Priority GV
\namespace*{pgv}{
  % Types
  \providecommand{\tyunit}[0]{\ensuremath{\mathbf{1}}}
  \providecommand{\tyvoid}[0]{\ensuremath{\mathbf{0}}}
  \providecommand{\typrod}[2]{\ensuremath{{#1}\mathbin{\times}{#2}}}
  \providecommand{\tysum}[2]{\ensuremath{{#1}\mathbin{+}{#2}}}
  \providecommand{\tylolli}[3][]{\ensuremath{{#2}\mathbin{\multimap^{#1}}{#3}}}
  \providecommand{\co}[1]{\ensuremath{\overline{#1}}}
  % Session types
  \providecommand{\tysend}[3][]{\ensuremath{!^{#1}{#2}.{#3}}}
  \providecommand{\tyrecv}[3][]{\ensuremath{?^{#1}{#2}.{#3}}}
  \providecommand{\tyends}[1][]{\ensuremath{\mathbf{end}_{!}^{#1}}}
  \providecommand{\tyendr}[1][]{\ensuremath{\mathbf{end}_{?}^{#1}}}
  \providecommand{\tyselect}[3][]{\ensuremath{{#2}\mathbin{\oplus^{#1}}{#3}}}
  \providecommand{\tyoffer}[3][]{\ensuremath{{#2}\mathbin{\with^{#1}}{#3}}}
  \providecommand{\tyselectemp}[1][]{\ensuremath{{\oplus}^{#1}\{\}}}
  \providecommand{\tyofferemp}[1][]{\ensuremath{{\with}^{#1}\{\}}}
  % Terms
  \providecommand{\andthen}[2]{\ensuremath{#1;#2}}
  \providecommand{\letbind}[3]{\ensuremath{\mathbf{let}\;#1\mathbin{=}#2\;\mathbf{in}\;#3}}
  \providecommand{\pair}[2]{\ensuremath{(#1,#2)}}
  \providecommand{\letpair}[4]{\ensuremath{\letbind{\pair{#1}{#2}}{#3}{#4}}}
  \providecommand{\labinl}[0]{\ensuremath{\mathbf{inl}}}
  \providecommand{\labinr}[0]{\ensuremath{\mathbf{inr}}}
  \providecommand{\inl}[1]{\ensuremath{\labinl\;#1}}
  \providecommand{\inr}[1]{\ensuremath{\labinr\;#1}}
  \providecommand{\casesum}[5]{\ensuremath{\mathbf{case}\;#1\;\left\{\inl{#2}\mapsto{#3};\;\inr{#4}\mapsto{#5}\right\}}}
  \providecommand{\unit}[0]{\ensuremath{()}}
  \providecommand{\letunit}[2]{\ensuremath{\letbind{\unit}{#1}{#2}}}
  \providecommand{\absurd}[1]{\ensuremath{\mathbf{absurd}\;#1}}
  \providecommand{\link}[0]{\ensuremath{\mathbf{link}}}
  \providecommand{\new}[0]{\ensuremath{\mathbf{new}}}
  \providecommand{\halt}[0]{\ensuremath{\mathbf{halt}}}
  \providecommand{\spawn}[0]{\ensuremath{\mathbf{spawn}}}
  \providecommand{\send}[0]{\ensuremath{\mathbf{send}}}
  \providecommand{\recv}[0]{\ensuremath{\mathbf{recv}}}
  \providecommand{\fork}[0]{\ensuremath{\mathbf{fork}}}
  \providecommand{\wait}[0]{\ensuremath{\mathbf{wait}}}
  \providecommand{\close}[0]{\ensuremath{\mathbf{close}}}
  \providecommand{\select}[1]{\ensuremath{\mathbf{select}\;#1}}
  \providecommand{\offer}[5]{\ensuremath{\mathbf{offer}\;#1\;\{\inl{#2}\mapsto{#3};\inr{#4}\mapsto{#5}\}}}
  \providecommand{\offeremp}[1]{\ensuremath{\mathbf{offer}\;#1\;\{\}}}
  % Configurations
  \providecommand{\ppar}[2]{\ensuremath{#1\parallel#2}}
  \providecommand{\res}[3]{\ensuremath{(\nu#1#2)#3}}
  % Flags
  \providecommand{\main}[0]{\ensuremath{\bullet}}
  \providecommand{\child}[0]{\ensuremath{\circ}}
  % Typing judgements
  \providecommand{\tseq}[4][]{\ensuremath{#2\vdash^{#1}\tmty{#3}{#4}}}
  \providecommand{\cseq}[3][]{\ensuremath{#2\vdash^{#1}\tm{#3}}}
  % Reduction relations
  \providecommand{\tred}[0]{\ensuremath{\longrightarrow_{M}}}
  \providecommand{\cred}[0]{\ensuremath{\longrightarrow_{\conf{C}}}}
  % Cyclic Scheduler
  \newcommand{\sched}{\ensuremath{\mathbf{sched}}}
  \newcommand{\agent}[1]{\ensuremath{\mathbf{agent}_{#1}}}
  \newcommand{\proc}[1]{\ensuremath{\mathbf{proc}_{#1}}}
  \newcommand{\echo}[0]{\ensuremath{\mathbf{echo}}}
}{}
\newcommand{\pgv}[1]{\namespace*{pgv}{}{#1}}

% References
\usepackage{varioref}
\usepackage[hidelinks]{hyperref}
\usepackage{cleveref}
\usepackage{doi}

\begin{document}
%
\renewcommand{\onlyinsubfile}[1]{}
\renewcommand{\notinsubfile}[1]{#1}
%
\title{Prioritise the Best Variation%
  \thanks{Supported by the UK EPSRC grant EP/K034413/1, ``From Data Types to Session Types: A Basis for Concurrency and Distribution'' (ABCD), and by the EU HORIZON 2020 MSCA RISE project 778233 ``Behavioural Application Program Interfaces'' (BehAPI).}
}
%
\author{
  Wen Kokke%
  \inst{1}%
  \and%
  Ornela Dardha%
  \inst{2}}
%
\authorrunning{W. Kokke \and O. Dardha}
%
\institute{%
  University of Edinburgh, Edinburgh
  \email{wen.kokke@ed.ac.uk}
  \\
  University of Glasgow, Glasgow
  \email{ornela.dardha@glasgow.ac.uk}}
%
\maketitle
%
\begin{abstract}
Session types are a type formalism used to specify and verify communication protocols in concurrent settings. They have been successfully integrated in the $\pi$-calculus~\cite{honda93,takeuchihonda94,hondavasconcelos98} and a concurrent $\lambda$-calculus, called Good Variation (GV)~\cite{wadler15,lindleymorris15}, among other paradigms. Session type systems guarantee communication safety and session fidelity, but cannot guarantee deadlock freedom\simon{Too strong IMO. They guarantee deadlock-freedom within a single session, using binary duality. MPSTs extend this to DF within a single multiparty session.}. Deadlock freedom in GV is guaranteed by combining channel creation and thread spawning under the same operation, called fork. This is in line with Classical Processes (CP), a~process calculus with strong ties to classical linear logic~\cite{cairespfenning10,wadler12}, which combines channel creation and parallel composition in the $\pi$-calculus under the same logical cut rule. In both GV and CP, deadlock freedom is achieved at the expense of expressivity as the only communication structures allowed are trees. Dardha and Gay~\cite{dardhagay18} define Priority CP (PCP) which allows for cyclic structures and restores deadlock freedom by adding priorities to types following Kobayashi~\cite{kobayashi06} and Padovani~\cite{padovani14}.
Following PCP, we present Priority GV (PGV), a~variant of GV which decouples channel creation from thread spawning, and restores deadlock freedom by adding priorities. We show our type system is sound by proving subject reduction and progress. We define an encoding from PCP to PGV and prove that the encoding well-typed and sound and complete with respect to the operational semantics. We illustrate PGV and the encoding via Milner's Cyclic Scheduler~\cite{milner89}.
\keywords{Session types \and Process calculus}
\end{abstract}

\subfile{introduction}
\subfile{priority-gv}
\subfile{priority-cp}
\subfile{conclusion}

\bibliographystyle{splncs04}
\bibliography{main}

\clearpage\appendix
% change the margins of appendix
\changetext{}{10em}{-5em}{-5em}{}
\section{Priority GV}
{
  \usingnamespace{pgv}

  % Proofs that the syntactic sugar is well-typed.
  \begin{figure}[t]
  \paragraph*{Typing Rules for Syntactic Sugar}
  \begin{mathpar}
    \inferrule*[lab=T-Seq]{
      \tseq[\cs{p}]{\ty{\Gamma}}{M}{\tyunit}
      \\
      \tseq[\cs{q}]{\ty{\Delta}}{N}{T}
      \\
      \cs{p}<\pr(\ty{\Delta})
    }{\tseq[\cs{p}\sqcup\cs{q}]{\ty{\Gamma},\ty{\Delta}}{\andthen{M}{N}}{T}}
    
    \inferrule*[lab=T-Let]{
      \tseq[\cs{p}]{\ty{\Gamma}}{M}{T}
      \\
      \tseq[\cs{q}]{\ty{\Delta},\tmty{x}{T}}{N}{U}
      \\
      \cs{p}<\pr(\ty{\Delta})
    }{\tseq[\cs{p}\sqcup\cs{q}]{\ty{\Gamma},\ty{\Delta}}{\letbind{x}{M}{N}}{U}}
    \\
    \inferrule*[lab=T-LamUnit]{
      {\tseq[\cs{o}]{\ty{\Gamma}}{M}{T}}
    }{\tseq[\cs{\pbot}]
      {\ty{\Gamma}}
      {\lambda\unit.M}
      {\tylolli[\cs{\pr(\ty{\Gamma})},\cs{o}]{\tyunit}{T}}}
    
    \inferrule*[lab=T-LamPair]
    {\tseq[\cs{o}]
      {\ty{\Gamma},\tmty{x}{T},\tmty{y}{T'}}
      {M}
      {U}}
    {\tseq[\cs{\pbot}]
      {\ty{\Gamma}}
      {\lambda\pair{x}{y}.M}
      {\tylolli[\cs{\pr(\ty{\Gamma})},\cs{o}]{\typrod{T}{T'}}{U}}}
    \\
    \inferrule*[lab=T-Fork]{
    }{\tseq[\cs{\pbot}]
      {\emptyenv}
      {\fork}
      {\tylolli[]{(\tylolli[\cs{p},\cs{q}]{S}{\tyunit})}{\co{S}}}}
    \\
    \inferrule*[lab=T-Select-Inl]{
      \pr(\ty{S})=\pr(\ty{S'})
    }{\tseq[\cs{\pbot}]{\emptyenv}{\select{\labinl}}{\tylolli[\cs{\ptop},\cs{o}]{\tyselect[\cs{o}]{S}{S'}}{S}}}
    
    \inferrule*[lab=T-Select-Inr]{
      \pr(\ty{S})=\pr(\ty{S'})
    }{\tseq[\cs{\pbot}]{\emptyenv}{\select{\labinr}}{\tylolli[\cs{\ptop},\cs{o}]{\tyselect[\cs{o}]{S}{S'}}{S'}}}
    
    \inferrule*[lab=T-Offer]{
      {\tseq[\cs{p}]
        {\ty{\Gamma}}
        {L}
        {\tyoffer[\cs{o}]{S}{S'}}}
      \\
      {\tseq[\cs{q}]
        {\ty{\Delta},\tmty{x}{S}}
        {M}
        {T}}
      \\
      {\tseq[\cs{q}]
        {\ty{\Delta},\tmty{y}{S'}}
        {N}
        {T}}
      \\
      \cs{o}\sqcup\cs{p}<\pr(\ty{\Delta},\ty{S},\ty{S'})
    }{\tseq[\cs{o}\sqcup\cs{p}\sqcup\cs{q}]
      {\ty{\Gamma},\ty{\Delta}}
      {\offer{L}{x}{M}{y}{N}}
      {T}}
    
    \inferrule*[lab=T-Offer-Absurd]{
      \tseq[\cs{p}]
      {\ty{\Gamma}}
      {L}
      {\tyofferemp[\cs{o}]}
      \\
      \cs{o}\sqcup\cs{p}<\pr(\ty{\Delta})
    }{\tseq[\cs{o}\sqcup\cs{p}]
      {\ty{\Gamma},\ty{\Delta}}
      {\offeremp{L}}
      {T}}
  \end{mathpar}
  \caption{Typing Rules for Syntactic Sugar for PGV.}
  \label{fig:pgv-typing-sugar}
\end{figure}
%%% Local Variables:
%%% TeX-master: "../priorities"
%%% End:


  \restatelemma{lempgvvaluedone}
  \begin{proof}
  \label{prf:lem-pgv-value-done}
  By induction on the derivation of $\tseq[\cs{o}]{\ty{\Gamma}}{V}{T}$.

  \begin{case*}[\LabTirName{T-Lam}]
    Immediately.
    \begin{mathpar}
      \inferrule*{
        \tseq[\cs{q}]{\ty{\Gamma},\tmty{x}{T}}{M}{U}
      }{\tseq[\cs{\pbot}]{\ty{\Gamma}}{\lambda x.M}{\tylolli[\cs{\pr(\ty{\Gamma})},\cs{q}]{T}{U}}}
    \end{mathpar}
  \end{case*}
  \begin{case*}[\LabTirName{T-Unit}]
    Immediately.
    \begin{mathpar}
      \inferrule*{
      }{\tseq[\cs{\pbot}]{\emptyenv}{\unit}{\tyunit}}
    \end{mathpar}
  \end{case*}
  \begin{case*}[\LabTirName{T-Pair}]
    The induction hypotheses give us $\cs{p}=\cs{q}=\cs{\pbot}$, hence $\cs{p}\sqcup\cs{q}=\cs{\pbot}$, and $\pr(\ty{\Gamma})=\pr(\ty{T})$ and $\pr(\ty{\Delta})=\pr(\ty{U})$, hence $\pr(\ty{\Gamma},\ty{\Delta})=\pr(\ty{\Gamma})\sqcap\pr(\ty{\Delta})=\pr(\ty{T})\sqcap\pr(\ty{U})=\pr(\ty{\typrod{T}{U}})$.
    \begin{mathpar}
      \inferrule*{
        \tseq[\cs{p}]{\ty{\Gamma}}{V}{T}
        \\
        \tseq[\cs{q}]{\ty{\Delta}}{W}{U}
        \\
        \cs{p}<\pr(\ty{\Delta})
      }{\tseq[\cs{p}\sqcup\cs{q}]{\ty{\Gamma},\ty{\Delta}}{\pair{V}{W}}{\typrod{T}{U}}}
    \end{mathpar}
  \end{case*}
  \begin{case*}[\LabTirName{T-Inl}]
    The induction hypothesis gives us $\cs{p}=\cs{\pbot}$, and $\pr(\ty{\Gamma})=\pr{\ty{T}}$. We know $\pr(\ty{T})=\pr({\ty{U}})$, hence $\pr(\ty{\Gamma})=\pr(\ty{\tysum{T}{U}})$.
    \begin{mathpar}
      \inferrule*{
        \tseq[\cs{p}]{\ty{\Gamma}}{V}{T}
        \\
        \pr(\ty{T})=\pr(\ty{U})
      }{\tseq[\cs{p}]{\ty{\Gamma}}{\inl{V}}{\tysum{T}{U}}}
    \end{mathpar}
  \end{case*}
  \begin{case*}[\LabTirName{T-Inr}]
    The induction hypothesis gives us $\cs{p}=\cs{\pbot}$, and $\pr(\ty{\Gamma})=\pr{\ty{U}}$. We know $\pr(\ty{T})=\pr({\ty{U}})$, hence $\pr(\ty{\Gamma})=\pr(\ty{\tysum{T}{U}})$.
    \begin{mathpar}
      \inferrule*{
        \tseq[\cs{p}]{\ty{\Gamma}}{V}{U}
        \\
        \pr(\ty{T})=\pr(\ty{U})
      }{\tseq[\cs{p}]{\ty{\Gamma}}{\inr{V}}{\tysum{T}{U}}}
    \end{mathpar}
  \end{case*}
\end{proof}

%%% Local Variables:
%%% TeX-master: "../priorities"
%%% End:


  \restatelemma{lempgvsubstitution}
  \begin{proof}
  By induction on the derivation of $\tseq[\cs{p}]{\ty{\Gamma},\tmty{x}{U'}}{M}{T}$.
  \begin{case*}[\LabTirName{T-Var}]
    By \cref{lem:value-done}, $\cs{q}=\cs{\pbot}$.
    \begin{mathpar}
      \inferrule*{
      }{\tseq[\cs{\pbot}]{\tmty{x}{U'}}{x}{U'}}
      \substarrow{V}{x}
      \tseq[\cs{\pbot}]{\ty{\Theta}}{V}{U'}
    \end{mathpar}
  \end{case*}
  \begin{case*}[\LabTirName{T-Lam}]
    By \cref{lem:value-done}, $\pr(\ty{\Theta})=\pr(\ty{U'})$, hence $\pr(\ty{\Gamma},\ty{\Theta})=\pr(\ty{\Gamma},\ty{U'})$.
    \begin{mathpar}
      \inferrule*{
        \tseq[\cs{o}]{\ty{\Gamma},\tmty{x}{U'},\tmty{y}{T}}{M}{U}
      }{\tseq[\cs{\pbot}]{\ty{\Gamma},\tmty{x}{U'}}
        {\lambda y.M}
        {\tylolli[\cs{\pr(\ty{\Gamma},\ty{U'})},\cs{o}]{T}{U}}}
      \substarrow{V}{x}
      \inferrule*{
        \tseq[\cs{o}]{\ty{\Gamma},\ty{\Theta},\tmty{y}{T}}{\subst{M}{V}{x}}{U}
      }{\tseq[\cs{\pbot}]{\ty{\Gamma},\ty{\Theta}}
        {\lambda y.\subst{M}{V}{x}}
        {\tylolli[\cs{\pr(\ty{\Gamma},\ty{\Theta})},\cs{o}]{T}{U}}}
    \end{mathpar}
  \end{case*}
  \begin{case*}[\LabTirName{T-App}]
    There are two subcases:
    \begin{subcase*}[$\tm{x}\in\tm{M}$]
      Immediately, from the induction hypothesis.
      \begin{mathpar}
        \inferrule*{
          \tseq[\cs{p}]{\ty{\Gamma},\tmty{x}{U'}}{M}{\tylolli[\cs{o},\cs{r}]{T}{U}}
          \\
          \tseq[\cs{q}]{\ty{\Delta}}{N}{T}
          \\
          \cs{p}<\pr(\ty{\Delta})
          \\
          \cs{q}<\cs{o}
        }{\tseq[\cs{p}\sqcup\cs{q}\sqcup\cs{r}]{\ty{\Gamma},\ty{\Delta},\tmty{x}{U'}}{M\;N}{U}}
        \substarrow{V}{x}
        \inferrule*{
          \tseq[\cs{p}]{\ty{\Gamma},\ty{\Theta}}{\subst{M}{V}{x}}{\tylolli[\cs{o},\cs{r}]{T}{U}}
          \\
          \tseq[\cs{q}]{\ty{\Delta}}{N}{T}
          \\
          \cs{p}<\pr(\ty{\Delta})
          \\
          \cs{q}<\cs{o}
        }{\tseq[\cs{p}\sqcup\cs{q}\sqcup\cs{r}]{\ty{\Gamma},\ty{\Delta},\ty{\Theta}}{(\subst{M}{V}{x})\;N}{U}}
      \end{mathpar}
    \end{subcase*}
    \begin{subcase*}[$\tm{x}\in\tm{N}$]
      By \cref{lem:value-done}, $\pr(\ty{\Theta})=\pr(\ty{U'})$, hence $\pr(\ty{\Delta},\ty{\Theta})=\pr(\ty{\Delta},\ty{U'})$.
      \begin{mathpar}
        \inferrule*{
          \tseq[\cs{p}]{\ty{\Gamma}}{M}{\tylolli[\cs{o},\cs{r}]{T}{U}}
          \\
          \tseq[\cs{q}]{\ty{\Delta},\tmty{x}{U'}}{N}{T}
          \\
          \cs{p}<\pr(\ty{\Delta},\ty{U'})
          \\
          \cs{q}<\cs{o}
        }{\tseq[\cs{p}\sqcup\cs{q}\sqcup\cs{r}]{\ty{\Gamma},\ty{\Delta},\tmty{x}{U'}}{M\;N}{U}}
        \substarrow{V}{x}
        \inferrule*{
          \tseq[\cs{p}]{\ty{\Gamma}}{M}{\tylolli[\cs{o},\cs{r}]{T}{U}}
          \\
          \tseq[\cs{q}]{\ty{\Delta},\ty{\Theta}}{\subst{N}{V}{x}}{T}
          \\
          \cs{p}<\pr(\ty{\Delta},\ty{\Theta})
          \\
          \cs{q}<\cs{o}
        }{\tseq[\cs{p}\sqcup\cs{q}\sqcup\cs{r}]{\ty{\Gamma},\ty{\Delta},\ty{\Theta}}{M\;(\subst{N}{V}{x})}{U}}
      \end{mathpar}
    \end{subcase*}
  \end{case*}
  \begin{case*}[\LabTirName{T-LetUnit}]
    There are two subcases:
    \begin{subcase*}[$\tm{x}\in\tm{M}$]
      Immediately, from the induction hypothesis.
      \begin{mathpar}
        \inferrule*{
          \tseq[\cs{p}]{\ty{\Gamma},\tmty{x}{U'}}{M}{\tyunit}
          \\
          \tseq[\cs{q}]{\ty{\Delta}}{N}{T}
          \\
          \cs{p}<\pr(\ty{\Delta})
        }{\tseq[\cs{p}\sqcup\cs{q}]{\ty{\Gamma},\ty{\Delta},\tmty{x}{U'}}{\letunit{M}{N}}{T}}
        \substarrow{V}{x}
        \inferrule*{
          \tseq[\cs{p}]{\ty{\Gamma},\ty{\Theta}}{\subst{M}{V}{x}}{\tyunit}
          \\
          \tseq[\cs{q}]{\ty{\Delta}}{N}{T}
          \\
          \cs{p}<\pr(\ty{\Delta})
        }{\tseq[\cs{p}\sqcup\cs{q}]{\ty{\Gamma},\ty{\Delta},\ty{\Theta}}{\letunit{\subst{M}{V}{x}}{N}}{T}}
      \end{mathpar}
    \end{subcase*}
    \begin{subcase*}[$\tm{x}\in\tm{N}$]
      By \cref{lem:value-done}, $\pr(\ty{\Theta})=\pr(\ty{U'})$, hence $\pr(\ty{\Delta},\ty{\Theta})=\pr(\ty{\Delta},\ty{U'})$.
      \begin{mathpar}
        \inferrule*{
          \tseq[\cs{p}]{\ty{\Gamma}}{M}{\tyunit}
          \\
          \tseq[\cs{q}]{\ty{\Delta},\tmty{x}{U'}}{N}{T}
          \\
          \cs{p}<\pr(\ty{\Delta},\ty{U'})
        }{\tseq[\cs{p}\sqcup\cs{q}]{\ty{\Gamma},\ty{\Delta},\tmty{x}{U'}}{\letunit{M}{N}}{T}}
        \substarrow{V}{x}
        \inferrule*{
          \tseq[\cs{p}]{\ty{\Gamma}}{M}{\tyunit}
          \\
          \tseq[\cs{q}]{\ty{\Delta},\ty{\Theta}}{\subst{N}{V}{x}}{T}
          \\
          \cs{p}<\pr(\ty{\Delta},\ty{\Theta})
        }{\tseq[\cs{p}\sqcup\cs{q}]{\ty{\Gamma},\ty{\Delta},\ty{\Theta}}{\letunit{M}{\subst{N}{V}{x}}}{T}}
      \end{mathpar}
    \end{subcase*}
  \end{case*}
  \begin{case*}[\LabTirName{T-Pair}]
    There are two subcases:
    \begin{subcase*}[$\tm{x}\in\tm{M}$]
      Immediately, from the induction hypothesis.
      \begin{mathpar}
        \inferrule*{
          \tseq[\cs{p}]{\ty{\Gamma},\tmty{x}{U'}}{M}{T}
          \\
          \tseq[\cs{q}]{\ty{\Delta}}{N}{U}
          \\
          \cs{p}<\pr(\ty{\Delta},\ty{U'})
        }{\tseq[\cs{p}\sqcup\cs{q}]{\ty{\Gamma},\ty{\Delta},\tmty{x}{U'}}{\pair{M}{N}}{\typrod{T}{U}}}
        \substarrow{V}{x}
        \inferrule*{
          \tseq[\cs{p}]{\ty{\Gamma},\ty{\Theta}}{\subst{M}{V}{x}}{T}
          \\
          \tseq[\cs{q}]{\ty{\Delta}}{N}{U}
          \\
          \cs{p}<\pr(\ty{\Delta},\ty{\Theta})
        }{\tseq[\cs{p}\sqcup\cs{q}]{\ty{\Gamma},\ty{\Delta},\ty{\Theta}}{\pair{\subst{M}{V}{x}}{N}}{\typrod{T}{U}}}
      \end{mathpar}
    \end{subcase*}
    \begin{subcase*}[$\tm{x}\in\tm{N}$]
      By \cref{lem:value-done}, $\pr(\ty{\Theta})=\pr(\ty{U'})$, hence $\pr(\ty{\Delta},\ty{\Theta})=\pr(\ty{\Delta},\ty{U'})$.
      \begin{mathpar}
        \inferrule*{
          \tseq[\cs{p}]{\ty{\Gamma}}{M}{T}
          \\
          \tseq[\cs{q}]{\ty{\Delta},\tmty{x}{U'}}{N}{U}
          \\
          \cs{p}<\pr(\ty{\Delta},\ty{U'})
        }{\tseq[\cs{p}\sqcup\cs{q}]{\ty{\Gamma},\ty{\Delta},\tmty{x}{U'}}{\pair{M}{N}}{\typrod{T}{U}}}
        \substarrow{V}{x}
        \inferrule*{
          \tseq[\cs{p}]{\ty{\Gamma}}{M}{T}
          \\
          \tseq[\cs{q}]{\ty{\Delta},\ty{\Theta}}{\subst{N}{V}{x}}{U}
          \\
          \cs{p}<\pr(\ty{\Delta},\ty{\Theta})
        }{\tseq[\cs{p}\sqcup\cs{q}]{\ty{\Gamma},\ty{\Delta},\ty{\Theta}}{\pair{M}{\subst{N}{V}{x}}}{\typrod{T}{U}}}
      \end{mathpar}
    \end{subcase*}
  \end{case*}
  \begin{case*}[\LabTirName{T-LetPair}]
    There are two subcases:
    \begin{subcase*}[$\tm{x}\in\tm{M}$]
      Immediately, from the induction hypothesis.
      \begin{mathpar}
        \inferrule*{
          \tseq[\cs{p}]{\ty{\Gamma},\tmty{x}{U'}}{M}{\typrod{T}{T'}}
          \\
          \tseq[\cs{q}]{\ty{\Delta},\tmty{y}{T},\tmty{z}{T'}}{N}{U}
          \\
          \cs{p}<\pr(\ty{\Delta},\ty{T},\ty{T'})
        }{\tseq[\cs{p}\sqcup\cs{q}]{\ty{\Gamma},\ty{\Delta},\tmty{x}{U'}}{\letpair{y}{z}{M}{N}}{U}}
        \substarrow{V}{x}
        \inferrule*{
          \tseq[\cs{p}]{\ty{\Gamma},\ty{\Theta}}{\subst{M}{V}{x}}{\typrod{T}{T'}}
          \\
          \tseq[\cs{q}]{\ty{\Delta},\tmty{y}{T},\tmty{z}{T'}}{N}{U}
          \\
          \cs{p}<\pr(\ty{\Delta},\ty{T},\ty{T'})
        }{\tseq[\cs{p}\sqcup\cs{q}]{\ty{\Gamma},\ty{\Delta},\ty{\Theta}}{\letpair{y}{z}{\subst{M}{V}{x}}{N}}{U}}
      \end{mathpar}
    \end{subcase*}
    \begin{subcase*}[$\tm{x}\in\tm{N}$]
      By \cref{lem:value-done}, $\pr(\ty{\Theta})=\pr(\ty{U'})$, hence $\pr(\ty{\Delta},\ty{\Theta},\ty{T},\ty{T'})=\pr(\ty{\Delta},\ty{U'},\ty{T},\ty{T'})$.
      \begin{mathpar}
        \inferrule*{
          \tseq[\cs{p}]{\ty{\Gamma}}{M}{\typrod{T}{T'}}
          \\
          \tseq[\cs{q}]{\ty{\Delta},\tmty{x}{U'},\tmty{y}{T},\tmty{z}{T'}}{N}{U}
          \\
          \cs{p}<\pr(\ty{\Delta},\ty{U'},\ty{T},\ty{T'})
        }{\tseq[\cs{p}\sqcup\cs{q}]{\ty{\Gamma},\ty{\Delta},\tmty{x}{U'}}{\letpair{y}{z}{M}{N}}{U}}
        \substarrow{V}{x}
        \inferrule*{
          \tseq[\cs{p}]{\ty{\Gamma}}{M}{\typrod{T}{T'}}
          \\
          \tseq[\cs{q}]{\ty{\Delta},\ty{\Theta},\tmty{y}{T},\tmty{z}{T'}}{\subst{N}{V}{x}}{U}
          \\
          \cs{p}<\pr(\ty{\Delta},\ty{\Theta},\ty{T},\ty{T'})
        }{\tseq[\cs{p}\sqcup\cs{q}]{\ty{\Gamma},\ty{\Delta},\ty{\Theta}}{\letpair{y}{z}{M}{\subst{N}{V}{x}}}{U}}
      \end{mathpar}
    \end{subcase*}
  \end{case*}
  \begin{case*}[\LabTirName{T-Absurd}]
    \begin{mathpar}
      \inferrule*{
        \tseq[o]{\ty{\Gamma},\tmty{x}{U'}}{M}{\tyvoid}
      }{\tseq[o]{\ty{\Gamma},\ty{\Delta},\tmty{x}{U'}}{\absurd{M}}{T}}
      \substarrow{V}{x}
      \inferrule*{
        \tseq[o]{\ty{\Gamma},\ty{\Theta}}{\subst{M}{V}{x}}{\tyvoid}
      }{\tseq[o]{\ty{\Gamma},\ty{\Delta},\ty{\Theta}}{\absurd{\subst{M}{V}{x}}}{T}}
    \end{mathpar}
  \end{case*}
  \begin{case*}[\LabTirName{T-Inl}]
    \begin{mathpar}
      \inferrule*{
        \tseq[o]{\ty{\Gamma},\tmty{x}{U'}}{M}{T}
        \\
        \pr(\ty{T})=\pr(\ty{U})
      }{\tseq[o]{\ty{\Gamma},\tmty{x}{U'}}{\inl{M}}{\tysum{T}{U}}}
      \substarrow{V}{x}
      \inferrule*{
        \tseq[o]{\ty{\Gamma},\ty{\Theta}}{\subst{M}{V}{x}}{T}
        \\
        \pr(\ty{T})=\pr(\ty{U})
      }{\tseq[o]{\ty{\Gamma},\ty{\Theta}}{\inl{\subst{M}{V}{x}}}{\tysum{T}{U}}}
    \end{mathpar}
  \end{case*}
  \begin{case*}[\LabTirName{T-Inr}]
    \begin{mathpar}
      \inferrule*{
        \tseq[o]{\ty{\Gamma},\tmty{x}{U'}}{M}{U}
        \\
        \pr(\ty{T})=\pr(\ty{U})
      }{\tseq[o]{\ty{\Gamma},\tmty{x}{U'}}{\inr{M}}{\tysum{T}{U}}}
      \substarrow{V}{x}
      \inferrule*{
        \tseq[o]{\ty{\Gamma},\ty{\Theta}}{\subst{M}{V}{x}}{U}
        \\
        \pr(\ty{T})=\pr(\ty{U})
      }{\tseq[o]{\ty{\Gamma},\ty{\Theta}}{\inr{\subst{M}{V}{x}}}{\tysum{T}{U}}}
    \end{mathpar}
  \end{case*}
  \begin{case*}[\LabTirName{T-CaseSum}]
    There are two subcases:
    \begin{subcase*}[$\tm{x}\in\tm{L}$]
      Immediately, from the induction hypothesis.
      \begin{mathpar}
        \inferrule*{
          \tseq[\cs{p}]{\ty{\Gamma},\tmty{x}{U'}}{L}{\tysum{T}{T'}}
          \\
          \tseq[\cs{q}]{\ty{\Delta},\tmty{y}{T}}{M}{U}
          \\
          \tseq[\cs{q}]{\ty{\Delta},\tmty{z}{T'}}{N}{U}
          \\
          \cs{p}<\pr(\ty{\Delta})
        }{\tseq[\cs{p}\sqcup\cs{q}]{\ty{\Gamma},\ty{\Delta},\tmty{x}{U'}}{\casesum{L}{y}{M}{z}{N}}{U}}
        \substarrow{V}{x}
        \inferrule*{
          \tseq[\cs{p}]{\ty{\Gamma},\ty{\Theta}}{\subst{L}{V}{x}}{\tysum{T}{T'}}
          \\
          \tseq[\cs{q}]{\ty{\Delta},\tmty{y}{T}}{M}{U}
          \\
          \tseq[\cs{q}]{\ty{\Delta},\tmty{z}{T'}}{N}{U}
          \\
          \cs{p}<\pr(\ty{\Delta})
        }{\tseq[\cs{p}\sqcup\cs{q}]{\ty{\Gamma},\ty{\Delta},\ty{\Theta}}{\casesum{\subst{L}{V}{x}}{y}{M}{z}{N}}{U}}
      \end{mathpar}
    \end{subcase*}
    \begin{subcase*}[$\tm{x}\in\tm{M}$ and $\tm{x}\in\tm{N}$]
      By \cref{lem:value-done}, $\pr(\ty{\Theta})=\pr(\ty{U'})$, hence $\pr(\ty{\Delta},\ty{\Theta},\ty{T})=\pr(\ty{\Delta},\ty{U'},\ty{T})$ and $\pr(\ty{\Delta},\ty{\Theta},\ty{T'})=\pr(\ty{\Delta},\ty{U'},\ty{T'})$.
      \begin{mathpar}
        \inferrule*{
          \tseq[\cs{p}]{\ty{\Gamma}}{L}{\tysum{T}{T'}}
          \\
          \tseq[\cs{q}]{\ty{\Delta},\tmty{x}{U'},\tmty{y}{T}}{M}{U}
          \\
          \tseq[\cs{q}]{\ty{\Delta},\tmty{x}{U'},\tmty{z}{T'}}{N}{U}
          \\
          \cs{p}<\pr(\ty{\Delta},\ty{U'})
        }{\tseq[\cs{p}\sqcup\cs{q}]{\ty{\Gamma},\ty{\Delta},\tmty{x}{U'}}{\casesum{L}{y}{M}{z}{N}}{U}}
        \substarrow{V}{x}
        \inferrule*{
          \tseq[\cs{p}]{\ty{\Gamma}}{L}{\tysum{T}{T'}}
          \\
          \tseq[\cs{q}]{\ty{\Delta},\ty{\Theta},\tmty{y}{T}}{\subst{M}{V}{x}}{U}
          \\
          \tseq[\cs{q}]{\ty{\Delta},\ty{\Theta},\tmty{z}{T'}}{\subst{N}{V}{x}}{U}
          \\
          \cs{p}<\pr(\ty{\Delta},\ty{\Theta})
        }{\tseq[\cs{p}\sqcup\cs{q}]{\ty{\Gamma},\ty{\Delta},\ty{\Theta}}{\casesum{L}{y}{\subst{M}{V}{x}}{z}{\subst{N}{V}{x}}}{U}}
      \end{mathpar}
    \end{subcase*}
  \end{case*}
  \noindent
  We omit the cases where $\tm{x}\not\in\tm{M}$.
\end{proof}

%%% Local Variables:
%%% TeX-master: "../priorities"
%%% End:


  \restatelemma{lempgvsubjectreductionterms}
  \begin{proof}
  \label{prf:lem-pgv-subject-reduction-terms}
  By induction on the derivation of $\tm{M}\tred\tm{M'}$.

  \begin{case*}[\LabTirName{E-Lam}]
    By \cref{lem:pgv-substitution}.
    \begin{mathpar}
      \inferrule*{
        \inferrule*{
          \tseq[\cs{p}]{\ty{\Gamma},\tmty{x}{T}}{M}{U}
        }{\tseq[\cs{\pbot}]{\ty{\Gamma}}{\lambda x.M}{\tylolli[\cs{\pr(\ty{\Gamma})},\cs{p}]{T}{U}}}
        \\
        \tseq[\cs{\pbot}]{\ty{\Delta}}{V}{T}
      }{\tseq[\cs{p}]{\ty{\Gamma},\ty{\Delta}}{(\lambda x.M)\;V}{U}}
      \tred
      \tseq[\cs{p}]{\ty{\Gamma},\ty{\Delta}}{\subst{M}{V}{x}}{U}
    \end{mathpar}
  \end{case*}
  \begin{case*}[\LabTirName{E-Unit}]
    By \cref{lem:pgv-substitution}.
    \begin{mathpar}
      \inferrule*{
        \inferrule*{
        }{\tseq[\cs{\pbot}]{\emptyenv}{\unit}{\tyunit}}
        \\
        \tseq[\cs{p}]{\ty{\Gamma}}{M}{T}
      }{\tseq[\cs{p}]{\ty{\Gamma}}{\letunit{\unit}{M}}{T}}
      \tred
      \tseq[\cs{p}]{\ty{\Gamma}}{M}{T}
    \end{mathpar}
  \end{case*}
  \begin{case*}[\LabTirName{E-Pair}]
    By \cref{lem:pgv-substitution}.
    \begin{mathpar}
      \inferrule*{
        \inferrule*{
          \tseq[\cs{\pbot}]{\ty{\Gamma}}{V}{T}
          \\
          \tseq[\cs{\pbot}]{\ty{\Delta}}{W}{T'}
        }{\tseq[\cs{\pbot}]{\ty{\Gamma},\ty{\Delta}}{\pair{V}{W}}{\typrod{T}{T'}}}
        \\
        \tseq[\cs{p}]{\ty{\Theta},\tmty{x}{T},\tmty{y}{T'}}{M}{U}
      }{\tseq[]{\ty{\Gamma},\ty{\Delta},\ty{\Theta}}{\letpair{x}{y}{\pair{V}{W}}{M}}{U}}
      \\
      \begin{turn}{270}
        \tred
      \end{turn}
      \\
      \tseq[\cs{p}]{\ty{\Gamma},\ty{\Delta},\ty{\Theta}}{\subst{\subst{M}{V}{x}}{W}{y}}{U}
    \end{mathpar}
  \end{case*}
  \begin{case*}[\LabTirName{E-Inl}]
    By \cref{lem:pgv-substitution}.
    \begin{mathpar}
      \inferrule*{
        \inferrule*{
          \tseq[\cs{\pbot}]{\ty{\Gamma}}{V}{T}
        }{\tseq[\cs{\pbot}]{\ty{\Gamma}}{\inl{V}}{\tysum{T}{T'}}}
        \\
        \tseq[\cs{p}]{\ty{\Delta},\tmty{x}{T}}{M}{U}
        \\
        \tseq[\cs{p}]{\ty{\Delta},\tmty{y}{T'}}{N}{U}
      }{\tseq[\cs{p}]{\ty{\Gamma},\ty{\Delta}}{\casesum{\inl{V}}{x}{M}{y}{N}}{U}}
      \\
      \begin{turn}{270}
        \tred
      \end{turn}
      \\
      \tseq[\cs{p}]{\ty{\Gamma},\ty{\Delta}}{\subst{M}{V}{x}}{U}
    \end{mathpar}
  \end{case*}
  \begin{case*}[\LabTirName{E-Inr}]
    By \cref{lem:pgv-substitution}.
    \begin{mathpar}
      \inferrule*{
        \inferrule*{
          \tseq[\cs{\pbot}]{\ty{\Gamma}}{V}{T'}
        }{\tseq[\cs{\pbot}]{\ty{\Gamma}}{\inr{V}}{\tysum{T}{T'}}}
        \\
        \tseq[\cs{p}]{\ty{\Delta},\tmty{x}{T}}{M}{U}
        \\
        \tseq[\cs{p}]{\ty{\Delta},\tmty{y}{T'}}{N}{U}
      }{\tseq[\cs{p}]{\ty{\Gamma},\ty{\Delta}}{\casesum{\inr{V}}{x}{M}{y}{N}}{U}}
      \\
      \begin{turn}{270}
        \tred
      \end{turn}
      \\
      \tseq[\cs{p}]{\ty{\Gamma},\ty{\Delta}}{\subst{N}{V}{y}}{U}
    \end{mathpar}
  \end{case*}
  \begin{case*}[\LabTirName{E-Lift}]
    By induction on the evaluation context $\tm{E}$.
  \end{case*}
\end{proof}

%%% Local Variables:
%%% TeX-master: "../priorities"
%%% End:


  \restatelemma{lempgvsubjectcongruence}
  \begin{proof}
  \label{prf:lem-pgv-subject-congruence}
  By induction on the derivation of $\tm{\conf{C}}\equiv\tm{\conf{C'}}$.

  \begin{case*}[\LabTirName{SC-LnkSwp}]
    \begin{mathpar}
      \inferrule*{
        \inferrule*[vdots=1.5em]{
          \inferrule*{
          }{\tmty{\link}{\tylolli{\typrod{S}{\co{S}}}{\tyunit}}}
          \\
          \inferrule*{
            \inferrule*{
            }{\tseq[\cs{\pbot}]{\tmty{x}{S}}{x}{S}}
            \\
            \inferrule*{
            }{\tseq[\cs{\pbot}]{\tmty{y}{\co{S}}}{y}{\co{S}}}
          }{\tseq[\cs{\pbot}]{\tmty{x}{S},\tmty{y}{\co{S}}}{\pair{x}{y}}{\typrod{S}{\co{S}}}}
        }{\tseq[\cs{\pbot}]{\tmty{x}{S},\tmty{y}{\co{S}}}{\link\;{\pair{x}{y}}}{\tyunit}}
      }{\cseq[\phi]{\ty{\Gamma},\tmty{x}{S},\tmty{y}{\co{S}}}{\plug{\conf{F}}{\link\;{\pair{x}{y}}}}}
      \\
      \begin{turn}{270}
        \ensuremath{\equiv}
      \end{turn}
      \\
      \inferrule*{
        \inferrule*[vdots=1.5em]{
          \inferrule*{
          }{\tmty{\link}{\tylolli{\typrod{\co{S}}{S}}{\tyunit}}}
          \\
          \inferrule*{
            \inferrule*{
            }{\tseq[\cs{\pbot}]{\tmty{y}{\co{S}}}{y}{\co{S}}}
            \\
            \inferrule*{
            }{\tseq[\cs{\pbot}]{\tmty{x}{S}}{x}{S}}
          }{\tseq[\cs{\pbot}]{\tmty{x}{S},\tmty{y}{\co{S}}}{\pair{y}{x}}{\typrod{S}{\co{S}}}}
        }{\tseq[\cs{\pbot}]{\tmty{x}{S},\tmty{y}{\co{S}}}{\link\;{\pair{y}{x}}}{\tyunit}}
      }{\cseq[\phi]{\ty{\Gamma},\tmty{x}{S},\tmty{y}{\co{S}}}{\plug{\conf{F}}{\link\;{\pair{y}{x}}}}}
    \end{mathpar}
  \end{case*}

  \begin{case*}[\LabTirName{SC-ResExt}]
    \begin{mathpar}
      \inferrule*{
        \inferrule*{
          \cseq{\ty{\Gamma}}{\conf{C}}
          \\
          \cseq{\ty{\Delta},\tmty{x}{S},\tmty{y}{\co{S}}}{\conf{D}}
        }{\cseq{\ty{\Gamma},\ty{\Delta},\tmty{x}{S},\tmty{y}{\co{S}}}{(\ppar{\conf{C}}{\conf{D}})}}
      }{\cseq{\ty{\Gamma},\ty{\Delta}}{\res{x}{y}{(\ppar{\conf{C}}{\conf{D}}})}}
      \equiv
      \inferrule*{
        \cseq{\ty{\Gamma}}{\conf{C}}
        \\
        \inferrule*{
          \cseq{\ty{\Delta},\tmty{x}{S},\tmty{y}{\co{S}}}{\conf{D}}
        }{\cseq{\ty{\Delta}}{\res{x}{y}{\conf{D}}}}
      }{\cseq{\ty{\Gamma},\ty{\Delta}}{\ppar{\conf{C}}{\res{x}{y}{\conf{D}}}}}
    \end{mathpar}
  \end{case*}

  \begin{case*}[\LabTirName{SC-ResSwp}]
    \todo{Write proof.}
  \end{case*}

  \begin{case*}[\LabTirName{SC-ResCom}]
    \begin{mathpar}
      \inferrule*{
        \inferrule*{
          \cseq[\phi]{\ty{\Gamma},\tmty{x}{S},\tmty{y}{\co{S}},\tmty{z}{S'},\tmty{w}{\co{S'}}}{\conf{C}}
        }{\cseq[\phi]{\ty{\Gamma},\tmty{x}{S},\tmty{y}{\co{S}}}{\res{z}{w}{\conf{C}}}}
      }{\cseq[\phi]{\ty{\Gamma}}{\res{x}{y}{\res{z}{w}{\conf{C}}}}}
      \equiv
      \inferrule*{
        \inferrule*{
          \cseq[\phi]{\ty{\Gamma},\tmty{x}{S},\tmty{y}{\co{S}},\tmty{z}{S'},\tmty{w}{\co{S'}}}{\conf{C}}
        }{\cseq[\phi]{\ty{\Gamma},\tmty{z}{S'},\tmty{w}{\co{S'}}}{\res{x}{y}{\conf{C}}}}
      }{\cseq[\phi]{\ty{\Gamma}}{\cseq{}{\res{z}{w}{\res{x}{y}{\conf{C}}}}}}
    \end{mathpar}
  \end{case*}

  \begin{case*}[\LabTirName{SC-ParNil}]
    \begin{mathpar}
      \inferrule*{
        \cseq[\phi]{\ty{\Gamma}}{\conf{C}}
        \\
        \inferrule*{
          \inferrule*{
          }{\tseq[\cs{\pbot}]{\emptyenv}{\unit}{\tyunit}}
        }{\cseq[\child]{\emptyenv}{\child{\unit}}}
      }{\cseq[\phi]{\ty{\Gamma}}{\ppar{\conf{C}}{\child{\unit}}}}
      \equiv
      \cseq[\phi]{\ty{\Gamma}}{\conf{C}}
    \end{mathpar}
  \end{case*}

  \begin{case*}[\LabTirName{SC-ParCom}]
    \begin{mathpar}
      \inferrule*{
        \cseq[\phi]{\ty{\Gamma}}{\conf{C}}
        \\
        \cseq[\phi']{\ty{\Delta}}{\conf{D}}
      }{\cseq[\phi+\phi']{\ty{\Gamma},\ty{\Delta}}{(\ppar{\conf{C}}{\conf{D}})}}
      \equiv
      \inferrule*{
        \cseq[\phi']{\ty{\Delta}}{\conf{D}}
        \\
        \cseq[\phi]{\ty{\Gamma}}{\conf{C}}
      }{\cseq[\phi'+\phi]{\ty{\Gamma},\ty{\Delta}}{(\ppar{\conf{D}}{\conf{C}})}}
    \end{mathpar}
  \end{case*}


  \begin{case*}[\LabTirName{SC-ParAsc}]
    \begin{mathpar}
      \inferrule*{
        \cseq[\phi]{\ty{\Gamma}}{\conf{C}}
        \\
        \inferrule*{
          \cseq[\phi']{\ty{\Delta}}{\conf{D}}
          \\
          \cseq[\phi'']{\ty{\Theta}}{\conf{E}}
        }{\cseq[\phi'+\phi'']{\ty{\Delta},\ty{\Theta}}{(\ppar{\conf{D}}{\conf{E}})}}
      }{\cseq[\phi+\phi'+\phi'']{\ty{\Gamma},\ty{\Delta},\ty{\Theta}}{\ppar{\conf{C}}{(\ppar{\conf{D}}{\conf{E}})}}}
      \equiv
      \inferrule*{
        \inferrule*{
          \cseq[\phi]{\ty{\Gamma}}{\conf{C}}
          \\
          \cseq[\phi']{\ty{\Delta}}{\conf{D}}
        }{\cseq[\phi+\phi']{\ty{\Gamma},\ty{\Delta}}{(\ppar{\conf{C}}{\conf{D}})}}
        \\
        \cseq[\phi'']{\ty{\Theta}}{\conf{E}}
      }{\cseq[\phi+\phi'+\phi'']{\ty{\Gamma},\ty{\Delta},\ty{\Theta}}{\ppar{(\ppar{\conf{C}}{\conf{D}})}{\conf{E}}}}
    \end{mathpar}
  \end{case*}
\end{proof}

%%% Local Variables:
%%% TeX-master: "../priorities"
%%% End:


  \restatetheorem{thmpgvsubjectreductionconfs}
  \begin{proof}
  \begin{case}[\LabTirName{E-New}]
    \small
    \begin{mathpar}
      \inferrule*{
        \inferrule*[vdots=1.5em]{
          \inferrule*{
          }{\tmty{\new}{\tylolli{\tyunit}{\typrod{S}{\co{S}}}}}
          \\
          \inferrule*{
          }{\tseq[\cs{\pbot}]{\emptyenv}{\unit}{\tyunit}}
        }{\tseq[\cs{\pbot}]{\emptyenv}{\new\;\unit}{\typrod{S}{\co{S}}}}
      }{\cseq[\phi]{\ty{\Gamma}}{\plug{\conf{F}}{\new\;\unit}}}
      \cred
      \inferrule*{
        \inferrule*{
          \inferrule*[vdots=1.5em]{
            \inferrule*{
            }{\tseq[\cs{\pbot}]{\tmty{x}{S}}{x}{S}}
            \\
            \inferrule*{
            }{\tseq[\cs{\pbot}]{\tmty{y}{\co{S}}}{y}{\co{S}}}
          }{\tseq[\cs{\pbot}]{\tmty{x}{S},\tmty{y}{\co{S}}}{\pair{x}{y}}{\typrod{S}{\co{S}}}}
        }{\cseq[\phi]{\ty{\Gamma},\tmty{x}{S},\tmty{y}{\co{S}}}{\plug{\conf{F}}{\pair{x}{y}}}}
      }{\cseq[\phi]{\ty{\Gamma}}{\res{x}{y}{\plug{\conf{F}}{\pair{x}{y}}}}}
  \end{mathpar}
  \end{case}
  \begin{case}[\LabTirName{E-Spawn}]
    \small
    \begin{mathpar}
      \inferrule*{
        \inferrule*[vdots=1.5em]{
          \inferrule*{
          }{\tmty{\spawn}{\tylolli{(\tylolli[\cs{p},\cs{q}]{\tyunit}{\tyunit})}{\tyunit}}}
          \tseq[\cs{\pbot}]{\ty{\Delta}}{V}{\tylolli[\cs{p},\cs{q}]{\tyunit}{\tyunit}}
        }{\tseq[\cs{\pbot}]{\ty{\Delta}}{\spawn\;V}{\tyunit}}
      }{\cseq[\phi]{\ty{\Gamma},\ty{\Delta}}{\plug{\conf{F}}{\spawn\;V}}}
      \\
      \begin{turn}{270}
        \cred
      \end{turn}
      \\
      \inferrule*{
        \inferrule*{
          \inferrule*[vdots=1.5em]{
          }{\tseq[\cs{\pbot}]{\emptyenv}{\unit}{\tyunit}}
        }{\cseq[\phi]{\ty{\Gamma}}{\plug{\conf{F}}{\unit}}}
        \\
        \inferrule*{
          \inferrule*{
            \tseq[\cs{\pbot}]{\ty{\Delta}}{V}{\tylolli[\cs{p},\cs{q}]{\tyunit}{\tyunit}}
            \\
            \inferrule*{
            }{\tseq[\cs{\pbot}]{\emptyenv}{\unit}{\tyunit}}
          }{\tseq[\cs{q}]{\ty{\Delta}}{V\;\unit}{\tyunit}}
        }{\cseq[\child]{\ty{\Delta}}{\child\;(V\;\unit)}}
      }{\cseq[\phi]{\ty{\Gamma},\ty{\Delta}}{\ppar{\plug{\conf{F}}{\unit}}{\child\;(V\;\unit)}}}
    \end{mathpar}
  \end{case}
  \begin{case}[\LabTirName{E-Send}]
    See \cref{fig:pgv-subject-reduction-cred-send}.
  \end{case}
  \begin{case}[\LabTirName{E-Close}]
    See \cref{fig:pgv-subject-reduction-cred-close}.
  \end{case}
  \begin{case}[\LabTirName{E-LiftC}]
    \todo{By induction on the evaluation context $\conf{G}$.}
  \end{case}
  \begin{case}[\LabTirName{E-LiftM}]
    By \cref{thm:pgv-subject-reduction-terms}.
  \end{case}
  \begin{case}[\LabTirName{E-LiftE}]
    By \cref{thm:pgv-subject-congruence}.
  \end{case}
\end{proof}
\begin{landscape}
\begin{figure}[ht!]
     \small
    \begin{mathpar}
      \inferrule*{
        \inferrule*{
          \inferrule*{
            \inferrule*[vdots=1.5em]{
              \inferrule*{
              }{\tmty{\send}{\tylolli[\cs{\ptop},\cs{o}]{\typrod{T}{\tysend[\cs{o}]{T}{S}}}{S}}}
              \inferrule*{
                \tseq[\cs{p}]{\ty{\Delta}}{V}{T}
                \\
                \inferrule*{
                }{\tseq[\cs{\pbot}]{\tmty{x}{\tysend[\cs{o}]{T}{S}}}{x}{\tysend[\cs{o}]{T}{S}}}
              }{\tseq[\cs{p}]{\ty{\Delta},\tmty{x}{\tysend[\cs{o}]{T}{S}}}
                {\pair{V}{x}}{\typrod{T}{\tysend[\cs{o}]{T}{S}}}}
            }{\tseq[\cs{p}\sqcup\cs{o}]{\ty{\Delta},\tmty{x}{\tysend[\cs{o}]{T}{S}}}{\send\;{\pair{V}{x}}}{S}}
          }{\cseq[\phi]
            {\ty{\Gamma},\ty{\Delta},\tmty{x}{\tysend[\cs{o}]{T}{S}}}
            {\plug{\conf{F}}{\send\;{\pair{V}{x}}}}}
          \\
          \inferrule*{
            \inferrule*[vdots=1.5em]{
              \inferrule*{
              }{\tmty{\recv}{\tylolli[\cs{\ptop},\cs{o}]{\tyrecv[\cs{o}]{T}{\co{S}}}{\typrod{T}{\co{S}}}}}
              \\
              \inferrule*{
              }{\tseq[\cs{\pbot}]{\tmty{y}{\tyrecv[\cs{o}]{T}{\co{S}}}}{y}{\tyrecv[\cs{o}]{T}{\co{S}}}}
            }{\tseq[\cs{o}]
              {\tmty{y}{\tyrecv[\cs{o}]{T}{\co{S}}}}
              {\recv\;y}
              {\typrod{T}{\co{S}}}}
          }{\cseq[\phi']
            {\ty{\Theta},\tmty{y}{\tyrecv[\cs{o}]{T}{\co{S}}}}
            {\plug{\conf{F'}}{\recv\;{y}}}}
        }{\cseq[\phi+\phi']
          {\ty{\Gamma},\ty{\Delta},\ty{\Theta},
            \tmty{x}{\tysend[\cs{o}]{T}{S}},\tmty{y}{\tyrecv[\cs{o}]{T}{\co{S}}}}
          {\ppar{\plug{\conf{F}}{\send\;{\pair{V}{x}}}}{\plug{\conf{F'}}{\recv\;{y}}}}}
      }{\cseq[\phi+\phi']
        {\ty{\Gamma},\ty{\Delta},\ty{\Theta}}
        {\res{x}{y}{(\ppar
            {\plug{\conf{F}}{\send\;{\pair{V}{x}}}}
            {\plug{\conf{F'}}{\recv\;{y}}})}}}
      \\
      \begin{turn}{270}
        \cred
      \end{turn}
      \\
      \inferrule*{
        \inferrule*{
          \inferrule*{
            \inferrule*[vdots=1.5em]{
            }{\tseq[\cs{\pbot}]{\tmty{x}{S}}{x}{S}}
          }{\cseq[\phi]{\ty{\Gamma},\tmty{x}{S}}{\plug{\conf{F}}{x}}}
          \\
          \inferrule*{
            \inferrule*[vdots=1.5em]{
              \tseq[\cs{p}]{\ty{\Delta}}{V}{T}
              \\
              \inferrule*{
              }{\tseq[\cs{\pbot}]{\ty{\Delta},\tmty{y}{\co{S}}}{y}{\co{S}}}
            }{\tseq[\cs{p}]{\ty{\Delta},\tmty{y}{\co{S}}}{\pair{V}{y}}{\typrod{T}{\co{S}}}}
          }{\cseq[\phi']
            {\ty{\Delta},\ty{\Theta},\tmty{y}{\co{S}}}
            {\plug{\conf{F'}}{\pair{V}{y}}}}
        }{\cseq[\phi+\phi']
          {\ty{\Gamma},\ty{\Delta},\ty{\Theta},
            \tmty{x}{S},\tmty{y}{\co{S}}}
          {\ppar
            {\plug{\conf{F}}{x}}
            {\plug{\conf{F'}}{\pair{V}{y}}}}}
      }{\cseq[\phi+\phi']
        {\ty{\Gamma},\ty{\Delta},\ty{\Theta}}
        {\res{x}{y}{(\ppar
            {\plug{\conf{F}}{x}}
            {\plug{\conf{F'}}{\pair{V}{y}}})}}}
    \end{mathpar}
  \caption{Subject Reduction (\LabTirName{E-Send})}
  \label{fig:pgv-subject-reduction-cred-send}
\end{figure}
\begin{figure}[ht!]
     \small
    \begin{mathpar}
      \inferrule*{
        \inferrule*{
          \inferrule*{
            \inferrule*[vdots=1.5em]{
              \inferrule*{
              }{\tmty{\close}{\tylolli[\cs{\ptop},\cs{o}]{\tyends[\cs{o}]}{\tyunit}}}
              \\
              \inferrule*{
              }{\tseq[\cs{\pbot}]{\tmty{x}{\tyends[\cs{o}]}}{x}{\tyends[\cs{o}]}}
            }{\tseq[\cs{o}]{\tmty{x}{\tyends[\cs{o}]}}{\close\;{x}}{\tyunit}}
          }{\cseq[\phi]
            {\ty{\Gamma},\tmty{x}{\tyends[\cs{o}]}}
            {\plug{\conf{F}}{\close\;{x}}}}
          \\
          \inferrule*{
            \inferrule*[vdots=1.5em]{
              \inferrule*{
              }{\tmty{\wait}{\tylolli[\cs{\ptop},\cs{o}]{\tyendr[\cs{o}]}{\tyunit}}}
              \\
              \inferrule*{
              }{\tseq[\cs{\pbot}]{\tmty{y}{\tyendr[\cs{o}]}}{y}{\tyendr[\cs{o}]}}
            }{\tseq[\cs{o}]{\tmty{y}{\tyendr[\cs{o}]}}{\wait\;{y}}{\tyunit}}
          }{\cseq[\phi']
            {\ty{\Delta},\tmty{y}{\tyendr[\cs{o}]}}
            {\plug{\conf{F'}}{\wait\;{y}}}}
        }{\cseq[\phi+\phi']
          {\ty{\Gamma},\ty{\Delta},
            \tmty{x}{\tyends[\cs{o}]},\tmty{y}{\tyendr[\cs{o}]}}
          {\ppar{\plug{\conf{F}}{\close\;{x}}}{\plug{\conf{F'}}{\wait\;{y}}}}}
      }{\cseq[\phi+\phi']
        {\ty{\Gamma},\ty{\Delta}}
        {\res{x}{y}{(\ppar
            {\plug{\conf{F}}{\close\;{x}}}
            {\plug{\conf{F'}}{\wait\;{y}}})}}}
      \\
      \begin{turn}{270}
        \cred
      \end{turn}
      \\
      \inferrule*{
        \inferrule*{
          \inferrule*[vdots=1.5em]{
          }{\tseq[\cs{\pbot}]{\emptyenv}{\unit}{\tyunit}}
        }{\cseq[\phi]{\ty{\Gamma}}{\plug{\conf{F}}{\unit}}}
        \\
        \inferrule*{
          \inferrule*[vdots=1.5em]{
          }{\tseq[\cs{\pbot}]{\emptyenv}{\unit}{\tyunit}}
        }{\cseq[\phi']
          {\ty{\Delta}}
          {\plug{\conf{F'}}{\unit}}}
      }{\cseq[\phi+\phi']
        {\ty{\Gamma},\ty{\Delta}}
        {\ppar{\plug{\conf{F}}{\unit}}{\plug{\conf{F'}}{\unit}}}}
    \end{mathpar}
  \caption{Subject Reduction (\LabTirName{E-Close})}
  \label{fig:pgv-subject-reduction-cred-close}
\end{figure}
\end{landscape}

%%% Local Variables:
%%% TeX-master: "../priorites"
%%% End:


  \begin{restatablelemma}{lempgvreadypriority}
    \label{lem:pgv-ready-priority}
    If $\tseq[\cs{p}]{\ty{\Gamma}}{L}{T}$ is ready to act on $\tmty{x}{S}\in\ty{\Gamma}$, then the priority bound $\cs{p}$ is some priority $\cs{o}$, \ie not $\cs{\pbot}$ or $\cs{\ptop}$.
  \end{restatablelemma}
  \begin{proof}
    Let $\tm{L}=\tm{\plug{E}{M}}$. By induction on the structure of $\tm{E}$. $\tm{M}$ has priority $\pr({\ty{S}})$, and each constructor of the evaluation context $\tm{E}$ passes on the \emph{maximum} of the priorities of its premises. No rule introduces the priority bound $\cs{\ptop}$ on the sequent.
  \end{proof}
}
\section{Relation to Priority CP}
{
  \usingnamespace{pcp}

  % Proofs that the syntactic sugar is well-typed.
  \begin{figure}[t]
  \begin{mathpar}
    \inferrule*[lab=T-Unbound-Send]{
      \seq{P}{\ty{\Gamma},\tmty{x}{B}}
    }{\seq{\usend{x}{y}{P}}{\ty{\Gamma},\tmty{x}{\tytens[\cs{o}]{A}{B}},\tmty{y}{\co{A}}}}
    \elabarrow
    \inferrule*{
      \inferrule*{
        \inferrule*{
        }{\seq{\link{y}{z}}{\tmty{y}{\co{A}},\tmty{z}{A}}}
        \\
        \seq{P}{\ty{\Gamma},\tmty{x}{B}}
      }{\seq{\ppar{\link{y}{z}}{P}}{\ty{\Gamma},\tmty{x}{B},\tmty{y}{\co{A}},\tmty{z}{A}}}
      \\
      \cs{o}<\minpr(\ty{\Gamma},\ty{A},\ty{B},\ty{\co{A}})
    }{\seq{\send{x}{z}{(\ppar{\link{y}{z}}{P})}}{\ty{\Gamma},\tmty{x}{\tytens[\cs{o}]{A}{B}},\tmty{y}{\co{A}}}}
  \end{mathpar}
  \caption{Typing Rules for Syntactic Sugar for PCP.}
  \label{fig:pcp-typing-sugar}
\end{figure}
%%% Local Variables:
%%% TeX-master: "../priorities"
%%% End:


  \begin{definition}[Actions]
    A~process acts on an endpoint $\tm{x}$ if it is in one of the following forms:
    \begin{multicols}{3}
      \begin{itemize}[noitemsep,topsep=0pt,parsep=0pt,partopsep=0pt]
      \item $\tm{\link{x}{y}}$ 
      \item $\tm{\link{y}{x}}$
      \item $\tm{\send{x}{y}{P}}$
      \item $\tm{\recv{x}{y}{P}}$
      \item $\tm{\close{x}{P}}$
      \item $\tm{\wait{x}{P}}$
      \item $\tm{\inl{x}{P}}$
      \item $\tm{\inr{x}{P}}$
      \item $\tm{\offer{x}{P}{Q}}$
      \item $\tm{\absurd{x}}$
      \end{itemize}
    \end{multicols}
    \noindent
    A~process is an action if it acts on some endpoint $\tm{x}$.
  \end{definition}
  
  \begin{definition}[Canonical Forms]
    \label{def:pcp-canonical-forms}
    A~process $\tm{P}$ is in canonical form if it is in one of the following forms:
    \begin{itemize}[noitemsep,topsep=0pt,parsep=0pt,partopsep=0pt]
    \item
      $\tm{\halt}$
    \item
      $\tm{\res{x_1}{x'_1}{\dots\res{x_n}{x'_n}{(P_1 \parallel\dots\parallel P_m)}}}$
      where $m>0$ and each $\tm{P_j}$ is an action.
    \end{itemize}
  \end{definition}
  
  \begin{restatablelemma}{lempcpcanonicalforms}[Canonical Forms]
    \label{lem:pcp-canonical-forms}
    If $\seq{P}{\ty{\Gamma}}$, there exists some $\tm{Q}$ such that $\tm{P}\equiv\tm{Q}$ and $\tm{Q}$ is in canonical form.
  \end{restatablelemma}
  \begin{proof}
    If $\tm{P}=\tm{\halt}$, we are done. Otherwise, we move any $\nu$-binders to the top using \LabTirName{SC-ResExt}, and discard any superfluous occurrences of $\tm{\halt}$ using \LabTirName{SC-ParNil}.
  \end{proof}

  \restatetheorem{thmpcpclosedprogress}
  \begin{proof}
  \label{prf:thm-pcp-closed-progress}
  By~\cref{lem:pcp-canonical-forms}, we rewrite $\tm{P}$ to canonical form. If the resulting process is $\tm{\halt}$, we are done. Otherwise, it is of the form
  \[
    \seq{\res{x_1}{x'_1}{\dots\res{x_n}{x'_n}{(P_1 \parallel\dots\parallel P_m)}}}{\emptyenv}
  \]
  where $m>0$ and each $\seq{P_i}{\ty{\Gamma_i}}$ is an action.

  Our proof follows the same reasoning by Kobayashi~\cite{kobayashi06} used in the proof of deadlock freedom for closed processes (Theorem 2). 

  Consider processes $\tm {P_1 \parallel\dots\parallel P_m}$. Among them, we pick the process with the smallest priority $\minpr{(\ty{\Gamma_i})}$ for all $i$. Let this process be let this be $\tm{P_i}$ and the priority of the top prefix be $\cs o$. $\tm{P_i}$ acts on some endpoint $\tmty{y}{A}\in\ty{\Gamma_i}$. We must have $\minpr{(\ty{\Gamma_i})}=\pr{(\ty{A})} = \cs o$, since the other actions in $\tm{P_i}$ are guarded by the action on $\tmty{y}{A}$, thus satisfying law (i) of priorities.

  If $\tm{P_i}$ is a link $\tm{\link{y}{z}}$ or $\tm{\link{z}{y}}$, we apply \LabTirName{E-Link}.

  Otherwise, $\tm{P_i}$ is an input/branching or output/selection action on endpoint $\tm y$ of type $\ty A$ with priority $\cs o$. Since process $\tm P$ is closed and consequently it respects law (ii) of priorities, there must be a co-action $y'$ of type $\ty{\co{A}}$  where $\tm{y}$ and $\tm{y'}$ are dual endpoints of the same channel (by application of rule \textsc{T-Res}). By duality, $\pr{(\ty{A})}=\pr{(\ty{\co{A}})}= \cs o$. In the following we show that: $y'$ is the subject of a top level action of a process $P_j$ with $i\neq j$. This allows for the communication among $\tm{P_i}$ and $\tm{P_j}$ to happen immediately over channel endpoints $\tm{y}$ and $\tm{y'}$.
  
  Suppose that $\tm{y'}$ is an action not in a different parallel process $P_j$ but rather of $P_i$ itself. That means that the action on $\tm{y'}$ must be prefixed by the action on $\tm{y}$, which is top level in $\tm{P_i}$. To respect law (i) of priorities we must have $\cs o < \cs o$, which is absurd. This means that $\tm{y'}$ is in another parallel process $\tm{P_j}$ for $i\neq j$.

  Suppose that $\tm{y'}$ in $\tm{P_j}$ is not at top level. In order to respect law (i) of priorities, it means that $\tm{y'}$ is prefixed by actions that are smaller than its priority $\cs o$. This leads to a contradiction because stated that $\cs o$ is the smallest priority. Hence, $y'$ must be the subject of a top level action.
  
  We have two processes, acting on dual endpoints. We apply the appropriate reduction rule, \ie \LabTirName{E-Send}, \LabTirName{E-Close}, \LabTirName{E-Select-Inl}, or \LabTirName{E-Select-Inr}.
\end{proof}

%%% Local Variables:
%%% TeX-master: "../priorities"
%%% End:

}
{
  \restatelemma{lempcptopgvtermspreservation}
  \begin{proof}
  \label{prf:lem-pcp-to-pgv-terms-preservation}
  By induction on the derivation of $\pcp{\seq{P}{\ty{\Gamma}}}$.
  \begin{case*}[\LabTirName{T-Link}, \LabTirName{T-Res}, \LabTirName{T-Par}, and \LabTirName{T-Halt}]
    See \cref{fig:pcp-to-pgv-preservation}.
  \end{case*}
  \begin{case*}[\LabTirName{T-Close}, and \LabTirName{T-Wait}]
    See \cref{fig:pcp-to-pgv-preservation-close-and-wait}.
  \end{case*}
  \begin{case*}[\LabTirName{T-Send}]
    See \cref{fig:pcp-to-pgv-preservation-send}.
  \end{case*}
  \begin{case*}[\LabTirName{T-Recv}]
    See \cref{fig:pcp-to-pgv-preservation-recv}.
  \end{case*}
  \begin{case*}[\LabTirName{T-Select-Inl}, \LabTirName{T-Select-Inr}, and \LabTirName{T-Offer}]
    See \cref{fig:pcp-to-pgv-preservation-select-and-offer}.
  \end{case*}
\end{proof}
\begin{landscape}
\begin{figure}
\small
\begin{mathpar}
  % Translation for Link
  \pcp{\inferrule*[lab=T-Link]{
    }{\seq{\link[\ty{A}]{x}{y}}{\tmty{x}{A}, \tmty{y}{\co{A}}}}}
  \cpgvMarrow
  \pgv{\inferrule*{
      \inferrule*{
      }{\tmty{\link}{\tylolli[]{\typrod{\cpgvT{A}}{\co{\cpgvT{A}}}}{\tyunit}}}
      \\
      \inferrule*{
        \inferrule*{
        }{\tseq[\cs{\pbot}]
          {\tmty{x}{\cpgvT{A}}}
          {x}
          {\cpgvT{A}}}
        \\
        \inferrule*{
        }{\tseq[\cs{\pbot}]
          {\tmty{y}{\co{\cpgvT{A}}}}
          {y}
          {\co{\cpgvT{A}}}}
      }{\tseq[\cs{\pbot}]
        {\tmty{x}{\cpgvT{A}},\tmty{y}{\co{\cpgvT{A}}}}
        {\pair{x}{y}}
        {\typrod{\cpgvT{A}}{\co{\cpgvT{A}}}}}
    }{\tseq[\cs{\pbot}]
      {\tmty{x}{\cpgvT{A}},\tmty{y}{\co{\cpgvT{A}}}}
      {\link\;{\pair{x}{y}}}
      {\tyunit}}}

  % Translation for Res
  \pcp{\inferrule*[lab=T-Res]{
      \seq{P}{\ty{\Gamma},\tmty{x}{A},\tmty{y}{\co{A}}}
    }{\seq{\res{x}{y}{P}}{\ty{\Gamma}}}}
  \cpgvMarrow
  \pgv{\inferrule*{
      \inferrule*{
        \inferrule*{
        }{\tmty
          {\new}
          {\tylolli{\tyunit}{{\typrod{\cpgvT{A}}{\co{\cpgvT{A}}}}}}}
        \\
        \inferrule*{
        }{\tseq[\cs{\pbot}]
          {\emptyenv}
          {\unit}
          {\tyunit}}
      }{\tseq[\cs{\pbot}]
        {\emptyenv}
        {\new\;\unit}
        {\typrod{\cpgvT{A}}{\co{\cpgvT{A}}}}}
      \\
      {\tseq[\cs{p}]
        {\ty{\cpgvT{\Gamma}},\tmty{x}{\cpgvT{A}},\tmty{y}{\co{\cpgvT{A}}}}
        {\cpgvM{P}}
        {\tyunit}}
    }{\tseq[\cs{p}]
      {\ty{\cpgvT{\Gamma}}}
      {\letpair{x}{y}{\new\;\unit}{\cpgvM{P}}}
      {\tyunit}}}

  % Translation for Mix
  \pcp{\inferrule*[lab=T-Par]{
      \seq{\tm{P}}{\ty{\Gamma}}
      \\
      \seq{\tm{Q}}{\ty{\Delta}}
    }{\seq{\tm{\ppar{P}{Q}}}{\ty{\Gamma},\ty{\Delta}}}}
  \cpgvMarrow
  \pgv{\inferrule*{
      \inferrule*{
        \inferrule*{
        }{\tmty{\spawn}{\tylolli{(\tylolli[\cs{\pr{(\ty{\Gamma})}},\cs{p}]{\tyunit}{\tyunit})}{\tyunit}}}
        \\
        \inferrule*{
          \tseq[\cs{p}]
          {\ty{\cpgvT{\Gamma}}}
          {\tm{\cpgvM{P}}}
          {\tyunit}
        }{\tseq[\cs{\pbot}]
          {\ty{\cpgvT{\Gamma}}}
          {\lambda\unit.\cpgvM{P}}
          {\tylolli[\cs{\pr{(\ty{\Gamma})}},\cs{p}]{\tyunit}{\tyunit}}}
      }{\tseq[\cs{\pbot}]
        {\ty{\cpgvT{\Gamma}}}
        {\tm{\spawn\;{(\lambda\unit.\cpgvM{P})}}}
        {\tyunit}}
      \\
      \tseq[\cs{q}]
      {\ty{\cpgvT{\Delta}}}
      {\tm{\cpgvM{Q}}}
      {\tyunit}
    }{\tseq[\cs{q}]
      {\ty{\cpgvT{\Gamma}},\ty{\cpgvT{\Delta}}}
      {\andthen{\spawn\;{(\lambda\unit.\cpgvM{P})}}{\cpgvM{Q}}}
      {\tyunit}}}
  \\
  % Translation for Halt
  \pcp{\inferrule*[lab=T-Halt]{
    }{\seq{\tm{\halt}}{\emptyenv}}}
  \cpgvMarrow
  \pgv{\inferrule*{
    }{\tseq[\cs{\pbot}]{\emptyenv}{\unit}{\tyunit}}}
\end{mathpar}
\caption{Translation $\cpgvM{\cdot}$ preserves typing (\LabTirName{T-Link}, \LabTirName{T-Res}, \LabTirName{T-Par}, and \LabTirName{T-Halt}).}
\label{fig:pcp-to-pgv-preservation}
\end{figure}
\begin{figure}
\begin{mathpar}
  % Translation for Close
  \pcp{\inferrule*[lab=T-Close]{
      \seq{\tm{P}}{\ty{\Gamma}}
      \\
      \cs{o}<\pr(\ty{\Gamma})
    }{\seq{\tm{\close{x}{P}}}{\ty{\Gamma},\tmty{x}{\tyone[\cs{o}]}}}}
  \cpgvMarrow
  \pgv{\inferrule*{
      \inferrule*{
        \inferrule*{
        }{\tmty{\close}{\tylolli[\cs{\ptop},\cs{o}]{\tyends[\cs{o}]}{\tyunit}}}
        \\
        \inferrule*{
        }{\tseq[\cs{\pbot}]
          {\tmty{x}{\tyends[\cs{o}]}}
          {x}
          {\tyends[\cs{o}]}}
      }{\tseq[\cs{o}]
        {\tmty{x}{\tyends[\cs{o}]}}
        {\close\;{x}}
        {\tyunit}}
      \\
      \tseq[\cs{p}]{\ty{\cpgvT{\Gamma}}}{\cpgvM{P}}{\tyunit}
      \\
      \cs{o}<\pr(\ty{\cpgvT{\Gamma}})
    }{\tseq[\cs{o}\sqcup\cs{p}]
      {\ty{\cpgvT{\Gamma}},\tmty{x}{\tyends[\cs{o}]}}
      {\andthen{\close\;{x}}{\cpgvM{P}}}
      {\tyunit}}}
    \\
  % Translation for Wait
  \pcp{\inferrule*[lab=T-Wait]{
      \seq{\tm{P}}{\ty{\Gamma}}
      \\
      \cs{o}<\pr(\ty{\Gamma})
    }{\seq{\tm{\wait{x}{P}}}{\ty{\Gamma},\tmty{x}{\tyone[\cs{o}]}}}}
  \cpgvMarrow
  \pgv{\inferrule*{
      \inferrule*{
        \inferrule*{
        }{\tmty{\wait}{\tylolli[\cs{\ptop},\cs{o}]{\tyendr[\cs{o}]}{\tyunit}}}
        \\
        \inferrule*{
        }{\tseq[\cs{\pbot}]
          {\tmty{x}{\tyendr[\cs{o}]}}
          {x}
          {\tyendr[\cs{o}]}}
      }{\tseq[\cs{o}]
        {\tmty{x}{\tyendr[\cs{o}]}}
        {\wait\;{x}}
        {\tyunit}}
      \\
      \tseq[\cs{p}]{\ty{\cpgvT{\Gamma}}}{\cpgvM{P}}{\tyunit}
      \\
      \cs{o}<\pr(\ty{\cpgvT{\Gamma}})
    }{\tseq[\cs{o}\sqcup\cs{p}]
      {\ty{\cpgvT{\Gamma}},\tmty{x}{\tyendr[\cs{o}]}}
      {\andthen{\wait\;{x}}{\cpgvM{P}}}
      {\tyunit}}}
\end{mathpar}
\caption{Translation $\cpgvM{\cdot}$ preserves typing (\LabTirName{T-Close} and \LabTirName{T-Wait}).}
\label{fig:pcp-to-pgv-preservation-close-and-wait}
\end{figure}
\begin{figure}
\begin{mathpar}
  % Translation for Send
  \pcp{\inferrule*[lab=T-Send]{
      \seq{\tm{P}}{\ty{\Gamma},\tmty{y}{A},\tmty{x}{B}}
      \\
      \cs{o}<\pr(\ty{\Gamma},\ty{A},\ty{B})
    }{\seq{\tm{\send{x}{y}{P}}}{\ty{\Gamma},\tmty{x}{\tytens[\cs{o}]{A}{B}}}}}
  \cpgvMarrow
  \\
  \pgv{\inferrule*[lab=(a)]{
      \inferrule*{
      }{\tmty
        {\new}
        {\tylolli{\tyunit}{{\typrod{\cpgvT{A}}{\co{\cpgvT{A}}}}}}}
      \\
      \inferrule*{
      }{\tseq[\cs{\pbot}]
        {\emptyenv}
        {\unit}
        {\tyunit}}
    }{\tseq[\cs{\pbot}]
      {\emptyenv}
      {\new\;\unit}
      {\typrod{\cpgvT{A}}{\co{\cpgvT{A}}}}}}
  \\
  \pgv{\inferrule*[lab=(b)]{
      \inferrule*{
      }{\tmty{\send}{\tylolli[\cs{\ptop},\cs{o}]
          {\typrod{\co{\cpgvT{A}}}{\tysend[\cs{o}]{\co{\cpgvT{A}}}{\cpgvT{B}}}}
          {\cpgvT{B}}}}
      \\
      \inferrule*{
        \inferrule*{
        }{\tseq[\cs{\pbot}]
          {\tmty{z}{\co{\cpgvT{A}}}}
          {x}
          {\co{\cpgvT{A}}}}
        \\
        \inferrule*{
        }{\tseq[\cs{\pbot}]
          {\tmty{x}{\tysend[\cs{o}]{\co{\cpgvT{A}}}{\cpgvT{B}}}}
          {x}
          {\tysend[\cs{o}]{\co{\cpgvT{A}}}{\cpgvT{B}}}}
      }{\tseq[\cs{\pbot}]
        {\tmty{x}{\tysend[\cs{o}]{\co{\cpgvT{A}}}{\cpgvT{B}}},\tmty{z}{\co{\cpgvT{A}}}}
        {\pair{z}{x}}
        {\typrod{\co{\cpgvT{A}}}{\tysend[\cs{o}]{\co{\cpgvT{A}}}{\cpgvT{B}}}}}
    }{\tseq[\cs{o}]
      {\tmty{x}{\tysend[\cs{o}]{\co{\cpgvT{A}}}{\cpgvT{B}}},\tmty{z}{\co{\cpgvT{A}}}}
      {\send\;{\pair{z}{x}}}
      {\cpgvT{B}}}}
  \\
  \pgv{\inferrule*{
      \LabTirName{(a)}
      \\
      \inferrule*{
        \LabTirName{(b)}
        \\
        {\tseq[\cs{p}]
          {\ty{\cpgvT{\Gamma}},\tmty{y}{\cpgvT{A}},\tmty{x}{\cpgvT{B}}}
          {\cpgvM{P}}
          {\tyunit}}
        \\
        \cs{o}<\pr(\ty{\cpgvT{\Gamma}},\ty{\cpgvT{A}},\ty{\cpgvT{B}})
      }{\tseq[\cs{o}\sqcup\cs{p}]
        {\ty{\cpgvT{\Gamma}},%
          \tmty{x}{\tysend[\cs{o}]{\co{\cpgvT{A}}}{\cpgvT{B}}},%
          \tmty{y}{\cpgvT{A}},\tmty{z}{\co{\cpgvT{A}}}}
        {\letbind{x}{\send\;{\pair{z}{x}}}{\cpgvM{P}}}
        {\tyunit}}
    }{\tseq[\cs{o}\sqcup\cs{p}]
      {\ty{\cpgvT{\Gamma}},\tmty{x}{\tysend[\cs{o}]{\co{\cpgvT{A}}}{\cpgvT{B}}}}
      {\letpair{y}{z}{\new\;\unit}{\letbind{x}{\send\;{\pair{z}{x}}}{\cpgvM{P}}}}
      {\tyunit}}}
\end{mathpar}
\caption{Translation $\cpgvM{\cdot}$ preserves typing (\LabTirName{T-Send}).}
\label{fig:pcp-to-pgv-preservation-send}
\end{figure}
\begin{figure}
\begin{mathpar}
  % Translation for Recv
  \pcp{\inferrule*[lab=T-Recv]{
      \seq{P}{\ty{\Gamma},\tmty{y}{A},\tmty{x}{B}}
      \\
      \cs{o}<\pr(\ty{\Gamma},\ty{A},\ty{B})
    }{\seq{\recv{x}{y}{P}}{\ty{\Gamma},\tmty{x}{\typarr[\cs{o}]{A}{B}}}}}
  \cpgvMarrow
  \\
  \pgv{\inferrule*[lab=(a)]{
      \inferrule*{
      }{\tmty{\recv}{\tylolli[\cs{\ptop},\cs{o}]
          {\tyrecv[\cs{o}]{{\cpgvT{A}}}{\cpgvT{B}}}
          {\typrod{{\cpgvT{A}}}{\cpgvT{B}}}}}
      \\
      \inferrule*{
      }{\tseq[\cs{\pbot}]
        {\tmty{x}{\tyrecv[\cs{o}]{{\cpgvT{A}}}{\cpgvT{B}}}}
        {x}
        {\tyrecv[\cs{o}]{{\cpgvT{A}}}{\cpgvT{B}}}}
    }{\tseq[\cs{o}]
      {\tmty{x}{\tyrecv[\cs{o}]{{\cpgvT{A}}}{\cpgvT{B}}}}
      {\recv\;{x}}{\typrod{\cpgvT{A}}{\cpgvT{B}}}}}
  \\
  \pgv{\inferrule*{
      \LabTirName{(a)}
      \\
      {\tseq[\cs{p}]
        {\ty{\cpgvT{\Gamma}},\tmty{y}{\cpgvT{A}},\tmty{x}{\cpgvT{B}}}
        {\cpgvM{P}}
        {\tyunit}}
      \\
      \cs{o}<\pr(\ty{\cpgvT{\Gamma}},\ty{\cpgvT{A}},\ty{\cpgvT{B}})
    }{\tseq[\cs{o}\sqcup\cs{p}]
      {\ty{\cpgvT{\Gamma}},%
        \tmty{x}{\tyrecv[\cs{o}]{{\cpgvT{A}}}{\cpgvT{B}}},\tmty{y}{\cpgvT{A}},\tmty{z}{{\cpgvT{A}}}}
      {\letbind{x}{\recv{x}}{\cpgvM{P}}}{\tyunit}}}
\end{mathpar}
\caption{Translation $\cpgvM{\cdot}$ preserves typing (\LabTirName{T-Recv}).}
\label{fig:pcp-to-pgv-preservation-recv}
\end{figure}
\begin{figure}
\begin{mathpar}
  % Translation for Select-Inl
  \pcp{\inferrule*[lab=T-Select-Inl]{
      \seq{\tm{P}}{\ty{\Gamma},\tmty{x}{A}}
      \\
      \cs{o}<\pr(\ty{\Gamma})
    }{\seq{\inl{x}{P}}{\ty{\Gamma},\tmty{x}{\typlus[\cs{o}]{A}{B}}}}}
  \cpgvMarrow
  \pgv{\inferrule*{
      \inferrule*{
        \inferrule*{
        }{\tmty
          {\select{\labinl}}
          {\tylolli[\cs{\ptop},\cs{o}]{\tyselect[\cs{o}]{\cpgvT{A}}{\cpgvT{B}}}{\cpgvT{A}}}}
        \\
        \inferrule*{
        }{\tseq[\cs{\pbot}]
          {\tmty{x}{\tyselect[\cs{o}]{\cpgvT{A}}{\cpgvT{B}}}}
          {x}
          {\tyselect[\cs{o}]{\cpgvT{A}}{\cpgvT{B}}}}
      }{\tseq[\cs{o}]
        {\tmty{x}{\tyselect[\cs{o}]{\cpgvT{A}}{\cpgvT{B}}}}
        {\select{\labinl}\;{x}}
        {\cpgvT{A}}}
      \\
      {\tseq[\cs{p}]
        {\ty{\Gamma},\tmty{x}{\cpgvT{A}}}
        {\cpgvM{P}}
        {\tyunit}}
      \\
      \cs{o}<\pr(\ty{\Gamma})
    }{\tseq[\cs{o}\sqcup\cs{p}]
      {\ty{\Gamma},\tmty{x}{\tyselect[\cs{o}]{\cpgvT{A}}{\cpgvT{B}}}}
      {\letbind{x}{\select{\labinl}\;{x}}{\cpgvM{P}}}
      {\tyunit}}}
  \\
  % Translation for Select-Inl
  \pcp{\inferrule*[lab=T-Select-Inr]{
      \seq{\tm{P}}{\ty{\Gamma},\tmty{x}{A}}
      \\
      \cs{o}<\pr(\ty{\Gamma})
    }{\seq{\inr{x}{P}}{\ty{\Gamma},\tmty{x}{\typlus[\cs{o}]{A}{B}}}}}
  \cpgvMarrow
  \pgv{\inferrule*{
      \inferrule*{
        \inferrule*{
        }{\tmty
          {\select{\labinr}}
          {\tylolli[\cs{\ptop},\cs{o}]{\tyselect[\cs{o}]{\cpgvT{A}}{\cpgvT{B}}}{\cpgvT{B}}}}
        \\
        \inferrule*{
        }{\tseq[\cs{\pbot}]
          {\tmty{x}{\tyselect[\cs{o}]{\cpgvT{A}}{\cpgvT{B}}}}
          {x}
          {\tyselect[\cs{o}]{\cpgvT{A}}{\cpgvT{B}}}}
      }{\tseq[\cs{o}]
        {\tmty{x}{\tyselect[\cs{o}]{\cpgvT{A}}{\cpgvT{B}}}}
        {\select{\labinr}\;{x}}
        {\cpgvT{B}}}
      \\
      {\tseq[\cs{p}]
        {\ty{\Gamma},\tmty{x}{\cpgvT{B}}}
        {\cpgvM{P}}
        {\tyunit}}
      \\
      \cs{o}<\pr(\ty{\Gamma})
    }{\tseq[\cs{o}\sqcup\cs{p}]
      {\ty{\Gamma},\tmty{x}{\tyselect[\cs{o}]{\cpgvT{A}}{\cpgvT{B}}}}
      {\letbind{x}{\select{\labinr}\;{x}}{\cpgvM{P}}}
      {\tyunit}}}
  \\
  % Translation for Offer
  \pcp{\inferrule*[lab=T-Offer]{
      \seq{P}{\ty{\Gamma},\tmty{x}{A}}
      \\
      \seq{Q}{\ty{\Gamma},\tmty{x}{B}}
      \\
      \cs{o}<\pr(\ty{\Gamma},\ty{A},\ty{B})
    }{\seq{\offer{x}{P}{Q}}{\ty{\Gamma},\tmty{x}{\tywith[\cs{o}]{A}{B}}}}}
  \cpgvMarrow
  \pgv{\inferrule*{
      \inferrule*{
      }{\tseq[\cs{\pbot}]
        {\tmty{x}{\tyoffer[\cs{o}]{\cpgvT{A}}{\cpgvT{B}}}}
        {x}
        {\tyoffer[\cs{o}]{\cpgvT{A}}{\cpgvT{B}}}}
      \\
      \tseq[\cs{p}]{\ty{\cpgvT{\Gamma}},\tmty{x}{\cpgvT{A}}}{\cpgvM{P}}{\tyunit}
      \\
      \tseq[\cs{p}]{\ty{\cpgvT{\Gamma}},\tmty{x}{\cpgvT{B}}}{\cpgvM{Q}}{\tyunit}
      \\
      \cs{o}<\pr(\ty{\cpgvT{\Gamma}},\ty{\cpgvT{A}},\ty{\cpgvT{B}})
    }{\tseq[\cs{o}\sqcup\cs{p}]
      {\ty{\cpgvT{\Gamma}},\tmty{x}{\tyoffer[\cs{o}]{\cpgvT{A}}{\cpgvT{B}}}}
      {\offer{x}{x}{\cpgvM{P}}{x}{\cpgvM{Q}}}
      {\tyunit}}}
\end{mathpar}
\caption{Translation $\cpgvM{\cdot}$ preserves typing (\LabTirName{T-Select-Inl}, \LabTirName{T-Select-Inr}, and \LabTirName{T-Offer}).}
\label{fig:pcp-to-pgv-preservation-select-and-offer}
\end{figure}
\end{landscape}

%%% Local Variables:
%%% TeX-master: "../priorities"
%%% End:

  \restatetheorem{thmpcptopgvconfspreservation}
  \begin{proof}
  \begin{case*}[\LabTirName{T-Cut}]
    Immediately, from the induction hypothesis.
    \small
    \begin{mathpar}
      \pcp{\inferrule*[lab=T-Cut]{
          \seq{\ty{\Gamma},\tmty{x}{A},\tmty{y}{\co{A}}}{\tm{P}}
        }{\seq{\ty{\Gamma}}{\tm{\res{x}{y}{P}}}}}
      \cpgvcarrow
      \pgv{\inferrule*{
          \cseq[\child]{\ty{\cpgv{\Gamma}},\tmty{x}{\cpgv{A}},\tmty{y}{\cpgv{B}}}{\tm{\cpgvc{P}}}
        }{\cseq[\child]{\ty{\cpgv{\Gamma}}}{\tm{\res{x}{y}{\cpgvc{P}}}}}}
    \end{mathpar}
  \end{case*}
  \begin{case*}[\LabTirName{T-Mix}]
    Immediately, from the induction hypotheses.
    \begin{mathpar}
      \pcp{\inferrule*[lab=T-Mix]{
          \seq{\ty{\Gamma}}{\tm{P}}
          \\
          \seq{\ty{\Delta}}{\tm{Q}}
        }{\seq{\ty{\Gamma},\ty{\Delta}}{\ppar{P}{Q}}}}
      \cpgvcarrow
      \pgv{\inferrule*{
          \cseq[\child]{\ty{\cpgv{\Gamma}}}{\tm{\cpgvc{P}}}
          \\
          \cseq[\child]{\ty{\cpgv{\Delta}}}{\tm{\cpgvc{Q}}}
        }{\cseq[\child]{\ty{\cpgv{\Gamma}},\ty{\cpgv{\Delta}}}{\tm{\ppar{\cpgvc{P}}{\cpgvc{Q}}}}}}
    \end{mathpar}
  \end{case*}
  \begin{case*}[\LabTirName{*}]
    By \cref{lem:pcp-to-pgv-terms-preservation}
    \begin{mathpar}
      \pcp{\seq{\ty{\Gamma}}{\tm{P}}}
      \cpgvcarrow
      \pgv{\inferrule*{
          \tseq[\cs{p}]{\ty{\cpgv{\Gamma}}}{\tm{\cpgv{P}}}{\tyunit}
        }{\cseq[\child]{\ty{\cpgv{\Gamma}}}{\tm{\child\;\cpgv{P}}}}}
    \end{mathpar}
  \end{case*}
\end{proof}

%%% Local Variables:
%%% TeX-master: "../priorities"
%%% End:


  \restatetheorem{thmpcptopgvoperationalcorrespondencesoundness}
  \begin{proof}
  \label{prf:thm-pcp-to-pgv-operational-correspondence-soundness}
  By induction on the derivation of $\pgv{\tm{\cpgvC{P}}\cred\tm{\conf{C}}}$.
  We omit the cases which cannot occur as their left-hand side term forms are not in the image of the translation function, \ie \LabTirName{E-New}, \LabTirName{E-Spawn}, and \LabTirName{E-LiftM}.

  \begin{case*}[\LabTirName{E-Link}]
    \[\pgv{%
        \tm{\res{x}{x'}{(\ppar{\plug{\conf{F}}{\link\;\pair{w}{x}}}{\conf{C}})}}
        \cred
        \tm{\ppar{\plug{\conf{F}}{\unit}}{\subst{\conf{C}}{w}{x'}}}
      }\]
    The source for $\pgv{\tm{\link\;\pair{w}{x}}}$ \emph{must} be $\pcp{\tm{\link{w}{x}}}$. None of the translation rules introduce an evaluation context around the recursive call, hence $\pgv{\tm{\conf{F}}}$ must be the empty context. Let $\pcp{\tm{P}}$ be the source term for $\pgv{\tm{\conf{C}}}$, \ie $\pgv{\tm{\cpgvC{P}}=\tm{\conf{C}}}$. Hence, we have:
    \begin{mathpar}
      \begin{tikzcd}[cramped, column sep=tiny]
        \pcp{\tm{\res{x}{x'}{(\ppar{\link{w}{x}}{P})}}}
        \arrow[r, "\pcp{\red}"]
        \arrow[d, "\cpgvC{\cdot}"]
        &
        \pcp{\tm{\subst{P}{w}{x'}}}
        \arrow[dd, "\cpgvC{\cdot}"]
        \\
        \pgv{\tm{\res{x}{x'}{(\ppar{\child\;\link\;\pair{w}{x}}{\cpgvC{P}})}}}
        \arrow[d, "\pgv{\cred^+}"]
        \\
        \pgv{\tm{\subst{\cpgvC{P}}{w}{x'}}}
        \arrow[r, "\pcp{=}"]
        &
        \pgv{\tm{\cpgvC{\subst{P}{w}{x'}}}}
      \end{tikzcd}
    \end{mathpar}
  \end{case*}
  \begin{case*}[\LabTirName{E-Send}]
    \[\pgv{%
        \tm{\res{x}{x'}{(\ppar{\plug{\conf{F}}{\send\;{\pair{V}{x}}}}{\plug{\conf{F'}}{\recv\;{x'}}})}}
        \cred
        \tm{\res{x}{x'}{(\ppar{\plug{\conf{F}}{x}}{\plug{\conf{F'}}{\pair{V}{x'}}})}}
      }\]
    There are three possible sources for $\pgv{\tm{\send}}$ and $\pgv{\tm{\recv}}$: $\pcp{\tm{\send{x}{y}{P}}}$ and $\pcp{\tm{\recv{x'}{y'}{Q}}}$; $\pcp{\tm{\inl{x}{P}}}$ and $\pcp{\tm{\offer{x'}{Q}{R}}}$; or $\pcp{\tm{\inr{x}{P}}}$ and $\pcp{\tm{\offer{x'}{Q}{R}}}$.
    \begin{subcase*}[$\pcp{\tm{\send{x}{y}{P}}}$ and $\pcp{\tm{\recv{x'}{y'}{Q}}}$]
      None of the translation rules introduce an evaluation context around the recursive call, hence $\pgv{\tm{\conf{F}}}$ must be $\pgv{\tm{\child\;\letbind{x}{\hole}{\cpgvM{P}}}}$. Similarly, $\pgv{\tm{\conf{F'}}}$ must be $\pgv{\tm{\child\;\letpair{y'}{x'}{\hole}{\cpgvM{Q}}}}$. The value $\pgv{\tm{V}}$ must be an endpoint $\tm{y}$, bound by the name restriction $\pgv{\tm{\res{y}{y'}{}}}$ introduced by the translation. Hence, we have:
      \begin{mathpar}
          \begin{tikzcd}[cramped, column sep=tiny]
            \pcp{\tm{\res{x}{x'}{(\ppar{\send{x}{y}{P}}{\recv{x'}{y'}{Q}})}}}
            \arrow[d, "\cpgvC{\cdot}"]
            \arrow[r, "\pcp{\red}"]
            &
            \pcp{\tm{\res{x}{x'}{\res{y}{y'}{(\ppar{P}{Q})}}}}
            \arrow[dd, "\cpgvC{\cdot}"]
            \\
            \pgv{\tm{\res{x}{x'}{\res{y}{y'}{}}\left(
                  \begin{array}{l}
                    \child\;\letbind{x}{\send\;{\pair{y}{x}}}{\cpgvM{P}}
                    \parallel
                    \\
                    \child\;\letpair{y'}{x'}{\recv\;{x'}}{\cpgvM{Q}}
                  \end{array}
                \right)}}
            \arrow[d, "\pgv{\equiv\cred^+}"]
            \\
            \pgv{\tm{\res{x}{x'}{\res{y}{y'}{(\ppar{\child\;\cpgvM{P}}{\child\;\cpgvM{Q}})}}}}
            \arrow[r, "\pgv{\cred^\star}", "\text{(by \cref{lem:pcp-to-pgv-cpgvM-to-cpgvC})}"']
            &
            \pgv{\tm{\res{x}{x'}{\res{y}{y'}{(\ppar{\cpgvC{P}}{\cpgvC{Q}})}}}}
          \end{tikzcd}
      \end{mathpar}
    \end{subcase*}
    \begin{subcase*}[$\pcp{\tm{\inl{x}{P}}}$ and $\pcp{\tm{\offer{x'}{Q}{R}}}$]
      None of the translation rules introduce an evaluation context around the recursive call, hence $\pgv{\tm{\conf{F}}}$ must be $$\pgv{\tm{\child\;\letbind{x}{\andthen{\close\;\hole}{y}}{\cpgvM{P}}}}.$$ Similarly, $\pgv{\tm{\conf{F'}}}$ must be $$\pgv{\tm{\child\;\letpair{y'}{x'}{\hole}{\andthen{\wait\;x'}{\casesum{y'}{y'}{\cpgvM{Q}}{y'}{\cpgvM{R}}}}}}.$$ Hence, we have:
      \begin{mathpar}
        \begin{tikzcd}[cramped, column sep=tiny]
          \pcp{\tm{\res{x}{x'}{(\ppar{\inl{x}{P}}{\offer{x}{Q}{R}})}}}
          \arrow[d, "\cpgvM{\cdot}"]
          \arrow[r, "\pcp{\red}"]
          &
          \pcp{\tm{\res{x}{x'}{(\ppar{P}{Q})}}}
          \arrow[dd, "\cpgvC{\cdot}"]
          \\
          \pgv{\tm{\res{x}{x'}{}\left(
                \begin{array}{l}
                  \child\;\letbind{x}{\select{\labinl}\;{x}}{\cpgvM{P}}
                  \parallel
                  \\
                  \child\;\offer{x'}{x'}{\cpgvM{Q}}{x'}{\cpgvM{R}}
                \end{array}
              \right)}}
          \arrow[d, "\pgv{\cred^+}"]
          \\ 
          \pgv{\tm{\res{x}{x'}{(\ppar{\child\;\cpgvM{P}}{\child\;\cpgvM{Q}})}}}
          \arrow[r, "\pgv{\cred^\star}", "\text{(by \cref{lem:pcp-to-pgv-cpgvM-to-cpgvC})}"']
          & 
          \pgv{\tm{\res{x}{x'}{(\ppar{\cpgvC{P}}{\cpgvC{Q}})}}}
        \end{tikzcd}
      \end{mathpar}
    \end{subcase*}
    \begin{subcase*}[$\pcp{\tm{\inr{x}{P}}}$ and $\pcp{\tm{\offer{x'}{Q}{R}}}$]
      None of the translation rules introduce an evaluation context around the recursive call, hence $\pgv{\tm{\conf{F}}}$ must be $$\pgv{\tm{\child\;\letbind{x}{\andthen{\close\;\hole}{y}}{\cpgvM{P}}}}.$$ Similarly, $\pgv{\tm{\conf{F'}}}$ must be $$\pgv{\tm{\child\;\letpair{y'}{x'}{\hole}{\andthen{\wait\;x'}{\casesum{y'}{y'}{\cpgvM{Q}}{y'}{\cpgvM{R}}}}}}.$$ Hence, we have:
      \begin{mathpar}
        \begin{tikzcd}[cramped, column sep=tiny]
          \pcp{\tm{\res{x}{x'}{(\ppar{\inr{x}{P}}{\offer{x}{Q}{R}})}}}
          \arrow[d, "\cpgvM{\cdot}"]
          \arrow[r, "\pcp{\red}"]
          &
          \pcp{\tm{\res{x}{x'}{(\ppar{P}{Q})}}}
          \arrow[dd, "\cpgvC{\cdot}"]
          \\
          \pgv{\tm{\res{x}{x'}{}\left(
                \begin{array}{l}
                  \child\;\letbind{x}{\select{\labinr}\;{x}}{\cpgvM{P}}
                  \parallel
                  \\
                  \child\;\offer{x'}{x'}{\cpgvM{Q}}{x'}{\cpgvM{R}}
                \end{array}
              \right)}}
          \arrow[d, "\pgv{\cred^+}"]
          \\ 
          \pgv{\tm{\res{x}{x'}{(\ppar{\child\;\cpgvM{P}}{\child\;\cpgvM{R}})}}}
          \arrow[r, "\pgv{\cred^\star}", "\text{(by \cref{lem:pcp-to-pgv-cpgvM-to-cpgvC})}"']
          & 
          \pgv{\tm{\res{x}{x'}{(\ppar{\cpgvC{P}}{\cpgvC{R}})}}}
        \end{tikzcd}
      \end{mathpar}
    \end{subcase*}
  \end{case*}
  \begin{case*}[\LabTirName{E-Close}]
    \[\pgv{%
        \tm{\res{x}{x'}{(\ppar{\plug{\conf{F}}{\wait\;{x}}}{\plug{\conf{F'}}{\close\;{x'}}})}}
        \cred
        \tm{\ppar{\plug{\conf{F}}{\unit}}{\plug{\conf{F'}}{\unit}}}
      }\]
    The source for $\pgv{\tm{\wait}}$ and $\pgv{\tm{\close}}$ \emph{must} be $\pcp{\tm{\wait{x}{P}}}$ and $\pcp{\tm{\close{x'}{Q}}}$.

    (The translation for $\pcp{\tm{\offer{x}{P}{Q}}}$ also introduces a $\pgv{\tm{\wait}}$, but it is blocked on another communication, and hence cannot be the first communication on a translated term. The translations for $\pcp{\tm{\inl{x}{P}}}$ and $\pcp{\tm{\inr{x}{P}}}$ also introduce a $\pgv{\tm{\close}}$, but these are similarly blocked.)

    None of the translation rules introduce an evaluation context around the recursive call, hence $\pgv{\tm{\conf{F}}}$ must be $\pgv{\tm{\andthen{\hole}{\cpgvM{P}}}}$. Similarly, $\pgv{\tm{\conf{F'}}}$ must be $\pgv{\tm{\andthen{\hole}{\cpgvM{Q}}}}$. Hence, we have:
    \begin{mathpar}
      \begin{tikzcd}
        \pcp{\tm{\res{x}{x'}{(\ppar{\close{x}{P}}{\wait{x'}{Q}})}}} 
        \arrow[d, "\cpgvM{\cdot}"]
        \arrow[r, "\pcp{\red}"]
        &
        \pcp{\tm{\ppar{P}{Q}}}
        \arrow[dd, "\cpgvC{\cdot}"]
        \\
        \pgv{\tm{\res{x}{x'}{(\ppar
              {\child\;\andthen{\close\;{x}}{\cpgvM{P}}} 
              {\child\;\andthen{\wait\;{x'}}{\cpgvM{Q}}}
              )}}}
        \arrow[d, "\pgv{\cred^+}"]
        \\ 
        \pgv{\tm{\ppar{\child\;\cpgvM{P}}{\child\;\cpgvM{Q}}}}
        \arrow[r, "\pgv{\cred^\star}", "\text{(by \cref{lem:pcp-to-pgv-cpgvM-to-cpgvC})}"']
        & 
        \pgv{\tm{\ppar{\cpgvC{P}}{\cpgvC{Q}}}}
      \end{tikzcd}
    \end{mathpar}
  \end{case*}
  \begin{case*}[\LabTirName{E-LiftC}]
    By the induction hypothesis and \LabTirName{E-LiftC}.
  \end{case*}
  \begin{case*}[\LabTirName{E-LiftSC}]
    By the induction hypothesis, \LabTirName{E-LiftSC},
    and \cref{lem:pcp-to-pgv-confs-operational-correspondence-equiv}.
  \end{case*}
\end{proof}

%%% Local Variables:
%%% TeX-master: "../priorities"
%%% End:



  \restatelemma{lempcptopgvcpgvMtocpgvC}
  \begin{proof}
  \label{prf:lem-pcp-to-pgv-cpgvM-to-cpgvC}
  By induction on the structure of $\pcp{\tm{P}}$.

  \begin{case*}[$\pcp{\tm{\res{x}{y}{P}}}$]
    We have:
    \begin{mathpar}
      \begin{array}{lrll}
        \pcp{\tm{\cpgvM{\res{x}{y}{P}}}}
        & =
        & \pgv{\tm{\child\;\letpair{x}{y}{\new}{\cpgvM{P}}}}
        \\
        & \pgv{\cred^+}
        & \pgv{\tm{\res{x}{y}{(\child\;\cpgvM{P})}}}
        \\
        & \pgv{\cred^\star}
        & \pgv{\tm{\res{x}{y}{\cpgvC{P}}}}
        \\
        & =
        & \pcp{\tm{\cpgvC{\res{x}{y}{P}}}}
      \end{array}
    \end{mathpar}
  \end{case*}
  \begin{case*}[$\pcp{\tm{\ppar{P}{Q}}}$]
    \begin{mathpar}
      \begin{array}{lrll}
        \pcp{\tm{\cpgvM{\ppar{P}{Q}}}}
        & =
        & \pgv{\tm{\child\;\andthen{\spawn\;(\lambda\unit.\cpgvM{P})}{\cpgvM{Q}}}}
        \\
        & \pgv{\cred^+}
        & \pgv{\tm{\ppar{\child\;\cpgvM{P}}{\child\;\cpgvM{Q}}}}
        \\
        & \pgv{\cred^\star}
        & \pgv{\tm{\ppar{\cpgvC{P}}{\cpgvC{Q}}}}
        \\
        & =
        & \pcp{\tm{\cpgvC{\ppar{P}{Q}}}}
      \end{array}
    \end{mathpar}
  \end{case*}
  \begin{case*}[$\pcp{\tm{\send{x}{y}{P}}}$]
    \begin{mathpar}
      \begin{array}{lrll}
        \pcp{\tm{\cpgvM{\send{x}{y}{P}}}}
        & =
        & \pgv{\tm{\letpair{y}{z}{\new}{\letbind{x}{\send\;{\pair{z}{x}}}{\cpgvM{P}}}}}
        \\
        & \pgv{\cred^+}
        & \pgv{\tm{\res{y}{z}{(\child\;\letbind{x}{\send\;\pair{z}{x}}{\cpgvM{P}})}}}
        \\
        & =
        & \pcp{\tm{\cpgvC{\send{x}{y}{P}}}}
      \end{array}
    \end{mathpar}
  \end{case*}
  \begin{case*}[$\pcp{\tm{\inl{x}{P}}}$]
    \begin{mathpar}
      \begin{array}{lrll}
        \pcp{\tm{\cpgvM{\send{x}{y}{P}}}}
        & =
        & \pgv{\tm{\letbind{x}{\select{\labinl}\;{x}}{\cpgvM{P}}}}
        \\
        & \elabarrow
        & \pgv{\tm{\letbind{x}{\letpair{y}{z}{\new}{\andthen{\close\;(\send\;{\pair{\inl{y}}{x}})}{z}}}{\cpgvM{P}}}}
        \\
        & \pgv{\cred^+}
        & \pgv{\tm{\res{y}{z}{(\child\;\letbind{x}{\andthen{\close\;(\send\;\pair{\inl{y}}{x})}{z}}{\cpgvM{P}})}}}
        \\
        & =
        & \pcp{\tm{\cpgvC{\send{x}{y}{P}}}}
      \end{array}
    \end{mathpar}
  \end{case*}
  \begin{case*}[$\pcp{\tm{\inr{x}{P}}}$]
    \begin{mathpar}
      \begin{array}{lrll}
        \pcp{\tm{\cpgvM{\send{x}{y}{P}}}}
        & =
        & \pgv{\tm{\letbind{x}{\select{\labinr}\;{x}}{\cpgvM{P}}}}
        \\
        & \elabarrow
        & \pgv{\tm{\letbind{x}{\letpair{y}{z}{\new}{\andthen{\close\;(\send\;{\pair{\inr{y}}{x}})}{z}}}{\cpgvM{P}}}}
        \\
        & \pgv{\cred^+}
        & \pgv{\tm{\res{y}{z}{(\child\;\letbind{x}{\andthen{\close\;(\send\;\pair{\inr{y}}{x})}{z}}{\cpgvM{P}})}}}
        \\
        & =
        & \pcp{\tm{\cpgvC{\send{x}{y}{P}}}}
      \end{array}
    \end{mathpar}
  \end{case*}
  \begin{case*}[$*$]
    In all other cases, $\pgv{\tm{\child\;\cpgvM{P}}=\tm{\cpgvC{P}}}$.
  \end{case*}
\end{proof}

%%% Local Variables:
%%% TeX-master: "../priorities"
%%% End:


  \begin{restatablelemma}{lempcptopgvoperationalcorrespondenceequiv}%
    \label{lem:pcp-to-pgv-confs-operational-correspondence-equiv}
    If $\pcp{\seq{P}{\ty{\Gamma}}}$ and $\pcp{\tm{P}\equiv\tm{Q}}$,
    then $\pgv{\tm{\cpgvC{P}}\equiv\tm{\cpgvC{Q}}}$.
  \end{restatablelemma}
  \begin{proof}
    Every axiom of the structural congruence in PCP maps directly to the axiom of the same name in PGV.
  \end{proof}

  \restatetheorem{thmpcptopgvoperationalcorrespondencecompleteness}
  \begin{proof}
  \label{prf:thm-pcp-to-pgv-operational-correspondence-completeness}
  By induction on the derivation of $\pcp{\tm{P}\red\tm{Q}}$.

  \begin{case*}[\LabTirName{E-Link}]
    \begin{mathpar}
      \begin{tikzcd}[cramped]
        \pcp{\tm{\res{x}{x'}{(\ppar{\link{w}{x}}{P})}}}
        \arrow[r, "\pcp{\red}"]
        \arrow[d, "\cpgvC{\cdot}"]
        &
        \pcp{\tm{\subst{P}{w}{x'}}}
        \arrow[dd, "\cpgvC{\cdot}"]
        \\
        \pgv{\tm{\res{x}{x'}{(\ppar{\child\;\link\;\pair{w}{x}}{\cpgvC{P}})}}}
        \arrow[d, "\pgv{\cred^+}"]
        \\
        \pgv{\tm{\subst{\cpgvC{P}}{w}{x'}}}
        \arrow[r, "\pcp{=}"]
        &
        \pgv{\tm{\cpgvC{\subst{P}{w}{x'}}}}
      \end{tikzcd}
    \end{mathpar}
  \end{case*}
  \begin{case*}[\LabTirName{E-Send}]
    \begin{mathpar}
      \begin{tikzcd}
        \pcp{\tm{\res{x}{x'}{(\ppar{\send{x}{y}{P}}{\recv{x'}{y'}{Q}})}}}
        \arrow[d, "\cpgvM{\cdot}"]
        \arrow[r, "\pcp{\red}"]
        &
        \pcp{\tm{\res{x}{x'}{\res{y}{y'}{(\ppar{P}{Q})}}}}
        \arrow[dd, "\cpgvC{\cdot}"]
        \\
        \pgv{\tm{
            \setlength{\arraycolsep}{0pt}
            \res{x}{x'}{}\left(
              \begin{array}{l}
                \child\bigg(
                  \begin{array}{l}
                    \letpair{y}{y'}{\new}{}
                    \\
                    \letbind{x}{\send\;{\pair{y}{x}}}{\cpgvM{P}}
                  \end{array}
                \bigg)
                \parallel
                \\
                \child\hphantom{\bigg(}
                \letpair{y'}{x'}{\recv\;{x'}}{\cpgvM{Q}}
              \end{array}
            \right)}}
        \arrow[d, "\pgv{\cred^+}"]
        \\
        \pgv{\tm{\res{x}{x'}{\res{y}{y'}{(\ppar{\child\;\cpgvM{P}}{\child\;\cpgvM{Q}})}}}}
        \arrow[r, "\pgv{\cred^\star}", "\text{(by \cref{lem:pcp-to-pgv-cpgvM-cpgvC-completeness})}"']
        &
        \pgv{\tm{\res{x}{x'}{\res{y}{y'}{(\ppar{\cpgvC{P}}{\cpgvC{Q}})}}}}
      \end{tikzcd}
    \end{mathpar}
  \end{case*}
  \begin{case*}[\LabTirName{E-Close}]
    \begin{mathpar}
      \begin{tikzcd}
        \pcp{\tm{\res{x}{x'}{(\ppar{\close{x}{P}}{\wait{x'}{Q}})}}} 
        \arrow[d, "\cpgvM{\cdot}"]
        \arrow[r, "\pcp{\red}"]
        &
        \pcp{\tm{\ppar{P}{Q}}}
        \arrow[dd, "\cpgvC{\cdot}"]
        \\
        \pgv{\tm{\res{x}{x'}{(\ppar
              {\child\;\andthen{\close\;{x}}{\cpgvM{P}}} 
              {\child\;\andthen{\wait\;{x'}}{\cpgvM{Q}}}
              )}}}
        \arrow[d, "\pgv{\cred^+}"]
        \\ 
        \pgv{\tm{\ppar{\child\;\cpgvM{P}}{\child\;\cpgvM{Q}}}}
        \arrow[r, "\pgv{\cred^\star}", "\text{(by \cref{lem:pcp-to-pgv-cpgvM-cpgvC-completeness})}"']
        & 
        \pgv{\tm{\ppar{\cpgvC{P}}{\cpgvC{Q}}}}
      \end{tikzcd}
    \end{mathpar}
  \end{case*}
  \begin{case*}[\LabTirName{E-Select-Inl}]
    \begin{mathpar}
      \begin{tikzcd}
        \pcp{\tm{\res{x}{x'}{(\ppar{\inl{x}{P}}{\offer{x}{Q}{R}})}}}
        \arrow[d, "\cpgvM{\cdot}"]
        \arrow[r, "\pcp{\red}"]
        &
        \pcp{\tm{\res{x}{x'}{(\ppar{P}{Q})}}}
        \arrow[dd, "\cpgvC{\cdot}"]
        \\
        \pgv{\tm{\res{x}{x'}{}\left(
              \begin{array}{l}
                \child\;\letbind{x}{\select{\labinl}\;{x}}{\cpgvM{P}}
                \parallel
                \\
                \child\;\offer{x'}{x'}{\cpgvM{Q}}{x'}{\cpgvM{R}}
              \end{array}
            \right)}}
        \arrow[d, "\pgv{\cred^+}"]
        \\ 
        \pgv{\tm{\res{x}{x'}{(\ppar{\child\;\cpgvM{P}}{\child\;\cpgvM{Q}})}}}
        \arrow[r, "\pgv{\cred^\star}", "\text{(by \cref{lem:pcp-to-pgv-cpgvM-cpgvC-completeness})}"']
        & 
        \pgv{\tm{\res{x}{x'}{(\ppar{\cpgvC{P}}{\cpgvC{Q}})}}}
      \end{tikzcd}
    \end{mathpar}
  \end{case*}
  \begin{case*}[\LabTirName{E-Select-Inr}]
    \begin{mathpar}
      \begin{tikzcd}
        \pcp{\tm{\res{x}{x'}{(\ppar{\inr{x}{P}}{\offer{x}{Q}{R}})}}}
        \arrow[d, "\cpgvM{\cdot}"]
        \arrow[r, "\pcp{\red}"]
        &
        \pcp{\tm{\res{x}{x'}{(\ppar{P}{R})}}}
        \arrow[dd, "\cpgvC{\cdot}"]
        \\
        \pgv{\tm{\res{x}{x'}{}\left(
              \begin{array}{l}
                \child\;\letbind{x}{\select{\labinr}\;{x}}{\cpgvM{P}}
                \parallel
                \\
                \child\;\offer{x'}{x'}{\cpgvM{Q}}{x'}{\cpgvM{R}}
              \end{array}
            \right)}}
        \arrow[d, "\pgv{\cred^+}"]
        \\ 
        \pgv{\tm{\res{x}{x'}{(\ppar{\child\;\cpgvM{P}}{\child\;\cpgvM{R}})}}}
        \arrow[r, "\pgv{\cred^\star}", "\text{(by \cref{lem:pcp-to-pgv-cpgvM-cpgvC-completeness})}"']
        & 
        \pgv{\tm{\res{x}{x'}{(\ppar{\cpgvC{P}}{\cpgvC{R}})}}}
      \end{tikzcd}
    \end{mathpar}
  \end{case*}
  \begin{case*}[\LabTirName{E-LiftRes}]
    By the induction hypothesis and \LabTirName{E-LiftC}.
  \end{case*}
  \begin{case*}[\LabTirName{E-LiftPar}]
    By the induction hypotheses and \LabTirName{E-LiftC}.
  \end{case*}
  \begin{case*}[\LabTirName{E-LiftSC}]
    By the induction hypothesis, \LabTirName{E-LiftSC}, and \cref{lem:pcp-to-pgv-confs-operational-correspondence-equiv}.
  \end{case*}
\end{proof}

%%% Local Variables:
%%% TeX-master: "../priorities"
%%% End:

}

\end{document}

%%% Local Variables:
%%% mode: latex
%%% TeX-master: t
%%% End: