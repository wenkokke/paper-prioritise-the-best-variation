\begin{proof}
  \label{prf:thm-pcp-closed-progress}
  By~\cref{lem:pcp-canonical-forms}, we rewrite $\tm{P}$ to canonical form. If the resulting process is $\tm{\halt}$, we are done. Otherwise, it is of the form
  \[
    \seq{\res{x_1}{x'_1}{\dots\res{x_n}{x'_n}{(P_1 \parallel\dots\parallel P_m)}}}{\emptyenv}
  \]
  where $m>0$ and each $\seq{P_i}{\ty{\Gamma_i}}$ is an action.

  Our proof follows the same reasoning by Kobayashi~\cite{kobayashi06} used in the proof of deadlock freedom for closed processes (Theorem 2). 

  Consider processes $\tm {P_1 \parallel\dots\parallel P_m}$. Among them, we pick the process with the smallest priority $\minpr{(\ty{\Gamma_i})}$ for all $i$. Let this process be let this be $\tm{P_i}$ and the priority of the top prefix be $\cs o$. $\tm{P_i}$ acts on some endpoint $\tmty{y}{A}\in\ty{\Gamma_i}$. We must have $\minpr{(\ty{\Gamma_i})}=\pr{(\ty{A})} = \cs o$, since the other actions in $\tm{P_i}$ are guarded by the action on $\tmty{y}{A}$, thus satisfying law (i) of priorities.

  If $\tm{P_i}$ is a link $\tm{\link{y}{z}}$ or $\tm{\link{z}{y}}$, we apply \LabTirName{E-Link}.

  Otherwise, $\tm{P_i}$ is an input/branching or output/selection action on endpoint $\tm y$ of type $\ty A$ with priority $\cs o$. Since process $\tm P$ is closed and consequently it respects law (ii) of priorities, there must be a co-action $y'$ of type $\ty{\co{A}}$  where $\tm{y}$ and $\tm{y'}$ are dual endpoints of the same channel (by application of rule \textsc{T-Res}). By duality, $\pr{(\ty{A})}=\pr{(\ty{\co{A}})}= \cs o$. In the following we show that: $y'$ is the subject of a top level action of a process $P_j$ with $i\neq j$. This allows for the communication among $\tm{P_i}$ and $\tm{P_j}$ to happen immediately over channel endpoints $\tm{y}$ and $\tm{y'}$.
  
  Suppose that $\tm{y'}$ is an action not in a different parallel process $P_j$ but rather of $P_i$ itself. That means that the action on $\tm{y'}$ must be prefixed by the action on $\tm{y}$, which is top level in $\tm{P_i}$. To respect law (i) of priorities we must have $\cs o < \cs o$, which is absurd. This means that $\tm{y'}$ is in another parallel process $\tm{P_j}$ for $i\neq j$.

  Suppose that $\tm{y'}$ in $\tm{P_j}$ is not at top level. In order to respect law (i) of priorities, it means that $\tm{y'}$ is prefixed by actions that are smaller than its priority $\cs o$. This leads to a contradiction because stated that $\cs o$ is the smallest priority. Hence, $y'$ must be the subject of a top level action.
  
  We have two processes, acting on dual endpoints. We apply the appropriate reduction rule, \ie \LabTirName{E-Send}, \LabTirName{E-Close}, \LabTirName{E-Select-Inl}, or \LabTirName{E-Select-Inr}.
\end{proof}

%%% Local Variables:
%%% TeX-master: "../priorities"
%%% End:
