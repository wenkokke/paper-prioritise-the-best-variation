% !TeX root = priorities.tex
\documentclass[main.tex]{subfiles}

\begin{document}
\section{Related Work and Conclusion}

\paragraph{On deadlock freedom and progress in $\pi$-calculus and $\lambda$-calculus}
Deadlock freedom and progress have been studied thoroughly in the $\pi$-calculus with or without session types.
In the standard typed $\pi$-calculus a foundational line of work is by Kobayashi and his original type-theoretic approach to deadlock-freedom~\cite{kobayashi98}, where priorities are abstract tags defined over a partially ordered set. Later on these abstract tags were simplified to natural numbers, and pairs of obligations and capabilities were used in the type system for deadlock or livelock freedom~\cite{kobayashi02,kobayashi06}. This allowed more $\pi$-calculus processes to be typed. Following, Kobayashi's line of work, Padovani~\cite{padovani13} adapted the obligation and capability pairs to session types, and later on he further simplified them to a single priority for linear $\pi$-calculus~\cite{padovani14}. As discussed by the author, by using the encoding of session types into linear types~\cite{kobayashi07,dardhagiachino12,dardha14beat,dardha16}, the priority-based technique for deadlock freedom can be transferred onto the $\pi$-calculus with session types.

The groundbreaking work on progress for session-typed $\pi$-calculus, by Dezani \emph{et al.}~\cite{dezani-ciancaglinimostrous06}, guarantees progress by allowing only one active session at a time. In~\cite{dezani-ciancagliniliguoro09progress} the authors introduce a partial order on channels in line with Kobayashi's work~\cite{kobayashi98}. Carbone and Debois~\cite{carbonedebois10} define progress for session typed $\pi$-calculus in terms of a \emph{catalyser} used to provide a missing counterpart to a process, thus guaranteeing progress.
Carbone \emph{et al.}~\cite{carbonedardha14} studied further the use of catalysers and showed that progress is a compositional form of lock-freedom. As in~\cite{padovani14} the authors show that by using the encoding of session types~\cite{dardhagiachino12} and Kobayashi's obligations/capabilities, we can obtain progress for session types. Vieira and Vasconcelos~\cite{vieiravasconcelos13} used single priorities and an abstract partial order to guarantee deadlock freedom in a session-typed $\pi$-calculus.

Gay \emph{et al.} \cite{gaynagarajan03} and Vasconcelos \emph{et al.} \cite{vasconcelosravara04,vasconcelosgay06} were the first to introduce a functional language with session types. However, such works and including the GV line \cite{gayvasconcelos10,gayvasconcelos12} did not guarantee deadlock freedom. This property was later on addressed \cite{lindleymorris15,wadler15} via syntactic restrictions allowing only communication in tree structures, as discussed in the introduction. Toninho \emph{et al.} \cite{toninhocaires12} present an encoding of simply-typed $\lambda$-calculus into session-typed $\pi$-calculus investigating the concurrency features of $\lambda$-calculus and providing a new logical explanation. Again, their focus is not on deadlock freedom.


\paragraph{On Curry-Howard correspondences}
The Curry-Howard correspondence between logic and types laid the foundation for functional programming~\cite{wadler15}. With the rise of linear logic~\cite{girard87} a Curry-Howard correspondence was established between linear logic and the linear $\pi$-calculus~\cite{abramsky94,bellinscott94}. With the rise of session types, a new correspondence between linear logic and the $\pi$-calculus was proposed first by Caires and Pfenning~\cite{cairespfenning10}, for dual intuitionistic linear logic (DILL), and later by Wadler~\cite{wadler15} for classical linear logic (CLL). The underlying language in both correspondences satisfy deadlock freedom by design, by forbidding cyclic connections via the logical cut rule. Dardha and Gay follow Padovani's work~\cite{padovani14} and use priorities in CP to define a new calculus which is deadlock-free by design, based on classical linear logic correspondence with session types~\cite{wadler12}. Dardha and P\'{e}rez~\cite{dardhaperez15} compare Kobayashi-style typing and CLL typing, and prove that CLL corresponds to Kobayashi's system with the restriction that only single cuts are allowed. Balzer \emph{et al.} study sharing~\cite{balzerpfenning17} and later on deadlock freedom~\cite{balzertoninho19} in a session-typed $\pi$-calculus based on the correspondence with DILL. In~\cite{balzerpfenning17} types are either linear or shared with modal operators connecting them, which allows sharing, but not deadlock freedom. In~\cite{balzertoninho19} types are enhanced with information regarding the order of acquiring resources, which guarantees deadlock freedom.

\paragraph{Conclusion and Future Work}
In this paper we presented Priority GV, which is a functional language with session types. We used priorities in session types, following Priority CP~\cite{dardhagay18} to guarantee deadlock freedom statically via our priority-based type system. We proved the progress property of PGV in \Cref{lem:pgv-closed-progress-confs}. We also presented an updated version of PCP, where we moved from commuting conversions to structural congruence in the reduction relation, thus remaining faithful to the reduction relation of the $\pi$-calculus. Finally, we presented an encoding of Priority CP to Pririty GV in \Cref{sec:pcp-to-pgv}. For this translation we proved that it is sound with respect to typing (\Cref{thm:pcp-to-pgv-confs-preservation}) and operational semantics, given by operational correspondence soundness (\Cref{thm:pcp-to-pgv-operational-correspondence-soundness}) and completeness (\Cref{thm:pcp-to-pgv-operational-correspondence-completeness}).
We identify the following main directions for future work: explore recursion in PGV; define the inverse encoding from Priority GV to Priority CP; add sharing to PGV and PCP following the lines of \cite{balzerpfenning17}.
\end{document}