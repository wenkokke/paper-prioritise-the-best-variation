% !TeX root = priorities.tex
\documentclass[main.tex]{subfiles}

\begin{document}
\section{Related Work and Conclusion}

\paragraph{On GV}

\paragraph{On deadlock freedom and progress in the $\pi$-calculus}
Deadlock freedom and progress have been studied thoroughly in the $\pi$-calculus with or without session types.

In the standard typed $\pi$-calculus a seminal line of work is by Kobayashi and his original type-theoretic approach to deadlock-freedom \cite{K98}, where priorities are abstract tags defined over a partially ordered set. Later on these abstract tags were simplified to natural numbers, and pairs of obligations and capabilities were used in the type system for deadlock freedom \cite{K02,K06}. This allowed more $\pi$-calculus processes to be typed. Following, Kobayashi's line of work, Padovani \cite{P13} adapted obligation/capability pairs to session types, and later on he further simplified them to a single priority for linear $\pi$-calculus \cite{P14}. As discussed by the author, by using the encoding of session types into linear types \cite{Koba07,DGS12,D14,Dardha16}, the deadlock freedom technique of the linear $\pi$-calculus with priorities can be transferred onto the $\pi$-calculus with session types.

The seminal work on progress for session-typed $\pi$-calculus, by Dezani \emph{et al.} \cite{DMYD06}, guarantees progress by allowing only one active session at a time. In \cite{DLY07} the authors introduce a partial order on channels in line with Kobayashi's work \cite{K98}. Carbone and Debois \cite{CD10} define progress for session typed $\pi$-calculus in terms of a \emph{catalyser} used to provide a missing counterpart to a process, thus guaranteeing progress.
Carbone \emph{et al.} \cite{CDM14} studied further the use of catalysers and showed that progress is a compositional form of lock-freedom. As in \cite{P14} the authors show that by using the encoding of session types \cite{DGS12} and Kobayashi's obligations/capabilities, we can obtain progress in for session types. Vieira and Vasconcelos \cite{VieiraV13} used single priorities and an abstract partial order to guarantee deadlock freedom in session-typed $\pi$-calculus.

\paragraph{On Curry-Howard correspondences}
The Curry-Howard correspondence between logic and types laid the foundation for functional programming \cite{Wadler15}. With the rise of linear logic \cite{Girard87} a Curry-Howard correspondence was established between linear logic and the linear $\pi$-calculus \cite{Abramsky94,BellinS94}. With the rise of session types, a new correspondence between linear logic and the $\pi$-calculus was proposed first by Caires and Pfenning \cite{CP10}, for dual intuitionistic linear logic (DILL), and later by Wadler \cite{wadler2012} for classical linear logic (CLL). The underlying language in both correspondences satisfy deadlock freedom by design, by forbidding cyclic connections via typing. Dardha and Gay follow Padovani's work \cite{P14} and use priorities in CP to define a new calculus which is deadlock-free by design, based on classical linear logic \cite{Girard87}  and the Curry-Howard correspondence with session types \cite{wadler2012}. Dardha and P\'{e}rez \cite{DardhaP15} compare Kobayashi-style typing and CLL typing, and prove that CLL corresponds to Kobayashi's system with the restriction that only single cuts are allowed. Balzer \emph{et al.} study sharing \cite{BalzerP17} and later on deadlock freedom \cite{BalzerTP19} in session-typed $\pi$-calculus based on DILL. In \cite{BalzerP17} types are either linear or shared with modal operators connecting them, which allows sharing, but not deadlock freedom. Later on in \cite{BalzerTP19} types are enhanced with information regarding the order of acquiring resources, which guarantees deadlock freedom.

\paragraph{Conclusion}
In this paper...
\end{document}