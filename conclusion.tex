% !TeX root = priorities.tex
\documentclass[main.tex]{subfiles}

\begin{document}
\section{Related Work and Conclusion}
\textit{Simon: I think you really need some mention of MPSTs here, since a key motivation for MPSTs is ensuring deadlock-freedom between participants in a single MPST session. You can then talk about the global progress work by Coppo~\etal, too, which incorporates an 'interaction typing system' (that few people understand). How does that relate?}

\subsubsection*{Deadlock freedom}
Deadlock freedom and progress are well-studied in the $\pi$-calculus.
For the standard typed $\pi$-calculus, one line of work is Kobayashi's approach to deadlock-freedom~\cite{kobayashi98}, where priorities are abstract tags defined over a partially ordered set. These abstract tags were later simplified to natural numbers. Pairs of obligations and capabilities were used in the type system for lock freedom~\cite{kobayashi02,kobayashi06}, which allowed more $\pi$-calculus processes to be typed. Padovani~\cite{padovani13} adapted the obligation and capability pairs to session types, and later simplified them to a single priority for linear $\pi$-calculus~\cite{padovani14}. Using the encoding of session types into linear types~\cite{kobayashi07,dardhagiachino12,dardha14beat,dardha16}, the priority-based technique for deadlock freedom can be transferred onto the $\pi$-calculus with session types.

For the session-typed $\pi$-calculus, the foundational work by Dezani~\etal~\cite{dezani-ciancaglinimostrous06} guarantees progress by allowing only one active session at a time. Dezani later~\cite{dezani-ciancagliniliguoro09progress} introduces a partial order on channels in line with Kobayashi's work~\cite{kobayashi98}. Carbone and Debois~\cite{carbonedebois10} define progress for session typed $\pi$-calculus in terms of a \emph{catalyser} used to provide a missing counterpart to a process, thus guaranteeing deadlock freedom.
Carbone~\etal~\cite{carbonedardha14} studied the use of catalysers and showed that progress is a compositional form of lock-freedom. Following Padovani~\cite{padovani14}, Dardha~\etal~\cite{dardhagiachino12} show that by using the encoding of session types and Kobayashi's obligations/capabilities, we can obtain progress for session types.\wen{I'm not sure what this means.} Vieira and Vasconcelos~\cite{vieiravasconcelos13} used single priorities and an abstract partial order to guarantee deadlock freedom in a session-typed $\pi$-calculus.

Gay~\etal~\cite{gaynagarajan03} and Vasconcelos~\etal~\cite{vasconcelosravara04,vasconcelosgay06} were the first to introduce a functional language with session types. However, such works, including early GV~\cite{gayvasconcelos10,gayvasconcelos12} did not guarantee deadlock freedom, until it was addressed via syntactic restrictions~\cite{lindleymorris15,wadler15}. Toninho~\etal~\cite{toninhocaires12} present a translation of simply-typed $\lambda$-calculus into session-typed $\pi$-calculus. However, their focus is not on deadlock freedom.

\subsubsection*{Ties with logic}
The correspondence between logic and types lays the foundation for functional programming~\cite{wadler15}. Since its inception by Girard~\cite{girard87}, linear logic has been a candidate for a foundational correspondence for concurrent programs. A~correspondence with linear $\pi$-calculus was established early on by Abramsky~\cite{abramsky94,bellinscott94}. A~correspondence between session-typed $\pi$-calculus and dual intuitionistic linear logic was developed by Caires and Pfenning~\cite[$\pi\text{DILL}$]{cairespfenning10}, and with classical linear logic by Wadler~\cite[CP]{wadler15}, both guaranteeing deadlock freedom as a result of their connection to logic. Dardha and Gay~\cite[PCP]{dardhagay18} integrate Padovani's work on priorities~\cite{padovani14} with CP, creating the first calculus which combines priorities and strong ties with logic. Dardha and P\'{e}rez~\cite{dardhaperez15} compare Kobayashi-style typing and CLL typing, and show that CLL corresponds to Kobayashi's system with the restriction that only single cuts are allowed.\wen{What do you mean by ``single cuts''?} Balzer~\etal~\cite[$\text{SILL}_S$]{balzerpfenning17} introduce shared state, which breaks deadlock freedom. They later restore deadlock freedom using priorities~\cite[$\text{SILL}_{S+}$]{balzertoninho19}.

\subsubsection*{Conclusion and Future Work}
We presented Priority GV, a~session-typed functional language which uses priorities to ensure deadlock freedom, and showed its relation to Priority CP~\cite{dardhagay18} via an operational correspondence.

Our formalism so far only captures the core of GV. In future work, we plan to explore: recursion in PGV, by integrating the works of Lindley and Morris~\cite{lindleymorris16} and Padovani and Novara~\cite{padovaninovara15}; a~translation from Priority GV to Priority CP; and sharing, following Balzer and Pfenning~\cite{balzerpfenning17}.
\end{document}