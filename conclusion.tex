% !TeX root = priorities.tex
\documentclass[main.tex]{subfiles}

\begin{document}
\section{Related Work and Conclusion}
\subsubsection*{Deadlock freedom and progress}
Deadlock freedom and progress are well studied properties in the $\pi$-calculus.
For the `standard' typed $\pi$-calculus, one line of work is Kobayashi's approach to deadlock freedom~\cite{kobayashi98}, where priorities are abstract tags defined over a partially ordered set. These tags were later simplified to pairs of natural numbers, called obligations and capabilities~\cite{kobayashi06}, which allowed more $\pi$-calculus processes to be typed. Padovani~\cite{padovani13} adapted the obligation and capability pairs to session types, and later simplified them to a single priority for the linear $\pi$-calculus~\cite{padovani14}.  Furthermore, by exploiting the encoding of session types into linear types~\cite{kobayashi07,dardhagiachino12,dardha14beat} and the priority-based linear $\pi$-calculus,
we can obtain deadlock freedom for the $\pi$-calculus with session types.

For the session-typed $\pi$-calculus, Dezani~\etal~\cite{dezani-ciancaglinimostrous06} guarantee progress by allowing only one active session at a time. Dezani later~\cite{dezani-ciancagliniliguoro09progress} introduces a partial order on channels similar to Kobayashi~\cite{kobayashi98}. Carbone and Debois~\cite{carbonedebois10} define progress for session typed $\pi$-calculus in terms of a \emph{catalyser}, which provides a missing counterpart to a process, thus guaranteeing deadlock freedom.
Carbone~\etal~\cite{carbonedardha14} use such catalysers to show that progress is a compositional form of lock-freedom and that it can be lifted to session types via the encoding of session types to linear types. Vieira and Vasconcelos~\cite{vieiravasconcelos13} use single priorities and an abstract partial order to guarantee deadlock freedom in a binary session-typed $\pi$-calculus and building on conservation types.

While our work focuses on \emph{binary} session types, it is worth to discuss related work on Multiparty Session Types (MPST) in order to give a broader context. The original work on MPST by Honda~\etal~\cite{hondayoshida08} guarantees deadlock freedom \emph{within a single} session, but not for session interleaving.
Later on, several techniques for deadlock freedom in MPST were defined.
Bettini~\etal~\cite{bettinicoppo08} follow a similar technique to Kobayashi's for the MPST setting. The main difference with our work is that we associate priorities with communication actions, whether Bettini~\etal~\cite{bettinicoppo08} with channels. Carbone and Montesi~\cite{carbonemontesi13} explore and combine MPST and choreographies to obtain a formalism that satisfies deadlock freedom by design.
Deni{\'{e}}lou and Yoshida \cite{DenielouY13} introduce the notion of \emph{multiparty compatibility} which generalises that of duality in binary session types and they synthesise safe and deadlock-free global types from local session types via the use of LTSs and communicating automata (CFSA).
Castellani~\etal~\cite{CastellaniDGH20} guarantee lock freedom--a stronger property than deadlock freedom--in the context of MPST with \emph{internal delegation}, where participants are allowed to delegate tasks to each other as long as they fall within the same multiparty session. This allows for the internal delegation to be captured by the global type.
Scalas and Yoshida \cite{scalasyoshida19} provide a revision of the theoretical foundations of MPST leading to a less complicated and more general theory, by completely removing the notion of duality/consistency. The type systems is parametric and ensures decidability of type checking, while allowing for a novel integration of model checking techniques in MPST. In this new theory, more protocols and processes can be typed that are guaranteed to be free of deadlocks. 

Gay~\etal~\cite{gaynagarajan03} and Vasconcelos~\etal~\cite{vasconcelosravara04,vasconcelosgay06} were the first to introduce a functional language with session types. However, such works, including early GV~\cite{gayvasconcelos10,gayvasconcelos12} did not guarantee deadlock freedom, until it was addressed via syntactic restrictions~\cite{lindleymorris15,wadler14}. Toninho~\etal~\cite{toninhocaires12} present a translation of simply-typed $\lambda$-calculus into session-typed $\pi$-calculus. However, their focus is not on deadlock freedom.

\subsubsection*{Ties with logic}
The correspondence between logic and types lays the foundation for functional programming~\cite{wadler15}. Since its inception by Girard~\cite{girard87}, linear logic has been a candidate for a foundational correspondence for concurrent programs. A~correspondence with linear $\pi$-calculus was established early on by Abramsky~\cite{abramsky94} and Bellin and Scott~\cite{bellinscott94}. Many years later, a correspondence between the $\pi$-calculus with binary session types and linear logic was proposed: Caires and Pfenning~\cite[$\pi\text{DILL}$]{cairespfenning10} proposed a correspondence with dual intuitionistic linear logic, while Wadler~\cite[CP]{wadler14} proposed a correspondence with classical linear logic. Both works guarantee deadlock freedom as result of their connection to logic and in particular due to the fact that the cut rule only allows tree-structured processes. Dardha and Gay~\cite[PCP]{dardhagay18} integrate Kobayashi and Padovani's work on priorities~\cite{kobayashi06,padovani14} with CP, creating the first calculus which combines priorities and logic. They prove cycle-elimination, in the same lines as cut-elimination for linear logic, from which the meta-theory and deadlock freedom follow. Dardha and P\'{e}rez~\cite{dardhaperez15} compare Kobayashi-style typing and CLL typing, and show that CLL corresponds to a subsystem of Kobayashi's where restriction is applied once. Balzer~\etal~\cite[$\text{SILL}_S$]{balzerpfenning17} introduce sharing, which breaks deadlock freedom. They later restore deadlock freedom using priorities~\cite[$\text{SILL}_{S+}$]{balzertoninho19}.
Carbone~\etal~\cite{CarboneMSY15,carbonelindley16} give a logical view of MPST with a generalised duality, called \emph{coherence}.
Caires and P\'{e}rez~\cite{CairesP16} give a novel presentation of MPST in terms of binary session types and the use of a \emph{medium process} which guarantee protocol fidelity and deadlock freedom. Their binary session types are rooted in linear logic following the proofs-as-processes correspondences.
Ciobanu and Horne~\cite{CiobanuH15} give the first processes-as-formulae correspondence of MPST and \emph{the calculus of structures} by \cite{Guglielmi07}, and give a calculus where global and local session types are specified inspired by Scribble\footnote{\url{http://www.scribble.org/}}. The calculus is then used to determine whether a set of local session types can be composed safely and in a deadlock free manner.
Horne \cite{Horne20} presents a proof system for subtyping and multiparty compatibility where compatible processes are race free and deadlock free. This work also gives a processes-as-formulae correspondence--similar to the approach in~\cite{CiobanuH15}--with a non-commutative extension of linear logic. As such, the meta-theoretical results of this work are obtained in terms of cut-elimination.

\subsubsection*{Conclusion and future work}
We answered our research question by presenting Priority GV, a~session-typed functional language which allows cyclic communication structures and uses priorities to ensure deadlock freedom. We showed its relation to Priority CP~\cite{dardhagay18} via an operational correspondence.

Our formalism so far only captures the core of GV. In future work, we plan to explore: recursion in PGV, by integrating the works of Lindley and Morris~\cite{lindleymorris16} and Padovani and Novara~\cite{padovaninovara15}; or in the lines of Kobayashi and Laneve for unbounded networks~\cite{KL17}, which is a refinement of previous work by Kobayashi. %a~translation from Priority GV to Priority CP;
Another interesting future work is on sharing, following Balzer and Pfenning~\cite{balzerpfenning17}, or Voinea \etal~\cite{VoineaDG19}.

\subsubsection*{Acknowledgement}
The authors would like to thank Simon Fowler, April Gon\c{c}alves, and Philip Wadler for their comments on the manuscript.

\end{document}
